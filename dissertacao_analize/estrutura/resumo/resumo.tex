Diversas pessoas são submetidas a algum tipo de risco ao executar atividades profissionais relacionadas à área da eletricidade, petroquímica, transportes e outras. Com base em entrevista com um Engenheiro de Manutenção, participação \textbf{in loco} em uma atividade de manutenção na área de transmissão de energia elétrica e revisão bibliográfica exploratória, esse estudo sintetiza as informações coletadas, constrói e avalia um modelo conceitual destinado à construção de sistemas computacionais de simulação de riscos e acidentes com fins diversos, tais como, treinamento de operadores ou estudo de riscos associados a uma atividade laboral. O modelo conceitual construído fundamenta-se nos conceitos de agentes, artefatos, sistemas multiagentes normas e riscos/consequências. Atividades podem ser expressas em termos de objetivos com pré-requisitos e efeitos. A preocupação central é a modelagem de riscos e de suas consequências. Deste modo, pré-requisitos de atividades que não estão presentes quando os agentes decidem executá-las levam a violações que podem causar consequências para o objetivo atual e/ou subsequentes (ex. parada) assim como para os agentes envolvidos (ex. acidente). O modelo, formalizado em Lógica de Primeira Ordem, foi implementado em PROLOG e aplicado a um estudo de caso permitindo avaliar se as inferências produzidas correspondiam ao esperado bem como compreender o estado do problema particular modelado. Além disso, esta formalização permitiu realizar uma análise comparativa (em relação a cenários de acidentes) entre quatro modelos, MOISE+, Modelo de Agentes Normativos de DASTANI, V3S e NORMMAS. Como conclusão, o modelo é uma síntese de vários modelos existentes reunindo em um só as características relevantes para tal tipo de modelagem. Também, se revelou capaz de representar diversos cenários atrelados a atividades profissionais de risco, em especial, mostrou-se capaz de produzir resultados coerentes com o esperado no estudo de caso.