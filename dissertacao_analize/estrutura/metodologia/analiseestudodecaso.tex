Um estudo de caso é de crucial importância para analisar esse modelo. Isso só pode ser feito se o caso escolhido se o caso está dentro desses requisitos desse estudo. Uma situação onde isso se enquadra perfeitamente é a manutenção de equipamentos elétricos em linha viva. Nesse tipo de situação, os procedimentos de manutenção acontecem sobre esses equipamentos sem que o fluxo de potência seja devidamente interrompido. Esses procedimentos acontecem, normalmente, em subestações e em linhas de transmissão. 

Esse caso é interessante porque apresenta a situação onde profissionais devem executar algum tipo de procedimento operacional colaborativo em um ambiente que contem diversos riscos a vida desses profissionais (como de ser eletrocutado sobre altos níveis de potência).

Uma vez feita a identificação desse estudo de caso, o autor fez esforço em compreender como ocorre os procedimentos em linha viva e isso se deu de três maneiras: 
estudo de manuais técnicos (privativos a uma dada concenionária de energia), entrevista com o engenheiro de manutenção de linha viva de uma dada conceionária de energia e acompanhamento de um procedimento de manutenção (em linha viva) de campo. Isso forneceu o entendimento necessário para o autor delimitar um caso específico de manutenção. Durante todo o procedimento de especificação no modelo em análise, o autor teve acesso aos profissionais da área de linha viva e isso ocasionou diversas correções de especificação.  