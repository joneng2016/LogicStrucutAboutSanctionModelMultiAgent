Existe uma gama de possibilidades de linguagens (de programação ou não) que o autor pode usar para implementar esse estudo de caso. Nesta pesquisa, o Prolog foi escolhido para essa possibilidade por ser uma linguagem lógico-matemática simples, enchuta e que estrutura o código em termos de variáveis (representam os Conceitos ou Conjuntos), átomos (representam os elementos dos conjuntos),  fatos (são os predicados em relação aos átomos) e regras (corresponde a regras de implicabilidade). A maneira de verificar um dado raciocíncio se dá por realizar consultas a estrutura especificada nessa linguage. 

Essas características tornam o \textit{Prolog} uma linguagem apropriada para os interesses dessa pesquisa que consiste na realização de prova de conceito sobre os raciocínios que devem ser demonstrados. 