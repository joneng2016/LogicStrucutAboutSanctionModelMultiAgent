Em síntese é possível concluir que o modelo conceitual concebido nesse estudo contém os conceitos e relações relevantes para representar ambientes de acidentes averiguando causas e consequências. No que tange ao estudo de caso e a validação é possível concluir que o modelo foi capaz de abranger uma dada quantidade de cenários (sendo esses os mais relevantes para os contextos em análise) possíveis dentro de uma certa situação. Sobre averiguação de um entendimento formal do problema é possível concluir que uma análise computacional (seja por meio de modelos computacionais, ou seja por meio de algoritmos), que tenha como por meta trabalhar com acidentes de trabalho, deve considerar pelo menos alguns dos seguintes conceitos: Agente, \textit{SMA}, Artefato, Norma, Violação, Sanção, Riscos, Objetivos, Condições Ambientes, Interações entre Entidades (Agentes, Artefatos) e Eventos Probabilísticos. Sobre os tipos de representação computacional que se enquadram no domínio em estudo, conclui-se o que se encontra na lista que se segue:
\begin{itemize}
    \item \textit{MOISE+} é mais apropriado para representar os seguintes conceitos: \textit{Agente, SMA, objetivos}.
    \item \textit{Dastani} é mais apropriado para representar os seguintes conceitos \textit{Normas, Violações, Sanções}.
    \item \textit{V3S} é mais apropriado para representar \textit{SMA, Artefato, Riscos e contém estruturas otimizadas para descrever dinamicamente os cenários de acidentes}.
    \item \textit{NORMMAS} é mais apropriado para representar \textit{Normas, Violações, Sanções}.
\end{itemize}

No que tange ao contraste do modelo conceitual proposto nesse estudo em relação aos arcabouços verificados nesse texto, conclui-se que aquele unifica em uma única estrutura concepções que são tratadas de formas isoladas nestes, no que diz respeito a tratativa de acidentes em ambientes de trabalho. Então, a estrutura proposta nesse texto apresenta mecanismos para tratar de Agente, SMA, Artefato, Norma, Violação, Sanção, Risco, Possibilidades de ocorrer algum evento ruim, Objetivos, Integração entre Agentes e Artefatos e Descrição dos Cenários de Acidentes. No \textit{MOISE+}, por exemplo, os mecanismos deste arcabouço tratam apenas de Agente, SMA e Objetivos.