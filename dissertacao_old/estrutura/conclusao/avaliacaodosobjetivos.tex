Essa Seção têm o propósito de gerar uma reflexão sobre se o estudo presente nesta dissertação atingiu o objetivo geral bem como os objetivos específicos a que se propõem. A Subseção \ref{reflexobgeral} apresenta essa reflexão para o objetivo geral e a subseção \ref{reflexobespecifico} apresenta as reflexões a respeito dos objetivos específicos.

\subsection{Reflexão sobre Objetivo Geral}\label{reflexobgeral}

Essa seção apresenta uma análise detalhada do correspondente do objetivo geral proposto ao texto do estudo. A seguinte parte do objetivo geral: ``\textit{Sintetizar, construir e avaliar, por intermédio de observações, de análises de documentos técnicos, de análises de modelos computacionais e de entrevista com profissionais da área}'', foi abordada no estudo dos documentos técnicos, conversas com profissionais da área, entrevistas com engenheiro de manutenção em linha viva, acompanhamento de um procedimento de manutenção de linha viva, realização da descrição de cenários, participação de \textit{workshops} na área e realização de uma revisão exploratória de textos sobre arcabouços que podem ser aplicados ao contexto desse estudo. 

A seguinte parte do objetivo geral: ``\textit{um modelo conceitual que define os conceitos e as relações para representar os cenários de ambientes de atividades, bem como os respectivos acidentes que podem acontecer}'', foi trabalhada na construção de um modelo conceitual capaz de abordar cenários de manutenção. 

A seguinte parte do objetivo geral: ``\textit{em que a validação ocorre por verificar se os raciocínios (para um dado estudo de caso do setor de energia elétrica)}'', foi tratada aplicação do modelo obtido em um estudo de caso cujo qual consiste em procedimentos de manutenção em linha viva. Esse estudo de caso apresenta tanto a modelagem como o desenvolvimento de raciocínios de determinados cenários que são previstos pelo modelo. Por intermédio de observações anteriores, foi possível avaliar se a depuração desses raciocínios está de acordo com a realidade.  

A seguinte parte do objetivo geral: ``\textit{a fim de levantar um entendimento formal do problema para a comunidade acadêmica no que tange a que tipo de representação computacional é mais apropriada para determinado contexto}'' foi abordada na avaliação do modelo conceitual em relação a arcabouços computacionais consolidados pela comunidade acadêmica e verificação  do estado do problema no que tange a determinada conjuntura. 

\subsection{Reflexão sobre Objetivos Específicos}\label{reflexobespecifico}


O objetivo específico ``\textit{Identificar os pontos essenciais que devem fazer parte da estrutura do modelo em relação aos riscos e consequências (acidentes) para os atores e atividades (continuidade), que sejam relevantes na prática da atividade de manutenção, em caso de falha na operação}'' foi tratado nos estudos de manuais técnicos, entrevista com um engenheiro da área, participação em \textit{workshops}, levantamento de manuais técnicos e acompanhamento da execução de um certo procedimento de manutenção em linha viva.  

O objetivo específico ``\textit{Construir um modelo conceitual que seja implementável computacionalmente e que produza as inferências que respondam às questões definidas como essenciais}'' foi atingido no momento da obtenção de um modelo conceitual com a capacidade de produzir raciocínios dentro do formalismo Lógica de Predicados. 

O objetivo específico ``\textit{Validar o modelo por aplicá-lo a um dado estudo de caso a fim de averiguar se os raciocínios produzidos nessa situação estão de acordo com a realidade}'' foi verificado na aplicação do modelo a um dado estudo de caso como produção de raciocínios sobre certos cenários a fim de confronta-los contra a realidade para sintetizar uma avaliação se ambos são correspondentes e em quais proporções isso ocorre. 

O objetivo específico ``\textit{Analisar modelos computacionais em relação ao modelo conceitual desse estudo a fim de ter um levantamento formal do estado do problema}'' foi atingido como na comparação do modelo conceitual com os arcabouços computacionais propostos pela comunidade acadêmica. Nessa comparação foi possível observar em quais aspectos o modelo se assemelha e se distancia de cada arcabouço permitindo mapear em termos formais o estado desse problema. 

Assim sendo, é possível concluir que esse estudo concebeu um modelo conceitual com a capacidade de representar e definir raciocínios sobre acidentes e que, por conta disso, possibilita esclarecer o contexto do problema a ser analisado a luz da computação. 