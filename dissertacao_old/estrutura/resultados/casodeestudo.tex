Esse capítulo apresenta os resultados atrelados à etapa metodológica apresentada na Seção \ref{inferencias} que tem por finalidade validar o modelo conceitual no que tange a um cenário real de manutenção com alta potencialidade para a ocorrência de acidentes. 

\section{Introdução ao Problema}

O estudo de caso desta pesquisa consiste em sete profissionais de linha viva (profissionais que realizam manutenção em equipamentos elétricos energizados) designados com o propósito de realizar a substituição de um isolador de pedestal. Os papéis desses profissionais são: um supervisor e seis executores. A manutenção deve ser executada apenas sob as seguintes condições: céu ensolarado e umidade relativa do ar menor que 70 por cento. Todos os profissionais devem possuir os EPI's necessários: capacete, óculos de sol, roupa isolante e antichamas, luvas isolantes e botas isolantes. Os profissionais que entram no potencial devem estar vestidos com roupa condutiva e cabo guarda. As ferramentas necessárias para resolver esse problema são: bastão garra de diâmetro 64 x 3600 mm, sela de diâmetro 65, colar, corda de fibra sintética, carretilha, chave com catraca, bastão universal, soquete adequado, locador de pino e bastão com soquete multiangular. O método selecionado para esse tipo de manutenção é a distância onde o eletricista não acessa diretamente o potencial, mas faz isso por intermédio de um bastão isolante. A substituição do isolador 
de pedestal pode ser escrita nos seguintes objetivos: 

\begin{enumerate}
	\item Limpar, secar e testar corda.
	\item Instalar Bastão Garra na estrutura com o pedestal a ser substituído.
	\item Instalar sela com colar na estrutura
	\item Amarrar o bastão na parte superior da estrutura com a corda.
	\item Amarrar o olhal do bastão ao cavalo da sela atrás de uma corda.
	\item Instalar um segundo conjunto bastão e sela no lado oposto da estrutura.
	\item Enforcar um estropo de Náilon no corpo do isolador.
	\item Colocar a extremidade do estropo no gancho da corda de serviço.
	\item Afrouxar os parafusos do conector que prendem a barra ao isolador.
	\item Terminar de retirar os parafusos com o bastão com o soquete multiangular.
	\item Elevar a barra através da corda que une a sela ao bastão.
	\item Apertar o colar através da porca borboleta.
	\item Segurar firmemente a corda de serviço.
	\item Sacar parafusos da base da coluna.
	\item Baixar o isolador ao solo
	\item Içar o Isolador
	\item Colocar Parafusos na base da coluna.
	\item Baixar a barra para que a mesma apoie no novo isolador.
	\item Colocar os parafusos do conector que prende a barra ao novo isolador. 
	\item Retirar Equipamentos
\end{enumerate}

\subsection{Especificação dos Agentes e seus Papéis}

A tabela \ref{agents} apresenta todos os agentes que fazem parte da manutenção. 
\begin{table}[H]
\scalefont{0.8}
\centering
\begin{tabular}{|l|l|}
\hline
\textbf{símbolo} & \textbf{significado} \\ \hline
agente1 & Um dos agentes participantes da manutenção \\ \hline
agente2 & Um dos agentes participantes da manutenção \\ \hline
agente3 & Um dos agentes participantes da manutenção \\ \hline
agente4 & Um dos agentes participantes da manutenção \\ \hline
agente5 & Um dos agentes participantes da manutenção \\ \hline
agente6 & Um dos agentes participantes da manutenção \\ \hline
agente7 & Um dos agentes participantes da manutenção \\ \hline
\end{tabular}
\caption{Os agentes que constituem uma manutenção}
\label{agents}
\end{table}

 A tabela \ref{roles} apresenta todas as funções que deverão ser exercidas pelos agentes.

\begin{table}[H]
\scalefont{0.8}
\centering
\begin{tabular}{|l|l|}
\hline
\textbf{papel} & \textbf{descrição} \\ \hline
supervisor & Atribui papel a outros profissionais \\ \hline
executor1 & Tem como por finalidade executar certas atividades manuais vinculadas a manutenção \\ \hline
executor2 & Tem como por finalidade executar certas atividades manuais vinculadas a manutenção \\ \hline
executor3 & Tem como por finalidade executar certas atividades manuais vinculadas a manutenção \\ \hline
executor4 & Tem como por finalidade executar certas atividades manuais vinculadas a manutenção \\ \hline
executor5 & Tem como por finalidade executar certas atividades manuais vinculadas a manutenção \\ \hline
\end{tabular}
\caption{Os papéis relevantes para a ocorrência da manutenção}
\label{roles}
\end{table}

A tabela \ref{agentsroles} define o predicado $adoptsRole(ag_n,\rho_m)$ onde $ag_n$ é representado pela coluna agente e $\rho_m$ é representado pela coluna papel.

\begin{table}[H]
\scalefont{0.8}
\centering
\begin{tabular}{|l|l|}
\hline
\textbf{agente} & \textbf{papel} \\ \hline
agente1 & supervisor \\ \hline
agente2 & executor1 \\ \hline
agente3 & executor1 \\ \hline
agente4 & executor2 \\ \hline
agente5 & executor3 \\ \hline
agente6 & executor4 \\ \hline
agente7 & executor5 \\ \hline
\end{tabular}
\caption{Relação $adoptsRole(ag_n,\rho_m)$}
\label{agentsroles}
\end{table}

\subsection{Especificação das Ferramentas e Objetivos}

A tabela \ref{artefacts} apresenta todos artefatos que fazem parte da descrição deste estudo de caso.

\begin{table}[H]
\scalefont{0.8}
\centering
\begin{tabular}{|l|p{0.8\linewidth}|}
\hline
\textbf{Artefato} & \textbf{Descrição} \\ \hline
capacete & EPI usado pelo profissional para proteger a cabeça \\ \hline
óculos & Óculos usado para evitar dificuldades de enxergar presentes em dias claros \\ \hline
roupagem & Consiste em roupas isolantes e anti-chamas \\ \hline
luva & Luvas Isolantes \\ \hline
bota & Botas Isolantes para evitar que o profissional seja eletrocutado \\ \hline
bastaoGarra & Bastão isolante que possui uma ferramenta em estrutura de garra. 64 X 3600 mm \\ \hline
sela & Possui diâmetro 65 mm, é fixada na torre para sustentar o bastão. \\ \hline
colar & Estrutura que fica fixa na sela, bastão isolante é travado no colar. \\ \hline
corda & Corda Isolante. \\ \hline
carretilha & Carretilha que, em conjunto com a corda, é usada para mover material na vertical. \\ \hline
bastaoUniversal & Bastão isolante que permite o acoplamento de múltiplas ferramentas. \\ \hline
soquete & Usado na manipulação de parafusos. \\ \hline
locador & Usado como pino direcional em alinhamento de furo de parafusos, auxiliado na inserção de pinos e parafusos. \\ \hline
bastaoGarra & Bastão universal que possui uma garra. \\ \hline
isoladorVelho & Isolador de pedestal danificado a ser substituído \\ \hline
isoladorNovo & Isolador de pedestal novo que será posicionado no local do isolador velho. \\ \hline
torre & Estrutura metálica onde fica fixo o isolador \\ \hline
condutor & Em formato de cabo, fica fixo sobre o topo do isolador.e é por onde passa grandes quantidades de energia elétrica. \\ \hline
estropo & Pano firme usado para segurar Isolador quando estiver suspenso \\ \hline
pano & Pano usado para limpar ferramentas \\ \hline
glicerina & Substância usada para limpar as ferramentas adequadamente \\ \hline
condutímetro & Medidor de corrente de fuga sobre o bastão universal. \\ \hline
parafuso & Parafusos prendem o conector condutor-Isolador e também prendem o Isolador a base \\ \hline
conector & Estrutura que tem como por finalidade manter condutor,cabeçote do isolador em conjunto. \\ \hline
\end{tabular}
\caption{Definição de todos os artefatos presentes na manutenção}
\label{artefacts}
\end{table} 

As etapas da atividade anteriormente posta foram analisadas em conjunto com engenheiros da área e foram estruturadas com base nos objetivos expostos na tabela \ref{g}. Essa tabela também apresenta as especificações para o predicado $nextGoal(g_i,g_j)$ onde \textit{Objetivo} representa $g_i$, \textit{Próximo} $g_j$ e \textit{Descrição} é referente a $g_i$.

\begin{table}[H]
\scalefont{0.8}
\centering
\begin{tabular}{|l|l|p{0.6\linewidth}|}
\hline
\textbf{Objetivo} & \textbf{Próximo} & \textbf{Descrição} \\ \hline
gSupervisor & g1,g6 & Atribui objetivos aos demais agentes. \\ \hline
g0 &  gSupervisor & Vestir os EP'Is \\ \hline
g1 &  g2 & Limpar, secar e testar ferramentas com material isolante. \\ \hline
g2 &  g3 & Medir a corrente de fuga de ferramentas isolantes \\ \hline
g3 &  g4 & Instalar sela com colar na estrutura \\ \hline
g4 &  g5 & Passar o bastão garra por dentro do olhal do colar. \\ \hline
g5 &  g12 & Amarrar o bastão garra na parte superior da estrutura com a corda, fixar no condutor \\ \hline
g6 &  g7 & Amarrar o olhal do bastão garra ao cavalo da sela atrás de uma corda. \\ \hline
g7 &  g8 & Instalar sela com colar no outro lado da estrutura estrutura \\ \hline
g8 &  g9 & Passar o bastão universal por dentro do olhal do colar \\ \hline
g10 & g11 & Pender carretilha no bastão Universal. \\ \hline
g11 & g12 & Amarrar o bastão universal na parte superior da estrutura com a corda: \\ \hline
g12 & g13 & Rotacionar estrutura olhal garra em 45 graus. \\ \hline
g13 & g14 & Enforcar um estropo de Náilon no corpo do isolador velho. \\ \hline
g14 & g15 & Colocar a extremidade do estropo no gancho da corda de serviço. \\ \hline
g15 & g16 & Afrouxar os parafusos do conector que prendem a barra ao isolador. \\ \hline
g16 & g17 & Terminar de retirar os parafusos com o bastão com o soquete multiangular. \\ \hline
g17 & g18 & Elevar o condutor através da corda que une a sela ao bastão. \\ \hline
g18 & g19 & Apertar o colar através da porca borboleta. \\ \hline
g19 & g20 & Sacar parafusos da base da coluna. \\ \hline
g20 & g21 & Segurar firmemente a corda de serviço,baixar o isolador ao solo \\ \hline
g21 & g22 & Passar Estropo no Isolador Novo \\ \hline
g22 & g23 & Colocar a extremidade do estropo no gancho da corda de serviço. \\ \hline
g23 & g24 & Içar o Isolador \\ \hline
g24 & g25 & Colocar Parafusos na base da coluna. \\ \hline
g25 & g26 & Baixar o condutor para que a mesma se sustente no novo isolador. \\ \hline
g26 & g27 & Colocar os parafusos do conector que prende a barra ao novo isolador. \\ \hline
g27 &     & Retirar Equipamentos \\ \hline
\end{tabular}
\caption{Definição e descrição dos objetivos bem como dos respectivos pré-requisitos}
\label{g}
\end{table}

\subsection{Especificação das Condições e Relações}

A tabela \ref{condition} apresenta $c_k$ dado pela coluna condição e pela coluna descrição. Essa tabela define $hasRisk(c_k,risk_j,cs_m)$ onde $risk_j$ é descrito pela coluna risco e $cs_m$ é descrito como consequência. 

\begin{table}[H]
\scalefont{0.8}
\centering
\begin{tabular}{|l|p{0.6\linewidth}|l|l|}
\hline
\textbf{condição} & \textbf{descrição} & \textbf{risco} & \textbf{consequência} \\ \hline
umidade70 & Umidade Relativa do Ar deve ser inferior a setenta porcento. & eletrocutado & morte \\ \hline
noVento & Não deve haver vento durante os procedimentos de manutenção. & eletrocutado & morte \\ \hline
noChuva & Não deve haver chuva durante o ato da manutenção & eletrocutado & morte \\ \hline
sol & O dia deve estar ensolarado & eletrocutado & morte \\ \hline
\end{tabular}
\caption{Define as condições necessárias para que a manutenção tenha possibilidade de acontecer}
\label{condition}
\end{table}


A tabela \ref{relationEntEnt1} apresenta a especificação para dois predicados onde um deles é $possEntityRel(r_l,e_i,e_k)$ tal que $r_l$ é definido pela coluna \textit{relacionamento}, $e_i$ e $e_k$ pelas \textit{entidades envolvidas}. 
O outro predicado é dado por $hasRisk(r_k,risk_j,cs_m)$ onde $risk_j$ é dado pela coluna risco e $cs_m$ é dado pela coluna consequência. 
Algumas relações (instâncias do conjunto $Relation$) serão apresentadas usando o termo $X$. O objetivo disto consiste em tornar as tabelas mais enxutas por intermédio de uma regra a qual é: 

A variável $X$ deve ser substituída pelo agente que tem a permissão de executar alguma ação em certo objetivo em prol a sua função. 
Essa regra pode ser sintetizada na seguinte expressão para um agente $ag_n$ que é referenciado por $AGENT$:

\begin{eqnarray}\label{whenusex} \nonumber
    adoptsRole(AGENT,\rho_m) \wedge hasPermission(\rho_m,g_i) \wedge  requiresEntity(g_i, otherEntity) \\ 
    \to requiresCirc(g_1, relAGENTotherEntity)  
\end{eqnarray}


Para mostrar como se dá o uso desta regra pode-se considerar um relacionamento $ relXCapacete$ entre $X$ e $capacete$ (essa relação será melhor descrita na tabela \ref{relationEntEnt1}). Essa relação acontece no objetivo $g_0$ onde o trabalhador deve colocar o capacete em sua cabeça, por conta disto se aplica a todos os agentes que representam esses profissionais. Nesta implementação, esses agentes são: $agente1$, $agente2$, $agente3$, $agente4$, $agente5$, $agente6$ e $agente7$. Aplicando a regra \ref{whenusex} para esse caso, obtêm-se as seguintes relações: $relAgente1Capacete$, $relAgente2Capacete$, $relAgente3Capacete$, $relAgente4Capacete$, $relAgente5Capacete$, $relAgente6Capacete$ e $relAgente7Capacete$. Portanto, ao ler a tabela \ref{relationEntEnt1}:  

\begin{eqnarray}
	relXCapacete | X,capacete |
\end{eqnarray}

Para o objetivo $g_0$, essa linha é equivalente a: 

\begin{eqnarray}
relAgente1Capacete | Agente1 ,capacete | \nonumber \\
relAgente2Capacete | Agente2 ,capacete | \nonumber \\ 
relAgente3Capacete | Agente3 ,capacete | \nonumber \\ 
relAgente4Capacete | Agente4 ,capacete | \nonumber \\
relAgente5Capacete | Agente5 ,capacete | \nonumber \\
relAgente6Capacete | Agente6 ,capacete | \nonumber \\
relAgente7Capacete | Agente7 ,capacete | \nonumber \\
\nonumber \\
\end{eqnarray}

É digno de nota observar que a regra \ref{whenusex} não faz parte da estrutura do modelo. A sua existência se deve única e exclusivamente neste estudo de caso a fim de simplificar e otimizar o grande volume de informações repetitivas.

\begin{center}
\begin{longtable}[H]{|l|l|l|l|}
\hline
\textbf{relacionamento}                  & \textbf{entidades envolvidas}                & \textbf{risco}                & \textbf{consequência}              \\ \hline
relXCapacete                             & X,capacete                                     & nenhum                          & nenhum                               \\ \hline
relXOculos                               & X,oculos                                       & nenhum                          & nenhum                               \\ \hline
relXRoupagem                             & X,roupagem                                     & nenhum                          & nenhum                               \\ \hline
relXLuva                                 & X,luva                                         & nenhum                          & nenhum                               \\ \hline
relXBotas                                & X,bota                                         & nenhum                          & nenhum                               \\ \hline
relXPano                                 & X,pano                                         & nenhum                          & nenhum                               \\ \hline
relPanoGlicerina                         & pano,glicerina                                 & nenhum                          & nenhum                               \\ \hline
relPanoCorda                             & pano,corda                                     & nenhum                          & nenhum                               \\ \hline
relPanoBastoaUniversal                   & pano,bastaoUniversal                           & nenhum                          & nenhum                               \\ \hline
relPanoSoquete                           & pano,soquete                                   & nenhum                          & nenhum                               \\ \hline
relPanoBastaoUniversal                   & pano,bastaoGarra                               & nenhum                          & nenhum                               \\ \hline
relXSela                                 & X,sela                                         & nenhum                          & nenhum                               \\ \hline
relXColar                                & X,colar                                        & nenhum                          & nenhum                               \\ \hline
relXBastaoGarra                          & X,bastaoGarra                                  & nenhum                          & nenhum                               \\ \hline
relTorreSela                             & torre,sela                                     & nenhum                          & nenhum                               \\ \hline
relSelaColar                             & sela,colar                                     & nenhum                          & nenhum                               \\ \hline
relColarBastaoGarra                      & colar,bastaoGarra                              & nenhum                          & nenhum                               \\ \hline
relBastaoGarraCondutor                   & bastaoGarra,condutor                           & eletrocutado                    & morte                                \\ \hline
relXBastaoUniversal                      & X,bastaoUniversal                              & nenhum                          & nenhum                               \\ \hline
relCordaBastaoUniversal                  & corda,bastaoUniversal                          & nenhum                          & nenhum                               \\ \hline
relCordaCarretilha                       & corda,carretilha                               & nenhum                          & nenhum                               \\ \hline
relBastaoUniversalCarretilha             & bastaoUniversal,carretilha                     & nenhum                          & nenhum                               \\ \hline
relBastaoUniversalColar                  & bastaoUniversal,colar                          & nenhum                          & nenhum                               \\ \hline
relBastaoUniversalEstopo                 & bastaoUniversal,estopo                         & nenhum                          & nenhum                               \\ \hline
relCordaEstropo                          & corda,estropo                                  & eletrocutado                    & morte                                \\ \hline
relEstropoIsoladorVelho                  & estropo,isoladorVelho                          & nenhum                          & nenhum                               \\ \hline
relXChaveCatraca                         & X,chaveCatraca                                 & nenhum                          & nenhum                               \\ \hline
relChaveCatracaBastaoUniversal           & chaveCatraca,bastaoUniversal                   & nenhum                          & nenhum                               \\ \hline
relChaveCatracaParafuso                  & chaveCatraca,parafuso                          & eletrocutado                    & morte                                \\ \hline
relParafusoConector                      & parafuso,conector                              & eletrocutado                    & morte                                \\ \hline
relXBastaoSoquete                        & X,bastaoSoquete                                & nenhum                          & nenhum                               \\ \hline
relSoqueteParafuso                       & soquete,parafuso                               & eletrocutado			        & morte                                \\ \hline
relXCorda                                & X,corda                                        & eletrocutado                    & morte                                \\ \hline
relXIsoladorVelho                        & X,isoladorVelho                                & nenhum                          & nenhum                               \\ \hline
relXIsoladorNovo                         & X,isoladorNovo                                 & nenhum                          & nenhum                               \\ \hline
relCordaBastaoGarra                      & corda,bastaoGarra                              & nenhum                          & nenhum                               \\ \hline
relBastaoGarraSela                       & bastaoGarra, sela                              & nenhum                          & nenhum                               \\ \hline
relXCarretilha                           & X,carretilha                                   & nenhum                          & nenhum                               \\ \hline
relBastaoUniversalCorda                  & bastaoUniversal,corda                          & nenhum                          & nenhum                               \\ \hline
relBastaoUniversalTorre                  & bastaoUniversal,torre                          & nenhum                          & nenhum                               \\ \hline
relEstropoCorda                          & estropo,corda                                  & eletrocutado                    & morte                                \\ \hline
relEstropoIsoladorNovo                   & estropo,isoladorNovo                           & nenhum                          & nenhum                               \\ \hline
relBastaoUniversalSela                   & universal,sela                                 & nenhum                          & nenhum                               \\ \hline
relBastaoGarraTorre                      & bastaoGarra,torre                              & nenhum                          & nenhum                               \\ \hline
relBastaoUniversalEstropo                & bastaoUniversal,estropo                        & nenhum                          & nenhum                               \\ \hline
relXColar                                & X,colar                                        & nenhum                          & nenhum                               \\ \hline
relParafusoTorre                         & parafuso,torre                                 & eletrocutado                    & morte                                \\ \hline
relCondutivimetroCorda                   & condutímetro,corda                             & nenhum                          & nenhum                               \\ \hline
relCondutivimetroBastaoUniversal         & condutímetro,bastaoUniversal                   & nenhum                          & nenhum                               \\ \hline
relCondutivimetroBastaoGarra             & condutímetro,bastaoGarra                       & nenhum                          & nenhum                               \\ \hline
relCondutivimetroSoquete                 & condutímetro,soquete                           & nenhum                          & nenhum                               \\ \hline
\caption{Descrição das entidades em função das relações}
\label{relationEntEnt1}
\end{longtable}
\end{center}

Tendo em vista o texto presente em \ref{predic}, todos os elementos de $Relation$ bem como de $Condition$ também são elementos de $Circumstance$.

A Tabela \ref{relation1}  apresenta a relação $affectsRels(r_k,r_n)$ onde $r_k$ é representado pela coluna relacionamento-errado e $r_n$ é representado pela coluna relacionamento-afetado. 

\begin{center}
\begin{longtable}[H]{|l|l|}
\hline
\textbf{Relacionamento-errado} 					& \textbf{Relacionamento-afetado}   			   \\ \hline
relXCapacete                                    & relBastaoGarraCondutor                           \\ \hline
relXCapacete                                    & relCordaEstropo                                  \\ \hline
relXCapacete                                    & relChaveCatracaParafuso                          \\ \hline
relXCapacete                                    & relParafusoConector                              \\ \hline
relXCapacete                                    & relSoqueteParafuso                               \\ \hline
relXCapacete                                    & relXCorda                                        \\ \hline
relXCapacete                                    & relEstropoCorda                                  \\ \hline
relXOculos                                      & relBastaoGarraCondutor                           \\ \hline
relXOculos                                      & relCordaEstropo                                  \\ \hline
relXOculos                                      & relChaveCatracaParafuso                          \\ \hline
relXOculos                                      & relParafusoConector                              \\ \hline
relXOculos                                      & relSoqueteParafuso                               \\ \hline
relXOculos                                      & relXCorda                                        \\ \hline
relXOculos                                      & relEstropoCorda                                  \\ \hline
relXLuva                                        & relBastaoGarraCondutor                           \\ \hline
relXLuva                                        & relCordaEstropo                                  \\ \hline
relXLuva                                        & relChaveCatracaParafuso                          \\ \hline
relXLuva                                        & relParafusoConector                              \\ \hline
relXLuva                                        & relSoqueteParafuso                               \\ \hline
relXLuva                                        & relXCorda                                        \\ \hline
relXLuva                                        & relEstropoCorda                                  \\ \hline
relXBotas                                       & relBastaoGarraCondutor                           \\ \hline
relXBotas                                       & relCordaEstropo                                  \\ \hline
relXBotas                                       & relChaveCatracaParafuso                          \\ \hline
relXBotas                                       & relParafusoConector                              \\ \hline
relXBotas                                       & relSoqueteParafuso                               \\ \hline
relXBotas                                       & relXCorda                                        \\ \hline
relXBotas                                       & relEstropoCorda                                  \\ \hline
relXPano                                        & relBastaoGarraCondutor                           \\ \hline
relXPano                                        & relCordaEstropo                                  \\ \hline
relXPano                                        & relChaveCatracaParafuso                          \\ \hline
relXPano                                        & relParafusoConector                              \\ \hline
relXPano                                        & relSoqueteParafuso                               \\ \hline
relXPano                                        & relXCorda                                        \\ \hline
relXPano                                        & relEstropoCorda                                  \\ \hline
relPanoGlicerina                                & relBastaoGarraCondutor                           \\ \hline
relPanoGlicerina                                & relCordaEstropo                                  \\ \hline
relPanoGlicerina                                & relChaveCatracaParafuso                          \\ \hline
relPanoGlicerina                                & relParafusoConector                              \\ \hline
relPanoGlicerina                                & relSoqueteParafuso                               \\ \hline
relPanoGlicerina                                & relXCorda                                        \\ \hline
relPanoGlicerina                                & relEstropoCorda                                  \\ \hline
relPanoCorda                                    & relCordaEstropo                                  \\ \hline
relPanoCorda                                    & relXCorda                                        \\ \hline
relPanoCorda                                    & relEstropoCorda                                  \\ \hline
relPanoBastaoUniversal                          & relBastaoGarraCondutor                           \\ \hline
relPanoBastaoUniversal                          & relChaveCatracaParafuso                          \\ \hline
relPanoBastaoUniversal                          & relParafusoConector                              \\ \hline
relPanoBastaoUniversal                          & relBastaoGarraCondutor                           \\ \hline
relPanoSoquete                                  & relBastaoGarraCondutor                           \\ \hline
relPanoSoquete                                  & relCordaEstropo                                  \\ \hline
relPanoSoquete                                  & relChaveCatracaParafuso                          \\ \hline
relPanoSoquete                                  & relParafusoConector                              \\ \hline
relPanoSoquete                                  & relSoqueteParafuso                               \\ \hline
relPanoSoquete                                  & relXCorda                                        \\ \hline
relPanoSoquete                                  & relEstropoCorda                                  \\ \hline
relCondutivimetroCorda                          & relBastaoGarraCondutor                           \\ \hline
relCondutivimetroCorda                          & relCordaEstropo                                  \\ \hline
relCondutivimetroCorda                          & relChaveCatracaParafuso                          \\ \hline
relCondutivimetroCorda                          & relParafusoConector                              \\ \hline
relCondutivimetroCorda                          & relSoqueteParafuso                               \\ \hline
relCondutivimetroCorda                          & relXCorda                                        \\ \hline
relCondutivimetroCorda                          & relEstropoCorda                                  \\ \hline
relCondutivimetroCorda                          & relParafusoTorre                                 \\ \hline
relPanoBastaoUniversal                          & relBastaoGarraCondutor                           \\ \hline
relPanoBastaoUniversal                          & relChaveCatracaParafuso                          \\ \hline
relPanoBastaoUniversal                          & relParafusoConector                              \\ \hline
relPanoBastaoUniversal                          & relParafusoTorre                                 \\ \hline
relPanoBastaoUniversal                          & relBastaoGarraCondutor                           \\ \hline
relPanoSoquete                                  & relBastaoGarraCondutor                           \\ \hline
relPanoSoquete                                  & relCordaEstropo                                  \\ \hline
relPanoSoquete                                  & relChaveCatracaParafuso                          \\ \hline
relPanoSoquete                                  & relParafusoConector                              \\ \hline
relPanoSoquete                                  & relSoqueteParafuso                               \\ \hline
relPanoSoquete                                  & relXCorda                                        \\ \hline
relPanoSoquete                                  & relEstropoCorda                                  \\ \hline
relPanoSoquete                                  & relParafusoTorre                                 \\ \hline
relCondutivimetroCorda                          & relBastaoGarraCondutor                           \\ \hline
relCondutivimetroCorda                          & relCordaEstropo                                  \\ \hline
relCondutivimetroCorda                          & relChaveCatracaParafuso                          \\ \hline
relCondutivimetroCorda                          & relParafusoConector                              \\ \hline
relCondutivimetroCorda                          & relSoqueteParafuso                               \\ \hline
relCondutivimetroCorda                          & relXCorda                                        \\ \hline
relCondutivimetroCorda                          & relEstropoCorda                                  \\ \hline
relCondutivimetroCorda                          & relParafusoTorre                                 \\ \hline
\caption{Define o impacto que o erro em um relacionamento gera em outro relacionamento por mudar a possibilidade de algo errado acontecer.}
\label{relation1}
\end{longtable}
\end{center}

\subsection{Relação entre Objetivos e Papéis}

A tabela \ref{deontic1}  apresentam a relação $hasObligation(\rho_m,g_i)$ onde $\rho_m$ é representado pela coluna papel e $g_i$ é representado pela coluna objetivo. 


\begin{center}
\begin{longtable}[H]{|l|l|}
\hline
\textbf{Papel} & \textbf{Objetivo} \\ \hline
executor1 & g0 \\ \hline
executor2 & g0 \\ \hline
executor3 & g0 \\ \hline
executor4 & g0 \\ \hline
executor5 & g0 \\ \hline
supervisor & g0 \\ \hline
supervisor & gSupervisor \\ \hline
executor1 & g1 \\ \hline
executor2 & g1 \\ \hline
executor1 & g2 \\ \hline
executor2 & g2 \\ \hline
executor1 & g3 \\ \hline
executor2 & g2 \\ \hline
executor1 & g4 \\ \hline
executor2 & g4 \\ \hline
executor1 & g5 \\ \hline
executor2 & g5 \\ \hline
executor3 & g6 \\ \hline
executor4 & g6 \\ \hline
executor5 & g6 \\ \hline
executor3 & g7 \\ \hline
executor4 & g7 \\ \hline
executor5 & g7 \\ \hline
executor3 & g8 \\ \hline
executor4 & g8 \\ \hline
executor5 & g8 \\ \hline
executor3 & g9 \\ \hline
executor4 & g9 \\ \hline
executor5 & g9 \\ \hline
executor3 & g10 \\ \hline
executor4 & g10 \\ \hline
executor5 & g10 \\ \hline
executor3 & g11 \\ \hline
executor4 & g11 \\ \hline
executor5 & g11 \\ \hline
executor1 & g12 \\ \hline
executor2 & g12 \\ \hline
executor3 & g12 \\ \hline
executor4 & g12 \\ \hline
executor1 & g13 \\ \hline
executor2 & g13 \\ \hline
executor3 & g13 \\ \hline
executor4 & g13 \\ \hline
executor1 & g14 \\ \hline
executor2 & g14 \\ \hline
executor3 & g14 \\ \hline
executor4 & g14 \\ \hline
executor2 & g15 \\ \hline
executor3 & g15 \\ \hline
executor4 & g15 \\ \hline
executor5 & g15 \\ \hline
executor2 & g16 \\ \hline
executor3 & g16 \\ \hline
executor4 & g16 \\ \hline
executor5 & g16 \\ \hline
executor1 & g17 \\ \hline
executor3 & g17 \\ \hline
executor4 & g17 \\ \hline
executor5 & g17 \\ \hline
executor1 & g18 \\ \hline
executor3 & g18 \\ \hline
executor4 & g18 \\ \hline
executor5 & g18 \\ \hline
executor1 & g19 \\ \hline
executor3 & g19 \\ \hline
executor4 & g19 \\ \hline
executor5 & g19 \\ \hline
executor1 & g20 \\ \hline
executor3 & g20 \\ \hline
executor4 & g20 \\ \hline
executor5 & g20 \\ \hline
executor1 & g21 \\ \hline
executor3 & g21 \\ \hline
executor4 & g21 \\ \hline
executor5 & g21 \\ \hline
executor1 & g22 \\ \hline
executor2 & g22 \\ \hline
executor3 & g22 \\ \hline
executor5 & g22 \\ \hline
executor1 & g23 \\ \hline
executor2 & g23 \\ \hline
executor3 & g23 \\ \hline
executor5 & g23 \\ \hline
executor1 & g24 \\ \hline
executor2 & g24 \\ \hline
executor3 & g24 \\ \hline
executor5 & g24 \\ \hline
executor1 & g25 \\ \hline
executor2 & g25 \\ \hline
executor3 & g25 \\ \hline
executor4 & g25 \\ \hline
executor1 & g26 \\ \hline
executor2 & g26 \\ \hline
executor3 & g26 \\ \hline
executor4 & g26 \\ \hline
executor1 & g27 \\ \hline
executor2 & g27 \\ \hline
executor3 & g27 \\ \hline
executor4 & g27 \\ \hline
executor5 & g27 \\ \hline
\caption{Objetivos que devem ser atingidos pelo agente que assumir um dada função}
\label{deontic1}
\end{longtable}
\end{center}

\subsection{Relacionamentos das Entidades, Relações e Condições com Objetivos}

As próximas tabelas têm como por finalidade apresentar como se dá a especificação para os seguintes predicados: $requiresCirc(goal_n, circ_m)$,$requiresEntity(goal_n, ent_m)$. Contudo, existe uma série de circunstâncias $circ \in Circumstance$ que apontam para o mesmo objetivo $g \in Goal$. Para tornar a apresentação desses resultados mais simples, o autor optou por agrupar todas as circunstâncias que se relacionam na mesma célula na coluna esquerda. A coluna direita contém o respectivo objetivo com o qual essas circunstâncias se relacionam.

Por exemplo, suponha que em uma determinada situação, um objetivo g1 necessita a presença das circunstâncias $circ_a$, $circ_b$ e $circ_c$. De acordo com o modelo, essa situação é resolvida da seguinte maneira: $requiresCirc(g1,circ_a)$, $requiresCirc(g1,circ_b)$ e $requireCirc(g_1,circ_c)$. Porém, para economizar linha em tabelas, o autor optou por representar esse cenário como na tabela \ref{tableexemplgroupdataone}:

\begin{table}[H]
\centering
\scalefont{0.8}
\begin{tabular}{|l|l|l|}
\hline
\textbf{Circunstância}		 	& \textbf{Tipo de Instância}	&	\textbf{Objetivo}	\\ \hline
$circ_a$,$circ_b$,$circ_c$		& instanceOfRel 				& 	 g1         		\\ \hline
\end{tabular}
\caption{Exemplificação dos dados atrelados ao predicado $requiresCirc(goal_n, circ_m)$,$requiresEntity(goal_n, ent_m)$ serão organizados nas tabelas a seguir.}
\label{tableexemplgroupdataone}
\end{table}

A coluna \textit{Circunstância} apresenta todas as circunstâncias atreladas ao objetivo g1 descritas pelos predicados $requiresCirc$ expostos um pouco antes apresentar a tabela \ref{tableexemplgroupdataone}. A coluna \textit{Tipo de Instância} define se a circnstância é uma instância de \textit{Relation} ou se é uma instância de \textit{Condition}. Assim sendo, ao analisar a tabela \ref{tableexemplgroupdataone} pode-se concluir que $circ_a$, $circ_b$ e $circ_c$ são todos $Relations$. A coluna \textit{Objetivo} apresenta o objetivo no qual essas circunstâncias estão relacionadas. A Tabela \ref{relationsgroup1} apresenta essa mesma estrutura, porém aplicadas ao estudo de caso em interesse.

\begin{center}
\begin{longtable}[H]{|p{0.4\linewidth}|l|l|}
\hline
\textbf{Relacionamentos}																									& \textbf{Tipo de Instância}          & \textbf{Objetivo} \\ \hline
relXcapacete relXoculos relXroupagem relXluva relXbotas                                                                                                                                                                                                                                                                    & instanceOfRel              		 & g0         \\ \hline
relXPano, relPanoGlicerina, relPanoCorda, relPanoBastaoUniversal, relPanoBastaoGarra, relPanoSoquete                             						& instanceOfRel                       & g1         \\ \hline
umidade70,noVento,noChuva,sol 																														& instanceOfCond					  & g1        \\ \hline
relCondutivimetroCorda, relCondutivimetroBastaoUniversal, relCondutivimetroBastaoGarra, relCondutivimetroSoquete               						& instanceOfRel              		 & g2         \\ \hline
relCondutivimetro                                                                                                           						& instanceOfRel              		 & g3         \\ \hline
relXBastaoGarra, relColarBastaoGarra                                                                                         						& instanceOfRel              		 & g4         \\ \hline
relXBastaoGarra, relXCordarelCordaBastaoGarra, relBastaoGarraTorre, relBastaoGarraCondutor                                     						& instanceOfRel              		 & g5         \\ \hline
relBastaoGarraSela, relXBastaoGarra, relXSela                                                                                 						& instanceOfRel              		 & g6         \\ \hline
relXSela, relXColar, relTorreSela                                                                                             						& instanceOfRel              		 & g7         \\ \hline
relBastaoUniversalColar, relXBastaoUniversal                                                                                 						& instanceOfRel              		 & g8         \\ \hline
relXBastaoUniversal, relXCarretilha, relBastaoUniversalCarretilha                                                             						& instanceOfRel              		 & g9         \\ \hline
relXCorda, relXBastaoUniversal, relBastaoUniversalCorda, relBastaoUniversalTorre                                               						& instanceOfRel              		 & g10        \\ \hline
relXCorda, relXBastaoUniversal, relXColar, relBastaoUniversalColar, relBastaoUniversalSela                                      					& instanceOfRel              		 & g11        \\ \hline
relXColar                                                                                                                   						& instanceOfRel              		 & g12        \\ \hline
relXBastaoUniversal, relBastaoUniversalEstropo, relEstropoIsoladorVelho                                                       						& instanceOfRel              		 & g13        \\ \hline
relXBastaoUniversal, relBastaoUniversalCordarelCordaEstropo, relEstropoCorda                                                  						& instanceOfRel              		 & g14        \\ \hline
relChaveCatracaBastaoUniversal, relXChaveCatraca, relXBastaoUniversal, relChaveCatracaParafuso                                 						& instanceOfRel              		 & g15        \\ \hline
relXBastaoSoquete relSoqueteParafuso                                                                                        						& instanceOfRel              		 & g16        \\ \hline
relXCorda relCordaBastaoGarra, relBastaoGarraCondutor                                                                        						& instanceOfRel              		 & g17        \\ \hline
relXColar                                                                                                       									& instanceOfRel              		 & g18        \\ \hline
relChaveCatracaBastaoUniversal, relXChaveCatraca, relXBastaoUniversal, relChaveCatracaParafuso, relParafusoTorre, relXBastaoSoqueterelSoqueteParafuso    & instanceOfRel              		 & g19        \\ \hline
relXCorda                                                                                                       									& instanceOfRel              		 & g20        \\ \hline
relXEstropo, relEstropoIsoladorNovo                                                                              									& instanceOfRel              		 & g21        \\ \hline
relXBastaoUniversal, relBastaoUniversalCorda, relCordaEstropo, relEstropoCorda                                     									& instanceOfRel              		 & g22        \\ \hline
relXCorda                                                                                                       									& instanceOfRel              		 & g23        \\ \hline
relChaveCatracaBastaoUniversal, relXChaveCatraca, relXBastaoUniversal, relChaveCatracaParafuso, relParafusoTorrerelXBastaoSoquete, relSoqueteParafuso 	& instanceOfRel						 & g24        \\ \hline
relXCorda, relCordaBastaoGarra, relBastaoGarraCondutor                                                            									& instanceOfRel						 & g25        \\ \hline
relChaveCatracaBastaoUniversal, relXChaveCatraca, relXBastaoUniversal, relChaveCatracaParafuso                     									& instanceOfRel						 & g26        \\ \hline
relXSela, relXColarrelXBastaoGarrarelXBastaoUniversal, relXBastaoSoquete, relXCorda, relXCarretilha, relXChaveCatraca, relColarBastaoGarra, relCordaBastaoGarra, relBastaoGarraTorre, relBastaoGarraCondutor, relBastaoUniversalCarretilha, relBastaoGarraSela, relBastaoUniversalSela, relSelaColar, relTorreSela, relBastaoUniversalCorda, relBastaoGarraCorda 									& instanceOfRel						 & g27  \\ \hline
\caption{Especificação do predicado $requiresCirc(goal_i,circ_j)$, do predicado $instanceOfRel(circ_n)$ e do predicado $instanceOfCond(circ_n)$}
\label{relationsgroup1}
\end{longtable}
\end{center}

A Tabela \ref{entitygoals} apresenta os dados em relação ao predicado $requiresEntity(goal_i, e_j)$ e segue o mesmo padrão apresentado pelo exemplo na tabela \ref{tableexemplgroupdataone}. Contudo, tendo em vista o fato de que está implicito no conceito do predicado $requiresEntity(goal_i, e_j)$ a natureza da instância atrelada ao segundo argumento (ou seja, não há dúvida que $e_j$ é uma entidade e está contida em $Entity$), a Tabela \ref{entitygoals} não apresenta coluna \textit{Tipo de Instância}.

\begin{center}
\begin{longtable}[H]{|p{0.8\linewidth}|l|}
\hline
\textbf{Entidades}                                                                                                 & \textbf{Objetivo} \\ \hline
capacete,óculos,roupagem,luvas,botas X = \{agentes em relação aos objetivos\}                                      & g0         \\ \hline
pano,glicerina,carretilha,bastaoUniversal,corda,bastaoGarra,X = \{agentes em relação aos objetivos\}               & g1         \\ \hline
pano,glicerina,carretilha,bastaoUniversal,corda,bastaoGarra,condutímetro,X = \{agentes em relação aos objetivos\}  & g2         \\ \hline
sela,colarX = \{agentes em relação aos objetivos\}                                                                 & g3         \\ \hline
colar,bastaoGarraX = \{agentes em relação aos objetivos\}                                                          & g4         \\ \hline
corda,bastaoGarra,bastaoGarraTorre,condutorX = \{agentes em relação aos objetivos\}                                & g5         \\ \hline
bastaoGarra,selaX = \{agentes em relação aos objetivos\}                                                           & g6         \\ \hline
sela,colarX = \{agentes em relação aos objetivos\}                                                                 & g7         \\ \hline
sela,bastaoUniversal,Colar,X = \{agentes em relação aos objetivos\}                                                & g8         \\ \hline
bastaoUniversal,carretilha,X = \{agentes em relação aos objetivos\}                                                & g9         \\ \hline
corda,bastaoUniversal,corda,torre,X = \{agentes em relação aos objetivos\}                                         & g10        \\ \hline
bastaoUniversal,corda,colar,selaX = \{agentes em relação aos objetivos\}                                       	   & g11        \\ \hline
colar,X = \{agentes em relação aos objetivos\}                                                                     & g12        \\ \hline
bastaoUniversal,estropo,isoladorVelhoX = \{agentes em relação aos objetivos\}                                      & g13        \\ \hline
bastaoUniversal,corda,estropoX = \{agentes em relação aos objetivos\}                                              & g14        \\ \hline
chaveCatraca,bastaoUniversal,prafusoX = \{agentes em relação aos objetivos\}                                       & g15        \\ \hline
bastaoSoquete,parafuso,X = \{agentes em relação aos objetivos\}                                                    & g16        \\ \hline
bastaoGarra,condutorcordaX = \{agentes em relação aos objetivos\},                                                 & g17        \\ \hline
colar,X = \{agentes em relação aos objetivos\},                                                                    & g18        \\ \hline
chaveCatraca,bastaoUniversal,prafusobastaoSoquete,parafuso,torreX = \{agentes em relação aos objetivos\}           & g19        \\ \hline
cordaX = \{agentes em relação aos objetivos\}                                                                      & g20        \\ \hline
estropo, isoladorNovo,X = \{agentes em relação aos objetivos\}                                                     & g21        \\ \hline
bastaoUniversal,corda,estropoX = \{agentes em relação aos objetivos\}                                              & g22        \\ \hline
cordaX = \{agentes em relação aos objetivos\}                                                                      & g23        \\ \hline
chaveCatraca,bastaoUniversal,prafusobastaoSoquete,parafuso,torreX = \{agentes em relação aos objetivos\}           & g24        \\ \hline
bastaoGarra,condutorcordaX = \{agentes em relação aos objetivos\},                                                 & g25        \\ \hline
chaveCatraca,bastaoUniversal,prafusoX = \{agentes em relação aos objetivos\}                                       & g26        \\ \hline
sela, colar, bastaoGarra, bastaoUniversal, bastaoSoquete, corda, carretilha, chaveCatraca, torre, condutor		   & g27        \\ \hline

\caption{Especificação do estudo de caso atrelado ao predicado $requiresEntity(goal_i, e_j)$}
\label{entitygoals}
\end{longtable}
\end{center}




