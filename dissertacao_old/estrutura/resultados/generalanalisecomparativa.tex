Uma vez estruturada a problemática da representação de cenários de acidentes em termos de um modelo computacional, é possível discutir semelhanças e diferenças de arcabouços existentes na literatura. Isso possibilita inferir implicações que ocorrem ao fazer uso de um arcabouço em específico (tendo o modelo conceitual desse estudo como parâmetro). Os aracabouços selecionados para comparação são: \textit{MOISE+} \cite{moiseframework}, Modelo presente no texto do Dastani \cite{dastaniframework}, o \textit{V3S} \cite{v3sframework} e o modelo \textit{NormMAS Framework} \cite{normas}. Portanto, as próximas seções apresentam os seguintes aspectos: análise da estrutura do modelo para aqueles onde isso não foi feito na fundamentação, Tabelas que realizam uma análise comparativa entre os arcabouços selecionados e o modelo conceitual proposto (os atributos foram escolhidos caso a caso com base nas características de cada modelo) e uma análise única de todos os modelos em uma Tabela para avaliar a expressividade em certos atributos.