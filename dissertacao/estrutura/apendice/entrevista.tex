\section{Perguntas Gerais}
\begin{itemize}
    \item Dentro do contexto da manutenção em linha viva, quais são todos os papéis envolvidos?
    \item Quais são as funções de cada papel? Tentar especificar em uma lista todas as funções relacionadas a cada papel.
    \item Entre os eletricistas, existe alguma divisão de tipos de eletricistas, por exemplo, eletricista sênior, eletricista pleno ou eletricista júnior?
    \item Entre os Engenheiros e os Gerentes existe algum tipo de subdivisão?
    \item Quais são as relações de Link de autoridades (ou seja, quem tem autoridade sobre quem) existente entre os papéis?
    \item Esses papéis se mantêm para diferentes tipos de manutenção?
    \item Quais são as relações de compatibilidade entre os papéis, ou seja, se não houver nenhuma pessoa de um papel, qual é a pessoa de outro papel que pode substituir ela?
    \item Se fosse necessário realizar uma divisão em grupo de diferentes papéis, você acha que os grupos: Grupo dos Engenheiros, Grupo dos Gerentes, Grupo dos Eletricistas são suficientes para descrever todos os elementos da manutenção?
    \item O número de pessoas necessárias para cada papel varia com relação ao tipo de manutenção? Isso ocorre em todos os papéis?
    \item Se não varia (ou seja, temos um número mínimo e um número máximo para qualquer tipo de manutenção em linha viva) então qual é o número de pessoas mínimo e o número de pessoas máximo para cada um dos papéis?
    \item Se se varia, então qual é o número mínimo e o número máximo de pessoas para cada papel em relação a esses tipos de manutenção: substituição do isolador de pedestal (método a distância), substituição de pingo aéreo (mét dist e mét potencial), substituição do conector (métd a distância e métd. ao potencial)?
    \item Quais sãos os critérios que influenciam nesses números?
    \item Levando em consideração esses papéis, quais são as relações mínimas e máximas entre cada papel? Ou seja, quantos eletricistas eu preciso para cada gerente? Quantos eletricistas eu preciso para cada Engenheiro? Quantos gerentes eu preciso para cada Engenheiro? Esses números são fixos ou variam para cada manutenção? Se variam, quais são os critérios que influenciam nessas relações?
    \item Durante o planejamento da manutenção, quais são as variáveis ambientais que afetam as tomadas de decisão? Como essas variáveis influenciam nessa tomada de decisão dos agentes?
    \item Durante o desenvolvimento da manutenção, quais são as situações imprevisíveis ou difíceis de prever que influenciam no decorrer da manutenção (ou seja, o que pode mudar o comportamento da manutenção durante a execução) - essas situações mudam muito conforme o tipo de manutenção?
    \item O que influencia no método da manutenção (distância ou ao potencial)?
    \item O planejamento da manutenção acontece com quantos dias de antecedência da execução da manutenção em si?
    \item Como é feita a análise das distâncias seguras? Quais são os critérios usados?
    \item A simulação da manutenção é feita? O que impacta nessa necessidade?  
    \item Como é feita a previsão do tempo de manutenção? 
    \item Como é feita a análise dos fatores inerentes de intervenção?  
    \item As etapas como Sinalização e Isolamento da Área de Trabalho, Afastamento de Fases, Teste de Corrente de Fuga em Andaime e Escada isolante e Instalação de By-Pass Provisórios podem ser entendidas como uma pré-manutenção ou já estão dentro do escopo da manutenção?
    \item Como é feita essa inspeção visual? Quais são os critérios que influenciam nessa inspeção? 
    \item O que são as condições necessárias de segurança? 
    \item No que diz respeito a ferramenta há um eletricista responsável de cuidar somente elas? O trabalho dele é focado apenas nisso ou ele revessa de trabalho com outro eletricista? 
    \item Existe um eletricista responsável apenas para fazer a medida do isolamento e das estruturas ou cada eletricista deve tomar cuidado para fazer a medida dos seus equipamentos em específico?
    \item Confere que o trabalho é iniciado apenas depois que a operação libera o circuito para a intervenção? Quando essa liberação acontece? Antes da manutenção, quando a equipe chega no local?
    \item O gerente ou o responsável pela equipe fica o tempo inteiro coordenando a manutenção? Dizendo a todo momento a todos o que cada um deve fazer? 
    \item O responsável pela equipe também pode trabalhar como um eletricista? Quando isso acontece, porque isso acontece?
    \item Quais são os critérios que influenciam no revezamento dos eletricistas? 
    \item O que define, no método ao potencial, se a abordagem será feita através da escada ou através do andaime?
    \item As distâncias no manual Linha Vida de Subestação Método a Distância e ao Potencial são informações atualizada? Onde consigo referências mais atuais? 
    \item Os procedimentos Gerais para Manutenção valem em qualquer situação?
\end{itemize}


\section{Perguntas sobre Substituição do Isolador de Pedestal}

A lista a seguir descreve os passos desse procedimento.
\begin{enumerate}
	\item Instalar a sela e o colocar na coluna do lado 1
	\item Fixar o bastão garra no colar e sela lado 1
	\item Fixar o topo do bastão garra do lado 1 no Condutor
	\item Instalar a sela e o colar na coluna base do lado 2 (Esquerdo)
	\item Fixar o bastão garra invertido no colar e na sela do lado 2 (Esquerdo)
	\item Fixar a carretilha com a corda de fibra sintética na ponta do bastão
	\item Girar a cela de forma a erguer o bastão com a carretilha e corda na ponta
	\item Instalar e enforcar com um estopo o alto do isolador usando um bastão universal
	\item Conectar a corda no estopo
	\item Usar um bastão universal com a chave catraca para soltar os parafusos de cima do conector
	\item Retirar a parte de cima do conector com a chave catraca para soltar os parafusos de cima do conector
	\item Suspender o cabo condutor através do bastão garra do lado 1 (Direito)
	\item Tencionar a corda presa ao estopo
	\item Tirar os parafusos de baixo do isolador com a chave catraca
	\item Passar corda na base do isolador para auxiliar na descida
	\item Puxar a corda para erguer o isolador 
	\item Fazer a retirada completa do isolador até o chão
	\item Enforcar estopo no novo isolador
	\item Suspender novo isolador
	\item Engatar corda guia na base do novo isolador
	\item Içar novo isolador até a posição correta com o auxilio da corda guia
	\item Parafusar base do isolador na coluna usando a chave catraca
	\item Soltar o colar do bastão garra do lado 1 (Direito) para baixar o cabo condutor no conector do isolador
	\item Fechar os parafusos do conector usando o bastão universal com  a catraca
	\item Desconectar o bastão garra do conector
	\item Retirar o bastão garra do lado 1 (Direto)
	\item Retirar cela e colar garra do lado 1 (Direito)
	\item Retirar estopo do isolador
	\item Retirar cela e colar garra do lado 2 (Esquerdo)
	\item Recolher equipamentos.
\end{enumerate}

As perguntas são: 

\begin{itemize}
    \item Por que a atividade é feita apenas ao método a distância? 
    \item Existe alguma situação em que pode ser feita através do potencial?
    \item Esses passos estão certos? 
    \item Para um número razoável de eletricistas, quais passos estariam associados a quais eletricistas? 
\end{itemize}

\section{Perguntas sobre Substituição do Pingo Aéreo pelo acesso a Dinstância} 

A lista a seguir descreve os passos desse procedimento.
\begin{enumerate}
	\item Proceder uma análise preliminar do espaço físico, das condições da conexão e condutores
	\item  Medir com uma corda limpa, seca e previamente testada, o comprimento do novo condutor
	\item Escovar os pontos de fixação do by-pass provisório
	\item Instalar o by-pass provisório através de bastões de manobras nas duas extremidades do pingo a ser substituído
	\item Soltar os parafusos que prendem o pingo ao condutor nas duas extremidades. Caso haja algum parafuso fundido ou espanado, cortá-lo com a serra em um bastão universal.
	\item Mover as conexões para liberar o espaço para os novos conectores
	\item Escovar os pontos das novas conexões
	\item Prender novo pingo com os conectores no torque correto deixando somente as partes que serão colocadas no barramento com apenas dois parafusos provisórios em cada conector. De preferência que esses parafusos sejam ligeiramente maiores que os originais, a fim de facilitar a instalação.
	\item Colocar no condutor uma carretilha com uma corda através de um bastão universal com gancho espiral. A corda deverá estar limpa, seca e testada através do testador de bastão. Este teste deverá ser feito no próprio local
	\item Colocar na parte superior do novo pingo, próximo ao conector, um grampo manipulador de condutor instalado em um bastão universal
	\item Amarrar uma das pontas da corda no bastão junto ao grampo manipulador de condutor (no caso de condutor de peso e/ou comprimento elevado).
	\item Içar o pingo tomando todos os cuidados necessários para que o mesmo não toque ou se aproxime de potenciais diferentes. 
	\item Com auxílio de uma chave de fenda presa em um bastão universal, abrir a conexão da parte superior colocando a mesma no condutor
	\item Encostar parcialmente as porcas, através do bastão com soquete multiangular.
	\item Com um sacador e colocador de pingo, colocar restante dos parafusos
	\item Prender as porcas e arruelas com fita crepe em um soquete, colocando-as nos parafusos com o bastão com soquete multiangular.
	\item Através deste mesmo processo, substituir os parafusos provisórios. 
	\item Utilizar a chave com catraca para apertar aos poucos, um a um, para então dar o torque final.
	\item Repetir a operação na conexão inferior
	\item Fazer uma última inspeção para prevenir-se de eventuais rachaduras durante a operação de torque. 
	\item Transferir o bastão com grampo manipulador de condutor amarrado à corda para o pingo antigo
	\item Segurar firmemente a corda.
	\item Retirar totalmente os parafusos necessários para liberar o pingo antigo. 
	\item Retirar os jumpers provisórios.
	\item Com auxílio de um outro grampo manipulador de condutor instalado em um bastão universal, sacar o condutor da conexão na parte inferior.
	\item Baixar o pingo, tomando todos os cuidados necessários para que o mesmo não toque nem se aproxime de potenciais diferentes. O mesmo só poderá ser tocado com as mãos depois que estiver totalmente estendido no solo.
	\item Retirar a carretilha com a corda do condutor.
	\item Recolher o material.
\end{enumerate}

As perguntas sobre esse procedimento. 

\begin{itemize}
    \item Esses passos estão corretos?
    \item O que é um pingo aéreo? 
    \item Como um pingo aéreo é usado na subestação?
    \item Como é feita a escovação dos condutores?
    \item Nos passos, dá a entender que o pingo novo é colocando antes da remoção do pingo antigo, por quê isso acontece? Como é feito, em maiores detalhes, esse procedimento?
    \item Como é feita a instalação da carretilha? Há sub-passos?
    \item Quais são as ferramentas usadas em cada passo?
    \item Dado um certo número de eletricistas, quais passos estariam relacionados com quais eletricistas?
\end{itemize}

\section{Perguntas sobre Substituição do Pingo Aéreo pelo acesso ao Potencial} 


A lista a seguir descreve os passos desse procedimento.
\begin{enumerate}
	\item Proceder uma análise preliminar do espaço físico, das condições da conexão e condutores
	\item Medir com uma corda limpa, seca e previamente testada, o comprimento do novo condutor
	\item Escovar os pontos de fixação do by-pass provisório
	\item Instalar o by-pass provisório através de bastões de manobras nas duas extremidades do pingo a ser substituído
	\item Soltar os parafusos que prendem o pingo ao condutor nas duas extremidades. Caso haja algum parafuso fundido ou espanado, cortá-lo com a serra em um bastão universal.
	\item Mover as conexões para liberar o espaço para os novos conectores
	\item Escovar os pontos das novas conexões
	\item Prender novo pingo com os conectores no torque correto deixando somente as partes que serão colocadas no barramento com apenas dois parafusos provisórios em cada conector. De preferência que esses parafusos sejam ligeiramente maiores que os originais, a fim de facilitar a instalação.
	\item Colocar no condutor uma carretilha com uma corda através de um bastão universal com gancho espiral. A corda deverá estar limpa, seca e testada através do testador de bastão. Este teste deverá ser feito no próprio local
\end{enumerate}

As perguntas são:
\begin{itemize}
    \item Quais são as ferramentas usadas em cada passo?
    \item Dado um certo número de eletricistas, quais passos estariam relacionados com quais eletricistas? 
\end{itemize}


\section{Substituição do Conector - Método a Distância}

Descrição dos passos:

\begin{enumerate}
	\item Colocar no condutor, uma carretilha com uma corda através de um bastão universal com gancho espiral.
	\item Instalar próximo ao conector no jumper a ser liberado, um grampo manipulador de condutor instalado em um bastão universal
	\item Amarrar uma das pontas da corda no bastão, junto ao grampo manipulador de condutor.
	\item Segurar firmemente a corda e o bastão com grampo manipulador de condutor
	\item Soltar os parafusos do conector com uma chave catraca e soquete adequado instalados em um bastão universal.
	\item Havendo parafusos espanados ou fundidos, serrá-los com o arco de serra em um bastão universal
	\item Sacar a conexão
	\item Soltar parcialmente a corda que suporta o condutor, cuidado para não transferir o esforço para o by-pass provisório.
	\item Amarrar a corda em um ponto firme no solo
	\item Com um grampo manipulador de condutor em um bastão universal, segurar o condutor e escovar os pontos da nova conexão
	\item Instalar o novo conector
	\item Fazer última inspeção para prevenir-se de eventuais rachaduras durante a operação de torque.
	\item Retirar o by-pass provisório
	\item Retirar o bastão com a corda de segurança
	\item Retirar a carretilha com a corda.
	\item Recolher o material.
\end{enumerate}

As perguntas são:
\begin{itemize}
    \item O que é um Conector?
    \item Em quais condições um conector é usado?
    \item Nos passos, dá a entender que o conector antigo é removido antes da instalação do conector antigo, como se dá esse processo?
    \item Quais são as ferramentas usadas em cada passo?
    \item Como é feita a instalação da carretilha? Há sub-passos?
    \item Dado um certo número de eletricistas, quais passos estariam relacionados com quais eletricistas?
\end{itemize}

\section{Substituição do Conector - Método ao Potencial}

Descrição dos passos:

\begin{enumerate}
	\item Determinar maneira mais adequada para o acesso ao potencial.
	\item Verificar distâncias entre fase terra e fase fase, e se estas permitem a execução segura do trabalho.
	\item Determinar o melhor ponto para montar o equipamento de acesso, levando sempre em consideração as distâncias de segurança do eletricista na escalada e durante o trabalho ao potencial bem como prever possíveis a amarrações do condutor à escada ou andaime, após a desconexão do mesmo.
	\item Montar o equipamento.
	\item Escalar o andaime ou escada, sempre mantendo as distãncias seguras.
	\item Entrar no potencial.
	\item Fazer uma análise rigorosa das condições da conexão e do condutor.
	\item Escovar os pontos de fixação do by-pass provisório.
	\item Instalar o by-pass provisório.
	\item Colocar no condutor a ser liberado, um grampo de torção com um pedaço de corda amarrada ao mesmo.
	\item Amarrar esta corda no andaime ou escada para evitar o balanço ou queda do mesmo.
	\item Sacar a conexão. Caso haja algum parafuso fundido ou espanado, cortá-lo com serra.
	\item Retirar o conector com defeito.
	\item Ajustar a corda que prende o condutor à escada ou andaime.
	\item Escovar os pontos de fixação do da nova conexão.
	\item Soltar parcialmente a corda que prende o condutor à escada ou andaime.
	\item Instalar o novo conector.
	\item Fazer uma última inspeção para prevenir-se de eventuais rachaduras durante a operação de torque.
	\item Retirar o by-pass provisório.
	\item Retirar a corda de segurança.
	\item Descer ferramentas e materiais.
	\item Sair do potencial.
	\item Desfazer ferramentas e materiais
	\item Enrolar as fitas condutivas colocando-as no bolso da jaqueta e fechando-o e seguida.
	\item Desmontar os equipamentos.
	\item Recolher o material
\end{enumerate}

Perguntas: 
\begin{itemize}
    \item Quais são as ferramentas usadas em cda passo?
    \item Dado um certo número de eletricistas, quais passos estariam relacionados com quais eletricistas?
\end{itemize}