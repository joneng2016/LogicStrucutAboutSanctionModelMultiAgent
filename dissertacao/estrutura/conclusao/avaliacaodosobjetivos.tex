A primeira parte da sentença que define o objetivo geral desse estudo é "Obter um modelo conceitual". Os autores entendem que essa parte do objetivo foi atingido com êxito, por que parte dos resultados apresentam uma estrutura formalizada em termos de conjuntos e predicados que definem a estrutura de um modelo. 

A segunda parte da sentença sobre do objetivo aqui posto é: "que define os conceitos, as relações e as operações". Esse estudo também cumpriu com esse objetivo porque os conjuntos fazem referência a conceitos (respaldado por observações empíricas e análise da literatura) e tanto as relações como as operações são conceitualizadas por meio dos predicados. Algo que é agregado à essa observação, é o fato de que essas estruturas foram usadas para usar um dado caso de estudo.

A terceira parte da sentença sobre o objetivo geral é: "para representar cenários de ambientes de acidentes". Os autores entendem que esse objetivo foi alcançado com sucesso porque os fundamentos dos vocábulos (conjuntos e predicados) bem como as regras são baseados em observações de cenários de acidentes reais e  em estudos que lidam com elementos correspondente a essa problemática. Outro argumento que pode ser usado em prol da veracidade da proposição reside no fato de que a estrutura conceitual resultante foi usada para representar um dado cenário de acidente. Tanto a formalização como os raciocínios, correspondem, não em totalidade, mas (ao menos nos elementos principais) em boa parte da realidade no que tange ao estudo de caso. 

Os autores concluem que esse modelo conceitual apresenta uma formalização e conceitualização da problemática que há em situações onde existe riscos de acidentes (permitindo entender a profundidade do tema), mapeia a rede de conceitos que devem ser especificados por algum dado arcabouço e possibilita arbitrar a utilização especifica de um dado arcabouço.