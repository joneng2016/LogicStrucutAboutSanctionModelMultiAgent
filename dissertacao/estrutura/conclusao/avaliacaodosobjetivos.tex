O autor entende que o objetivo geral do estudo foi atingido com êxito. Para demostrar isso, será feito uma análise detalhada do correspondente do objetivo geral no que tange ao texto do estudo. A seguinte parte do objetivo geral: "\textit{Sintetizar, construir e avaliar, por intermédio de observações, de análises de documentos técnicos, de análises de modelos computacionais e de entrevista com profissionais da área}, foi trabalhada nas seguintes partes do texto; \ref{revexpanalcamp} (metodologia usada para investigar essa questão), \ref{resrevisaoexploratoria} (resultados atrelados a essa parte do objetivo).  A seguinte parte do objetivo geral: "\textit{um modelo conceitual que define os conceitos e as relações para representar os cenários de ambientes de atividades, bem como os respectivos acidentes que podem acontecer}", foi trabalhada nas seguintes partes do texto; \ref{modconceitual} (metodologia), \ref{estconceitual} (resultado e discussão) e \ref{conscritmetcasoestudo} (discussão do modelo referente a estudo de caso). A seguinte parte do objetivo geral; "\textit{em que a validação ocorre por verificar se os raciocínios (para um dado estudo de caso do setor de energia elétrica) resultantes desse modelo são correspondentes com a realidade}" foi trabalhada nas seguintes partes do texto; \ref{inferencias} (metodologia), \ref{casestudy} (resultados), \ref{conscritmetrac} (discussão). A seguinte parte do objetivo geral: "\textit{a fim de levantar um entendimento formal do problema para a comunidade acadêmica no que tange a que tipo de representação computacional é mais apropriada para determinado contexto}" foi trabalhada nas seguintes partes do texto: \ref{possarc} e \ref{analisecomparativa}.

No que tange aos objetivos específicos, pode-se concluir que o objetivo \textit{Identificar os pontos essenciais que devem ser avaliados pelo modelo em relação aos riscos e consequências (acidentes) para os atores e atividades (continuidade), que sejam relevantes na prática da atividade de manutenção, em caso de falha na operação} foi verificado nas seguintes partes do texto: \ref{revexpanalcamp} (metodologia usada para investigar essa questão), \ref{resrevisaoexploratoria} (resultados). O objetivo específico \textit{Construir um modelo conceitual que seja implementável computacionalmente e que produza as inferências que respondam às questões definidas como essenciais} foi avaliado em \ref{modconceitual} (metodologia), \ref{estconceitual} (resultado e discussão) e \ref{conscritmetcasoestudo} (discussão do modelo referente a estudo de caso). O objetivo específico \textit{Validar o modelo por aplica-lo a um dado estudo de caso a fim de averiguar se os raciocínios produzidos nessa situação estão de acordo com a realidade} foi avaliado em \ref{inferencias} (metodologia), \ref{casestudy} (resultados), \ref{conscritmetrac} (discussão). O objetivo específico \textit{Analisar modelos computacionais em relação ao modelo conceitual desse estudo a fim de ter um levantamento formal do estado do problema} foi averiguado em \ref{possarc} e \ref{analisecomparativa}.

Assim sendo, é possível concluir que esse estudo concebeu um modelo conceitual com a capacidade de representar e definir raciocínios sobre acidentes e que, por conta disso, possibilita esclarecer o contexto do problema a ser analisado a luz da computação. 