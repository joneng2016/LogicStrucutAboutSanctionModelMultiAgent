Esse estudo abre margem para muitos trabalhos futuros. Alguns desses residem no fato de que este texto apresenta análises de certos arcabouços na representação do modelo conceitual aqui posto. Tendo em vista isso, para cada arcabouço é possível derivar um estudo futuro a fim de usar o modelo conceitual aqui posto para conceber a formulação de requisitos e especificações. 

Outro estudo futuro possível reside numa reformulação e em um refinamento da estrutura conceitual aqui posta a fim de analisar com maior riqueza de detalhes a discussão presente em \ref{conscritmetcasoestudo} sobre $hasEntity(g_j,e_i)$ e $hasRelation(g_j,r_i)$. A invetigação dos conceitos presentens nesse modelo para formular análises de probabilidades discutido em \ref{conscritmetrac} também apresenta um grande potencial para estudos futuros.

O texto presente em \ref{conscritmetrac} discuti, para cada um dos cinco primeiros raciocínios, elementos que foram muito bem representados assim como elementos que não foram representados adequadamente. Verificar como cada um desses problemas podem ser resolvidos também resultam em estudos futuros. 