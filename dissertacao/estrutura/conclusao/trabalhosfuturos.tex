Esse estudo abre margem para muitos trabalhos futuros. Alguns desses residem no fato de que este texto apresenta análises de certos arcabouços na representação do modelo conceitual aqui posto. Para cada arcabouço é possível derivar um estudo futuro a fim de usar o modelo conceitual aqui posto para conceber a formulação de requisitos e especificações com o propósito de estudar aspectos inerentes ao arcabouço em si (sendo o levantamento de requisitos e especificações deste modelo conceitual um fim em si mesmo de estudo acadêmico). Por exemplo, supondo que em dada circunstância há o interesse de usar o \textit{MOISE+} para especificar um determinado problema, como o modelo conceitual aqui posto pode ser usado para levantar os requisitos no \textit{MOISE+}? Quais são os pontos positivos e negativos dessa situação? Qual cenário é o mais apropriado para essa situação? Qual cenário é o mais desapropriado? Existem adaptações que devem ser feitas no modelo a fim de melhorar a formalização de requisitos? Se sim, quais são, como e por quais motivos isso deve ocorrer? 

O texto presente em \ref{conscritmetrac} discute, para cada um dos cinco primeiros raciocínios, elementos que foram muito bem representados assim como elementos que não foram representados adequadamente. Verificar como cada um desses problemas podem ser resolvidos também resultam em estudos futuros. A listagem a seguir sintetiza esses problemas em sentenças objetivas; 

\begin{enumerate}
    \item Em vez de trabalhar conceitos de possibilidade, sintetizar um modelo estatístico probabilístico na estrutura conceitual proposta neste estudo.
    \item Investigar novas estruturas conceituais para tratar cenários onde os agentes buscam técnicas alternativas para resolver um determinado problema. 
    \item Investigar novas estruturas conceituais onde uma violação pode ou não gerar uma sanção para as condições ambientes (ou seja, em vez de tratar a possibilidade sobre um relacionamento futuro, tratar a possibilidade sobre um relacionamento presente).
    \item Investigar novas estruturas conceituais que considerem a violação de relacionamento em termos de possibilidades e não de um efeito direto no que tange a uma dada causa.   
\end{enumerate}

Uma outra abordagem futura de estudo consiste em produzir um modelo conceitual mais enxuto com estruturas conceituais e de relacionamento mais genéricas. Por exemplo, criar formas genéricas de violações.