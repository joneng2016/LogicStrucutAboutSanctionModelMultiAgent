O interesse desse estudo reside conceber um modelo que possa ser aplicado a uma certa classe de problemas. Assim sendo é importante que os requisitos não sejam focados com base em casos específicos mas que resultem em algo com certa capacidade de generalização. Nessa linha de raciocínio os requisitos são necessariamente os objetivos propostos nesse estudo. 

Os objetivos desse estudo não definem um caso de estudo em específico, mas estruturam o que deve ser modelado e quais são os compomentes que devem necessariamente fazer parte desse sistema. A natureza desses requisitos é estática, ou seja - uma ves que foram definidos (processo obtido na busca do que se deseja investigar), esses elementos não foram modificados. 

Outro ponto interessante reside no fato de que, em muitos casos, os requisitos não são claramente definidos (diversos são os fatores que ocasionam nisso, mas em Engenharia de \textit{Software} isso se dá tendo em vista uma certa falta de comunicação entre o implementador e o solicitante) \cite{softwareeng}. Contudo, esse problema não acontece nesse estudo de caso porque os requisitos são definidos pelos próprios implementadores e os termos são construidos em um vocabulário comum para toda comunidade acadêmica da ciência da computação (agentes, sma, normas, violações, objetivos e entre outros). 
