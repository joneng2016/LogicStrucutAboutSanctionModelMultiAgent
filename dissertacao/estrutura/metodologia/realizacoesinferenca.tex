A próxima etapa dessa pesquisa se deu por construir as regras que definem como se dá a transição de estados do sistema. Os raciocínios portanto, são definidos com base nessas regras que, em linguagem natural, correspondem a \textit{Se ..., Então }. Em termos formais, os pesquisadores optaram por usar o formalismo lógico de implicabilidade dado por $\to$. Também adotaram os seguintes critérios para construir essas regras; elaboração de raciocínios práticos, análise das regras dos outros modelos e verificação da semântica dos vocábulos.   

Com a obtenção dos conjuntos, dos predicados e das regras, os pesquisadores analisaram um caso de manutenção específico, definiram esse caso na estrutura do modelo conceitual e realizaram as inferências com a finalidade de verificar se os cenários que podem ser derivados dessas regras correspondem com a realidade dos fatos. 