Com a formulação do modelo conceito e com o desenvolvimento de inferências que correspondem, em partes, a realidade, os pesquisadores obtiveram um modelo conceitual criterioso suficiente para avaliar a correspondência com os arcabouços disponíveis. Isso é feito por intermédio de uma análise da estrutura conceitual desses modelos a fim de averiguar a correspondência dos mesmos com o modelo resultante.

Uma análise da estrutura conceitual consiste entender a linguagem na qual o modelo é formulado, averiguar a semântica de todos os elementos e comprender as regras de sintaxe. Tendo em vista o fato de que o modelo conceitual possui as suas estruturas semânticas alicerçadas em uma literatura acadêmica que é em comum com esses arcabouços, a equivalência semântica se torna algo relativamente notório de ser feita (existe alguns modelos onde averiguar essa correspondência semântica é algo relativamente complicado de ser feita, contudo essas questões são tratadas e justificadas na discussão).

Como o modelo conceitual permitiu a formulação de regras e raciocínios, a análise comparativa também verifica a correspondência que o modelo conceitual tem com os arcabouços no que tange a conclusões de riscos e acidentes.