Com a formulação do modelo conceito e com o desenvolvimento de inferências que correspondem em partes a realidade, obteve-se um modelo conceitual que permitiu estabelecer correspondências com os arcabouços disponíveis. Isso foi feito por intermédio de uma análise da estrutura conceitual desses modelos, a fim de averiguar a correspondência dos mesmos com o modelo resultante.

Essa análise da estrutura conceitual consiste em entender a linguagem na qual o modelo é formulado, compreender as regras de sintaxe e averiguar a semântica de todos os elementos. Tendo em vista o fato de que o modelo conceitual possui as suas estruturas semânticas alicerçadas em uma literatura acadêmica, que é um ponto em comum com esses arcabouços, a equivalência semântica se torna algo difícil de ser feito (existem alguns modelos onde averiguar essa correspondência semântica é algo relativamente complicado de ser feito, contudo essas questões são tratadas e justificadas na discussão).

Como o modelo conceitual permitiu a formulação de regras e raciocínios, a análise comparativa também verificou a correspondência que o modelo conceitual tem com os arcabouços no que tange a conclusões de riscos e acidentes.