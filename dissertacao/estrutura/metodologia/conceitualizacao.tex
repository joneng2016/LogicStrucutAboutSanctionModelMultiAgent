Uma vez identificado quais são esses modelos (esses arcabouços são apresentados na seção \ref{resrevisaoexploratoria}), e uma vez realizada as observações em campo, o autor formulou conceitos e relações interessantes para o estudo em análise. Feito isso, ocorre a etapa da adaptação, onde essas categorias são redefinidas em estruturas conceituais mais específicas com objetivo de construir um vocabulário especializado para as condições de interesse desse estudo. Como resultado desse processo, o autor obteve uma lista de conceitos e suas relações específicas para representar cenários delineados pelos resultados das observações. 

A próxima etapa reside em escrever esses conceitos em algum tipo de formalismo. Nesse estudo, o autor optou por usar teoria de conjuntos e lógica de predicados. Isso possibilitou estruturar o modelo em voga numa linguagem formal onde cada parte (conjunto, elemento e predicados) é um vocábulo com uma semântica clara. 

Os conjuntos são usados para representar um certo conceito. Os elementos de um conjunto representam os objetos atrelados ao conceito. Por exemplo, supondo que uma atividade de manutenção utilize os seguintes equipamentos de proteção individual; \textit{óculos}, \textit{bota}, \textit{capacete}.
Nesta situação, o conceito de carro é representado pelo conjunto $C$ e os modelos são elementos do mesmo. Portanto, numa linguagem matemática formal tem-se a seguinte situação $C = \{oculos,bota,capacete\}$. 

Dentro do conceito matemático, uma relação é uma correspondência entre elementos de conjuntos não vazio, sendo dada por $R \subseteq  A \times B = \{(a,b)| a \in A \wedge b \in B \}$. A Lógica de Predicados foi usada para representar essas
relações.

O \textit{UML} também é uma ferramenta que foi usada para criar representações do modelo. O propósito disto consiste nos seguintes aspectos: apresentar a perspectiva global do modelo, definir com maior precisão os critérios existenciais (agregação, composição), tornar o processo de apresentação mais didático e aproxima-lo de mecanismos de implementação (ex. linguagens de programação). 

Ainda nessa parte da pesquisa o autor deste estudo construiu as regras que definem como se dá a transição de estados do sistema. Os raciocínios portanto, são definidos com base nessas regras que, em linguagem natural, correspondem a \textit{Se ..., Então }. Em termos formais, o autor optou por usar o formalismo lógico de implicabilidade dado por $\to$. Também foram adotados os seguintes critérios para construir essas regras; elaboração de raciocínios práticos, análise das regras dos outros modelos e verificação da semântica dos vocábulos.