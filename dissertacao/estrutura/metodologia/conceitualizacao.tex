Os pesquisadores, nesse etapa, identificaram quais são os conceitos que devem fazer parte do modelo. Isso se deu por intermédio da análise (e adequação) das estruturas conceitos contidas em modelos existentes.  

O processo de investigação dos modelos se deu por analisar representações que contêm (em sua estrutura) os conceitos dados como necessários pelos requisitos (ao menos em partes) e também que apresentam capacidade de representar acidentes de trabalho (não havia necessidade de encontrar ambas características, apenas uma delas era dado como suficiente para o modelo ser selecionado). Alguns modelos são conglomerados de outros modelos menores. Assim sendo, os pesquisadores preferiram as estruturas de representação abrangentes, pois isso permitiu averiguar diversos elementos relevantes em uma estrutura única.  

Uma vez identificado quais são esses modelos, os pesquisadores isolaram todos os conceitos e relações interessantes para o estudo em análise. Feito isso, ocorre a etapa da adaptação, onde essas categorias são redefinidas em estruturas conceituais mais específicas com objetivo de construir um vocabulário especializado para as condições de interesse desse estudo. Para exemplificar pode-se considerar o conceito de violação dado pelo estudo \cite{dastaniframework}. Na construção do modelo proposto nesta pesquisa, esse conceito de violação é especificado em três tipos; condição, entidade e relação (isso será aprofundado com maior riqueza de detalhes mais adiante). A ocorrência dos três tipos violação é baseado nas observações dos pesquisadores sobre o estudo de caso (visando uma generalização para problemas de mesma ordem), é baseado nos tipos de raciocínios que podem trazer alguma finalidade prática e é baseado na natureza presente nos requisitos. Como resultado dessa, os pesquisadores obtiveram uma lista de conceitos e suas relações específicas para representar cenários delineados pelos requisitos. 


A próxima etapa reside em escrever esses conceitos em algum tipo de formalismo. Nesse estudo, os pesquisadores optaram por usar teoria de conjuntos e lógica de predicados. Isso possibilitou estruturar o modelo em voga numa linguagem formal onde cada parte (conjunto, elemento e predicados) é um vocabulo com uma semântica clara. Com uma representação assim não é possível construir raciocínios, mas já existe a possibilidade de especificar estudos de caso. 

Os conjuntos são usados para representar um dado conceito. Os elementos de um conjunto representam os objetos atrelados ao 
conceito. Por exemplo, supondo que uma loja de Carros venda os seguintes veículos; \textit{Jetta}, \textit{Gol}, \textit{Uno}.
Nesta situação, o conceito de carro é representado pelo conjunto $C$ e os modelos são elementos do mesmo. Portanto, numa linguagem
matemática formal tem-se a seguinte situação $C = \{Jetta,Gol,Uno\}$. 

Dentro do conceito matemático, uma relação é uma correspondência entre elementos de conjuntos não vazio, sendo dada por
$R \subseteq  A \times B = \{(a,b)| a \in A \wedge b \in B \}$. A Lógica de Predicados foi usada para representar essas
relações.

O \textit{UML} também é uma ferramenta que foi usada para criar representações do modelo. O propósito disto consiste nos
seguintes aspectos; apresentar perspectiva global do modelo, definir melhor os critérios existenciais (agregação, composição),
tornar o processo de apresentação mais didático e aproxima-lo de mecanismos de implementação (ex. linguagens de programação). 

A próxima etapa dessa pesquisa se deu por construir as regras que definem como se dá a transição de estados do sistema. Os raciocínios, portanto, são definidos com base nessas regras que, em linguagem natural, correspondem a \textit{Se ..., Então }. Em termos formais, os pesquisadores optaram por usar o formalismo lógico de implicabilidade dado por $\to$. Os pesquisadores adotaram os seguintes critérios para construir essas regras; elaboração de raciocínios práticos, análise das regras dos outros modelos e verificação da semântica dos vocábulos.   

