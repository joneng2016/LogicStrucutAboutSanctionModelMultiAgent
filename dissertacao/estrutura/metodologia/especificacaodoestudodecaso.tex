A especificação necessita de um cenário a ser modelado. Portanto, após tratar com profissionais da área, o autor resolveu trabalhar sobre o seguinte procedimento; "método a distância para a troca de um isolador de pedestal" (será detalhado com maior requisa de detalhes mais adiante em Resultados).

O próximo passo foi contruir um modelo chamado de "Modelagem Baseada em Cenários" \cite{softwareeng}. Esse modelo consiste conceber os eventos que devem ser processados pelo sistema definidos em termos de linguagem natural para cada ator. Esse tipo de modelo é interessante porque consiste em uma ponte de comunicação entre a equipe de desenvolvimento do sistema com os usuários em interesse.     

O Modelo Baseado em Cenários serviu de especficação para formalização da estrutura conceitual proposta neste estudo. Isso se traduz numa segundoa especificação, porém não em linguagem natural mas sim na estrutura formal determinada pelos conjuntos e predicados aqui em voga. Como resultado houve a obtenção do estudo de caso especificado nos vocabulos neste estudo.