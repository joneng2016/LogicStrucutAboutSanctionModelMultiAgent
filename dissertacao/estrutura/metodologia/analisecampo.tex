Para realizar um levantamento dos conceitos que estão atrelados a atividades envolvendo acidentes, se faz necessário analisar uma atividade onde esse tipo de situação acontece. Para isso, o autor optou por estudar manutenção em linha viva onde eletricistas executam atividades preventivas e corretivas em equipamentos elétricos energizados. Esses profissionais são submetidos a riscos de serem eletrocutados e, consequentemente, mortos. 

A análise das atividades se deu por meio dos seguintes pontos definidos na lista que se segue: 
\begin{enumerate}
	\item Estudo dos manuais técnicos privativos a uma companhia de energia, análise (teórica e prática) das ferramentas usadas na execução dessas atividades. \label{estudosprivativos}
	\item Conversas com profissionais que atuam diretamente na área de manutenção. \label{conversas}
	\item Entrevista com engenheiro de manutenção em linha viva (sobre os procedimentos de manutenção) de uma companhia de energia (\ref{pergent}). \label{entrevista}
	\item Acompanhamento de um procedimento de manutenção em linha viva (onde o autor desta dissertação esteve em uma certa subestação verificando de perto a execução dos procedimentos). \label{inloco}
	\item Criação de textos descrevendo cenários de manutenção.
	\item Participação em \textit{workshops} onde recebeu treinamentos sobre técnicas e procedimentos em linha viva.
	\item Revisão bibliográfica exploratória em busca de textos e pesquisas sobre arcabouços que apresentam o potencial para representar esse tipo de situação. \label{revisaoexploratoria}
\end{enumerate}

A etapa referente ao item \ref{conversas} se deu em diversos encontros do autor com os profissionais da área de linha viva tanto informalmente como em reuniões com propósito de realizar troca de informações sobre os procedimentos de manutenção. 

A etapa referente ao item \ref{entrevista} se deu por meio de uma entrevista que durou mais de duas horas com perguntas voltadas para o entendimento de diversos tipo de procedimentos de manutenção (questionário da entrevista em apêndice \ref{pergent}). O Engenheiro respondeu as perguntas do entrevistador de forma clara e didática diante de equipamentos de uma subestação. Toda entrevista foi gravada. A escuta do áudio da mesma foi realizada diversas vezes para a construção do modelo.  

A etapa referente ao item \ref{inloco} se deu através de observações em campo do autor em relação a um dado procedimento em linha viva (troca de um isolador de pedestal) que se deu em uma subestação na cidade de Maringá - Paraná. Durante o ato da manutenção, o autor acompanhou o procedimento de dentro da subestação (em uma região de segurança) a uma distância de aproximadamente 15 metros (adotando procedimentos e equipamentos de proteção individual).

A etapa referente ao item \ref{revisaoexploratoria} seu deu através de pesquisas em bases de dados especializadas em publicação acadêmica nos temas atrelados aspectos que foram identificados nas etapas anteriores sendo essas correlatas a computação. Esses aspectos são concepções relevantes para o entendimento do cenário em análise como, por exemplo,  o fato de que os profissionais se orientam por meio de uma organização ou que a ocorrência de acidentes acontece por descuidos e entre muitos outros aspectos que serão mehor trabalhados ao longo do texto. O termo 'correlatas a computação' faz referência a modelos computacionais que são apropriados para esse aspecto. Uma maneira de comprrender isso consiste em refletir sobre o aspecto de que os trabalhadores atuam de forma organizada por meio de papéis, hierarquias a fim de cumprir com missões e objetivos. Assim sendo, um arcabouço razoável para resolver essa situação é o MOISE+. 

Ainda sobre a etapa \ref{revisaoexploratoria} é importante frisar que os textos foram análisados em função da relevância. Não só isso como foi feita uma análise das referências a fim de encontrar outros textos com pontencial para contribuir a causa desse estudo.