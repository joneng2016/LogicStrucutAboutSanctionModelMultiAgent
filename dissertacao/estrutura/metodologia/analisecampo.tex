Para realizar um levantamento dos conceitos que estão atrelados a atividades envolvendo acidentes, se faz necessário analisar uma atividade onde esse tipo de situação acontece. Para isso, os autores optaram por estudar manutenção em linha viva onde eletricistas executam atividades preventivas e corretivas em equipamentos elétricos energizados. Esses profissionais são submetidos a riscos de serem eletrocutados e, consequentemente, mortos. 

A análise das atividades se deu por meio dos seguintes pontos definidos na lista que se segue: 
\begin{enumerate}
	\item Estudo dos manuais técnicos privativos a uma companhia de energia, análise (teórica e prática) das ferramentas usadas na execução dessas atividades. \label{estudosprivativos}
	\item Conversas com profissionais que atuam diretamente na área de manutenção. \label{conversas}
	\item Entrevista com engenheiro de manutenção em linha viva (sobre os procedimentos de manutenção) de uma companhia de energia. \label{entrevista}
	\item Acompanhamento de um procedimento de manutenção em linha viva (onde o pesquisador esteve em uma certa subestação verificando de perto a execução dos procedimentos). \label{inloco}
	\item Criação de textos descrevendo cenários de manutenção.
	\item Participação em \textit{workshops} onde recebeu treinamentos sobre técnicas e procedimentos em linha viva.
	\item Revisão exploratória em busca de textos e pesquisas sobre arcabouços que apresentam o potencial para representar esse tipo de situação.
\end{enumerate}

A etapa referente ao item \ref{conversas} se deu em diversos encontros dos autores com os profissionais da área de linha viva tanto de forma informal como em reuniões com propósito de realizar transferência de conhecimento sobre os procedimentos de manutenção. 

A etapa referente ao item \ref{entrevista} se deu por meio de uma entrevista que durou mais de duas horas com perguntas voltadas para o entendimento de diversos tipo de procedimentos de manutenção (questionário da entrevista em apêndice). O Engenheiro respondeu as perguntas do entrevistador de forma clara e didática diante de equipamentos de uma subestação. Toda entrevista foi gravada. A análise do áudio da mesma foi analisada diversas vezes para a construção do modelo.  

A etapa referente ao item \ref{inloco} se deu através de observações em campo dos autores em relação a um dado procedimento em linha viva (troca de um isolador de pedestal) que se deu em uma subestação na cidade de Maringá - Paraná. Durante o ato da manutenção, os autores acompanharam o procedimento de dentro da subestação (em uma região de segurança) a uma distância de aproximadamente 15 metros (tanto que tiveram de adotar procedimentos e equipamentos de proteção individual).   