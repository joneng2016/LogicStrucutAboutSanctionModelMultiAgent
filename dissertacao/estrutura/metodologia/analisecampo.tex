Para realizar um levantamento dos conceitos que estão atrelados a atividades envolvendo acidentes, se faz necessário analisar uma atividade onde esse tipo de situação acontece. Para isso, os pesquisadores optaram por estudar manutenção em linha viva onde eletricistas executam atividades preventivas e corretivas em equipamentos elétricos energizados. Esses profissionais são submetidos a riscos de serem eletrocutados e, consequentemente, mortos. 

A análise de atividades se deu por meio dos seguintes pontos; estudo dos manuais técnicos privativos a uma dada companhia de energia, análise (teórica e prática) das ferramentas usadas na execução dessas atividades, conversas com profissionais que atuam diretamente na área de manutenção, entrevista com engenheiro de manutenção em linha viva de uma dada companhia de energia, acompanhamento de um procedimento de manutenção em linha viva (o pesquisador esteve em uma dada subestação com os EPI'S verificando de perto a execução dos procedimentos), criação de textos descrevendo cenários de manuntenção e solicitando a correção desses por profissionais especializados na área. Além disso, a equipe fez parte de diversos \textit{workshops} onde recebou treinamentos sobre técnicas e procedimentos em linha viva. 

Nessa etapa, em paralelo aos estudos de como ocorre as manutenções, os pesquisadores fizeram uma revisão exploratória em busca de textos e pesquisas sobre aracabouços que apresentam o potencial para representar esse tipo de situação. Apesar de não saberem ao certo qual é o modelo e quais são os conceitos mais apropriados para esse propósito, os pesquisadores entendiam a necessidade de uma estrutura que leve em consideração questões organizacionais, normativas e punitivas. Por conta disso, o alvo da revisão exploratória eram os modelos relacionados a sistemas multiagentes (normativos ou não). Isso se deve ao fato de que as observações em campo apontavam claramente para questões relacionadas a pessoas que trabalham de forma colaborativa, contudo apresentam o potencial para realizar um dado comportamento inapropriado. Sobre certas condições os comportamentos inapropriados geram implicações negativas sobre si mesmos sofrendo, portanto, acidentes. Contudo, as observações também mostraram que esses comportamentos podem resultar consequências inadvetidas a outros profissionais que não cometeram nenhum tipo de conduta inapropriada. É dingo de nota que os pesquisadores obtiveram ciência dessas questões por análisar as atividades em linha viva.