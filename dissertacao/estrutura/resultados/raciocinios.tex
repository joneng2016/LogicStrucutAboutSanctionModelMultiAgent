\label{racs}

Uma vez que o modelo foi definido e que foi implementado em um estudo de caso, é possível avaliar as conclusões possíveis dado certa condição de mundo. Essa seção demonstra 
como esse modelo cumpre o proposto por demonstrar certos raciocínios tendo em vista o caso de estudo em análise. 

\subsubsection{Raciocínio - 1} 
\label{raciocinio1}

O raciocínio a seguir mostra o que acontece se o $agente4$ esquecer de passar glicerina no pano, dado pela relação $relPanoGlicerina$, designados a ele no objetivo $g1$. 
Todos os predicados vinculados a essa situação são;

\begin{enumerate}
	\item $hasRole(agente4,executor2)$ 
	\item $hasObligation(executor2,g1)$
	\item $hasRelation(g1,rg1)$ 
	\item $relPanoCorda \in rg1$
	\item $ x = agente4 $
	\item $tryReach(agente4,g1)$
	\item $affectsOtherRelations(relPanoGlicerina,relBastaoGarraCondutor)$
	\item $affectsOtherRelations(relPanoGlicerina,relCordaEstropo)$  
	\item $affectsOtherRelations(relPanoGlicerina,relChaveCatracaParafuso)$
	\item $affectsOtherRelations(relPanoGlicerina,relParafusoConector)$ 
	\item $affectsOtherRelations(relPanoGlicerina,relSoqueteParafuso)$ 
	\item $affectsOtherRelations(relPanoGlicerina,relAgente4Corda)$ 
	\item $affectsOtherRelations(relPanoGlicerina,relEstropoCorda)$	
\end{enumerate}

Com base nisso, as relações de implicabilidade resultantes são;

\begin{eqnarray}\nonumber
	hasRelation(g1,rg1)\wedge \neg isPresent(relPanoGlicerina)  \nonumber \\ 
	\wedge (relPanoGlicerina\in rg_1) \wedge tryReach(agente4,g1) \nonumber \\ 
	\to \nonumber \\ 
	violationRelation(agente4,g1,relPanoGlicerina) \nonumber \\	
\end{eqnarray}

\begin{eqnarray}\nonumber
	violationRelation(agente4,g1,relPanoGlicerina)  \nonumber \\ 
	\wedge affectsOtherRelations(relPanoGlicerina,relBastaoGarraCondutor)   \nonumber \\ 
	\to \nonumber \\  
	possibilityHappensBadEvent(relBastaoGarraCondutor) 
\end{eqnarray}

\begin{eqnarray}\nonumber
	violationRelation(agente4,g1,relPanoGlicerina)  \nonumber \\ 
	\wedge affectsOtherRelations(relPanoGlicerina,relCordaEstropo)   \nonumber \\ 
	\to \nonumber \\  
	possibilityHappensBadEvent(relCordaEstropo) 
\end{eqnarray}

\begin{eqnarray}\nonumber
	violationRelation(agente4,g1,relPanoGlicerina)  \nonumber \\ 
	\wedge affectsOtherRelations(relPanoGlicerina,relParafusoConector)   \nonumber \\ 
	\to \nonumber \\  
	possibilityHappensBadEvent(relParafusoConector) 
\end{eqnarray}

\begin{eqnarray}\nonumber
	violationRelation(agente4,g1,relPanoGlicerina)  \nonumber \\ 
	\wedge affectsOtherRelations(relPanoGlicerina,relSoqueteParafuso)   \nonumber \\ 
	\to \nonumber \\  
	possibilityHappensBadEvent(relSoqueteParafuso) 
\end{eqnarray}

\begin{eqnarray}\nonumber
	violationRelation(agente4,g1,relPanoGlicerina)  \nonumber \\ 
	\wedge affectsOtherRelations(relPanoGlicerina,relAgente4Corda)   \nonumber \\ 
	\to \nonumber \\  
	possibilityHappensBadEvent(relAgente4Corda) 
\end{eqnarray}

\begin{eqnarray}\nonumber
	violationRelation(agente4,g1,relPanoGlicerina)  \nonumber \\ 
	\wedge affectsOtherRelations(relPanoGlicerina,relEstropoCorda)   \nonumber \\ 
	\to \nonumber \\  
	possibilityHappensBadEvent(relEstropoCorda) 
\end{eqnarray}

\begin{eqnarray}\nonumber
	violationRelation(agente4,g1,relPanoGlicerina)  \nonumber \\ 
	\wedge affectsOtherRelations(relPanoGlicerina,relEstropoCorda)   \nonumber \\ 
	\to \nonumber \\  
	possibilityHappensBadEvent(relEstropoCorda) 
\end{eqnarray}



\subsubsection{Raciocínio - 2} 
\label{raciocinio2}
O raciocínio a seguir mostra o que acontece se o pano não estiver presente no local da manutenção quando os eletricistas forem alcançar o $g1$. A lista a seguir exibe todos os predicados necessários para averiguar essa condição de mundo. 


\begin{enumerate}
	\item $hasRole(agente2,executor1)$ 
	\item $hasRole(agente3,executor1)$	 	
	\item $hasRole(agente4,executor2)$	 
	\item $hasObligation(executor1,g1)$
	\item $hasObligation(executor2,g1)$
	\item $tryReach(agente2,g1)$ 
	\item $tryReach(agente3,g1)$	 	
	\item $tryReach(agente4,g1)$	
	\item $hasEntity(g1,eg1)$		
	\item $pano \in eg1$
	\item $\neg isPresent(pano)$
\end{enumerate}

\begin{eqnarray}\nonumber
	hasEntity(g1,eg1) \nonumber \\ 
	\wedge \neg isPresent(pano) 	\nonumber \\ 
	\wedge (pano \in eg1) \wedge tryReach(agente2,g1) \to \nonumber \\ 
	violationEntity(agent2,g1,pano) \nonumber \\
\end{eqnarray}

\begin{eqnarray}\nonumber
	hasEntity(g1,eg1) \nonumber \\ 
	\wedge \neg isPresent(pano) 	\nonumber \\ 
	\wedge (pano \in eg1) \wedge tryReach(agente3,g1) \to \nonumber \\ 
	violationEntity(agente3,g1,pano) \nonumber \\
\end{eqnarray}

\begin{eqnarray}\nonumber
	hasEntity(g1,eg1) \nonumber \\ 
	\wedge \neg isPresent(pano) 	\nonumber \\ 
	\wedge (pano \in eg1) \wedge tryReach(agente4,g1) \to \nonumber \\ 
	violationEntity(agente4,g1,pano) \nonumber \\
\end{eqnarray}

\begin{eqnarray}
	violationEntity(agente4,g1,pano) \to stopIn(g1)
\end{eqnarray}



\subsubsection{Raciocínio - 3} 
\label{raciocinio3}
O raciocínio a seguir mostra o que acontece se o $agente5$ tentar alcançar o objetivo $g11$ com a umidade relativa do ar superior a setenta porcento. A lista a seguir exibe todos os predicados necessários para averiguar essa condição de mundo. 

\begin{enumerate}
	\item $hasRole(agente5,executor3)$
	\item $hasObligation(executor3,g11)$	
	\item $tryReach(agente5,g11)$ 
	\item $hasCondition(g11,cg1)$
	\item $umidade70 \in cg1$	
	\item $\neg isPresent(umidade70)$
	\item $hasRisk(umidade70,eletrocutado,morte)$
\end{enumerate}

\begin{eqnarray}
	hasCondition(g11,cg1) \nonumber \\ 
	\wedge \neg isPresent(umidade70) \nonumber \\
	\wedge umidade70 \in cg1 \nonumber \\
	\wedge tryReach(agente5,g11) \to \nonumber \\  
	violationCondition(agente5,g11,umidade70) \nonumber \\
\end{eqnarray}

\begin{eqnarray} \nonumber
	violationCondition(agente5,g11,umidade70) \nonumber \\
	\wedge hasRisk(umidade70,eletrocutado,morte) \to \nonumber \\  
	consequenceOfBadEvent(g11,agente5,eletrocutado,morte)
\end{eqnarray}


\begin{eqnarray}
	consequenceOfBadEvent(g11,agente5,eletrocutado,morte) \to stopIn(g11)
\end{eqnarray}
\subsubsection{Raciocínio - 4} 
\label{raciocinio4}
O raciocínio a seguir mostra o que acontece se o $agente3$ errar a forma adequada de realizar o relacionamento $relChaveCatracaParafuso$ no objetivo $g15$. Os predicados envolvidos são;

\begin{enumerate}
	\item $hasRole(agente4,executor2)$
	\item $hasObligation(executor4,g15)$	
	\item $tryReach(agente4,g15)$ 
	\item $hasRelation(g15,rg15)$
	\item $relChaveCatracaParafuso \in rg15$	
	\item $\neg isPresent(relChaveCatracaParafuso)$
	\item $hasRisk(relChaveCatracaParafuso,eletrocutado,morte)$
\end{enumerate}

\begin{eqnarray}
	hasRelation(g15,rg15) \nonumber \\
	\wedge \neg isPresent(relChaveCatracaParafuso)  \nonumber \\ 
	\wedge (relChaveCatracaParafuso \in rg15) \nonumber \\
	\wedge tryReach(agente4,g15) \nonumber \\ 
	\to \nonumber \\ 
	violationRelation(agente4,g15,relChaveCatracaParafuso) \nonumber \\
\end{eqnarray}

\begin{eqnarray}\nonumber
	violationRelation(agente4,g15,relChaveCatracaParafuso) \nonumber \\ 
	 \wedge hasRisk(relChaveCatracaParafuso,eletrocutado,morte) \nonumber \\ 
	\to \nonumber \\ 
	consequenceOfBadEvent(g15,agente4,eletrocutado,morte)
\end{eqnarray}

\begin{eqnarray}
	consequenceOfBadEvent(g15,agente4,eletrocutado,morte) \to stopIn(g15)
\end{eqnarray}

\subsubsection{Raciocínio - 5} 
\label{raciocinio5}
A finalidade dessa demonstração consiste em mostrar como um agente pode ser submetido a consequências ruins tendo em vista erros cometidos por outros profissionais. O raciocínio 1 mostra que o fato do $agente4$ não conseguir realizar o relacionamento $relPanoGlicerina$ resulta na violação $violationRelation(agente4,g1,relPanoGlicerina)$. Essa violação, por sua vez, impacta diversas outras relações, em que 
$possibilityHappensBadEvent(relParafusoConector)$ é uma delas. Assim sendo, antes do $agente4$ cometer o erro, a possibilidade da ocorrência de um evento ruim acontecer era 0, se o agente realizar a relação $relParafusoConector$ sem cometer violação alguma. Contudo, após a ocorrência do erro cometido pelo $agent4$, existe uma possibilidade de um evento ruim acontecer na relação $relParafusoConector$ mesmo que tudo seja feito de acordo com os conformes. Assim sendo, a lista de predicados e o raciocínio mostra o que acontece dado a seguinte situação; o possível evento ruim presente em $relParafusoConector$ 
se torna uma realidade;  

\begin{enumerate}
	\item $relParafusoConector \in rg19$	
	\item $hasRelation(g19,rg19)$		
	\item $hasObligation(executor3,g19)$
	\item $hasObligation(executor4,g19)$
	\item $hasObligation(executor5,g19)$		
	\item $tryReach(agente5,g19)$
	\item $tryReach(agente6,g19)$
	\item $tryReach(agente7,g19)$									
	\item $hasRole(agente5,executor3)$
	\item $hasRole(agente6,executor4)$
	\item $hasRole(agente7,executor5)$
	\item $hasRisk(relParafusoConector,eletrocutado,morte)$
	\item $possibilityHappensBadEvent(relParafusoConector)$
	\item $happensBadEvent(g19,relParafusoConector)$	
\end{enumerate}


\begin{eqnarray}\nonumber
    possibilityHappensBadEvent(relParafusoConector) \nonumber \\
    \wedge happensBadEvent(relParafusoConector) \nonumber \\ 
    \wedge hasRelation(g19,rg19)  \nonumber \\  
    \wedge (relParafusoConector \in rg19) \nonumber \\ 
    \wedge hasRisk(relParafusoConector,eletrocutado,morte) \nonumber \\  
    \wedge tryReach(agente5,g19) \nonumber \\ 
    \to consequenceOfBadEvent(g19,agente5,eletrocutado,morte) \\ \nonumber
\end{eqnarray}

\begin{eqnarray}
	consequenceOfBadEvent(g19,agente5,eletrocutado,morte) \to stopIn(g19)
\end{eqnarray}

\begin{eqnarray}\nonumber
   possibilityHappensBadEvent(relParafusoConector) \nonumber \\
    \wedge happensBadEvent(relParafusoConector) \nonumber \\ 
    \wedge hasRelation(g19,rg19)  \nonumber \\  
    \wedge (relParafusoConector \in rg19) \nonumber \\ 
    \wedge hasRisk(relParafusoConector,eletrocutado,morte) \nonumber \\  
    \wedge tryReach(agente6,g19) \nonumber \\ 
    \to consequenceOfBadEvent(g19,agente6,eletrocutado,morte) \\ \nonumber
\end{eqnarray}

\begin{eqnarray}
	consequenceOfBadEvent(g19,agente6,eletrocutado,morte) \to stopIn(g19)
\end{eqnarray}

\begin{eqnarray}\nonumber
   possibilityHappensBadEvent(relParafusoConector) \nonumber \\
    \wedge happensBadEvent(relParafusoConector) \nonumber \\ 
    \wedge hasRelation(g19,rg19)  \nonumber \\  
    \wedge (relParafusoConector \in rg19) \nonumber \\ 
    \wedge hasRisk(relParafusoConector,eletrocutado,morte) \nonumber \\  
    \wedge tryReach(agente7,g19) \nonumber \\ 
    \to consequenceOfBadEvent(g19,agente7,eletrocutado,morte) \\ \nonumber
\end{eqnarray}

\begin{eqnarray}
	consequenceOfBadEvent(g19,agente7,eletrocutado,morte) \to stopIn(g19)
\end{eqnarray}


\subsubsection{Raciocínio - 6}
\label{raciocinio6}
Esse raciocínio tem a finalidade de mostrar como se dá os raciocínios para quando se faz necessário verificar se um objetivo foi atingido. O objetivo $g23$ deve ser atingido pelos agentes com as funções de $executor1$,$executor2$,$executor3$ e $executor5$. Isso implica dizer que os agentes; $agente2$,$agente3$,$agente4$,$agente5$ e $agente7$ devem tentar alcançar esses resultados. Considerando que $agg23$ são todos os agentes que tentaram alcançar o objetivo e $ago23$ os agentes que são obrigados a fazer isso, segue o raciocínio;


\begin{enumerate}
	\item $agente2 \in agg23$	
	\item $agente3 \in agg23$
	\item $agente4 \in agg23$
	\item $agente5 \in agg23$
	\item $agente7 \in agg23$								
	\item $agente2 \in ago23$	
	\item $agente3 \in ago23$
	\item $agente4 \in ago23$
	\item $agente5 \in ago23$
	\item $agente7 \in ago23$	
	\item $ago23 \subset agg23$
	\item $\neg stopIn(g23,agg23)$										
\end{enumerate}

\begin{eqnarray}\label{rel15}
	\neg stopIn(g23,agg23) \wedge (ago23 \subset agg23) \to isReached(g23)
\end{eqnarray}

\subsubsection{Raciocínio - 7}
\label{raciocinio7}
O raciocínio para o caso onde $agente1$ tente alcançar o objetivo $g23$.  

\begin{enumerate}
	\item $hasRole(agente1,supervisor)$
	\item $hasObligation(agente1,g23) \to F$										
\end{enumerate}

Isso implica uma afirmação falsa, então esse mundo não é possível segundo o modelo implementado para este estudo de caso.