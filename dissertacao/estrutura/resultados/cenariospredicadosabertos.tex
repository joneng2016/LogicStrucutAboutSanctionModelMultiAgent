Esse estudo apresenta duas categorias de predicados: \textit{controle} e \textit{estrutura}. Os predicados de estrutura são aqueles, cujo usuário do modelo não possui a liberdade de definir sua estrutura interna por intermédio de outras regras lógicas ou por valores. Isso se dever ao fato de que esses vocábulos tem sua estrutura alicerçada nas concepções deste modelo sendo que são essenciais para que o modelo funcione como foi concebido para ser. Assim sendo, o modelador deve fazer uso delas, apenas com o propósito de especificar os objetos de interesse. Em termos práticos não existe dificuldade em identificar esses predicados, pois sua própria natureza não abre margem para que o modelador consiga escrever novos predicados e novas regras para determinar o seu respectivo valor. Esses predicados são:

\begin{itemize}
    \item $possEntityRel(r_l,e_i,e_k)$
    \item $adoptsRole(ag_n,\rho_m)$
    \item $hasObligation(\rho_m,g_j)$
    \item $hasPermission(\rho_m, g_j)$
    \item $reached(g_k)$
    \item $stopped(g_n, ag_m)$
    \item $nextGoal(g_i,g_j)$
    \item $requiresCirc(g_i,cir_k)$
    \item $requiresEntity(g_i,e_k)$
    \item $conditionViol(ag_i,g_j,c_k)$
    \item $relationViol(ag_i,g_j,r_k)$
    \item $entityViol(ag_i,g_j,e_k)$
    \item $hasRisk(circ_i, risk_j, cs_k)$
    \item $possOfNegConseqFor(r_l)$
    \item $affectsRels(r_k,r_n)$
    \item $negConseqFor(g_k, ag_i,risk_k,cs_m)$
    \item $lastGoal(g_i,\rho_m)$
    \item $instanceOfRel(circ_i)$
    \item $instanceOfCond(circ_i)$
    \item $enabledToStart(ag_i,g_j)$ 
\end{itemize}

Para exemplificar, pode-se considerar o predicado $possEntityRel(r_l,e_i,e_k)$. Se existir uma entidade $A$, uma entidade $B$ e uma relação entre $relAB$, então esse termo é escrito desta forma: $possEntityRel(relAB,A,B)$. O valor verdade deste predicado, não pode ser modificado para a criação de algum cenário e nem pode ser determinado por outras regras. Se modelador fizer isso então estará modificando a estrutura do modelo. Ou seja, esse é um predicado de estrutura no que tange aos aspectos fundamentais e semânticos desta representação. 

Por outro lado os predicados \textit{de controle} possuem um correspondente sintático e semântico no modelo, mas os seus valores devem ser forçados conforme o cenário que se deseja criar ou conforme outras regras de implicabilidade que não fazem parte da estrutura deste modelo. A não determinação dos valores destes predicados inviabilizam que o modelo seja analisado de forma procedural. Portanto, a aplicabilidade deste modelo em um dado contexto a fim de fazer uma averiguação procedural dos fatos só pode ser feita por determinar os valores Verdade para esses predicados. Faz parte deste conjunto os seguintes termos: 

\begin{itemize}
    \item $isPresent(X)$
    \item $starts(ag_i,g_j)$
    \item $happensNegConseqFor(r_k)$
\end{itemize}

Pode-se considerar o seguinte exemplo: Um agente $ag_a$ deve executar o objetivo $g_1$ e $g_2$, os predicados a seguir implementam este modelo para o exemplo:

\begin{itemize}
    \item $nextGoal(g_1,g_2)$
    \item $possEntityRel(rAB,entA,entB)$
    \item $possEntityRel(rCE,entC,entE)$
    \item $instanceOfRel(rAB)$,
    \item $instanceOfRel(rCE) $
    \item $instanceOfCond(cond_1)$
    \item $requiresCirc(g_1,rAB)$
    \item $requiresCirc(g_2,rCE)$
    \item $requiresCirc(g_1,cond_1)$
    \item $requiresCirc(g_2,cond_1)$    
    \item $requiresEntity(g_1,entA)$
    \item $requiresEntity(g_1,entB)$
    \item $requiresEntity(g_2,entC)$
    \item $requiresEntity(g_2,entE)$            
    \item $affecftsRels(rAB,rCE) $ 
    \item $hasRisk(cg_1,risk_1,cs_1)$
    \item $hasRisk(rCE,risk_2,cs_2)$
    \item $adoptsRole(ag_a,\rho_1)$
    \item $hasObligation(\rho_1,g_1)$
    \item $hasObligation(\rho_1,g_2)$
\end{itemize}

Apesar de todos os predicados denotarem uma certa condição, e de serem suficientes, para definir uma certa representação de mundo, não é possível fazer raciocínio algum. Isso, pois não se sabe quais são as ações dos agentes tão pouco quais condições e cenários se deseja representar.  

Para isso, se faz necessário definir um cenário de mundo. Por exemplo, pode-se definir o seguinte cenário: $starts(ag_a,g_1)$, $\neg isPresent(rAB)$, $starts(ag_b,g_2)$ e $ possOfNegConseqFor(rCE) \to happensNegConseqFor(rCE)$. 

Para esse caso é possível obter as seguintes relações de inferência: 

\begin{eqnarray}
    requiresCirc(g_1,r_AB)\wedge 
    \neg isPresent(rAB) \wedge 
    instanceOfRel(rAB) \wedge 
    starts(ag_a,g_1) \to \nonumber \\
    relationViol(ag_a,g_1,rAB) 
\end{eqnarray}

\begin{eqnarray}
    relationViol(ag_a,g_1,rAB)  \wedge 
    affectsRels(rAB,rCE) \nonumber \\
    \to possOfNegConseqFor(rCE)  
\end{eqnarray}

\begin{eqnarray}
    possOfNegConseqFor(rCE) \to happensNegConseqFor(rCE) 
\end{eqnarray}

\begin{eqnarray}\label{paybutiamnotguilty}
   possOfNegConseqFor(rCE) \nonumber \\
   \wedge happensNegConseqFor(rCE) \nonumber \\ 
   \wedge requiresCirc(g_2,rCE) \nonumber \\  
   \wedge instanceOfRel(rCE) \nonumber \\ 
   \wedge hasRisk(rCE,eletrocutado,cs_2) \nonumber \\  
   \wedge starts(ag_a,g_2) \nonumber \\ 
   \to negConseqFor(g_2,ag_a,risk_2,cs_2) \\ \nonumber
\end{eqnarray}


\begin{eqnarray}\label{badcons}
    negConseqFor(g_2,ag_a,risk_2,cs_2) \to stopped(g_2) 
\end{eqnarray}


Esses raciocínios e conclusões só foram possíveis porque o modelador forçou o valor de três predicados e definiu uma relação de implicação. Isso acontece por conta de três motivos: 1 - Esse é um modelo de \textit{SMA}, 2 - esse modelo apresenta grau de liberdade para escolher a disposição das entidades, condições e relações e 3 - não há como definir a solução de uma possibilidade. 

Para o primeiro caso o predicado $starts(ag_i,g_j)$ é resultado de estados internos do agente. Por exemplo, o desenvolvedor pode programar um agente que possui o estado de medo, então sob certas condições ele resolve não tentar alcançar o objetivo gerando valor falso para esse predicado, ou pode definir um agente que pondera pouco ao decidir se deve ou não tentar alcançar o objetivo. Isso pode ser feito por meio de modelos de agentes tais como: agentes lógicos, arquitetura BDI, agentes reativos e agentes em camada. Se for do interesse do modelador, o mesmo pode simplesmente definir o valor verdade para o predicado em certas condições. 

O mesmo se aplica para o $isPresent(X)$ em que o desenvolvedor pode definir um cenário por meio dos estados internos, por exemplo, os agentes esqueceram uma determinada ferramenta em um certo local ou, o agente apresenta um algoritmo para determinar qual ferramenta é a mais apropriada para uma condição específica. O modelador também é livre para gerar diferentes cenários simplesmente por definir valores diferentes para $isPresent(X)$. Por exemplo, supondo que uma equipe está desenvolvendo um jogo sério para avaliar profissionais de uma certa indústria. Para avaliar a competência dos trabalhadores, o modelador poderá usar este predicado por adicionar ou remover entidades e condições com base nas necessidades de avaliação.

O terceiro motivo reside no fato de que o predicado $possOfNegConseqFor(X)$ denota apenas que existe uma possibilidade de ocorrer algum evento ruim  $happensNegConseqFor(X)$. Contudo, se esse evento ocorrerá ou não, não é possível definir pois isso depende de questões estatísticas do objeto de estudo. Assim sendo, o usuário deste modelo possui algumas possibilidades de ação, tais como: quando $possOfNegConseqFor(X)$  for verdade, então definir $happensNegConseqFor(X)$ por meio de um número aleatório, para algumas situações onde ocorre $possOfNegConseqFor(X)$, tratar $happensNegConseqFor(X)$ como verdade e para dadas situações tratar $happensNegConseqFor(X)$ como falso, ou definir verdade para $happensNegConseqFor(X)$ como base em algum estudo probabilístico. Isso dependerá da finalidade dos modeladores. 