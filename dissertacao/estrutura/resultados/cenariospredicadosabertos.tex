Esse estudo apresenta duas categorias de predicados: \textit{abertos} e \textit{fechados}. Os predicados fechados são aqueles cujo usuário do modelo não possui a liberdade de definir sua estrutura interna por intermédio de outras regras lógicas ou por valores. Isso se dever ao fato de que esses vocábulos tem sua estrutura alicerçada nas concepções deste modelo sendo que são essenciais para que o modelo funcione como foi concebido para ser. Assim sendo, o modelador deve fazer uso delas apenas com o propósito de especificar os objetos de interesse. Em termos práticos não existe dificuldade em identificar esses predicados, pois sua própria natureza não abre margem para que o modelador consiga escrever novos predicados e novas regras para determinar o seu respectivo valor. 

Esses predicados são $thereIsRelation(r_l,e_i,e_k)$, $hasRole(ag_n,\rho_m)$, $hasObligation(\rho_m,g_j)$,
$hasPermission(\rho_m, g_j)$, $isReached(g_k)$, $stopIn(g_n, ag_m)$, $nextGoal(g_i,g_j)$, $hasCondition(g_i,cg_n)$,
$hasEntity(g_i,eg_m)$, $hasRelation(g_i,rg_m)$, $violationCondition(ag_i,g_j,c_k) $, $ violationRelation(ag_i,g_j,r_k) $,
$ violationEntity(ag_i,g_j,e_k) $,  $ hasRisk(X, risk_j, cs_k) $, $possibilityHappensBadEvent(r_l)$, 
$affectsOtherRelations(r_k,r_n)$, $consequenceOfBadEvenet(g_k, ag_i,risk_k,cs_m)$  e $lastGoal(g_i,\rho_m)$. 


Para exemplificar, pode-se considerar o predicado $thereIsRelation(r_l,e_i,e_k)$. Se existir uma entidade $A$, uma entidade $B$ e uma relação entre $relAB$, então esse termo é escrito desta forma: $thereIsRelation(relAB,A,B)$. O valor verdade deste predicado não pode ser modificado para a criação de algum cenário e nem pode ser determinado por outras regras. Se modelador fizer isso então estará modificando a estrutura do modelo. Ou seja, esse é um predicado fechado no que tange a aspectos fundamentais aos aspectos semânticos desta representação. 

Por outro lados os predicados \textit{abertos} possuem um correspondente sintático e semântico no modelo mas os seus valores devem ser forçados conforme o cenário que se deseja criar ou conforme outras regras de implicabilidade. A não determinação destes predicados inviabilizam que o modelo seja analisado de forma procedural. Faz parte deste conjunto os seguintes termos: $isPresent(X)$,$tryReach(ag_i,g_j)$ e $happensBadEvent(r_m)$.

Pode-se considerar o seguinte exemplo: Um agente $ag_a$ deve executar o objetivo $g_1$ e $g_2$, os predicados a seguir implementam este modelo para o exemplo:

\begin{itemize}
    \item $nextGoal(g_1,g_2)$
    \item $thereIsRelation(rAB,entA,entB)$
    \item $thereIsRelation(rCE,entC,entE)$
    \item $entA \in eg_1 ,entB \in eg_1$
    \item $entC \in eg_2, entD \in eg_2$
    \item $rAB \in rg_1$,
    \item $rCE \in rg_2 $
    \item $cond_1 \in {cg_1} $
    \item $ hasCondition(g_1,cg_1)$
    \item $ hasCondition(g_2,cg_2 )$
    \item $ hasEntity(g_1,eg_1) $
    \item $ hasEntity(g_2,eg_2) $
    \item $ hasRelation(g_1, rg_1 )$
    \item $ hasRelation(g_2, rg_2)$ 
    \item $ affectsOtherRelations(rAB,rCE) $ 
    \item $hasRisk(cg_1,risk_1,cs_1)$
    \item $hasRisk(rCE,risk_2,cs_2)$
    \item $hasRole(ag_a,\rho_1)$
    \item $hasObligation(\rho_1,g_1)$
    \item $hasObligation(\rho_1,g_2)$
\end{itemize}

Apesar de todos os predicados denotarem uma dada condição e de serem o suficientes para definir uma certa representação de mundo, não é possível fazer raciocínio algum. Isso, pois não se sabe quais são as ações dos agentes e não se sabe quais condições e cenários se deseja representar.  

Para isso, se faz necessário definir um cenário de mundo. Por exemplo, pode-se definir o seguinte cenário; $tryReach(ag_a,g_1)$, $\neg isPresent(rAB), tryReach(ag_b,g_2)$ e $ possibilityHappensBadEvent(rCE) \to $ $happensBadEvent(rCE)$. 

Para esse caso é possível obter as seguintes relações de inferência: 

\begin{eqnarray}
    hasRelation(g_1,rg_1)\wedge \neg isPresent(rAB) \wedge (rAB \in rg_1) \wedge tryReach(ag_a,g_1) \to \nonumber \\
    violationRelation(ag_a,g_1,rAB) 
\end{eqnarray}

\begin{eqnarray}
    violationRelation(ag_a,g_1,rAB)  \wedge affectsOtherRelations(rAB,rCE) \nonumber \\
    \to possibilityHappensBadEvent(rCE)  
\end{eqnarray}

\begin{eqnarray}
    possibilityHappensBadEvent(rCE) \to happensBadEvent(rCE) 
\end{eqnarray}

\begin{eqnarray}\label{paybutiamnotguilty}
    possibilityHappensBadEvent(rCE) \wedge  \nonumber \\
    happensBadEvent(rCE) \wedge  \nonumber \\
    hasRelation(g_2,rg_2) \wedge  \nonumber \\
    (rCE \in rg_2) \nonumber \wedge  \nonumber \\
    hasRisk(rCE,risk_2,cs_2) \wedge  \nonumber \\
    tryReach(ag_a,g_2) \nonumber  \nonumber \\
    \to consequenceOfBadEvent(g_2,ag_a,risk_2,cs_2) 
\end{eqnarray}


\begin{eqnarray}\label{badcons}
    consequenceOfBadEvent(g_2,ag_a,risk_2,cs_2) \to stopIn(g_2) 
\end{eqnarray}


Esses raciocínios e conclusões só foram possíveis porque o modelador forçou o valor de três predicados e definiu uma relação de implicação. Isso acontece por conta de três motivos: 1 - Esse é um modelo de \textit{SMA}, 2 - esse modelo apresenta grau de liberdade para escolher a disposição das entidades, condições e relações e 3 - não há como definir a solução de uma possibilidade. 

Para o primeiro caso o predicado $tryReach(ag_i,g_j)$ é resultado de estados internos do agente. Por exemplo, o desenvolvedor pode programar um agente que possui o estado de medo, então sobre certas condições ele resolve não tentar alcançar o objetivo gerando valor falso para esse predicado, ou pode definir um agente que pondera pouco ao decidir se deve ou não tentar alcançar um dado objetivo. Isso pode ser feito por meio de modelos de agentes tais como: agentes lógicos, arquitetura BDI, agentes reativos e agentes em camada. Se for do interesse do modelador, o mesmo pode simplesmente definir o valor verdade para o predicado em certas condições. 

O mesmo se aplica para o $isPresent(X)$ onde desenvolvedor pode definir um cenário que, por meio de estados internos os agentes esqueceram uma determinada ferramenta em um certo local ou, por exemplo, que o agente apresenta um algoritmo para determinar qual ferramenta é a mais apropriada para uma dada condição. Assim como o modelador é livre para gerar diferentes cenários simplesmente por definir valores diferentes para $isPresent(X)$. Por exemplo, supondo que uma equipe está desenvolvendo um jogo sério para avaliar profissionais de uma certa industria. Para avaliar a competência dos trabalhadores, o modelador poderá usar este predicado por adicionar ou remover entidades e condições com base nas necessidades de avaliação.

O terceiro motivo reside no fato de que o predicado $possibilityHappensBadEvent(X)$ denota apenas que existe uma possibilidade de ocorrer algum determinado evento ruim  $happensBadEvent(X)$. Contudo, se esse evento ocorrerá ou não, não é possível definir pois isso depende de questões estatísticas do objeto de estudo. Assim sendo, o usuário deste modelo possui algumas possibilidades de ação, tais como: quando $possibilityHappensBadEvent(X)$  for verdade, então definir $happensBadEvent(X)$ por meio de um número aleatório, para dadas situações onde ocorre $possibilityHappensBadEvent(X)$ tratar $happensBadEvent(X)$ como verdade e para dadas situações tratar $happensBadEvent(X)$ como falso ou definir verdade para $happensBadEvent(X)$ como base em algum estudo probabilístico. Isso dependerá da finalidade dos modeladores. 