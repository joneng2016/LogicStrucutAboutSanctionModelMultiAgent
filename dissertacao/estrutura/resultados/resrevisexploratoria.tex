Sobre as observações de campo, análise dos manuais, entrevista com Engenheiro da área e conversas com diversos tipos de profissionais que atuam no segmento da manutenção em linha viva, os autores concluiram que o modelo conceitual deve conter, em sua estrutura, elementos a capacidade representar os itens da lista que se segue:
\begin{enumerate}
	\item Os trabalhadores que executam os procedimentos.
	\item As diferentes funções desses trabalhadores.
	\item As ferramentas usadas pelos profissionais.
	\item Os equipamentos que são submetidos a manutenção.
	\item As interações entre todo tipo de entidade tais como trabalhadores,ferramentas e equipamentos.
	\item As etapas das tarefas que devem ser finalizadas. 
	\item As relações entre todas as entidades e interações com as tarefas (ex. Quais são as entidades e interações que definem que uma dada atividade foi alcançada?)
	\item Averiguar a incerteza presente em certos processos e equipamentos que podem contribuir como causas para a ocorrência de um acidente.
	\item Medidas que sao tomadas pelos profissionais para lidar com essas incertezas.
	\item Verificar cenários onde o trabalhador executa uma dada atividade que não corresponde com o esperado pelas práticas seguras \label{verificarcenariosonde}. 
	\item Na ocorrência de \ref{verificarcenariosonde} verificar os riscos gerados a todos os envolvidos na situação.
	\item Avaliar a ocorrência de incertezas e possibilidades do risco de um determinado acidente ocorrer a(s) um (alguns) profissional(is). 
\end{enumerate}