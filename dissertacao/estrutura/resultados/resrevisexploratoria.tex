As observações de campo, análise dos manuais, entrevista com Engenheiro da área e conversas com diversos tipos de profissionais que atuam no segmento da manutenção em linha viva, apontam que o modelo conceitual deve conter, em sua estrutura, elementos com a capacidade de representar os itens da lista que se segue:

\begin{enumerate}
	\item Os trabalhadores que executam os procedimentos.
	\item As diferentes funções desses trabalhadores.
	\item As ferramentas usadas pelos profissionais.
	\item Os equipamentos que são submetidos a manutenção.
	\item As interações entre todo tipo de entidade tais como trabalhadores,ferramentas e equipamentos.
	\item As etapas das tarefas que devem ser finalizadas. 
	\item As relações entre todas as entidades e interações com as tarefas (ex. Quais são as entidades e interações que definem que uma dada atividade foi alcançada?)
	\item Averiguar a incerteza presente em certos processos e equipamentos que podem contribuir como causas para a ocorrência de um acidente.
	\item Medidas que são tomadas pelos profissionais para lidar com essas incertezas.
	\item Verificar cenários onde o trabalhador executa uma dada atividade que não corresponde com o esperado pelas práticas seguras \label{verificarcenariosonde}. 
	\item Na ocorrência de \ref{verificarcenariosonde} verificar os riscos gerados a todos os envolvidos na situação.
	\item Avaliar a ocorrência de incertezas e possibilidades do risco de um determinado acidente ocorrer a(s) um (alguns) profissional(is).
\end{enumerate}

O resultado da análise exploratória dos arcabouços que apresentam potencial para representar os elementos contidos na lista anterior são apresentados na lista que se segue:

\begin{enumerate}
	\item MOISE+ 
	\item Modelo de Agentes Normativos de Dastani
	\item V3S
	\item NormMAS
\end{enumerate}

Os autores concluem que as atividades podem ser expressas em termos de objetivos com pré-requisitos e efeitos. Como a preocupação do modelo é representar falhas nas atividades, os pré-requisitos que não estão presentes levam a violações que podem causar consequências no objetivo atual ou em subsequentes.