Esse capítulo tem como finalidade exibir os resultados obtidos ao longo do estudo dessa pesquisa. Os primeiros resultados são exibidos em \ref{resrevisaoexploratoria} em que á apresentado o que deve ser considerado pelo modelo conceitual. Um outro resultado é exibido em \ref{estconceitual}, que consiste na exibição do modelo conceitual. A seção \ref{introdutorycase} mostra a aplicação deste modelo para um caso simples com a finalidade de introduzir didaticamente o leitor à estrutura do modelo. A seção  \ref{studycase} apresenta o modelo resultando sendo usado para descrever um dado estudo de caso do setor de energia elétrica. A seção \ref{rac} exibe os possíveis raciocínios sobre a descrição do estudo de caso. O último resultado é apresentado em \ref{validation} que mostra uma implementação da descrição do estudo de caso pelo modelo conceitual em \textit{Prolog} com o objetivo de verificar se depuraçao dos raciocínios feitos a mão estão de acordo com o computador.