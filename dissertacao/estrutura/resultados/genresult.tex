Esse capítulo tem como finalidade exibir os resultados obtidos ao longo do estudo dessa pesquisa. Os resultados presentes em \ref{resrevisaoexploratoria} são referentes a etapa metodológica descrita ta seção \ref{revexpanalcamp}, já os resultados presentes em \ref{estconceitual} são consequências da descrição metodológica apresentada em \ref{modconceitual}. A seção \ref{introdutorycase} mostra a aplicação deste modelo para um caso simples com a finalidade de introduzir didaticamente o leitor à estrutura do modelo. A seção \ref{studycase} apresenta a aplicação do modelo em um caso de estudo e esse procedimento metodológico é descrito pela seção \ref{inferencias}. O último resultado é apresentado em \ref{validation} que mostra uma implementação da descrição do estudo de caso pelo modelo conceitual em \textit{Prolog} com o objetivo de verificar se depuraçao dos raciocínios feitos a mão estão de acordo com o computador.