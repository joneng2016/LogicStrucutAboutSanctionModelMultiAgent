Esse capítulo tem como finalidade exibir os resultados obtidos ao longo do estudo dessa pesquisa. Um desses, é exibido em \ref{estconceitual}, que consiste na exibição do modelo conceitual. Um outro, é descrito pela seção \ref{introdutorycase} que mostra a aplicação deste modelo para um caso simples. O propósito disto consiste introduzir didaticamente o leitor à estrutura do modelo. Um terceiro é feito em \ref{studycase} que apresenta o modelo resultando sendo usado para descrever um dado estudo de caso. Um quarto ainda, é feito em \ref{rac} que exibe os possíveis raciocínios sobre a descrição do estudo de caso. O último resultado é apresentado em \ref{validation} que mostra uma implementação da descrição do estudo de caso pelo modelo conceitual em \textit{Prolog} com o objetivo de validar os raciocínios.