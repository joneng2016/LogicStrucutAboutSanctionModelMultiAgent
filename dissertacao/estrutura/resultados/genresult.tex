Esse capítulo tem como por finalidade exibir os resultados obtidos ao longo do estudo dessa pesquisa. Um desses resultados é exibido em \ref{estconceitual} que consiste na exibição do modelo conceitual. Um outro resultado é descrito pela seção \ref{studycase} que apresenta o modelo resultando sendo usado para descrever um dado estudo de caso. Um terceiro resultado é feito em \ref{rac} que exibe os possíveis raciocínios sobre a descrição do estudo de caso. O último resultado é exibido em \ref{validation} que apresenta uma implementação da descrição do estudo de caso pelo modelo conceitual em \textit{Prolog} com o objetivo de validar os raciocínios.