Sete profissionais de linha viva (profissionais que realizam manutenção em equipamentos elétricos energizados) são designados com o propósito de realizar 
a substituição de um isolador de pedestal. Os papeis desses desses profissionais são; 1 supervisor, 5 executores. A manutenção deve ser executada apenas 
sobre as seguintes condições: céu ensolarado e umidade relativa do ar menor que 70 porcento. Todos os profissionais devem possuir os EPI's necessários: 
capacete, óculos de sol, roupa isolante e antichamas, luvas isolantes e botas isolantes. Os profissionais que entram no potencial devem estar vestidos de 
roupa condutiva e cabo guarda. As ferramentas necessárias para resolver esse problema são: bastão garra de diâmetro 64 x 3600 mm, sela de diâmetro 65 , 
colar, corda de fibra sintética, carretilha, chave com catraca, bastão universal, soquete adequado, locador de pino e bastão com soquete multiangular. O método selecionado 
para esse tipo de manutenção é a distância onde o eletricista não acessa diretamente o potencial, mas faz isso por intermédio de um bastão isolante. A substituição do isolador 
de pedestal pode ser escrita nos seguintes objetivos: 

\begin{enumerate}
	\item Limpar, secar e testar corda.
	\item Instalar Bastão Garra na estrutura com o pedestal a ser substituído.
	\item Instalar sela com colar na estrutura
	\item Amarrar o bastão na parte superior da estrutura com a corda.
	\item Amarrar o olhal do bastão ao cavalo da sela atrás de uma corda.
	\item Instalar um segundo conjunto bastão e sela no lado oposto da estrutura.
	\item Enforcar um estropo de Náilon no corpo do isolador.
	\item Colocar a extremidade do estropo no gancho da corda de serviço.
	\item Afrouxar os parafusos do conector que prendem a barra ao isolador.
	\item Terminar de retirar os parafusos com o bastão com o soquete multiangular.
	\item Elevar a barra através da corda que une a sela ao bastão.
	\item Apertar o colar através da porca borboleta.
	\item Segurar firmemente a corda de serviço.
	\item Sacar parafusos da base da coluna.
	\item Baixar o isolador ao solo
	\item Içar o Isolador
	\item Colocar Parafusos na base da coluna.
	\item Baixar a barra para que a mesma apoie no novo isolador.
	\item Colocar os parafusos do conector que prende a barra ao novo isolador. 
	\item Retirar Equipamentos
\end{enumerate}

A tabela \ref{agents} apresenta todos os agentes que fazem parte da manutenção. 
\begin{table}[H]
\scalefont{0.8}
\centering
\begin{tabular}{|l|l|}
\hline
\textbf{símbolo} & \textbf{significado} \\ \hline
agente1 & Um dos agentes participantes da manutenção \\ \hline
agente2 & Um dos agentes participantes da manutenção \\ \hline
agente3 & Um dos agentes participantes da manutenção \\ \hline
agente4 & Um dos agentes participantes da manutenção \\ \hline
agente5 & Um dos agentes participantes da manutenção \\ \hline
agente6 & Um dos agentes participantes da manutenção \\ \hline
agente7 & Um dos agentes participantes da manutenção \\ \hline
\end{tabular}
\caption{Os agentes que constituem uma manutenção}
\label{agents}
\end{table}

 A tabela \ref{roles} apresenta todas as funções que deverão ser exercidas pelos agentes.

\begin{table}[H]
\scalefont{0.8}
\centering
\begin{tabular}{|l|l|}
\hline
\textbf{papel} & \textbf{descrição} \\ \hline
supervisor & Atribui papel a outros profissionais \\ \hline
executor1 & Tem como por finalidade executar certas atividades manuais vinculadas a manutenção \\ \hline
executor2 & Tem como por finalidade executar certas atividades manuais vinculadas a manutenção \\ \hline
executor3 & Tem como por finalidade executar certas atividades manuais vinculadas a manutenção \\ \hline
executor4 & Tem como por finalidade executar certas atividades manuais vinculadas a manutenção \\ \hline
executor5 & Tem como por finalidade executar certas atividades manuais vinculadas a manutenção \\ \hline
\end{tabular}
\caption{Os papeis relevantes para a ocorrência da manutenção}
\label{roles}
\end{table}

A tabela \ref{agentsroles} define a relação $hasRole(ag_n,\rho_m)$ onde $ag_n$ é representado pela coluna agente e $\rho_m$ é representado pela coluna papel.

\begin{table}[H]
\scalefont{0.8}
\centering
\begin{tabular}{|l|l|}
\hline
\textbf{agente} & \textbf{papel} \\ \hline
agente1 & supervisor \\ \hline
agente2 & executor1 \\ \hline
agente3 & executor1 \\ \hline
agente4 & executor2 \\ \hline
agente5 & executor3 \\ \hline
agente6 & executor4 \\ \hline
agente7 & executor5 \\ \hline
\end{tabular}
\caption{Relação $hasRole(ag_n,\rho_m)$}
\label{agentsroles}
\end{table}

A tabela \ref{artefacts1} e \ref{artefacts2} apresentam todos artefatos que fazem parte da descrição deste estudo de caso.

\begin{table}[H]
\scalefont{0.8}
\centering
\begin{tabular}{|l|p{0.8\linewidth}|}
\hline
\textbf{artefato} & \textbf{descrição} \\ \hline
capacete & EPI usado pelo profissional para proteger a cabeça \\ \hline
óculos & Óculos usado para evitar dificuldades de enxergar presentes em dias claros \\ \hline
roupagem & Consiste em roupas isolantes e anti-chamas \\ \hline
luva & Luvas Isolantes \\ \hline
bota & Botas Isolantes para evitar que o profissional seja eletrocutado \\ \hline
bastaoGarra & bastão isolante que possui uma ferramenta em estrutura de garra. 64 X 3600 mm \\ \hline
sela & Possui diâmetro 65 mm, é fixada na torre para sustentar o bastão. \\ \hline
colar & Estrutura que fica fixa na sela, bastão isolante é travado no colar. \\ \hline
corda & Corda Isolante. \\ \hline
carretilha & Carretilha que, em conjunto com a corda, é usada para mover material na vertical. \\ \hline
bastaoUniversal & Bastão isolante que permite o acoplamento de múltiplas ferramentas. \\ \hline
\end{tabular}
\caption{Definindo todos os artefatos presentes na manutenção}
\label{artefacts1}
\end{table} 

\begin{table}[H]
\scalefont{0.8}
\centering
\begin{tabular}{|l|p{0.8\linewidth}|}
\hline
\textbf{artefato} & \textbf{descrição} \\ \hline
soquete & Usado na manipulação de parafusos. \\ \hline
locador & Usado como pino direcional em alinhamento de furo de parafusos, auxiliado na inserção de pinos e parafusos. \\ \hline
bastaoGarra & Bastão Universal que possui uma garra. \\ \hline
isoladorVelho & Isolador de pedestal danificado a ser substituído \\ \hline
isoladorNovo & Isolador de pedestal novo que será posicionado no local do isolador velho. \\ \hline
torre & Estrutura metálica onde fica fixo o isolador \\ \hline
condutor & Em formato de cabo, fica fixo sobre o topo do isolador.e é por onde passa grandes quantidades de energia elétrica. \\ \hline
estropo & pano firme usado para segurar Isolador quando estiver suspenso \\ \hline
pano & pano usado para limpar ferramentas \\ \hline
glicerina & substância usada para limpar as ferramentas adequadamente \\ \hline
condutímetro & Medidor de corrente de fuga sobre o bastão universal. \\ \hline
parafuso & Parafusos prendem o conector condutor-Isolador e também prendem o Isolador a base \\ \hline
conector & Estrutura que tem como por finalidade manter condutor,cabeçote do isolador em conjunto. \\ \hline
\end{tabular}
\caption{Definindo todos os artefatos presentes na manutenção}
\label{artefacts2}
\end{table} 

A tabela \ref{g} apresenta os objetivos dados pela coluna $objetivo$ bem com sua descrição. Essa tabela também apresenta os conjuntos $gp_i$ dado pela coluna pré-requisitos. Assim sendo, essa tabela também apresenta a relação entre os objetivos e seus respectivos pré-requisitos, ou seja, a relação $isPresentRequisite(gp_i,g_j)$. 

\begin{table}[H]
\scalefont{0.8}
\centering
\begin{tabular}{|l|l|p{0.6\linewidth}|}
\hline
\textbf{objetivo} & \textbf{pré-requisito} & \textbf{Descrição} \\ \hline
gSupervisor & g0 & Atribui objetivos aos demais agentes. \\ \hline
g0 &  \O & Vestir os AP'Is \\ \hline
g1 &  gSupervisor & Limpar, secar e testar ferramentas com material isolante. \\ \hline
g2 &  g1 & Medir a corrente de fuga de ferramentas isolantes \\ \hline
g3 &  g2 & Instalar sela com colar na estrutura \\ \hline
g4 &  g3 & Passar o bastão garra por dentro do olhal do colar. \\ \hline
g5 &  g4 & Amarrar o bastão garra na parte superior da estrutura com a corda, fixar no condutor \\ \hline
g6 &  gSupervisor & Amarrar o olhal do bastão garra ao cavalo da sela atrás de uma corda. \\ \hline
g7 &  g6 & Instalar sela com colar no outro lado da estrutura estrutura \\ \hline
g8 &  g7 & Passar o bastão universal por dentro do olhal do colar \\ \hline
g9 &  g8 & Pender carretilha no bastão Universal. \\ \hline
g10 &  g9 & Amarrar o bastão universal na parte superior da estrutura com a corda; \\ \hline
g11 &  g10 & Amarrar o olhal do bastão universal ao cavalo da sela atrás de uma corda. \\ \hline
g12 &  g11,g5 & Rotacionar estrutura olhal garra em 45 graus. \\ \hline
g13 &  g12 & Enforcar um estropo de Náilon no corpo do isolador velho. \\ \hline
g14 &  g13 & Colocar a extremidade do estropo no gancho da corda de serviço. \\ \hline
g15 &  g14 & Afrouxar os parafusos do conector que prendem a barra ao isolador. \\ \hline
g16 &  g15 & Terminar de retirar os parafusos com o bastão com o soquete multiangular. \\ \hline
g17 &  g16 & Elevar o condutor através da corda que une a sela ao bastão. \\ \hline
g18 &  g17 & Apertar o colar através da porca borboleta. \\ \hline
g19 &  g18 & Sacar parafusos da base da coluna. \\ \hline
g20 &  g19 & Segurar firmemente a corda de serviço,baixar o isolador ao solo \\ \hline
g21 &  g20 & Passar Estropo no Isolador Novo \\ \hline
g22 &  g21 & Colocar a extremidade do estropo no gancho da corda de serviço. \\ \hline
g23 &  g22 & Içar o Isolador \\ \hline
g24 &  g23 & Colocar Parafusos na base da coluna. \\ \hline
g25 &  g24 & Baixar o condutor para que a mesma se sustente no novo isolador. \\ \hline
g26 &  g25 & Colocar os parafusos do conector que prende a barra ao novo isolador. \\ \hline
g27 &  g26 & Retirar Equipamentos \\ \hline
\end{tabular}
\caption{Define e descreve os objetivos bem como os respectivos pré-requisitos}
\label{g}
\end{table}

A tabela \ref{condition} apresenta $c_k$ dado pela coluna condição e pela coluna descrição. Essa tabela define a relação $hasRisk(c_k,risk_j,f_m)$ onde $risk_j$ é descrito pela coluna risco e $f_m$ é descrito pela coluna fatalidade. 

\begin{table}[H]
\scalefont{0.8}
\centering
\begin{tabular}{|l|p{0.6\linewidth}|l|l|}
\hline
\textbf{condição} & \textbf{descrição} & \textbf{risco} & \textbf{fatalidade} \\ \hline
umidade70 & Umidade Relativa do Ar deve ser inferior a setenta porcento. & eletrocutado & morte \\ \hline
noVento & Não deve haver vento durante os procedimentos de manutenção. & eletrocutado & morte \\ \hline
noChuva & Não deve haver chuva durante o ato da manutenção & eletrocutado & morte \\ \hline
sol & O dia deve estar ensolarado & eletrocutado & morte \\ \hline
\end{tabular}
\caption{Define as condições necessárias para que a manutenção tenha possibilidade de acontecer}
\label{condition}
\end{table}


A tabela \ref{relation} apresenta três relações onde uma delas é $thereIsRelation(r_l,e_i,e_k)$ onde $r_l$ é definido pela coluna $relacionamento$, $e_i$ e $e_k$ pelas entidades envolvidas. A outra relação é dada por $hasRisk(r_k,risk_j,f_m)$ onde $risk_j$ é dado pela coluna risco e $f_m$ é dado pela coluna fatalidade. A terceira relação é dada por $hasPossibility(gr_n,p_m)$. $X$ é uma variável que pode assumir os seguintes valores $agente1, agente2, agente3,agente4, agente5, agente6$ e $agente7$. Por exemplo, a primeira linha da tabela \ref{relation} é; 


\begin{eqnarray}
	relXCapacete | X,capacete | nenhum | nenhum | false
\end{eqnarray}


Substituindo o $X$ pelos valores, é possível obter todas essas relações; 


\begin{eqnarray}
relAgente1Capacete | Agente1 ,capacete | nenhum | nenhum | false \nonumber \\
relAgente2Capacete | Agente2 ,capacete | nenhum | nenhum | false \nonumber \\ 
relAgente3Capacete | Agente3 ,capacete | nenhum | nenhum | false \nonumber \\ 
relAgente4Capacete | Agente4 ,capacete | nenhum | nenhum | false \nonumber \\
relAgente5Capacete | Agente5 ,capacete | nenhum | nenhum | false \nonumber \\
relAgente6Capacete | Agente6 ,capacete | nenhum | nenhum | false \nonumber \\
relAgente7Capacete | Agente7 ,capacete | nenhum | nenhum | false\nonumber \\
\nonumber \\
\end{eqnarray}

\begin{table}[H]
\scalefont{0.8}
\centering
\begin{tabular}{|l|l|l|l|l|}
\hline
\textbf{relacionamento} & \textbf{entidades envolvidas} & \textbf{risco} & \textbf{fatalidade} & \textbf{possibilidade} \\  \hline
relXCapacete & X,capacete & nenhum & nenhum & false \\ \hline
relXOculos & X,oculos & nenhum & nenhum & false \\ \hline
relXRoupagem & X,roupagem & nenhum & nenhum & false \\ \hline
relXLuva & X,luva & nenhum & nenhum & false \\ \hline
relXBotas & X,bota & nenhum & nenhum & false \\ \hline
relXPano & X,pano & nenhum & nenhum & false \\ \hline
relPanoGlicerina & pano,glicerina & nenhum & nenhum & false \\ \hline
relPanoCorda & pano,corda & nenhum & nenhum & false \\ \hline
relPanoBastoaUniversal & pano,bastaoUniversal & nenhum & nenhum & false \\ \hline
relPanoSoquete & pano,soquete & nenhum & nenhum & false \\ \hline
relPanoBastaoUniversal & pano,bastaoGarra & nenhum & nenhum & false \\ \hline
relXSela & X,sela & nenhum & nenhum & false \\ \hline
relXColar & X,colar & nenhum & nenhum & false \\ \hline
relXBastaoGarra & X,bastaoGarra & nenhum & nenhum & false \\ \hline
relTorreSela & torre,sela & nenhum & nenhum & false \\ \hline
relSelaColar & sela,colar & nenhum & nenhum & false \\ \hline
relColarBastaoGarra & colar,bastaoGarra & nenhum & nenhum & false \\ \hline
relBastaoGarraCondutor & bastaoGarra,condutor & eletrocutado & morte & false \\ \hline
relXBastaoUniversal & X,bastaoUniversal & nenhum & nenhum & false \\ \hline
relCordaBastaoUniversal & corda,bastaoUniversal & nenhum & nenhum & false \\ \hline
relCordaCarretilha & corda,carretilha & nenhum & nenhum & false \\ \hline
relBastaoUniversalCarretilha & bastaoUniversal,carretilha & nenhum & nenhum & false \\ \hline
relBastaoUniversalColar & bastaoUniversal,colar & nenhum & nenhum & false \\ \hline
relBastaoUniversalEstopo & bastaoUniversal,estopo & nenhum & nenhum & false \\ \hline
\end{tabular}
\caption{Define os relacionamentos necessários que a manutenção aconteça}
\label{relation2}
\end{table}


\begin{table}[H]
\scalefont{0.8}
\centering
\begin{tabular}{|l|l|l|l|l|}
\hline
\textbf{relacionamento} & \textbf{entidades envolvidas} & \textbf{risco} & \textbf{fatalidade} & \textbf{possibilidade} \\  \hline
relCordaEstropo & corda,estropo & eletrocutado & morte & false \\ \hline
relEstropoIsoladorVelho & estropo,isoladorVelho & nenhum & nenhum & false \\ \hline
relXChaveCatraca & X,chaveCatraca & nenhum & nenhum & false \\ \hline
relChaveCatracaBastaoUniversal & chaveCatraca,bastaoUniversal & nenhum & nenhum & false \\ \hline
relChaveCatracaParafuso & chaveCatraca,parafuso & eletrocutado & morte & false \\ \hline
relParafusoConector & parafuso,conector & eletrocutado & morte & false \\ \hline
relXBastaoSoquete & X,bastaoSoquete & nenhum & nenhum & false \\ \hline
relSoqueteParafuso & soquete,parafuso & eletrocutado & morte & false \\ \hline
relXCorda & X,corda & eletrocutado & morte & false \\ \hline
relXIsoladorVelho & X,isoladorVelho & nenhum & nenhum & false \\ \hline
relXIsoladorNovo & X,isoladorNovo & nenhum & nenhum & false \\ \hline
relCordaBastaoGarra & corda,bastaoGarra & nenhum & nenhum & false \\ \hline
relBastaoGarraSela & bastaoGarra, sela & nenhum & nenhum & false \\ \hline
relXCarretilha & X,carretilha & nenhum & nenhum & false \\ \hline
relBastaoUniversalCorda & bastaoUniversal,corda & nenhum & nenhum & false \\ \hline
relBastaoUniversalTorre & bastaoUniversal,torre & nenhum & nenhum & false \\ \hline
relEstropoCorda & estropo,corda & eletrocutado & morte & false \\ \hline
relEstropoIsoladorNovo & estropo,isoladorNovo & nenhum & nenhum & false \\ \hline
relBastaoUniversalSela & universal,sela & nenhum & nenhum & false \\ \hline
relBastaoGarraTorre & bastaoGarra,torre & nenhum & nenhum & false \\ \hline
relBastaoUniversalEstropo & bastaoUniversal,estropo & nenhum & nenhum & false \\ \hline
relXColar & X,colar & nenhum & nenhum & false \\ \hline
relParafusoTorre & parafuso,torre & eletrocutado & morte & false \\ \hline
relCondutivimetroCorda & condutímetro,corda & nenhum & nenhum & false \\ \hline
relCondutivimetroBastaoUniversal & condutímetro,bastaoUniversal & nenhum & nenhum & false \\ \hline
relCondutivimetroBastaoGarra & condutímetro,bastaoGarra & nenhum & nenhum & false \\ \hline
relCondutivimetroSoquete & condutímetro,soquete & nenhum & nenhum & false \\ \hline
\end{tabular}
\caption{Define os relacionamentos necessários que a manutenção aconteça}
\label{relation}
\end{table}

As tabelas \ref{relation1},\ref{relation2},\ref{relation3} e \ref{relation4} apresentam a relação $affects(r_k,r_n)$ onde $r_k$ é representado pela coluna relacionamento-errado e $r_n$ é representado pela coluna relacionamento-afetado. 
A coluna nova possibilidade de algo errado tem como por finalidade representar que a possibilidade de ocorrer algum evento ruim atrelado ao relacionamento-afetado mudou de $false$ para $true$.

\begin{table}[H]
\centering
\scalefont{0.6}
\begin{tabular}{|l|l|l|}
\hline
\textbf{relacionamento-errado} & \textbf{relacionamento-afetado} & \textbf{nova possibilidade de algo errado} \\ \hline
relXCapacete & relBastaoGarraCondutor & true \\ \hline
relXCapacete & relCordaEstropo & true \\ \hline
relXCapacete & relChaveCatracaParafuso & true \\ \hline
relXCapacete & relParafusoConector & true \\ \hline
relXCapacete & relSoqueteParafuso & true \\ \hline
relXCapacete & relXCorda & true \\ \hline
relXCapacete & relEstropoCorda & true \\ \hline
relXCapacete & relParafusoTorre & true \\ \hline
relXOculos & relBastaoGarraCondutor & true \\ \hline
relXOculos & relCordaEstropo & true \\ \hline
relXOculos & relChaveCatracaParafuso & true \\ \hline
relXOculos & relParafusoConector & true \\ \hline
relXOculos & relSoqueteParafuso & true \\ \hline
relXOculos & relXCorda & true \\ \hline
relXOculos & relEstropoCorda & true \\ \hline
relXOculos & relParafusoTorre & true \\ \hline
relXLuva & relBastaoGarraCondutor & true \\ \hline
relXLuva & relCordaEstropo & true \\ \hline
relXLuva & relChaveCatracaParafuso & true \\ \hline
relXLuva & relParafusoConector & true \\ \hline
\end{tabular}
\caption{Define o impacto que o erro em um relacionamento gera em outro relacionamento}
\label{relation1}
\end{table}

\begin{table}[H]
\centering
\scalefont{0.6}
\begin{tabular}{|l|l|l|}
\hline
\textbf{relacionamento-errado} & \textbf{relacionamento-afetado} & \textbf{nova possibilidade de algo errado} \\ \hline
relXLuva & relSoqueteParafuso & true \\ \hline
relXLuva & relXCorda & true \\ \hline
relXLuva & relEstropoCorda & true \\ \hline
relXLuva & relParafusoTorre & true \\ \hline
relXBotas & relBastaoGarraCondutor & true \\ \hline
relXBotas & relCordaEstropo & true \\ \hline
relXBotas & relChaveCatracaParafuso & true \\ \hline
relXBotas & relParafusoConector & true \\ \hline
relXBotas & relSoqueteParafuso & true \\ \hline
relXBotas & relXCorda & true \\ \hline
relXBotas & relEstropoCorda & true \\ \hline
relXBotas & relParafusoTorre & true \\ \hline
relXPano & relBastaoGarraCondutor & true \\ \hline
relXPano & relCordaEstropo & true \\ \hline
relXPano & relChaveCatracaParafuso & true \\ \hline
relXPano & relParafusoConector & true \\ \hline
relXPano & relSoqueteParafuso & true \\ \hline
relXPano & relXCorda & true \\ \hline
relXPano & relEstropoCorda & true \\ \hline
relXPano & relParafusoTorre & true \\ \hline
relPanoGlicerina & relBastaoGarraCondutor & true \\ \hline
relPanoGlicerina & relCordaEstropo & true \\ \hline
relPanoGlicerina & relChaveCatracaParafuso & true \\ \hline
relPanoGlicerina & relParafusoConector & true \\ \hline
relPanoGlicerina & relSoqueteParafuso & true \\ \hline
relPanoGlicerina & relXCorda & true \\ \hline
relPanoGlicerina & relEstropoCorda & true \\ \hline
relPanoGlicerina & relParafusoTorre & true \\ \hline
relPanoCorda & relCordaEstropo & true \\ \hline
relPanoCorda & relXCorda & true \\ \hline
relPanoCorda & relEstropoCorda & true \\ \hline
\end{tabular}
\caption{Define o impacto que o erro em um relacionamento gera em outro relacionamento}
\label{relation2}
\end{table}

\begin{table}[H]
\centering
\scalefont{0.6}
\begin{tabular}{|l|l|l|}
\hline
\textbf{relacionamento-errado} & \textbf{relacionamento-afetado} & \textbf{nova possibilidade de algo errado} \\ \hline
relPanoBastaoUniversal & relBastaoGarraCondutor & true \\ \hline
relPanoBastaoUniversal & relChaveCatracaParafuso & true \\ \hline
relPanoBastaoUniversal & relParafusoConector & true \\ \hline
relPanoBastaoUniversal & relParafusoTorre & true \\ \hline
relPanoBastaoUniversal & relBastaoGarraCondutor & true \\ \hline
relPanoSoquete & relBastaoGarraCondutor & true \\ \hline
relPanoSoquete & relCordaEstropo & true \\ \hline
relPanoSoquete & relChaveCatracaParafuso & true \\ \hline
relPanoSoquete & relParafusoConector & true \\ \hline
relPanoSoquete & relSoqueteParafuso & true \\ \hline
relPanoSoquete & relXCorda & true \\ \hline
relPanoSoquete & relEstropoCorda & true \\ \hline
relPanoSoquete & relParafusoTorre & true \\ \hline
relCondutivimetroCorda & relBastaoGarraCondutor & true \\ \hline
relCondutivimetroCorda & relCordaEstropo & true \\ \hline
relCondutivimetroCorda & relChaveCatracaParafuso & true \\ \hline
relCondutivimetroCorda & relParafusoConector & true \\ \hline
relCondutivimetroCorda & relSoqueteParafuso & true \\ \hline
relCondutivimetroCorda & relXCorda & true \\ \hline
relCondutivimetroCorda & relEstropoCorda & true \\ \hline
relCondutivimetroCorda & relParafusoTorre & true \\ \hline
\end{tabular}
\caption{Define o impacto que o erro em um relacionamento gera em outro relacionamento por mudar a possibilidade de algo errado acontecer.}
\label{relation3}
\end{table}



\begin{table}[H]
\centering
\scalefont{0.6}
\begin{tabular}{|l|l|l|}
\hline
\textbf{relacionamento-errado} & \textbf{relacionamento-afetado} & \textbf{nova possibilidade de algo errado} \\ \hline
relCondutivimetroBastaoUniversal & relBastaoGarraCondutor & true \\ \hline
relCondutivimetroBastaoUniversal & relCordaEstropo & true \\ \hline
relCondutivimetroBastaoUniversal & relChaveCatracaParafuso & true \\ \hline
relCondutivimetroBastaoUniversal & relParafusoConector & true \\ \hline
relCondutivimetroBastaoUniversal & relSoqueteParafuso & true \\ \hline
relCondutivimetroBastaoUniversal & relXCorda & true \\ \hline
relCondutivimetroBastaoUniversal & relEstropoCorda & true \\ \hline
relCondutivimetroBastaoUniversal & relParafusoTorre & true \\ \hline
relCondutivimetroBastaoGarra & relBastaoGarraCondutor & true \\ \hline
relCondutivimetroBastaoGarra & relCordaEstropo & true \\ \hline
relCondutivimetroBastaoGarra & relChaveCatracaParafuso & true \\ \hline
relCondutivimetroBastaoGarra & relParafusoConector & true \\ \hline
relCondutivimetroBastaoGarra & relSoqueteParafuso & true \\ \hline
relCondutivimetroBastaoGarra & relXCorda & true \\ \hline
relCondutivimetroBastaoGarra & relEstropoCorda & true \\ \hline
relCondutivimetroBastaoGarra & relParafusoTorre & true \\ \hline
relCondutivimetroSoquete & relBastaoGarraCondutor & true \\ \hline
relCondutivimetroSoquete & relCordaEstropo & true \\ \hline
relCondutivimetroSoquete & relChaveCatracaParafuso & true \\ \hline
relCondutivimetroSoquete & relParafusoConector & true \\ \hline
relCondutivimetroSoquete & relSoqueteParafuso & true \\ \hline
relCondutivimetroSoquete & relXCorda & true \\ \hline
relCondutivimetroSoquete & relEstropoCorda & true \\ \hline
relCondutivimetroSoquete & relParafusoTorre & true \\ \hline
\end{tabular}
\caption{Define o impacto que o erro em um relacionamento gera em outro relacionamento por mudar a possibilidade de algo errado acontecer.}
\label{relation4}
\end{table}

As tabelas \ref{deontic1}, \ref{deontic2}, \ref{deontic3} e \ref{deontic4},  apresentam a relação $hasObligation(\rho_m,g_i)$ onde $\rho_m$ é representado pela coluna papel e $g_i$ é representado pela coluna objetivo. 

\begin{table}[H]
\centering
\scalefont{0.8}
\begin{tabular}{|l|l|}
\hline
\textbf{papel} & \textbf{objetivo} \\ \hline
executor1 & g0 \\ \hline
executor2 & g0 \\ \hline
executor3 & g0 \\ \hline
executor4 & g0 \\ \hline
executor5 & g0 \\ \hline
supervisor & g0 \\ \hline
supervisor & gSupervisor \\ \hline
executor1 & g1 \\ \hline
executor2 & g1 \\ \hline
executor1 & g2 \\ \hline
executor2 & g2 \\ \hline
executor1 & g3 \\ \hline
executor2 & g2 \\ \hline
executor1 & g4 \\ \hline
executor2 & g4 \\ \hline
executor1 & g5 \\ \hline
executor2 & g5 \\ \hline
executor3 & g6 \\ \hline
executor4 & g6 \\ \hline
executor5 & g6 \\ \hline
\end{tabular}
\caption{Objetivos que devem ser atingidos pelo agente que assumir um dada função}
\label{deontic1}
\end{table}

\begin{table}[H]
\centering
\scalefont{0.8}
\begin{tabular}{|l|l|}
\hline
\textbf{papel} & \textbf{objetivo} \\ \hline
executor3 & g7 \\ \hline
executor4 & g7 \\ \hline
executor5 & g7 \\ \hline
executor3 & g8 \\ \hline
executor4 & g8 \\ \hline
executor5 & g8 \\ \hline
executor3 & g9 \\ \hline
executor4 & g9 \\ \hline
executor5 & g9 \\ \hline
executor3 & g10 \\ \hline
executor4 & g10 \\ \hline
executor5 & g10 \\ \hline
executor3 & g11 \\ \hline
executor4 & g11 \\ \hline
executor5 & g11 \\ \hline
executor1 & g12 \\ \hline
executor2 & g12 \\ \hline
executor3 & g12 \\ \hline
executor4 & g12 \\ \hline
executor1 & g13 \\ \hline
executor2 & g13 \\ \hline
executor3 & g13 \\ \hline
executor4 & g13 \\ \hline
executor1 & g14 \\ \h\usepackage{scalefnt}
line
\end{tabular}
\caption{Objetivos que devem ser atingidos pelo agente que assumir um dada função}
\label{deontic2}
\end{table}


\begin{table}[H]
\centering
\scalefont{0.8}
\begin{tabular}{|l|l|}
\hline
\textbf{role} & \textbf{g} \\ \hline
executor2 & g14 \\ \hline
executor3 & g14 \\ \hline
executor4 & g14 \\ \hline
executor2 & g15 \\ \hline
executor3 & g15 \\ \hline
executor4 & g15 \\ \hline
executor5 & g15 \\ \hline
executor2 & g16 \\ \hline
executor3 & g16 \\ \hline
executor4 & g16 \\ \hline
executor5 & g16 \\ \hline
executor1 & g17 \\ \hline
executor3 & g17 \\ \hline
executor4 & g17 \\ \hline
executor5 & g17 \\ \hline
executor1 & g18 \\ \hline
executor3 & g18 \\ \hline
executor4 & g18 \\ \hline
executor5 & g18 \\ \hline
executor1 & g19 \\ \hline
executor3 & g19 \\ \hline
executor4 & g19 \\ \hline
executor5 & g19 \\ \hline
executor1 & g20 \\ \hline
executor3 & g20 \\ \hline
executor4 & g20 \\ \hline
executor5 & g20 \\ \hline
executor1 & g21 \\ \hline
executor3 & g21 \\ \hline
\end{tabular}
\caption{Objetivos que devem ser atingidos pelo agente que assumir um dada função}
\label{deontic3}
\end{table}

\begin{table}[H]
\centering
\scalefont{0.8}
\begin{tabular}{|l|l|}
\hline
\textbf{role} & \textbf{g} \\ \hline
executor4 & g21 \\ \hline
executor5 & g21 \\ \hline
executor1 & g22 \\ \hline
executor2 & g22 \\ \hline
executor3 & g22 \\ \hline
executor5 & g22 \\ \hline
executor1 & g23 \\ \hline
executor2 & g23 \\ \hline
executor3 & g23 \\ \hline
executor5 & g23 \\ \hline
executor1 & g24 \\ \hline
executor2 & g24 \\ \hline
executor3 & g24 \\ \hline
executor5 & g24 \\ \hline
executor1 & g25 \\ \hline
executor2 & g25 \\ \hline
executor3 & g25 \\ \hline
executor4 & g25 \\ \hline
executor1 & g26 \\ \hline
executor2 & g26 \\ \hline
executor3 & g26 \\ \hline
executor4 & g26 \\ \hline
executor1 & g27 \\ \hline
executor2 & g27 \\ \hline
executor3 & g27 \\ \hline
executor4 & g27 \\ \hline
executor5 & g27 \\ \hline
\end{tabular}
\caption{Objetivos que devem ser atingidos pelo agente que assumir um dada função}
\label{deontic4}
\end{table}

A tabela \ref{entities} apresenta as entidades que constituem os conjuntos \textbf{eg}.

\begin{table}[H]
\centering
\scalefont{0.8}
\begin{tabular}{|l|l|}
\hline
\textbf{entidades}                                                                                                    & \textbf{eg} \\ \hline
capacete,óculos,roupagem,luvas,botas X = \{agente que tenta alcançar o objetivo\}                                        & eg0         \\ \hline
pano,glicerina,carretilha,bastaoUniversal,corda,bastaoGarra,X = \{agente que tenta alcançar o objetivo\}               & eg1         \\ \hline
pano,glicerina,carretilha,bastaoUniversal,corda,bastaoGarra,condutímetro,X = \{agente que tenta alcançar o objetivo\} & eg2         \\ \hline
sela,colarX = \{agente que tenta alcançar o objetivo\}                                                                & eg3         \\ \hline
colar,bastaoGarraX = \{agente que tenta alcançar o objetivo\}                                                          & eg4         \\ \hline
corda,bastaoGarra,bastaoGarraTorre,condutorX = \{agente que tenta alcançar o objetivo\}                               & eg5         \\ \hline
bastaoGarra,selaX = \{agente que tenta alcançar o objetivo\}                                                           & eg6         \\ \hline
sela,colarX = \{agente que tenta alcançar o objetivo\}                                                                & eg7         \\ \hline
sela,bastaoUniversal,Colar,X = \{agente que tenta alcançar o objetivo\}                                               & eg8         \\ \hline
bastaoUniversal,carretilha,X = \{agente que tenta alcançar o objetivo\}                                               & eg9         \\ \hline
corda,bastaoUniversal,corda,torre,X = \{agente que tenta alcançar o objetivo\}                                        & eg10        \\ \hline
bastaoUniversal,corda,colar,selaX = \{agente que tenta alcançar o objetivo\}                                          & eg11        \\ \hline
colar,X = \{agente que tenta alcançar o objetivo\}                                                                    & eg12        \\ \hline
bastaoUniversal,estropo,isoladorVelhoX = \{agente que tenta alcançar o objetivo\}                                     & eg13        \\ \hline
bastaoUniversal,corda,estropoX = \{agente que tenta alcançar o objetivo\}                                             & eg14        \\ \hline
chaveCatraca,bastaoUniversal,prafusoX = \{agente que tenta alcançar o objetivo\}                                      & eg15        \\ \hline
bastaoSoquete,parafuso,X = \{agente que tenta alcançar o objetivo\}                                                   & eg16        \\ \hline
bastaoGarra,condutorcordaX = \{agente que tenta alcançar o objetivo\},                                                & eg17        \\ \hline
colar,X = \{agente que tenta alcançar o objetivo\},                                                                   & eg18        \\ \hline
chaveCatraca,bastaoUniversal,prafusobastaoSoquete,parafuso,torreX = \{agente que tenta alcançar o objetivo\}          & eg19        \\ \hline
cordaX = \{agente que tenta alcançar o objetivo\}                                                                     & eg20        \\ \hline
estropo, isoladorNovo,X = \{agente que tenta alcançar o objetivo\}                                                    & eg21        \\ \hline
bastaoUniversal,corda,estropoX = \{agente que tenta alcançar o objetivo\}                                             & eg22        \\ \hline
cordaX = \{agente que tenta alcançar o objetivo\}                                                                     & eg23        \\ \hline
chaveCatraca,bastaoUniversal,prafusobastaoSoquete,parafuso,torreX = \{agente que tenta alcançar o objetivo\}          & eg24        \\ \hline
bastaoGarra,condutorcordaX = \{agente que tenta alcançar o objetivo\},                                                & eg25        \\ \hline
chaveCatraca,bastaoUniversal,prafusoX = \{agente que tenta alcançar o objetivo\}                                      & eg26        \\ \hline
sela,colar,bastaoGarra,bastaoUniversal,bastaoSoquete,corda,carretilha,chaveCatraca,torre,condutor                     & eg27        \\ \hline
\end{tabular}
\caption{Entidades que formam os conjuntos $eg_n$. Cada conjunto destes estão relacionados com um objetivo e determinam as entidades necessárias para que o mesmo tenha codição de ser alcançado.}
\label{entities}
\end{table}


As tabelas \ref{relationsgroup1},\ref{relationsgroup2} apresentam as relações que constituem os conjuntos \textbf{rg}.
\begin{table}[H]
\centering
\scalefont{0.7}
\begin{tabular}{|p{0.8\linewidth}|l|}
\hline
\textbf{relacionamentos}                                                                                                                                                                                                                                                                                                                  & \textbf{rg} \\ \hline
relXcapacete relXoculos relXroupagem relXluva relXbotas                                                                                                                                                                                                                                                                                   & rg0         \\ \hline
relXPano relPanoGlicerina relPanoCorda relPanoBastaoUniversal relPanoBastaoGarra relPanoSoquete                                                                                                                                                                                                                                           & rg1         \\ \hline
relCondutivimetroCorda relCondutivimetroBastaoUniversal relCondutivimetroBastaoGarra relCondutivimetroSoquete                                                                                                                                                                                                                                & rg2         \\ \hline
relCondutivimetro                                                                                                                                                                                                                                                                                                                         & rg3         \\ \hline
relXBastaoGarra relColarBastaoGarra                                                                                                                                                                                                                                                                                                        & rg4         \\ \hline
relXBastaoGarra relXCordarelCordaBastaoGarra relBastaoGarraTorre relBastaoGarraCondutor                                                                                                                                                                                                                                                      & rg5         \\ \hline
relBastaoGarraSela relXBastaoGarra relXSela                                                                                                                                                                                                                                                                                                 & rg6         \\ \hline
relXSela relXColar relTorreSela                                                                                                                                                                                                                                                                                                             & rg7         \\ \hline
relBastaoUniversalColar relXBastaoUniversal                                                                                                                                                                                                                                                                                                & rg8         \\ \hline
relXBastaoUniversal relXCarretilha relBastaoUniversalCarretilha                                                                                                                                                                                                                                                                             & rg9         \\ \hline
relXCorda relXBastaoUniversal relBastaoUniversalCorda relBastaoUniversalTorre                                                                                                                                                                                                                                                             & rg10        \\ \hline
relXCorda relXBastaoUniversal relXColar relBastaoUniversalColar relBastaoUniversalSela                                                                                                                                                                                                                                                    & rg11        \\ \hline
\end{tabular}
\caption{Relacionamentos que formam os conjuntos $rg_n$. Cada conjunto $rg_n$ está relacionado com um objetivo A relação entre $rg_n$ e $goal_m$ determina os relacionaentos necessários para que um dado objetivo tenha condição de ser atingido.}
\label{relationsgroup1}
\end{table}


\begin{table}[H]
\centering
\scalefont{0.7}
\begin{tabular}{|p{0.8\linewidth}|l|}
\hline
\textbf{relacionamentos}                                                                                                                                                                                                                                                                                                                  & \textbf{rg} \\ \hline
relXColar                                                                                                                                                                                                                                                                                                                                 & rg12        \\ \hline
relXBastaoUniversal relBastaoUniversalEstropo relEstropoIsoladorVelho                                                                                                                                                                                                                                                                     & rg13        \\ \hline
relXBastaoUniversal relBastaoUniversalCordarelCordaEstropo relEstropoCorda                                                                                                                                                                                                                                                                & rg14        \\ \hline
relChaveCatracaBastaoUniversal relXChaveCatraca relXBastaoUniversal relChaveCatracaParafuso                                                                                                                                                                                                                                               & rg15        \\ \hline
relXBastaoSoquete relSoqueteParafuso                                                                                                                                                                                                                                                                                                      & rg16        \\ \hline
relXCorda relCordaBastaoGarra relBastaoGarraCondutor                                                                                                                                                                                                                                                                                      & rg17        \\ \hline
relXColar                                                                                                                                                                                                                                                                                                                                 & rg18        \\ \hline
relChaveCatracaBastaoUniversal relXChaveCatraca relXBastaoUniversal relChaveCatracaParafuso relParafusoTorre relXBastaoSoquete relSoqueteParafuso                                                                                                                                                                                           & rg19        \\ \hline
relXCorda                                                                                                                                                                                                                                                                                                                                 & rg20        \\ \hline
relXEstropo relEstropoIsoladorNovo                                                                                                                                                                                                                                                                                                         & rg21        \\ \hline
relXBastaoUniversal relBastaoUniversalCorda relCordaEstropo relEstropoCorda                                                                                                                                                                                                                                                                & rg22        \\ \hline
relXCorda                                                                                                                                                                                                                                                                                                                                 & rg23        \\ \hline
relChaveCatracaBastaoUniversal relXChaveCatraca relXBastaoUniversal relChaveCatracaParafuso relParafusoTorrerelXBastaoSoquete relSoqueteParafuso                                                                                                                                                                                           & rg24        \\ \hline
relXCorda relCordaBastaoGarra relBastaoGarraCondutor                                                                                                                                                                                                                                                                                      & rg25        \\ \hline
relChaveCatracaBastaoUniversal relXChaveCatraca relXBastaoUniversal relChaveCatracaParafuso                                                                                                                                                                                                                                               & rg26        \\ \hline
relXSela relXColarrelXBastaoGarrarelXBastaoUniversal relXBastaoSoquete relXCorda relXCarretilha relXChaveCatraca relColarBastaoGarra relCordaBastaoGarra relBastaoGarraTorre relBastaoGarraCondutor relBastaoUniversalCarretilha relBastaoGarraSela relBastaoUniversalSela relSelaColar relTorreSela relBastaoUniversalCorda relBastaoGarraCorda & rg27        \\ \hline
\end{tabular}
\caption{Relacionamentos que formam os conjuntos $rg_n$. Cada conjunto $rg_n$ está relacionado com um objetivo A relação entre $rg_n$ e $goal_m$ determina os relacionaentos necessários para que um dado objetivo tenha condição de ser atingido.}
\label{relationsgroup2}
\end{table}




A tabela \ref{conditions} apresenta as condições que constituem os conjuntos \textbf{cg}.

\begin{table}[H]
\centering
\scalefont{0.8}
\begin{tabular}{|l|l|}
\hline
\textbf{condições}         & \textbf{cg} \\ \hline
umidade70,noVento,noChuva,sol & cg1         \\ \hline
\end{tabular}
\caption{Todas as condições que constituem o conjunto $cg_n$. Este conjunto está relacionando com um ou mais objetivos e determina quais são as condições que devem ser mantidas para que o agente tenha uma situação razoável para tentar alcançar um certo objetivo}
\label{conditions}
\end{table}


A tabela \ref{goalsrelationsentity} define as relações $hasRelation(g_i,rg_n)$ onde a coluna objetivo é representada por $g_i$, $hasEntity(g_i,eg_m)$, $hasCondition(g_i,cg_n)$.  

		
\begin{table}[H]
\centering
\scalefont{0.6}
\begin{tabular}{|l|l|l|l|}
\hline
\textbf{objetivo}  & \textbf{rg} & \textbf{eg} & \textbf{cg} \\ \hline
goal0 & rg0 & eg0 & cg1 \\ \hline
goal0 & rg0 & eg0 & cg1 \\ \hline
goal1 & rg1 & eg1 & cg1 \\ \hline
goal2 & rg2 & eg2 & cg1 \\ \hline
goal3 & rg3 & eg3 & cg1 \\ \hline
goal4 & rg4 & eg4 & cg1 \\ \hline
goal5 & rg5 & eg5 & cg1 \\ \hline
goal6 & rg6 & eg6 & cg1 \\ \hline
goal7 & rg7 & eg7 & cg1 \\ \hline
goal8 & rg8 & eg8 & cg1 \\ \hline
goal9 & rg9 & eg9 & cg1 \\ \hline
goal10 & rg10 & eg10 & cg1 \\ \hline
goal11 & rg11 & eg11 & cg1 \\ \hline
goal12 & rg12 & eg12 & cg1 \\ \hline
goal13 & rg13 & eg13 & cg1 \\ \hline
goal14 & rg14 & eg14 & cg1 \\ \hline
goal15 & rg15 & eg15 & cg1 \\ \hline
goal16 & rg16 & eg16 & cg1 \\ \hline
goal17 & rg17 & eg17 & cg1 \\ \hline
goal18 & rg18 & eg18 & cg1 \\ \hline
goal19 & rg19 & eg19 & cg1 \\ \hline
goal20 & rg20 & eg20 & cg1 \\ \hline
goal21 & rg21 & eg21 & cg1 \\ \hline
goal22 & rg22 & eg22 & cg1 \\ \hline
goal23 & rg23 & eg23 & cg1 \\ \hline
goal24 & rg24 & eg24 & cg1 \\ \hline
goal25 & rg25 & eg25 & cg1 \\ \hline
goal26 & rg26 & eg26 & cg1 \\ \hline
goal27 & rg27 & eg27 & cg1 \\ \hline
\end{tabular}
\caption{Define a relação entre os objetivos, conjuntos $rg_n$, $eg_n$ e $cg_n$ }
\label{goalsrelationsentity}
\end{table}
