Isso se deu por escrever tanto as regras como as especificações dos predicados expostas em \textit{Caso de Estudo} nesta linguagem de programação. A seguir há um exemplo da regra \ref{reldeonticrole} escrita em Prolog.

hasPermission(RHO,GOAL) :- hasObligation(RHO,GOAL).

A seguir há um exemplo da especificação $adoptsRole(agente1,supervisor)$ implementada em Prolog.

adoptsRole(agente1,supervisor).

A validação dos raciocínios em \ref{racs} é feita por meio "queries" ao sistema. Em Prolog isso é feito por escrever a especificação do implicador ($implicador \to implicado$) e por meio do algoritmo de \textit{Backtracking}, é possível encontrar todos os predicados que são verdade. Por exemplo, para validar o Raciocínio 3 se faz necessário fazer as seguintes "queries": $? - stopped(g1)$, $? - entityViol(agente4,g1,pano)$, $? - entityViol(agente3,g1,pano)$, $? - entityViol(agent2,g1,pano)$. Se o programa retorna como verdade todos os predicados implicadores que se espera pela verificação analítica, então é razoável concluir que os raciocínios foram validados. Os sete raciocínios aqui expostos apresentaram o mesmo resultado na implementação em Prolog. Isso permitiu concluir que as regras foram usadas adequadamente para analisar como o sistema raciocina o estudo de caso. 