Como objetivo geral os pesquisadores propõem, dentro do contexto computacional, um modelo capaz de representar cenários onde os agentes podem cometer erros durante a execução de algum atividade. Tendo em vista esses erros, eles podem prejudicar a si mesmos como seus colegas. Assim sendo, esse estudo investiga esse cenário por meio de um modelo que incorpora normas (instruções claras que devem ser fornecidas a um agente), violação (ocorre quando um agente descumpri uma norma) e sanção (consequências decorrentes de uma violação desta norma). Não apenas isso, mas nesse cenário há a consideração de que essas questões estão atreladas aos artefatos que são usados pelos agentes. Esses artefatos são objetos passivos que estão sujeitos a influencia da ação dos agentes. Os artefatos são usados para representar objetos como ferramentas e máquinas, por exemplo. 

Os pesquisadores consideram, neste estudo, que as violações não estão relacionadas apenas aos artefatos, mas também nas relações entre todas as entidades existentes ao longo do cenário. As entidades compõem todos os elementos (agentes e artefatos) vinculados ao meio. As condições ambientais onde os agentes e os artefatos estão inseridos também são considerados neste estudo. O entendimento que se tem por condição ambiental consiste em um dado processo que existe no ambiente e que, de certa forma, pode influenciar a execução das atividades dos agentes. Como condições ambientais têm-se por exemplo os seguintes eventos; chuva, sol, vento e neve. 

Não é do interesse deste estudo investigar condições normativas cuja caráter punitivo advém de sanções administrativas e jurídicas que ocorre no que tange a decisão de uma certa autoridade. Assim sendo, o modelo resultante deve ser capaz de representar situações onde os agentes sofrem consequências físicas negativas resultantes de uma cadeia de causalidade que foi ocasionada pelo erro de alguém (usando, para isso, o conceito de norma, violação e sanção). Dada essa circunstância, a representação tratada neste estudo deve trabalhar, em sua estrutura interna, o conceito de risco. É digno de nota que o termo risco apresenta um espectro semântico bastante amplo podendo ser usado nos mais diferentes contextos possíveis (ex. risco financeiro de um dado ativo). Por conta disto, é necessário frisar que neste texto o vocábulo risco é usado no mesmo significado atribuído pela comunidade de Engenharia de Segurança. 

Neste estudo, os pesquisadores têm interesse em apresentar um vocabulário específico no que tange ao objeto em interesse a ser representado. Esse modelo, portanto, deve ser capaz de representar organizações tais como trabalhadores em uma obra, industrias, profissionais no âmbito hospitalar e estruturas deste gênero entendendo como se dá as violações em âmbitos específicos (falta de ferramentas, não conseguir executar procedimentos apropriados, executar uma atividade cujo momento não era adequado para isso). 

Para o propósito aqui posto, se não há infinitas maneiras, há ao menos um numero muito variado de formas para construir um modelo com os objetivos aqui definidos. Contudo, esse texto desbrava ao menos uma das formas de se realizar isso, analisa a abordagem sobre um estudo e define uma comparação com os modelos já existentes dada pela comunidade acadêmica. A análise comparativa é estruturada em termos de; conceitos (verificar quais são similares e quais são diferentes), relações entre os conceitos, capacidade de generalização (quanto mais mundos possíveis o modelo é capaz de descrever, mais genérico o é), capacidade de especificação tendo como referência circunstâncias que correspondem ao presente neste texto introdutório e uma verificação sobre como esses critérios se dão dos modelos em relação a um dado estudo de caso aqui presente. 