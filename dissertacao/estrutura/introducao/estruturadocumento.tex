O capítulo \ref{chap:fundteoric} apresenta os conceitos e estudos básicos que fornecem fundamentos para as pesquisas desenvolvidas ao longo desta pesquisa. O capítulo \ref{chap:metod} descreve as etapas metodológicas adotadas pelo autor para conceber, verificar e analisar o modelo conceitual. O capítulo \ref{chap:resul} mostra os resultados que consistem do: O modelo conceitual em si (conjuntos, predicados e regras de inferência) e da aplicação deste para uma situação simples a fim de introduzir o leitor ao uso do modelo. O capítulo \ref{casestudy} mostra o uso do modelo em um estudo descaso. Isso implica mostrar como os conjuntos (classes sobre a ótica de orientação a objeto) são usados para especificar as instâncias de estudo de caso. Esse capítulo apresenta também os raciocínios que podem ser feitos ao depurar a aplicação das regras. O capítulo \ref{chap:disc} realiza uma análise do modelo conceitual resultante neste estudo a luz de modelos computacionais bem verificados pela comunidade acadêmica. Nesse capítulo são feitas, também, discussões sobre a aplicação do modelo conceitual no estudo de caso verificando aspectos que ocorreram adequadamente bem como aspectos que devem ser analisados com mais profundidade. O capítulo \ref{chap:conc} realiza uma verificação geral do estudo, mostra como o objetivo geral bem como os objetivos específicos foram atingido(s) e apresenta quais são os próximos passos. 