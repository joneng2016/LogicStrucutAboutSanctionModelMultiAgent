Uma das motivações desse estudo reside em obter uma compreensão do problema em voga. Portanto essa situação é feita por obter um modelo conceitual que é concebido tendo como base dois fatores. Um desses fatores, consiste numa averiguação de modelos computacionais que são aplicados em sistemas multiagentes (inclusive os normativos). O outro fator consiste em averiguar um dado estudo de caso a fim de abstrair estruturas conceituais que possam ser do interesse dessa pesquisa.

Uma outra motivação desse estudo consiste em averiguar como o modelo conceitual resultante pode ser vinculado a arcabouços e modelos computacionais que fazem parte do \textit{mainframe} acadêmico no campo da computação em sistemas multiagentes normativos.

A motivação desse estudo, portanto, reside em explorar e entender a problemática atrelada a cenários de acidentes e riscos com a finalidade de avaliar em clareza as possibilidades existentes dentro do contexto computacional.