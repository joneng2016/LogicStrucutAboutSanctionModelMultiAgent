Nesse estudo os pesquisadores assumem que um acidente acontece porque alguém cometeu algum erro. Não é possível afirmar que essa proposição é verdade para todos os casos de acidente, contudo é possível verificar que existe situações de acidentes que se enquadram nessa condição. Acidentes de carro são exemplos disso, pois ocorrem (em muitos casos) porque pelo menos um condutor cometeu algum erro. 

Os pesquisadores entendem que uma representação baseada em sistemas multiagentes normativos é apropriada para tratar situações assim. Modelos baseados nesse tipo de representação tratam os seguintes conceitos; normas (instruções claras que devem ser fornecidas a um agente), violação (ocorre quando um agente descumpri uma norma) e sanção (consequências decorrentes de uma violação desta norma) \cite{dastaniframework} \cite{amodelmultiagentsystemdynamicrelationship}, \cite{ontologynormative}. Nessa linha de raciocínio é possível conceber os profissionais como os agentes, o que eles devem fazer é possível representar por meio de normas, os erros são as violações e as consequências são dadas por sanções.  

No que tange a investigação acadêmica, a motivação para realizar esse trabalho reside em entender como toda estrutura conceitual na área de agentes normativos (concebida pela comunidade acadêmica) pode ser usada para representar cenários de atividades que definem um potencial de acidentes. Outra motivação reside verificar quais são as vantagens e desvantagens perante os modelos que são apropriados para essa condição. Além desses duas motivações, é de grande interesse acadêmico analisar os raciocínios que podem se construídos e qual é o potencial desses no que diz respeito a correspondência com a realidade dos fatos. 