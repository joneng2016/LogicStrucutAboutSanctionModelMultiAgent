Realizar um estudo de campo e verificar como um dado modelo se adapta a esse estudo, em termos práticos, é uma atividade bastante complexa de ser feita que implica diversas análises (sendo que muitas vezes essas análises causam levantam novas questões para estudo). Cenários atrelados a verificação de acidentes e seguranças apresentam esse tipo de característica. 

Uma abordagem que se faz necessária reside entender quais são os arcabouços que podem ser usados para para especificar cenários que tratam de acidentes e riscos, tendo em vista a complexidade do problema e tendo em vista a existência de diversos modelos na computação muito bem consolidados que podem ser usados nessa classe de problemas. 

A motivação desse estudo, portanto, reside em um levantamento da complexidade do problema (por meio de um modelo conceitual) e entendimento de como as estruturas desse modelo podem ser usadas como especificações para arcabouços (considerando as vantagens e desvantagens).