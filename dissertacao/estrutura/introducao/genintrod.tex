Hoje em dia é comum que diversas pessoas se submetam a atividades (na grande maior das vezes, profissionais) que trazem algum tipo de risco a vida delas. Trabalhos como de eletricistas, bombeiros, área petro-química, serviços de telecomunicações e de transporte (motoristas de ônibus, carro) são apenas alguns exemplos onde profissionais são  expostos a condições de risco.

Acidentes em situações assim ocorrem pelos mais diversos motivos. Investigas as cadeias de causalidade que resultam nessas situações podem trazer um entendimento de como lidar com circunstâncias assim. Consequentemente, isso traz um potencial para dominuir o número de ocorrências desse gênero. Assim sendo, todas as técnicas possíveis para 
investigar esse tipo de situação é algo de grande interesse para a comunidade em geral. 

A computação é uma ciência que apresenta um grande potencial de contribuição para esse problema. Isso envolve criar representações do mundo real que permitam efetivar algum tipo de raciocínio relevante sobre o assunto. Essa situação levanta diversas reflexões interessantes, que são; Qual paradigma (dos muitos) que pode ser usado para constituir situações assim? Quais modelos já existem para tratar esse tipo de problema e quais são as concepções que eles não conseguem tratar com eficiência? Qual é as raizes dada pela comunidade acadêmica para os conceitos que podem ser usados para tratar esse tipo de problema? Existe a necessidade de criar algum conceito para tratar 
alguma problemática? Quais são as regras de inferências que podem ser construídas para tratativa de questões assim? 

Nesse estudo em específico, os pesquisadores têm o interesse de investigar essa questão usando sisteas multiagentes normativos. A seção \ref{motivation} apresenta os motivos pelos quais é razoável fazer uso de sistemas multiagentes normativos para estudar questões envolvendo riscos e acidentes de trabalho. O entendimento do porque 
esse trabalho é importante é algo que será feito em \ref{relevance}. Os objetivos desse estudo estarão sintetizados na seção \ref{goals}.
