Diversas pessoas são submetidas a algum tipo de risco na execução de atividades profissionais. Trabalhos como de eletricistas, bombeiros, área petroquímica, serviços de telecomunicações e de transporte (motoristas de ônibus e de carro) são apenas alguns exemplos onde profissionais são expostos a condições de risco.

Acidentes em situações assim ocorrem pelos mais diversos motivos. Investigar as cadeias de causalidade que resultam nessas situações pode trazer um entendimento de como lidar com essas circunstâncias. Consequentemente, isso traz um potencial para diminuir o número de acidentes. A computação é uma ciência que apresenta um grande potencial de contribuição para esse problema. Isso envolve criar representações que permitam formalizar raciocínios relevantes sobre o assunto.

Nesse estudo em específico, os pesquisadores têm o interesse de explorar, sobre uma perspectiva de representação computacional, cenários de manutenção que apresentam potencial de risco. A seção \ref{motivation} apresenta os motivos pelos quais é razoável fazer uso de sistemas multiagentes normativos para estudar questões envolvendo riscos e acidentes de trabalho. O entendimento do porque esse trabalho é importante é argumentado na seção \ref{relevance}. O objetivo geral desse estudo está sintetizado na seção \ref{goals}.