Diversas pessoas são submetidas a algum tipo de risco na execução de atividades profissionais. Trabalhos como de eletricistas, bombeiros, área petroquímica, serviços de telecomunicações e de transporte (motoristas de ônibus e de carro) são apenas alguns exemplos onde profissionais são expostos a condições de risco.

Acidentes em situações assim ocorrem pelos mais diversos motivos. Investigar as cadeias de causalidade que resultam nessas situações pode trazer um entendimento de como lidar com essas circunstâncias. Consequentemente, isso traz um potencial para diminuir o número de acidentes. A computação é uma ciência que apresenta um grande potencial de contribuição para esse problema. Isso envolve criar representações que descrevam atividades com riscos de acidentes e que permitam realizar raciocínios sobre as essas relações de causalidade.

Algo que exemplifica essa situação são as atividades praticadas por profissionais que trabalham com eletricidade em linha viva. Isso, pois esses trabalhadores realizam procedimentos de manutenção em componentes, estruturas e equipamentos elétricos energizados que demando algo fluxo de potência (ex. transformadores, condutores). A particularidade nos processos de linha viva consiste no fato de que esses equipamentos e estruturas se mantêm energizados durante a ação dos profissionais. Essa situação expõem esses profissionais a perigos de acidentes com eletricidade causando mortes ou ferimentos extremamente sérios com danos permanentes a vida dos acidentados.  