Diversas pessoas são submetidas a algum tipo de risco na execução de atividades profissionais. Trabalhos como de eletricistas, bombeiros, área petroquímica, serviços de telecomunicações e de transporte (motoristas de ônibus e de carro) são apenas alguns exemplos onde profissionais são expostos a condições de risco.

Acidentes em situações assim ocorrem pelos mais diversos motivos. Investigar as cadeias de causalidade que resultam nessas situações podem trazer um entendimento de como lidar com circunstâncias assim. Consequentemente, isso traz um potencial para diminuir o número de acidentes. Assim sendo, é grande contribuição para comunidade a realização de estudos desse tipo. A computação é uma ciência que apresenta um grande potencial de contribuição para esse problema. Isso envolve criar representações que permitam formalizar raciocínios relevantes sobre o assunto.

Nesse estudo em específico, os pesquisadores têm o interesse de investigar essa questão usando sistemas multiagentes normativos. A seção \ref{motivation} apresenta os motivos pelos quais é razoável fazer uso de sistemas multiagentes normativos para estudar questões envolvendo riscos e acidentes de trabalho. O entendimento do porque esse trabalho é importante é algo que será feito em \ref{relevance}. O objetivo geral desse estudo está sintetizado na seção \ref{goals}.