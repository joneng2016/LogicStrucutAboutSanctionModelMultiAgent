Parte dos objetivos específicos desse modelo reside na capacidade reside na capacidade de representação de cenários. Portanto, a lista a seguir apresenta os cenários aos quais o modelo proposto tem como por finalidade representar;
\begin{itemize}
    \item Uma organização entre pessoas que atuam em prol de objetivos em comum.
    \item Pesssoas que devem ser capaz de seguir instruções. 
    \item Situações onde alguém comete algum erro por fazer algo que viola o que deve ser feito. 
    \item Situações em que pessoas sofrem acidentes por consequência de suas próprias ações ou por consequência das ações de outros. 
    \item Apresentar as consequências a integridade física (ferimentos, morte) que um acidente ocasiona. 
    \item Aspectos aleatórios atrelados ao acidente (eventos ruins que apresentam uma possibilidade de ocorrer). 
    \item A relação entre condições ambientes, objetos (ferramentas, máquinas) e pessoas. 
    \item Conjuturas de condições ambientes, objetos e pessoas que contribuiem (e como se dá essa contribuição) para ocorrência de eventos desagradáveis. 
\end{itemize} 

Para realizar essa representação é crucial que alguns conceitos estejam presentes, esses são descritos na lista a seguir como objetivos específicos;

\begin{itemize}
	\item As pessoas são representadas usando conceito de agentes.
	\item Os objetos que são sujeitos a manipulação das pessoas são representados usando artefatos.
	\item Uma organização entre pessoas em prol de algum objetivo é representada por sistemas multiagentes. 
	\item As instruções sobre o que deve ou não deve ser feito é representado usando conceitos de normas.	
	\item Os erros cometidos por pessoas são representados por violações. 
	\item As consequências negativas decorrentes dos erros são representados por sanções.	
\end{itemize}

