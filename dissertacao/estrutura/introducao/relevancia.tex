Esse estudo apresenta contribuições para dois grandes campos do conhecimento. O primeiro campo é a representação computacional. O problema em análise apresenta uma complexidade de representação extremamente elevado. O uso de um arcabouço em específico fará com que certas características sejam realçadas a custo de um ofuscamento das demais. Assim sendo, esse trabalho é relevante ao conceber uma estrutura conceitual (definido em um formalismo lógico) que possibilite apontar para os aspectos positivos e negativos de cada arcabouço. 

Entendimento de como se dá a segurança do trabalho é o segundo campo para o qual esse estudo gera contribuição. O interesse desse texto é primeiramente computacional e não entra no mérito do campo Segurança no trabalho propriamente dito. Isso pois o interesse dos pesquisadores está relacionado a representar, em termos de conceitos, a área de segurança o que não implica elaborar um novo conceito de segurança propriamente dito. Contudo, o fato de conceber um sistema computacional para lidar com riscos e situações inesperadas em condições de trabalho consiste em um diálogo com a área de segurança por fornecer uma maneira formal de lidar com situações assim. Nessa situação, a relevância nessa área não é tão significativa como nos campos anteriores, contudo não é desprezível e merece ser notada.