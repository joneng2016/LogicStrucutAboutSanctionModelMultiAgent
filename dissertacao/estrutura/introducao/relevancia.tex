A computação pode contribuir de inúmeras maneiras por conceber sistemas que resolvam de diferentes formas os problemas atrelados a acidentes de manutenção. Contudo, isso não pode ser feito sem um entendimento, em termos conceituais e computacionais, do estado deste problema. Por conta disso, a relevância desse estudo reside na apresentação de um modelo conceitual que defina em termos de classes e relacionamento uma estrutura lógico-descritiva de fatores ambientais e organizacionais que resultam em acidentes. 

Esse estudo descreve a ocorrência de acidentes através de agentes, artefatos, violações e sanções. O que se define como inovador em relação aos demais estudos consiste no fato de que o esse modelo descreve a ocorrência de acidentes em termos de possibilidades em relação a uma violação. Em outros termos, isso define que, em determinados cenários, a ocorrência de uma violação em uma dada relação gera a possibilidade de ocorrer um acidente em algum outro momento futuro da manutenção.  