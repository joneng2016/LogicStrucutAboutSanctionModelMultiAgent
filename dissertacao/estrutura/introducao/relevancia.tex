A computação pode contribuir de inúmeras maneiras por conceber sistemas que resolvam de diferentes formas os problemas atrelados a acidentes de manutenção. Contudo, isso não pode ser feito sem um entendimento, em termos conceituais e computacionais, do estado deste problema. Por conta disso, a relevância desse estudo reside na apresentação de um modelo conceitual que defina em termos de classes e relacionamento uma estrutura lógico-descritiva de fatores ambientais e organizacionais que resultam em acidentes. 

Por mais que existam modelos computacionais altamente sofisticados que possam ser aplicados a situações assim, um modelo conceitual dessa problemática apresenta potencial de análise de novas perspectivas do problema. Assim sendo, esse estudo também é relevante por estimular o debate do uso da computação em cenários de acidentes explorando novos tópicos de reflexão. Sendo que esse estudo visa explorar esse problema a partir de uma análise criativa, contudo racional, lógica, transparente e científica. 