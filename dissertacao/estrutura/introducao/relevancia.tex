Esse estudo apresenta três grandes campos (sendo que dois são da computação) para os quais esse estudo é relevante. O primeiro campo a ser contribuído é a representação computacional. Isso se da pelo fato de que esse estudo se propõem a conceber uma representação que usa conceitos e estruturas (norma, violação e sanção) para resolver problemas envolvendo situações que apresentam potencial para acidente. Nesse aspecto existe duas condições a serem 
consideradas; a primeira situação consiste nos modelos que não são estruturados para representar cenários específicos de acidentes porém que podem ser usados para essa finalidade. Nessa situação a inovação e relevância desse estudo reside em aproveitar estruturas desses modelos para construir 
vocábulos e regras específicas para representar cenários dessa categoria. A segunda situação consiste em analisar modelos cujo foco implica representar cenários de acidentes. Nessa condição esse estudo é relevante porque aborda formas diferentes de fazer isso, o que gera consequências diferentes onde cada conjuntura específica apresenta vantagens e desfavantágeis únicas.    

O segundo campo para o qual esse estudo é relevante são Sistemas Multiagentes. Existe diversos modelos e aplicações onde esse tipo de sistema é interessante e isso está atrelado a um espectro que vai desde a criação de novas tecnologias a entedimento das dinâmicas de seres humanos em termos computacionais. Assim sendo, esse estudo apresenta uma forma específica de \textit{SMA} normativos a fim de entender, em termos computacionais, a dinâmica de trabalhadores que são expostos a riscos e acidentes. Como será tratado no texto adiante, existe modelos de \textit{SMA} que têm esse propósito. Contudo o modelo apresentado nesse estudo possui características específicas justamente por se basear em agentes normativos e por trazer regras específicas de violação e sanção para questões atreladas a acidentes tendo em vista análises e observações dos pesquisadores (sendo que essas análises foram feitas em estudos relacionados a temática de agentes e relacionados a acidentes de trabalho). É digno de nota que esse estudo não se preocupa em conceber representações que lidam com aspectos internos dos agentes, mas sim com a dinâmica sistemica dos mesmos.

Entendimento de como se dá a segurança do trabalho é o terceiro campo para o qual esse estudo gera contribuição. O interesse desse texto é primeiramente computacional e não entra no mérito da ciência da Segurança no trabalho propriamente dito. Isso pois o interesse dos pesquisadores está relaciondo a contruir um modelo usando, em partes, os conceitos presentes nessa área o que não envolve elaborar um novo conceito de segurança propriamente dito. Contudo, o fato de conceber um sistema computacional para lidar com riscos e situações inesperadas em condições de trabalho consiste em um dialogo com a área de segurança por fornecer uma maneira formal de lidar com situações assim. Nessa situação, a relevância nessa área não é tão significativa como nos campos anteriores, contudo não é desprezível e merece ser notada.