A definição de um artefato pode ser feita por analisar, antes, o comportamento de um agente. De maneira genérica, existe duas formas que podem ser
usadas para caracterizar o comportamento de um agente, e essas são \textit{goal-governed} e \textit{goal-oriented} \cite{relationwithagentprogram} \cite{programingagentartefact}.

Um agente que é caracterizado como \textit{goal-governed} são aqueles que apresentam capacidades cognitivas, que podem representar os seus respectivos objetivos
e, portanto, são capazes de definir seus objetivos em interesse \cite{relationwithagentprogram} \cite{programingagentartefact}. Em contraste com isso, \textit{goal-oriented} consiste nos agentes que são programados
com a finalidade de alcançar um determinado objetivo \cite{relationwithagentprogram} \cite{programingagentartefact}. Em um sistema multiagente muitas vezes
uma dada entidade não é adequadamente representada por nenhuma dessas duas categorias. 

Assim sendo, artefatos são enntidades que não se enquadram em  \textit{goal-governed}, não se enquadram em \textit{goal-oriented}, são explicitamente projetados com o propósito de serem explorados pelos agentes para que
possam alcançar seus objetivos individuais e sociais \cite{programingagentartefact}. 

Para ilustrar usando como referência a sociedade humana; se agentes (entidades autonomas) estão para seres humanos então artefatos são as ferramentas (não autonomas)
que são usadas para uma determinada finalidade (ex. um marceneiro (agente), usa um martelo (artefato) para pregar um prego (artefato)) \cite{programingagentartefact}.

Uma outra distinção entre agentes e artefatos se torna claramente explícita quando verificar sobre o ponto de vista conceitual e filosófico. Isso, pois
agentes apresentam a capacidade de se comunicar com outros agentes, em contraste com isso artefatos não apresentam essa condição \cite{programingagentartefact}.

Existem quatro elementos que podem ser usados para caracterizar um artefato \cite{programingagentartefact}, que são; \textit{interface de uso}, \textit{instruções de operação},
\textit{funcionalidade} e \textit{estrutura-comportamento}.

A \textit{interface de uso} consiste no conjunto de operações que podem ser invocadas pelo agente a fim explorar as suas funcionalidades \cite{programingagentartefact}. 
A \textit{instruções de operação} consiste na descrição de como fazer o uso das funcionalidades do artefato. Isso implica protocolos de uso das operações 
que devem ser invocadas pelo agente \cite{programingagentartefact}. A \textit{Funcinalidade} do artefato consiste no propósito definido pelo programador 
do sistema \cite{programingagentartefact}. A \textit{estrutra-comportamento} consiste nos aspectos internos do artefato, que é como o artefato é implimentado
a fim de providenciar as suas funcionalidades \cite{programingagentartefact}.    

\cite{cartago}
