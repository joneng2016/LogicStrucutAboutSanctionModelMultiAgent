A definição de um artefato pode ser feita por analisar antes o comportamento de um agente. De maneira genérica, existem duas formas que podem ser usadas para caracterizar o comportamento de um agente. Essas são: \textit{goal-governed} e \textit{goal-oriented} \cite{relationwithagentprogram} \cite{programingagentartefact}.

\textit{Goal-governed} são os agentes que apresentam capacidades cognitivas, que podem representar os seus respectivos objetivos e, portanto, são capazes de estruturar seu interesse \cite{relationwithagentprogram} \cite{programingagentartefact}. Em contraste, \textit{goal-oriented} consiste nos agentes que são programados com a finalidade de alcançar um determinado objetivo \cite{relationwithagentprogram} \cite{programingagentartefact}. Em um sistema multiagente muitas vezes
uma entidade não é adequadamente representada por nenhuma dessas duas categorias. 

Assim sendo, artefatos são entidades que não se enquadram em  \textit{goal-governed} e não se enquadram em \textit{goal-oriented}. Esses elementos são explicitamente projetados com o propósito de serem explorados pelos agentes para que possam alcançar seus objetivos individuais e sociais \cite{programingagentartefact} \cite{cartago}. 

Para ilustrar, usando como referência a sociedade humana: se agentes (entidades autônomas) estão para seres humanos, então artefatos estão para as ferramentas (não autônomas) que são usadas com uma determinada finalidade (ex. um marceneiro, agente, usa um martelo (artefato) para pregar um prego (artefato)) \cite{programingagentartefact}.

Uma outra distinção entre agentes e artefatos se torna claramente explícita quando verificada sob o ponto de vista conceitual e filosófico, pois os agentes apresentam a capacidade de se comunicar com outros agentes, em contraste, os artefatos não apresentam essa condição \cite{programingagentartefact}.

Existem quatro elementos que podem ser usados para caracterizar um artefato \cite{programingagentartefact}: \textit{interface de uso}, \textit{instruções de operação}, \textit{funcionalidade} e \textit{estrutura-comportamento}.

A \textit{interface de uso} consiste no conjunto de operações que podem ser invocadas pelo agente a fim explorar as suas funcionalidades \cite{programingagentartefact}. As \textit{instruções de operação} consistem na descrição de como fazer o uso das funcionalidades do artefato, implicando nos protocolos de uso das operações que devem ser invocadas pelo agente \cite{programingagentartefact}. A \textit{Funcionalidade} do artefato consiste no propósito definido pelo programador do sistema \cite{programingagentartefact}. A \textit{estrutura-comportamento} consiste nos aspectos internos do artefato. Isso define como o artefato é implementado a fim de providenciar as suas funcionalidades \cite{programingagentartefact}.    

Existem arcabouços para especificar artefatos. Um desses recebe o nome de \textit{Cartago} e será descrito com maior riqueza de detalhes na subseção que se segue.
 
\subsection{Cartago}

Cartago (\textit{Common "Artefacts for Agents" Open Framework}) é um arcabouço usado para especificar as relações entre agentes e artefatos. Tendo em vista o fato de ser um dos principais \textit{frameworks} na descrição dessas interações entre agentes e artefatos, não é possível falar no tema de artefatos em \textit{SMA} sem ao menos comentar a estrutura conceitual presente no Cartago. Esse modelo é composto de três blocos, que são: \textit{agent bodies} , \textit{artefact} e \textit{workspace} \cite{cartago}.

\textit{Agent bodies} é  o que possibilita a inteligência do agente de se relacionar com o ambiente. Para cada agente criado, háum \textit{agent bodies} construído. Um \textit{agent bodies} possui \textit{effectors} (efetores) que tem o propósito deexecutar ações sobre o ambiente de trabalho e possui sensores (captadores de sinais do ambiente em sua volta). A relação completaentre agente-\textit{agent bodies}-artefato é dada da seguinte forma: eventos observáveis são gerados pelos artefatos, sensores(componentes presentes no \textit{agent bodies}) são sensibilizados, esta informação é transmitida para a inteligência do  agenteque, por sua vez, realiza os raciocínios e toma as decisões necessárias). O agente comunica ao \textit{agent bodies} a ação quedeve ser feita ao meio e por último, através do \textit{effectors} uma ação é produzida no ambiente de trabalho. 

\textit{Artefacts} (artefatos) são os tijolos básicos na gerência do Cartago. Cada artefato apresenta um nome lógico e número de identificação (id) único que são definidos pelo \textit{artefact creator} no momento da instanciação. O nome lógico consiste num caminho ágil para o agente poder referenciar  e compartilhar o respectivo artefato com os demais agentes. O id é requisitado na identificação dos artefatos quando uma certa ação é executada. Os artefatos também apresentam nome completo que inclui o nome do(s) \textit{workspace(s)} onde está logicamente localizados \cite{cartago}.

A localização lógica dos artefatos fica dentro de \textit{workspace}, que pode ser usado para definir a topologia do ambiente de trabalho. Neste âmbito, o \textit{workspace} é feito com a finalidade de especificar nome lógico e id. Está dentro do escopo deste item definir uma topologia de ambiente possibilitando uma estrutura de interação entre agentes e artefatos. Assim sendo o agente só pode usar um artefato que está no mesmo \textit{workspace} onde ele está contido. 