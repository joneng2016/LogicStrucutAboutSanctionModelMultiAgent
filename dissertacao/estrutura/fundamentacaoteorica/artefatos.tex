A definição de um artefato pode ser feita por analisar, antes, o comportamento de um agente. De maneira genérica, existe duas formas que podem ser
usadas para caracterizar o comportamento de um agente, e essas são \textit{goal-governed} e \textit{goal-oriented} \cite{relationwithagentprogram} \cite{programingagentartefact}.

Um agente que é caracterizado como \textit{goal-governed} são aqueles que apresentam capacidades cognitivas, que podem representar os seus respectivos objetivos
e, portanto, são capazes de definir seus objetivos em interesse \cite{relationwithagentprogram} \cite{programingagentartefact}. Em contraste com isso, \textit{goal-oriented} consiste nos agentes que são programados
com a finalidade de alcançar um determinado objetivo \cite{relationwithagentprogram} \cite{programingagentartefact}. Em um sistema multiagente muitas vezes
uma dada entidade não é adequadamente representada por nenhuma dessas duas categorias. 

Assim sendo, artefatos são enntidades que não se enquadram em  \textit{goal-governed}, não se enquadram em \textit{goal-oriented}, são explicitamente projetados com o propósito de serem explorados pelos agentes para que
possam alcançar seus objetivos individuais e sociais \cite{programingagentartefact} \cite{cartago}. 

Para ilustrar usando como referência a sociedade humana; se agentes (entidades autonomas) estão para seres humanos então artefatos são as ferramentas (não autonomas)
que são usadas para uma determinada finalidade (ex. um marceneiro (agente), usa um martelo (artefato) para pregar um prego (artefato)) \cite{programingagentartefact}.

Uma outra distinção entre agentes e artefatos se torna claramente explícita quando verificar sobre o ponto de vista conceitual e filosófico. Isso, pois
agentes apresentam a capacidade de se comunicar com outros agentes, em contraste com isso artefatos não apresentam essa condição \cite{programingagentartefact}.

Existem quatro elementos que podem ser usados para caracterizar um artefato \cite{programingagentartefact}, que são; \textit{interface de uso}, \textit{instruções de operação},
\textit{funcionalidade} e \textit{estrutura-comportamento}.

A \textit{interface de uso} consiste no conjunto de operações que podem ser invocadas pelo agente a fim explorar as suas funcionalidades \cite{programingagentartefact}. 
A \textit{instruções de operação} consiste na descrição de como fazer o uso das funcionalidades do artefato. Isso implica protocolos de uso das operações 
que devem ser invocadas pelo agente \cite{programingagentartefact}. A \textit{Funcinalidade} do artefato consiste no propósito definido pelo programador 
do sistema \cite{programingagentartefact}. A \textit{estrutra-comportamento} consiste nos aspectos internos do artefato, que é como o artefato é implimentado
a fim de providenciar as suas funcionalidades \cite{programingagentartefact}.    

Cartago (\textit{Common "Artefacts for Agents" Open Framework}) é um framework usado para especificar as relações entre agentes e artefatos. Tendo em vista o fato 
de ser um dos principais \textit{frameworks} na descrição das interações entre agentes e artefatos, não é possível falar no tema de artafatos em \textit{SMA} 
sem ao menos comentar a estrutura conceitual presente no Cartago. Esse modelo é composto de três blocos, que são; \textit{agent bodies} , \textit{artefact} e \textit{workspace} \cite{cartago}.

 \textit{Agent bodies} é  o que possibilita a inteligência do agente se relacionar com o ambiente. Para cada agente criado, há um \textit{agent bodies} construído.
 Um \textit{agent bodies} possui \textit{effectors} (efetores) que tem o propósito de executar ações sobre o ambiente de trabalho e possui sensores (captam 
 sinais do ambiente em sua volta). A relação completa entre agente-\textit{agent bodies}-artefato, no Cartago, é dada da seguinte forma; eventos observáveis são 
 gerados pelos artefatos, sensores (componentes presentes no \textit{agent bodies}) são sensibilizados, essa informação é transmitida para a inteligência do 
 agente (esta, por sua vez, realiza os raciocínios e toma as  decisões necessárias), o agente comunica ao \textit{agent bodies} a ação que deve ser feita ao 
 meio e, por último, através do \textit{effectors} uma dada ação é produzida no ambiente de trabalho. 

 \textit{Artifacts} (artefatos) são os tijolos básicos na gerência do Cartago. Cada artefato apresenta um nome lógico e número de identificação (id)
 único que são definidos pelo \textit{artefat creator} no momento da instanciação. O nome lógico consiste num caminho ágil para o agente poder referenciar 
 e compartilhar o respectivo artefato com os demais agentes. O id é requisitado na identificaçao dos artefatos quando uma dada ação é executada. Os artefatos
 também apresentam nome completo que inclui o nome do(s) \textit{workspace(s)} onde está logicamente localizados \cite{cartago}.

 \textit{Workspaces} é a localização lógica dos artefatos ocorre dentro de \textit{workspaces}, que pode ser usado para definir a topologia do ambiente de trabalho.
 Neste ambito, o \textit{workspaces} são feitos com a finalidade de especificar nome lógico e id. Está dentro do escopo deste item definir uma topologia de ambiente
 possibilitando que uma estrutura de interação entre agentes e artefatos. Assim sendo o agente só pode usar
 um artefato que está no mesmo \textit{workspace} onde ele está contido. 