Quando se trata de normas em sistemas multiagentes é de crucial importância definir claramente este termo. Isso, pois há diversos estudos que tratam o conceito de norma sobre perspectivas
diferentes. Por exemplo, os estudos \cite{formalizeagent} \cite{formalizeagent2} apresentam normas, em sistemas multiagentes, para representar a presença de sociedades, institutos e organizações.
Há estudos que tratam normas como maneiras dos agentes trabalharem de forma coordenada com propósito de cumprir um objetivo global e também como uma maneira de obedecer as autoridades do 
sistema \cite{modelingnormsforautnomousagent} \cite{amodelmultiagentsystemdynamicrelationship}.

\cite{amodelmultiagentsystemdynamicrelationship}
\cite{modelingnormsforautnomousagent}
\cite{dastaniframework}
\begin{itemize}
    \item Definir o que é norma
    \item Tratar os tipos de normas
    \item Tratar o conceito de agentes normativos
    \item Tratar o conceito de sistemas multiagentes normativos 
    \item Apresentar o modelo do Dastani
    \item Enfatizar como ele apresenta os conceitos de violação e sanção
\end{itemize}