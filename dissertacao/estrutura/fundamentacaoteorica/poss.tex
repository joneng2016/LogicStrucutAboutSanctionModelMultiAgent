A lógica modal consiste em uma linguagem para tratar proposições que necessariamente ocorrem e proposições que possivelmente ocorrem. As proposições dadas como necessárias são aquelas que necessariamente são verdade. Por exemplo; A água sobre 1 atm e entre 0,1 ºC - 99 ºC se apresenta no estado líquido. O conceito de possibilidade é totalmente dependente do conceito de necessidade. Isso pois uma proposição possível é aquela que necessariamente não é falsa \cite{modallogic}. 

A lógica modal é do tipo \textbf{K} e isso significa que nela está condita símbolos $ \sim $ para não, $ \rightarrow$ para "se ... então" e $\Box$ para "Isto é necessário". 

De \textbf{K} e $\Box$, tem-se as seguintes regras;

Sendo que $isTheorem(A,\textbf{K})$ representa "Se A é teorema de \textbf{K}". 

\begin{equation} 
isTheorem(A,\textbf{K}) \rightarrow \Box A
\end{equation}
\label{ktheorema}

\begin{equation} 
 \Box (A \rightarrow B) \rightarrow (\Box A \rightarrow \Box B) 
\end{equation}
\label{boxdist}

O operador $\Diamond$ apresenta o seguinte correspondente semântico; "Isto é possível". A relação entre $\Box$ e $\Diamond$ é dada pela regra que se segue.

\begin{equation} 
 \Diamond A = \sim\Box\sim A
\end{equation}
\label{dianotboxnota}

As relações a seguir apresentam outras regras válidas para essa lógica;

\begin{equation} 
 \Box (A \wedge B)  \rightarrow \Box A \wedge \Box B
\end{equation}
\label{boxand}

\begin{equation} 
 \Box A \vee \Box B \rightarrow \Box (A \vee B)
\end{equation}
\label{boxaor}

\begin{equation} 
 \Box A \rightarrow A
\end{equation}
\label{boxtoa}

\begin{equation} 
    \Box A \rightarrow \Box\Box A
\end{equation}
\label{aboxbox}

\begin{equation} 
    \Diamond A \rightarrow \Box\Diamond A
\end{equation}
\label{diaaboxdiaa}

\begin{equation} 
    \Box\Box...\Box = \Box
\end{equation}
\label{alotbox}

\begin{equation} 
    \Diamond\Diamond...\Diamond = \Diamond
\end{equation}
\label{diamont}

\begin{equation} 
    A \rightarrow \Box\Diamond A
\end{equation}
\label{diamont}