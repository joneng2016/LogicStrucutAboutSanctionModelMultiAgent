A lógica modal consiste em uma linguagem para tratar proposições que necessariamente ocorrem e proposições que possivelmente ocorrem. Os proposições dados como necessarios são aqueles que necessariamente 
são verdade. Por exemplo, A água sobre 1 atm e entre 0,1 ºC - 99 ºC se apresenta no estado líquido. O conceito de possibilidade é totalmente dependente do conceito de necessidade. Isso pois 
uma proposição possível é aquela que necessariamente não é falsa \cite{modallogic}. 

A lógica modal é do tipo \textbf{K} e isso significa que nela está condita simbolos $ \sim $ para não, $ \rightarrow$ para "se ... então" e $\Box$ para "Isto é necessarário para que". 
De \textbf{K} e $\Box$, tem-se as seguintes regras;

Sendo que $isTheorem(A,\textbf{K})$ representa "Se A é teorema de \textbf{K}". 

\begin{equation} 
isTheorem(A,\textbf{K}) \rightarrow \Box A
\end{equation}
\label{ktheorema}