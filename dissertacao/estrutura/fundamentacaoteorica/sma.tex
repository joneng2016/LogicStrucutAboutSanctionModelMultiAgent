Um sistema multiagente(SMA) organizado é aquel constituido por agentes autonomos que interagem visando um propósito em comum tendo como consequência um comportamento global \cite{mosieframework} 
\cite{organiationofmultiagentsystem}. Assim sendo, uma organização com essas características deve ser capaz de manifestar conhecimento em comum, cultura, memória, história, distribuição de atividades 
e a capaciade de distinguir um  agente em espeçifico \cite{organiationofmultiagentsystem}. Deste fato é possível identificar o fenômeno "supra-individual" que implica em um comportamento que existe
além dos comportamentos e atributos particulares no que diz respeito as entidades constituintes do sistemas. 

Uma organização de um sistema multiagente deve conter relações sociais no que tange a agentes, institutos e grupos sociais \cite{organiationofmultiagentsystem}. Ainda sobre isso, uma organização 
\textit{SMA} deve apresentar uma \textit{extensão de um espaço abstrato}. Isso implica uma representação dos seguintes conceitos; estrutra espacial, estrutura temporal, símbolos, semântica e 
capacidade de dedução. Há organizações que não se enquadram em todas essas restrições, contudo são suficientes para tratar o problema dentro de uma perspectiva computacional \cite{organiationofmultiagentsystem}.

\subsection{Conceitos Gerais de uma Organização SMA}

A finalidade desta subseção consiste trabalhar com uma maior riqueza de detalhes todos os conceitos que constituem a ideia de uma organiozação de um sistema multiagente.  
  
\textbf{Divisão em tipos de atividades:} Uma organização não é uniformemente estruturada. Isso, pois as atividades são distribuidas de forma desigual entre as diferentes entidades.
Dentro do ponto de vista fenomenologico as atividades são sujeitas a classificação e ocorrem com diferentes frequências e em diferentes regiões dentro das definições espaciais da organização \cite{organiationofmultiagentsystem}.

\textbf{Integração:} Dentro de uma organização ocorre a presenção de interdependência entre diferentes espaços de ativiades. Essas, por sua vez, estão relacionadas em uma estrutura única definida
dentro de um contexto alinhado e integrado \cite{organiationofmultiagentsystem}.

\textbf{Composição} Uma organnização é composta por elementos menores. No caso dos multiagentes, os elementos atomicos que estruturam a organização são os agentes \cite{organiationofmultiagentsystem}.

\textbf{Estabilidade/Flexibilidade:} Uma organização apresenta padrões de atividades. Esses padrões possum cateristicas que podem ser enquadradas em dois aspectos; estaveis e flexiveis. 
No que tange as características estaveis, essas são constituidas por elementos/processos que definem o padrão em sí mesmo. Em constraste com isso um comportamento flexível acontece quando o 
sistema é submetido a situações incomuns \cite{organiationofmultiagentsystem} \textbf{?}.

\textbf{Coordenação:} Todo sistema é dependente de algum dado recurso. Assim sendo, se faz necessário que esse recurso seja utilizado de forma inteligente a fim de que possa se manter ao longo 
do tempo. Para que isso, se faz necessário que a organização se comporte como uma amplificadora de recursos a fim de que as estruturas operacionais tenham um comportamente cada vez mais organizado 
\cite{selforganization}, \cite{selforganizatioenvoriment}, \cite{defintionselforganization}, \cite{organiationofmultiagentsystem}. Ainda  



\cite{multiagentsystemmodernapproach}
\cite{multiagentsystemwhatis}
\cite{organiationofmultiagentsystem}
\cite{amodelmultiagentsystemdynamicrelationship}
\cite{mosieframework}
\cite{modelingsocialactionforaiagents}
\begin{itemize}
    \item Apresentar definições de SMA
    \item Aprestar conceito de objetivo
    \item Apresentar conceito de papel
    \item Apresentar os conceitos de relações deonticas
\end{itemize}