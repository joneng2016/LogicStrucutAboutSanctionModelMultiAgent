
O primeiro estudo teórico sobre acidentes de trabalho se da no texto \cite{riskoldschool}. Essa pesquisa conclui que os erros em industria não podem ser definidos apenas nas falhas de 
humanos, mas sim como consequência de um comportamento global da instituição. Ainda dentro deste ambito, esse comportamento advem de uma forte pressão tendo em vista eficiência e otimização 
dos processos de produção \cite{riskoldschool} \cite{safety}.

Com base neste entendimento, o estudo \cite{safety} apresenta um \textit{framework} a fim de identificar as redes de causalidade que resultam em acidentes de trabalho. Assim sendo, a gestão de 
segurança se dá com base nos seguintes fatores; 
\begin{itemize}
    \item Políticas; \textit{leis, diretrizes, padrões e regras},
    \item Corporativo; \textit{regras, estratégias, politicas internas, gerenciamento}
    \item Projeto de Equipamentos de Trabalho; \textit{especificação, integração de segurança}
\end{itemize}

O item Política é mais relavante que os intens Corpotativo e Projeto de Equipamentos de Trabalho. Esses dois últimos apresentam a mesma importância para uma estrutura de prevenção de acidentes 
bem sucedida. 

A primeira análise a ser feita diz respeito ao nível Corporativo-Projeto de Equipamentos de Trabalho. Muitas vezes a equipe adota um ativiades paleativas a fim de otimizar os processos 
de produção. Isso envolve assumir níveis de tolerância no que diz respeito ao desempenho e a segurança. Essa situação está dentro do conceito, para o \textit{framework}, de \textit{atividades
limites}, isso pois tratam de situações que trabalham no limiar com os riscos. Assim sendo, as decisões feitas pelo profissional podem resultar muito facilmente em acidentes ou incidentes \cite{safety}. 
Dentro desta perspectiva que se apresenta o ator \textit{BATU} - \textit{Boundary Activities Tolerated during Use} (Atividades Limites Toleradas Durante o Uso).

Existe dois tipos de \textit{BATU} que devem ser verificados tendo como base os processos de trabalho. Esses são; operacional e gerencial (administrativo - termo identificado no texto 
original; \textit{managerial}). Aquele faz referência as atividades relacionadas a melhoria da produtividade com o propósito de resultar em melhorias de produtividade tendo em vista as metas
de produção no que tange a produção, qualidade e segurança. Este diz respeito a decisões administrativas independentes dos processos operacionais mas que os impacta.

Outro conceito presente em \cite{safety} é o de \textit{Boundary Conditions Tolerated by Use} - \textit{BCTU} (Condições Limites Toleradas Durante o Uso). ..


