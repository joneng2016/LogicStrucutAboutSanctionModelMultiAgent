O primeiro estudo teórico sobre acidentes de trabalho se da no texto \cite{riskoldschool}. Essa pesquisa conclui que os erros em industria não podem ser definidos apenas nas falhas de 
humanos, mas sim como consequência de um comportamento global da instituição. Ainda dentro deste âmbito, esse comportamento advêm de uma forte pressão tendo em vista eficiência e otimização dos processos de produção \cite{riskoldschool} \cite{safety}.

Com base neste entendimento, o estudo \cite{safety} apresenta um arcabouço a fim de identificar as redes de causalidade que resultam em acidentes de trabalho. Assim sendo, a gestão de 
segurança se dá com base nos seguintes fatores; 
\begin{itemize}
    \item Políticas; \textit{leis, diretrizes, padrões e regras}
    \item Corporativo; \textit{regras, estratégias, politicas internas, gerenciamento}
    \item Projeto de Equipamentos de Trabalho; \textit{especificação, integração de segurança}
\end{itemize}

O item Política é mais relevante que os itens Corporativo e Projeto de Equipamentos de Trabalho. Esses dois últimos apresentam a mesma importância para uma estrutura de prevenção de acidentes 
bem sucedida. 

A primeira análise a ser feita diz respeito ao nível Corporativo-Projeto de Equipamentos de Trabalho. Muitas vezes a equipe adota atividades paliativas a fim de otimizar os processos de produção. Isso envolve assumir níveis de tolerância no que diz respeito ao desempenho e a segurança. Essa situação está dentro do conceito, para o arcabouço, de \textit{atividades
limites}, isso pois tratam de situações que trabalham no limiar com os riscos. Assim sendo, as decisões feitas pelo profissional podem resultar muito facilmente em acidentes ou incidentes \cite{safety}. 
Dentro desta perspectiva que se apresenta o ator \textit{BATU} - \textit{Boundary Activities Tolerated during Use} (Atividades Limites Toleradas Durante o Uso).

Existe dois tipos de \textit{BATU} que devem ser verificados tendo como base os processos de trabalho. Esses são; operacional e gerencial (administrativo - termo identificado no texto original; \textit{managerial}). Aquele faz referência as atividades relacionadas a melhoria da produtividade com o propósito de resultar em aumento das metas de produção, qualidade e segurança. Este diz respeito a decisões administrativas independentes dos processos operacionais mas que os impactam.

Outro conceito presente em \cite{safety} é o de \textit{Boundary Conditions Tolerated by Use} - \textit{BCTU} (Condições Limites Toleradas Durante o Uso). O termo condição faz referência a uma situação, um estado, circunstâncias externas às quais pessoas ou até mesmo entidades são afetados no que diz respeito a uma certa decisão. Assim sendo, \textit{BCTU} consiste em uma série de elementos e circunstâncias (ambiental, material, humana, produtos) que por conta de sua natureza ou de como se relaciona com as demais entidades e processos apresenta um certo potencial na geração de situações particulares, tendo em vista causas decorrentes de operações dinâmicas. Tanto os \textit{BATUs bem como os BCTUs} não podem ser analisados diretamente, mas devem ser analisados por intermédio das ações e escolhas dos operadores e dos atores que constituem esse trabalho \cite{safety}. 

Existe dois tipos de \textit{BCTU}. O primeiro consiste no \textit{BCTU} interno que se apresenta como uma concepção global de trabalhos e situações no que tange as relações de política da empresa. Nesta concepção, \textit{BCTU} interno faz referência as diferenças hierarquias em termos de nível e decisões centrais. Em contraste com esse ponto, o \textit{BCTU} externo aponta para o projeto da instalação. Como resultado, há o surgimento de quatro derivações, que são; soluções de segurança - funções de segurança (diz respeito as questões que podem fazer com que um dispositivo de segurança venha a falhar), soluções técnicas - requisições de trabalho (quando as soluções técnicas são incompatíveis com as requisições de trabalho), modelo de projeto - modelo de instalação (se da quando a solução final não é ótima ou está degradada quando comparada com a solução inicial) e condições nominais preventivos - condições reais de operação
\cite{safety}.

As relações entre \textit{BATU}s e \textit{BCTU}s são dinâmicas e são dependentes do processo. Para exemplificar, pode-se considerar o seguinte cenário; O projeto de uma máquina de dobra de papel obriga o operador a adotar uma dada posição que o faz assumir riscos para acessar determinados pontos da máquina.  Assim sendo, as escolhas do projeto da máquina (relacionada ao \textit{BCTU}) não levam em consideração todos os aspectos relacionados a dinâmica profissional-máquina fazendo o que o profissional envolvido tenha que atuar dentro de um certo intervalo de tolerância no que diz respeito a segurança profissional \textit{BATU}.
