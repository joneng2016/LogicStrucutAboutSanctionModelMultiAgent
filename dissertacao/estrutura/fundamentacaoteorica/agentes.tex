Não existe uma definição universal para tratar o conceito de agente sendo que esse tópicose encontra em meio a debates e controvérsias. Contudo, existe um entendimento generalizado de que 
um comportamento \textit{autonomo} é o cerne de noção que se tem por agente \cite{whatisagent}. 

Apesar de existir um debate no que tange ao conceito de agentes, a construção de um modelo computacional não pode ser feito sem uma definição. Assim sendo, nesse texto um agente é
 um sistema computacional que está situado em um dado ambiente e que apresenta comportamento autonomo \cite{definitionagent} \cite{whatisagent}. Não apenas isso, mas um agente faz uso
 so seu comportamento autonomo com o propósito de atingir objetivos que a ele é designado \cite{definitionagent} \cite{whatisagent}.

Dentro do contexto presente na definição de agentes, um ambiente apresenta as seguintes propriedades \cite{artificialinteligencemodermapproach} \cite{whatisagent}; 
\begin{itemize}
    \item \textit{Acessibilidade vs Inacessibilidade}; Um ambiente acessível é aquele onde um agente consegue ter informações claras, precisas e atualizadas no que tange a característica do ambiente.
    \item \textit{Determinístico vs Não-Determinístico}; Um comportamento determinístico é aquele onde uma ação possui um efeito claro e garantido, sem incertezas sobre o estado que irá resultar.
    \item \textit{Episódico vs Não-Episódico}; Um ambiente tende a ser o mais episódico possível tanto quando o desempenho do agente estiver associado a um episódio discreto e específico no ambiente.
    \item \textit{Estático vs Dinâmico}; Um ambiente é estático se não houver outros processos em parapelo aos eventos associados ao agente.
    \item \textit{Discreto vs Contínuo}; Um ambiente é discreto se existe um número finito de ações e percepções. 
\end{itemize}


Outro aspecto a ser considerado em um agente implica o comportamento autonomo. Uma entidade que possui essa natureza tem a capacidade agir por sí mesmo. Essa entidade não precisa de nenhuma outra entidade externa (ex. como um ser humano)
para realizar decisões \cite{whatisagent} \cite{definitionagent}. 
