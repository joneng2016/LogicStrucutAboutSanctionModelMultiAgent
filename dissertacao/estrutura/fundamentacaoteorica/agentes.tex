Não existe uma definição universal para tratar o conceito de agente sendo que esse tópico se encontra em meio a debates e controvérsias. Contudo, existe um entendimento generalizado de que um comportamento \textit{autônomo} é o cerne de noção que se tem por agente \cite{whatisagent}. Apesar disso, a construção de um modelo computacional não pode ser feita sem uma definição. Assim sendo, nesse texto um agente é um sistema computacional que está situado em um dado ambiente e que apresenta comportamento autônomo \cite{whatisagent}. Além disso um agente faz uso do seu comportamento autônomo com o propósito de atingir objetivos que a ele é designado \cite{whatisagent}.

Como a definição de agente faz uso do conceito de ambiente, não é possível ter uma compreensão plena daquele sem considerar aspectos deste. Assim sendo, o termo ambiente  é aquilo que apresenta as propriedades definidas na lista a seguir \cite{artificialinteligencemodermapproach} \cite{whatisagent}: 
\begin{itemize}
    \item \textit{Acessibilidade vs Inacessibilidade}: Um ambiente acessível é aquele em que um agente consegue ter informações claras, precisas e atualizadas no que tange a característica do ambiente.
    \item \textit{Determinístico vs Não-Determinístico}: Um comportamento determinístico é aquele em que uma ação possui um efeito claro e garantido, sem incertezas sobre o estado que irá resultar.
    \item \textit{Episódico vs Não-Episódico}: Um ambiente tende a ser o mais episódico possível tanto quanto o desempenho do agente estiver associado a um episódio discreto e específico no ambiente.
    \item \textit{Estático vs Dinâmico}: Um ambiente é estático se não houver outros processos em parapelo aos eventos associados ao agente.
    \item \textit{Discreto vs Contínuo}: Um ambiente é discreto se existeir um número finito de ações e percepções. 
\end{itemize}

Outro aspecto que está presente na definição de agente é o termo autônomo. Esse termo pode ser compreendido pela seguinte proposição: Uma entidade que possui essa natureza é uma entidade que tem a capacidade de agir por si mesma. Essa entidade não precisa de nenhum outro fator externo (ex. ser humano) para efetuar as suas decisões \cite{whatisagent}.

Há uma série de exemplos que se enquadram dentro dessa definição. Um termostato é um sistema de controle que está em um dado ambiente, como um quarto ou uma sala, \cite{whatisagent}, que gera dois sinais de saída, (um desses sinais indica que a temperatura está ou baixa ou alta demais dependendo da aplicação, e o outro, demonstra que a temperatura está no nível aceitável). O comportamento autônomo do termostato é baseado nas seguintes regras:

\begin{itemize}
    \item Se a temperatura estiver abaixo do nível de temperatura definido, então ligar o atuador.
    \item Se a temperatura estiver dentro do nível estabelecido, então desligar o atuador.
\end{itemize}

Outro exemplo de agentes consiste nos programas \textit{Daemons} em sistemas \textit{UNIX}. Esses algoritmos trabalham em segundo plano e monitoram um dado ambiente de \textit{software}. Com base em certas regras, na ocorrência de um certo evento no ambiente, esses programas realizam uma determinada atuação \cite{whatisagent}.   

Os exemplos presentes neste texto são apropriados dentro do conceito de agentes. Contudo, esses exemplos não estão de acordo com a definição de agentes inteligentes \cite{whatisagent}. Uma entidade que se enquadra dentro das características de um agente inteligente deve necessariamente respeitar a definição já apresentada e deve apresentar as seguintes propriedades: reatividade (capaz de perceber as mudanças
que ocorrem no ambiente e responder a elas de maneira apropriada no que condiz aos objetivos do agente), proatividade (apresenta comportamento orientado a objetivos sendo que o agente toma as decisões a fim de satisfazer os objetivos em interesse) e habilidades sociais (capacidade de interagir com outros agentes a fim de poder satisfazer os próprios objetivos) \cite{whatisagent} \cite{artificialinteligencemodermapproach}.

Os agentes podem ser definidos em categorias. Uma dessas consiste nos agentes que são puramente reativos. Esses agentes tomam decisões considerando apenas informações que estão no instante presente. Por consequência, o comportamento deles ocorre por respostas diretas ao ambiente \cite{whatisagent}. 

Os agentes que possuem estados estão contidos em outra categoria. Esses agentes possuem uma dada estrutura interna de dados que são consideradas em tomadas de decisão \cite{whatisagent}.

Uma outra maneira de analisar um agente se dá por meio das arquiteturas e modelos disponíveis para representar seus estados internos. Para o propósito do estudo que está sendo apresentando neste texto, é o suficiente considerar de forma sucinta quatro dessas arquiteturas. A primeira consiste nos agentes baseados em lógica que realizam deduções em função de uma decisão \cite{logicagent}, a segunda arquitetura consiste nos agentes reativos que tomam decisões com base em um mapeamento de uma certa situação em uma  ação \cite{reactiveagent}, a terceira arquitetura é \textit{BDI} cujo comportamento ocorre da manipulação de estruturas de dados que representam crenças, desejos e intenções do agente \cite{bdi} e a quarta arquitetura consiste em uma estrutura em camadas onde a tomada de decisão acontece por intermédio de diversos níveis de abstração a cerca do ambiente \cite{layeragent} \cite{whatisagent}.  

Existe um modelo de agentes que faz uso da arquitetura BDI com propósito de representar situações de risco. Esse modelo é denominado por \textit{MASVERP} que define o agente como tendo habilidades e objetivos a serem atingidos. Não só isso, mas o agente também pertence a certo círculo social. O autor deste estudo implementou o modelo \textit{COCOM} que define parâmetros físicos e cognitivos a fim de que os agentes articulem suas ações no que tange a execução de objetivos \cite{mavesp}. Esse modelo considera situações que delimitam quando um agente não tem condições de desempenhar uma certa atividade. Essas são: fome, sede, cansaço físico, carga cognitiva, cansaço em relação a vigilância, estresse, motivação para determinar agitação, motivação e condições regulares \cite{mavesp}. 

O \textit{MASVERP} incorpora o conceito de \textit{BTCU} (será tratado na seção \ref{risksec}). Em termos genéricos, \textit{BTCU} consiste em condições limites que são aceitas para que um agente possa realizar a atividade. No \textit{MASVERP} a decisão do agente no que tange a executar ou não uma atividade é parametrizada com base no \textit{BTCU}. 