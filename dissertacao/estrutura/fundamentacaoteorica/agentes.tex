Não existe uma definição universal para tratar o conceito de agente sendo que esse tópico se encontra em meio a debates e controvérsias. Contudo, existe um entendimento generalizado de que 
um comportamento \textit{autonomo} é o cerne de noção que se tem por agente \cite{whatisagent}.  Apesar disso, a construção de um modelo computacional não pode ser feito sem uma definição. Assim sendo, nesse texto um agente é
 um sistema computacional que está situado em um dado ambiente e que apresenta comportamento autonomo \cite{whatisagent} \cite{whatisagent}. Não apenas isso, mas um agente faz uso
 so seu comportamento autonomo com o propósito de atingir objetivos que a ele é designado \cite{whatisagent} \cite{whatisagent}.


Como a definição de agente faz uso do conceito de ambiente, não é possível tratar esse tópico de forma consiste sem considerar a semântica deste termo. 
Assim sendo, ambiente é aquilo que apresenta as proprieades listas as seguir \cite{artificialinteligencemodermapproach} \cite{whatisagent}; 
\begin{itemize}
    \item \textit{Acessibilidade vs Inacessibilidade}; Um ambiente acessível é aquele onde um agente consegue ter informações claras, precisas e atualizadas no que tange a característica do ambiente.
    \item \textit{Determinístico vs Não-Determinístico}; Um comportamento determinístico é aquele onde uma ação possui um efeito claro e garantido, sem incertezas sobre o estado que irá resultar.
    \item \textit{Episódico vs Não-Episódico}; Um ambiente tende a ser o mais episódico possível tanto quando o desempenho do agente estiver associado a um episódio discreto e específico no ambiente.
    \item \textit{Estático vs Dinâmico}; Um ambiente é estático se não houver outros processos em parapelo aos eventos associados ao agente.
    \item \textit{Discreto vs Contínuo}; Um ambiente é discreto se existe um número finito de ações e percepções. 
\end{itemize}

Outro aspecto que está contido na definição consiste no conceito expresso pelo termo autonomo. Assim sendo, a pesquisa retratada neste texto trabalha esse termo
da seguinte forma; Uma entidade que possui essa natureza tem a capacidade agir por sí mesmo. Essa entidade não precisa de nenhuma outra entidade externa 
(ex. ser humano) para realizar decisões \cite{whatisagent} \cite{whatisagent}.

Há uma série de exemplos que se enquadram dentro dessa deinição. Sistemas de controle é um desses exemplos. Um \textit{termostato} é um sistema de controle que está em um dado ambiente 
(como um quarto ou uma sala) \cite{whatisagent}, gera dois sinais de saída (um desses sinais indica que a temperatura está baixa demais (ou alta demais dependendo da aplicação) e o outro sinal
demonstra que a temperatura está no nível aceitável). O termostato tem o seu comportamento autonomo baseado em duas regras \cite{whatisagent}:

\begin{itemize}
    \item Se a temperatura estiver abaixo (ou acima) do nível de temperatura definido, então ligar o atuador.
    \item Se a temperatura estiver dentro do nível estabelecido, então desligar o atuador.
\end{itemize}

Outro exemplo de agentes consiste nos programas \textit{Daemons} em sistemas \textit{UNIX}. Esses algoritmos trabalham em segundo plano e monitoram um dado ambiente de \textit{software}. Com base
em certas regras, na ocorrência de um dado evento no ambiente, esses programas realizam uma dada atuação \cite{whatisagent}.   

Os exemplos presentes neste texto são apropriados dentro do conceito de agentes. Contudo, esses exemplos não ase equandram dentro da denição de agentes inteligentes \cite{whatisagent}. Uma entidade que se enquadra
dentro das caracterísiticas de um agente inteligente deve necessariamente respeitar a definição já apresenta e deve apresentar as seguintes propriedades; reatividade (capaz de perceber as mudanças
que ocorrem no ambiente e responder a delas de maneira apropriada no que condiz aos objetivos do agente), pro-atividade (apresenta comportamento orientado a objetivos sendo que o agente toma as decisões
a fim de satisfazer os objetivos em interesse) e habilidades sociais (capacidade de interagir com outros agentes (e possivelmente humanos) a fim de poder satisfazer os próprios objetivos) \cite{whatisagent} \cite{artificialinteligencemodermapproach}.

Os agentes podem ser definidos em categorias. Um desses são agentes puramente reativos. Esses agentes tomam decisões considerando apenas informações que estão no instante presente. Por consequência, 
o comportamento deles ocorre por respostas diretas ao ambiente \cite{whatisagent}. 

Outra categoria de agentes são aqueles que possuem estados. Essas entidades possum uma dada estrutura de dados internar que são considerados quando agente toma uma uma certa decisão \cite{whatisagent}.

Uma outra maneira de analisar os agentes se dá por meio das arquiteturas e modelos disponiveis para representar os agentes (tomada de decisão, estado interno). Para o propósito do estudo que está
sendo apresentando neste texto, é o suficiente considerar de forma sucinta quatros dessas arquiteturas. A primeira consiste nos agentes baseados em lógica onde os agentes realizam deduções lógicas
para tomar uma decisão \cite{logicagent}, a segunda arquitetura consiste nos agentes reativos que tomam decisões com base em um dado mapeamento de uma certa situação em uma dada ação \cite{reactiveagent}, 
a terceira arquitetura é \textit{BDI} cuja decisão ocorrem da manipulação de estruturas de dados que representam crenças, desejos e intenções do agente \cite{bdi} e a quarta arquitetura consiste em uma estrutura 
em camadas onde a tomada de decisão acontecem por intermédio de diversas camadas abstração a cerca do ambiente \cite{layeragent} \cite{whatisagent}.  

