Diversas pessoas são submetidas a algum tipo de risco ao executar atividades profissionais relacionadas à área da eletricidade, petroquímica, transportes e outras correlatas. Esse estudo sintetiza, constrói e avalia um modelo conceitual em função de um estudo de caso e de modelos computacionais com a finalidade de tratar essa situação.  A área de estudo desta pesquisa está relacionada com Agentes, Artefatos, Sistemas Multiagentes, Normas, Riscos e Lógica Modal. Pode-se entender como agente tudo aquilo que possui comportamento autônomo. Os conceitos presentes no modelo Cartago foram usados para tratar a concepção de Artefatos. No que tange a Sistemas Multiagentes, os autores optaram por modelos como MOISE+. Como resultado foi possível obter um modelo conceitual, formalizado em Lógica de Primeira Ordem, capaz de tratar diversos cenários atrelados a atividades profissionais de risco. Essa representação foi aplicada a um estudo de caso permitindo compreender o estado do problema. Não apenas isso, mas essa representação permitiu realizar uma análise comparativa (em relação a cenários de acidentes) entre cinco modelos, MOISE+, Modelo de Agentes Normativos de DASTANI,V3S e NORMMAS.