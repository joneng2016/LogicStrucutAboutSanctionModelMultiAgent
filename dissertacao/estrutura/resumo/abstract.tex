\textit{This dissertation presents part of the results of the R&D project code PD-06491-0299 / 2013 developed by the Lactec and UFPR Institutes for COPEL Geração e Transmissão SA under the Electric Energy Sector Research and Technological Development Program regulated by the National Agency Electric Energy (ANEEL).}


Many people are at risk when performing professional activities related to electricity, petrochemicals, transportation and others. Based on an interview with a Maintenance Engineer, on-site participation in a maintenance activity in the area of ​​electricity transmission and exploratory literature review, this study synthesizes the information collected, builds and assesses a conceptual model for the construction of computer simulation systems. hazards and accidents for various purposes, such as operator training or study of risks associated with a labor activity. The constructed conceptual model is based on the concepts of agents, artifacts, multi-agent systems, norms and risks / consequences. Activities can be expressed in terms of goals with prerequisites and effects. The central concern is risk modeling and its consequences. Thus, prerequisites for activities that are not present at the time agents decide to perform them lead to violations that may have consequences for the current and / or subsequent goals (e.g. stopping) as well as for the agents involved (e.g. accident). The model, formalized in First Order Logic, was implemented in PROLOG and applied to a case study allowing to evaluate if the inferences produced corresponded to the expected ones as well as to assess the state of the problem modeled. This representation made it possible to perform a comparative analysis (in relation to accident scenarios modeling) between four models, MOISE +, DASTANI Normative Agents Model, V3S and NORMS. In conclusion, the model is a synthesis of several existing models bringing together in one the relevant characteristics for this type of modeling. Besides, if it was able to represent several scenarios related to risky professional activities, in particular, it was able to produce coherent results with the expected ones in the case study.