Several people are at risk when performing professional activities related to electricity, petrochemicals, transportation and other related activities. This study synthesizes, builds and evaluates a conceptual model based on a case study and computational models in order to address this situation. The area of ​​study of this research is related to Agents, Artifacts, Multiagent Systems, Standards, Risks and Modal Logic. The agent is everything that has autonomous behavior. The concepts present in the Cartago model were used to address the design of Artifacts. The authors opted for models like MOISE + to define Multiagent Systems. As a result, it was possible to obtain a conceptual model, formalized in First Order Logic, capable of dealing with various scenarios linked to risky professional activities. This representation was applied to a case study allowing to understand the state of the problem. Not only that, but this representation allowed a comparative analysis (in relation to accident scenarios) between five models, MOISE +, DASTANI Normative Agents Model, V3S and NORMMAS.