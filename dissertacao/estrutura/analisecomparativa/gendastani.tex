\subsection{Modelo DASTANI}

A Tabela \ref{dastanitb1} apresenta uma comparação entre o \textit{DASTANI+} e o modelo proposto neste estudo.

\begin{center}
\begin{longtable}[H]{|l|p{0.3\linewidth}|p{0.3\linewidth}|}
\hline			
\textbf{Atributos}
& 
\textbf{DASTANI}
& 
\textbf{Modelo deste Estudo} 
\\ \hline
Finalidade
&
Modelo de \textit{SMA} que leva em consideração questões atreladas à normas, sanções e violações 
&
Representar cenários de acidentes 
\\ \hline
Agentes
&
Não define os estados mentais do agente
&
Não define os estados mentais do agente
\\ \hline
Generalização
&
Trata de cenários onde se tem o interesse de representar agentes, normas, violações e sanções 
&
Trata de cenários específicos para acidentes com a finalidade de identificar as causas destes
\\ \hline

\newpage

\hline

\textbf{Atributos}
& 
\textbf{DASTANI}
& 
\textbf{Modelo deste Estudo}

\\ \hline

Normas 
&
Presença de regras que permitem expressar violações e sanções para os mais diversos casos possíveis
&
Presença de regras que permitem expressar violações e sanções para cenários de acidentes
\\ \hline
Estrutura da Linguagem
& 
Linguagem de programação baseada em Lógica de Predicados de Primeira Ordem 
&
Teoria de Conjuntos e Lógica de Predicados de Primeira Ordem
\\ \hline
Raciocínios
& 
Permite representar raciocínios que considera agentes cometendo ou não violações e tendo que lidar com as respectivas sanções. Esses raciocínios levam em consideração conceitos como eventos e ações (estados que geram a transição de eventos)
& 
Presença de regras que possibilitam reproduzir cenários de acidentes entre os agentes. Não apenas isso, mas faz parte da estrutura dessas regras questões atreladas às normas, violações e sanções. Esses raciocínios também representam cenários onde agentes sofrem consequências negativas mesmo não sendo responsáveis pelas causas dos acidentes e fazem isso por levar em consideração questões possibilísticas e influências entre relações.
\\ \hline
\caption{Comparação entre o \textit{DASTANI+} e o modelo proposto neste estudo}
\label{dastanitb1}
\end{longtable}
\end{center}