\section{Modelo para Agentes Normativos de \textit{Dastani}}

O predicado $thereIsRelation(r_l,e_i,e_k)$ pode, em partes, ser especificado pelo modelo do \textit{Dastani}. Isso pode ser feito por trabalhar as artefatos, condições e relações como \textit{Facts}. Os agentes não podem entrar como \textit{Facts} porque existe uma categoria no modelo específico apenas para tratar esse tipo de entidade. Os pesquisadores entendem que esse predicado não pode ser representado na estrutura do \textit{Dastani} porque não existe correspondente semântico para isso. Uma primeira tentativa reside em considera-lo como pertencente ao conjunto \textit{Effects}. Contudo, esse conjunto é destinado a elementos que denotam característica 
de transição de estado o que não é a finalidade não somente deste como de todos os outros predicados do modelo conceitual. Assim sendo, a transição de estados em  \textit{Dastani} é o que corresponde, em partes, as regras na representação conceitual deste estudo. 

Os pesquisadores entendem que esse aracabouço não é formalizado para condicionar a idéia de objetivos e papeis ($\rho$). Assim sendo, todos os predicados que denotam o conceito de $objetivo$ em sua respectiva estrutura devem ser reformulados. Uma possibilidade de fazer isso reside em pensar todos os \textit{Facts} e \textit{Effects} a retratar as mesmas cadeias de causalidade que são traduzidas pelos objetivos. Os pesquisadores entendem que isso é possível, porem geram um aumento expressivo em termos de complexidade de representação. 

Dada as considerações da reformulação dos objetivos, é possível identificar correspondentes diretos dos $<i-liretals>$  e $<b-liretals>$ (que respectivamente depresentam violações e consequências) aos predicados que denotam violações bem como a parte dos predicados que apresentam as consequências. 

Algumas consequências das violações apresentam grandes complexidade de especificação em termos de \textit{Dastani}. Essas consequências são todas atreladas a $possibilityHappensBadEvenet(r_l)$ e $affectOtherRelations(r_k,r_n)$ pois esse arcabouço não é preparado para considerar as sucessões de eventos que trazem a ocorrência de aleatoriedades, logo possibilidades. 

Em termos de regras, considerando a problemática envolvendo os objetivos, é possível constatar que o \textit{Dastani} apresenta estrutura suficiente para representar as relações de inferência denotadas pelas regras \ref{violationcondition}, \ref{violationrelation}, \ref{violationentity} como \textit{Count-As Rules}. Isso pois no modelo conceitual essas regras definem as condições que ocasionam uma violação e o mesmo acontece com as \textit{Count-As Rules}. As regras presentes em \ref{consviolationcondition} e em \ref{consviolationrelation} podem ser representadas por \text{Sanction Rules} pois em ambas condições ocorre uma definição das consequências de uma violação. 