\subsection{Modelo DASTANI+}

As tables a seguir exibiem uma comparação entre o \textit{DASTANI+} e o modelo proposto neste estudo.

\begin{table}[H]
\centering
\begin{tabular}{|l|p{0.3\linewidth}|p{0.3\linewidth}|}
\hline			
\textbf{Atributos}
& 
\textbf{DASTANI}
& 
\textbf{Modelo deste Estudo} 
\\ \hline
Finalidade
&
Modelo de \textit{SMA} que leva em consideração questões atreladas a normas, sanções e violações. 
&
Representar cenários de Acidentes 
\\ \hline
Agentes
&
Não define os estados mentais do agente
&
Não define os estados mentais do agente
\\ \hline
Generalização
&
Trata de cenários onde se tem o interesse de representar agentes, normas, violações e sanções. 
&
Trata de cenários específicos para acidentes onde se deseja verificar culpados. 
\\ \hline
\end{tabular}
\end{table}


\begin{table}[H]
\centering
\begin{tabular}{|l|p{0.3\linewidth}|p{0.3\linewidth}|}
\hline			
\textbf{Atributos}
& 
\textbf{DASTANI}
& 
\textbf{Modelo deste Estudo} 
\\ \hline
Representação de Objetivos
& 
Não possui estrutura para representar objetivos
&
Possui uma estrutura simples para representar objetivos. Essa estrutura considera apenas questões de pré-requisitos. 
\\ \hline
Normas 
&
Presença de regras que permite expressar violações e sanções para os mais diversos casos possíveis. 
&
Presença de regras que permite expressar violações e sanções para cenários de acientes.
\\ \hline
Estrutura da Linguagem
& 
Usa linguagem EBNF 
&
Teoria de Conjuntos e Lógica de Predicados 
\\ \hline
Raciocínios
& 
Permite representar raciocínios que considera agentes comentendo ou não violações e tendo que lidar com as respectivas sanções. Esses raciocínios levam em consideração conceitos como eventos, ações (estados que geram a transição de eventos) e efetios (considerações generalizadas dos eventos e ações)
&
Possui regras de inferências que possibilitam reproduzir cenários de acidentes entre os agentes. Não apenas isso, mas faz parte da estrutura dessas regras questões atreladas a normas, violações e sanções. Esses raciocínios também representam cenários onde agentes sofrem consequências negativas mesmo não sendo responsáveis causas dos acidentes e fazem isso por levar em consideração questões possibilísticas e influências entre relações.
\\ \hline
\end{tabular}
\end{table}




