A discussão dos resultados que estão expostos nas sub-seções \ref{mods}, \ref{predic}, \ref{regras} foi feita na própria apresentação dos mesmos. Isso se deve a natureza desses resultados , não é possível realizar a exposição deles sem discutir os fundamentos conceituais que justifiquem a existência dos mesmos. Essa situação não se aplica no texto presente em \ref{studycase} e em \ref{rac} onde os resultados estão apenas expostos mas não foram discutidos. O mesmo acontece com a Metodologia usada para chegar nesses resultados, foi exposta mas não foi discutida. O texto a seguir fará uma discussão desses elementos apresentando as principais dificuldades que foram encontradas na realização desse estudo. 

\subsection{Considerações sobre Critérios Metodológicos ao estudo de caso} \label{conscritmetcasoestudo}

A primeira fase consistiu em descrever a manutenção em termos de objetivos que as vezes são organizados em série e as vezes em paralelo. Não houve grandes dificuldades para fazer isso, pois essa atividade é claramente composta de subatividades. O que foi um ponto relativamente complicado de se verificar nesse estudo, é que os profissionais não precisam executar os objetivos na estrutura em que o modelo foi apresentado. Inclusive, muitas vezes os técnicos planejam a manutenção de um jeito e ao chegar no ambiente de execução eles mudam o encadeamento dos objetivos. Há um número finito e relativamente pequeno (é difícil definir um número, mas as observações permitem concluir que este número está na ordem de 10 formas diferentes) sobre como esses objetivos podem ser organizados e isso ameniza a falta de previsibilidade de como a manutenção será realizada. 

O problema da organização dos objetivos pode ser resolvida de três maneiras diferentes. Em uma delas o engenheiro de manutenção modela o problema para todos os cenários possíveis. Portanto, se houver 10 formas diferentes de organizar esses objetivos, o engenheiro deverá refletir acerca dessas 10 formas. Outra forma consiste em definir todas as relações possíveis que o predicado $nextGoal(g_i,g_j)$ permite em uma única estrutura. Nessa implementação do modelo, o agente por intermédio dos seus estados internos, escolhe a qual objetivo ele deverá tentar alcançar. Essa questão não foi levada em consideração no estudo de caso em análise porque o autor estava interessados em realizar a análise necessária sobre a possibilidade de usar este modelo em um problema real. É possível argumentar que aquele arranjo de objetivos não é o único possível, contudo não deixa de ser um arranjo real e que pode servir de referência aos profissionais. A terceira solução a ser considerara consiste no estudo e uso de algoritmos de \textit{Partial Order Planning} aplicados a essa problemática \cite{planning}. 

O autor desse estudo não sabe afirmar se o arranjo dos objetivos interferem no predicado $affectsRels(r_k,r_n)$. Para que isso seja analisado se faz necessário aplicar esse modelo para diversas situações diferentes onde todas devem apresentar a problemática do arranjo de objetivos. Se em uma dessas situações a especificação do predicado $affectsRels(r_k,r_n)$ mudar, então a proposição - o arranjo de objetivos que afeta o predicado $affectsRels(r_k,r_n)$) - deve ser tida como verdadeira, contudo se não mudar, não é possível afirmar que essa proposição é falsa. 

A utilização do conceito de papel e da relação entre o agente e o seu papel foram muito adequados para as análises desse modelo. Isso se deu por observar como ocorre a distribuição dos objetivos aos agentes. Isso, pois na manutenção em linha viva todos os profissionais são tidos como executores, contudo existe uma distribuição de tarefas tendo em vista o conhecimento e a experiência de cada profissional ali envolvido. Portanto, para enquadrar essa questão nos moldes do modelo em análise se fez necessário encontrar um padrão de como os objetivos são distribuídos em função das atividades dos agentes. Com base nisso o autor concluiu o que está exposto na Tabela \ref{agentsroles}.

Os conceitos de artefato e de relação foram adequados para esse estudo de caso não tendo a necessidade de definir nenhuma outra abordagem para isso. Todo o rol de ferramentas e de equipamentos foram definidos como artefatos. A fim de tornar a modelagem mais expressiva, o autor desta dissertação criou subconjuntos de artefatos definindo um apenas para ferramentas e outro apenas para equipamentos. Isso não foi feito para não induzir os leitores desse estudo ao erro por entender que essa divisão pertence a estrutura conceitual do modelo propriamente dito. Uma taxonomia dessas seria adequada apenas para esse caso em estudo, caso contrário diminuiria o poder de generalização do modelo. 

Os conceitos de condição, risco e consequência bem como predicado $hasRisk(c_k,risk_k,cs_m)$ cujas instâncias e o relacionamento do predicado para o estudo de caso são apresentados na Tabela \ref{condition} foram muito apropriados tanto no contexto do modelo em si como também em aplicação ao caso de estudo. Isso se deve ao fato de que uma condição não é um agente e não é um artefato mas é algo que está presente no meio da atividade e interfere com grande intensidade no andamento dos processos, portanto desconsiderar esse conceito ou compactá-lo como parte de outras estruturas implicaria uma representação míope da realidade. O autor entende que essas condições são o suficiente para poder realizar a representação desse modelo. As relações entre condição, risco e consequência foram apropriadas para representar diversos cenários dentro do estudo de caso. 

Uma consideraçao que deve ser feita sobre os conceitos de risco e consequência (cujas instâncias para o estudo de caso estão apresentadas nas Tabelas \ref{condition}, \ref{relationEntEnt1}) em relação ao estudo de caso é a de que foi considerado apenas um único risco, que é o de ser eletrocutado e uma única consequência, que é a morte. Contudo, há considerações que devem ser feitas no que tange a realidade, pois essa atividade exibe outros riscos tais como: queda, animais peçonhentos, queimadura entre outros que, assim como eletrocutado, podem apresentar outras consequências além da morte. Esses riscos a mais não foram considerados no caso em estudo porque os raciocínios exibidos na subseção \ref{rac} puderam ser feitos sem a necessidade deles. Outro ponto que corroborou com isso consiste no fato de que o autor estava interessados em obter primeiramente uma versão mais simples do modelo para então, se necessário, torná-lo mais complexo. Isso implica em realizar algumas escolhas pragmáticas e uma delas consiste na verificação de qual risco é o mais importante e o mais temeroso na atividade. A análise com os profissionais mostram que o risco de ser eletrocutado é o principal e é mais preocupante ao executar uma atividade de manutenção em linha viva. Outro ponto reside na verificação das consequências desse risco o que remete a uma pergunta: Um profissional de linha viva ao executar manutenção em uma subestação de energia pode se envolver em um acidente em que ele é eletrocutado, e ainda sobreviver? A resposta a essa pergunta é sim, porém muito improvável. Descargas de equipamentos que operam a 69 kV 1500 kVA  (o que é relativamente baixo) costumam matar o profissional eletrocutado mesmo que os disjuntores atuem na ordem de milissegundos. Portanto, outras consequências além da morte tal como: queimaduras e perda de membros, contudo na grande maioria dos casos o profissional recairá no óbito. 

O predicado $possEntityRel(r_l,e_i,e_k)$, em que as instâncias para o estudo de caso são apresentada na tabela \ref{relationEntEnt1}, define como se dá a relação entre duas entidades. Essa estrutura se tornou muito útil para fazer diversos raciocínios interessantes que estão presentes nas regras. Portanto o autor desta dissertação conclui que ela foi adequada, necessária e importante para essa representação e para esse estudo de caso, contudo, ela tornou a especificação da modelagem um processo muito custoso, porque o modelador teve de refletir em todas as relações possíveis que são executadas na atividade e depois disso, teve que ver quais relações se enquadravam em cada objetivo. Esse custo também está presente nos raciocínios que devem ser feitos pois dependendo da situação há uma série de relações que devem ser avaliadas. Essa questão nos permite refletir sobre a viabilidade de um modelo assim, para situações onde o número de artefatos bem como o número de relações entre esses artefatos tendem ao infinito. Contudo, o fato do modelador ter que refletir sobre todas as relações bem como seus respectivos riscos, permite a realização de uma análise muito mais profunda da atividade e de como a segurança dos profissionais pode ser afetada de situação para situação. 

O predicado $affectsRels(r_k,r_n)$ (cujas instâncias dos conceitos bem como a relação está presente na Tabela \ref{relation1}) agregou dificuldades tanto na concepção como na aplicação ao estudo de caso. Houve muitas tentativas de resolver essa questão sem ter que abstrair tanto quanto esse predicado faz. Contudo, realizar um mapeamento minucioso de como se dão as atividades, resulta em uma carga de trabalho relativamente custosa e que pode apresentar diversas fragilidades no que tange à consistência lógica (ou seja, um sistema que se contradiz). Portanto, em uma primeira abordagem, admitir que a não execução (ou a má execução) de uma relação afeta negativamente outra relação implica uma visão pragmática e simples para resolver o problema onde um eletricista se envolve em um acidente sobre o qual ele não tem responsabilidade alguma. O autor desse estudo admite que esse é um ponto do modelo a ser melhorado a fim de se obter representações consistentes, expressivas e em com relativo baixo custo de modelagem. 

Os predicados $requiresCirc(circ_n,g_m)$, $requiresEntity(goal_i, e_j)$, $instanceOfRel(circ_n)$ e $instanceOfCond(circ_n)$ (em que as instâncias para esses predicados estão presentes em \ref{relationsgroup1} e \ref{entitygoals}) apresentam a especificação dos relacionamentos entre entidades, relações e condições com os objetivos. O lado positivo dessa abordagem é a possibilidade de ter uma descrição lógico-formal da manutenção altamente detalhada. Isso permite a equipe responsável pela manutenção avaliar o problema com mais profundidade e, portanto, tomar decisões mais eficientes. O problema dessa abordagem reside na mesma situação atrelada à Tabela \ref{relationEntEnt1} que consiste em um processo altamente custoso em termos de tempo e de trabalho a fim de especificar as relações, entidades e condições com os objetivos. 

\subsection{Considerações sobre Critérios Metodológicos ao Raciocínios} \label{conscritmetrac}

Os raciocínios feitos sobre o modelo são de crucial importância para definir a eficácia desse projeto pois é com base nisso que se torna possível avaliar o quão efetivo vem a ser essa representação. Os raciocínios 1 e 5, dados respectivamente pelas subsubseções \ref{raciocinio1} e \ref{raciocinio5} apresentam o problema com bastante expressividade. Tendo em vista o fato de que a Glicerina  é um composto químico relevante para manter o isolamento da parte não condutiva do bastão universal, esquecer de realizar isso gera um potencial acidente de ser eletrocutado em todas as outras situações onde o bastão será usado (a não ser nas situações onde o bastão universal não será usado em condutores energizados). Tanto os predicados como as regras que estão atreladas a violação de relação e suas respectivas consequências representaram essa condição com sucesso. Nesse caso não aconteceu nenhuma sanção sobre o agente 4, portanto nem toda violação de relação gera necessariamente uma sanção. O predicado $possOfNegConseqFor$ conseguiu trazer com êxito a sensação de possibilidade que existe em fenômeno desse gênero. 

Quando o autor concebeu esse modelo, considerou a possibilidade de trabalhar problemas dos raciocínios 1 e 5 por meio do conceito de \textit{Probabilidade}. Isso é interessante porque um predicado que consegue expressar fenômenos estatísticos válidos com excelência, apresenta possibilidades de aplicações extremamente elevadas. Contudo, trabalhar com probabilidade resulta em diversas complicações de modelagem. Uma dessas complicações consiste na coleta de dados de uma amostragem significativa de uma população de acidentes e na análise estatística adequada para descrever a probabilidade de um acidente acontecer. Entretanto, isso não é o suficiente pois essa probabilidade é condicionada a ocorrência de uma determinada relação. O raciocínio 1, por exemplo, demonstra que a não execução de $relPanoGlicerina$ resulta na possibilidade de ocorrer um acidente em $relBastaoGarraGondutor$. Se o autor estivesse trabalhando com o conceito de probabilidade, então é necessário desenvolver técnicas que verificam a probabilidade de acontecer algo ruim na relação $relBastaoGarraGondutor$ para o caso da relação $relPanoGlicerina$ não ser efetivada com sucesso. Contudo, se a não execução de uma outra relação também afetar $relBastaoGarraGondutor$, então também se faz necessário encontrar essa outra probabilidade. Além de aumentar a complexidade desse modelo, abre diversas indagações no que tange a como fazer isso, o que pode ser um potencial campo de investigação científica. Com a finalidade de viabilizar uma primeira versão do modelo, o autor optou por usar o conceito de possibilidade em vez de probabilidade. Apesar de diminuir a expressividade do modelo no que tange a questão que existe um componente sobre aleatoriedade, isso simplifica o processo de especificação, facilita o desenvolvimento de raciocínios e evita que o modelo seja estruturado sob proposições falsas (por exemplo, definir uma probabilidade para uma condição de mundo que não é precisamente valorada). 

O vocabulário definido neste modelo foi apropriado para representar a condição de mundo presente no Raciocínio 2 que está exposto na subsubseção \ref{raciocinio2}. Em uma situação onde não há um pano para poder limpar todas as ferramentas, a manutenção é interrompida e essa situação ficou claramente representada por esse raciocínio onde a geração da violação de entidades corresponde a finalização da manutenção. Há a possibilidade de existir um cenário onde os profissionais criam algum tipo de técnica alternativa para poder transpassar a falta de algum artefato, inclusive se esse não apresentar grande complexidade estrutural como é o caso de um pano. Contudo, o autor decidiu por não incorporar esse tipo de situação no modelo por conta de complexidades que isso pode trazer a estrutura da representação. Manter o modelo assim permite representar os cenários mais prováveis, tendo vista que a ausência de diversos tipos de artefatos muitas vezes não permite a continuidade da atividade.    

A execução de uma manutenção em linha viva deve seguir a risca as condições ambientais adequadas para essa finalidade. Uma dessas condições é a umidade relativa do ar, que deve estar necessariamente inferior a setenta por cento. O raciocínio 3 em \ref{raciocinio3} demonstra esse tipo de situação onde um agente tenta executar uma atividade com a umidade relativa dor ar em níveis inapropriados para isso, ocasionando o surgimento de uma violação de condição gerando uma sanção no agente que corresponde a ser eletrocutado e, consequentemente, morto. É interessante observar que nem toda violação de condição, no mundo real, resulta necessariamente em uma sanção ao violador. A umidade relativa do ar recai nessa situação, pois pode ser que o profissional cometa essa violação sem se envolver em um acidente. Isso pode ser resolvido por construir regras tratando condições em relação ao predicado $possOfNegConseqFor$. Contudo, a desobediência de condições ambientes normalmente resultam em acidentes. Portanto, essa condição - apesar de não tratar todos os cenários possíveis, trata um bom número dos mesmos.

A chave catraca é usada pelo profissional de linha viva para remover um parafuso que está preso ao conector. Uma execução inapropriada dessa relação resulta na ocorrência do eletricista ser eletrocutado e morto. Há diversas formas de como isso pode acontecer, sendo que uma delas consiste no profissional se posicionar de forma inapropriada para realizar essa relação e, por consequência, esbarrar tanto com o corpo quanto com a ferramenta em algum condutor de forma inapropriada. Portanto é de crucial importância que o profissional realize a execução com excelência. Esse comportamento é descrito pelo Raciocínio dado na subsubseção \ref{raciocinio4}. Assim como na situação relacionada condição, a realidade dos fatos pode produzir cenários possíveis "nesse caso" que não são adequadamente representados por esse modelo. Um possível cenário para essa situação consiste no fato do profissional simplesmente não conseguir executar a relação, sem que isso resulte em algum acidente. Contudo, a situação descrita pelo modelo apresenta o pior cenário possível. 

Em um acidente pessoas que não são responsáveis por atos cometidos podem sofrer duras consequências desses atos. Essa situação está demonstrada no raciocínio 5 presente na subsubseção \ref{raciocinio5}. Nessa situação, não passar glicerina no pano pode gerar um acidente ao montar a relação parafuso conector, porque o bastão isolante a ser usado nesse  processo, não estará em condições operacionais seguras, uma vez que a superfície dessa ferramenta pode conter algum tipo de impureza que corrobore com aumento de corrente de fuga em níveis suficientemente altos para matar alguém. O raciocínio 1 apresentou com excelência essa influência que a falta do uso de glicerina tem sobre a possibilidade de ocorrer algo errado no momento em que um certo profissional remover o parafuso usando o bastão. O autor entende, portanto que todos os predicados usados para representar essa situação foram necessários sendo que a ausência de um ou de outro, descaracteriza completamente essa representação. Nessa condição, se faz necessário saber com qual objetivo $relParafusoConector$ está atrelado, e isso é feito por intermédio do $\in$ e de $requiresCirc(goal_i,circ_j)$. Além disso, não é possível efetuar nenhum tipo de raciocínio sobre essa condição sem levar em consideração se os agentes tentam alcançar esse objetivo, e isso é feito por meio do predicado $starts(agent_n,g_m)$. O fato de ocorrer um acidente, ser independente do agente que está executando o objetivo, é muito bem representado pelos predicados $possOfNegConseqFor$ e $happensNegConseqFor$. Pela regra \ref{paybutiamnotguilty}, essa reunião de fatores em conjunto com os riscos associados ao evento dão como verdade para o predicado $possOfNegConseqForEven$ gerando a morte do profissional e a interrupção da manutenção. 

A presença do raciocínio 6 dado em \ref{raciocinio6} demonstrou que o modelo é capaz de interpretar quando o objetivo $g23$ for atingido. Em conjunto com o predicado $enabledToStart(ag_i,g_j)$, com a programação dos estados internos do agente e com a regra \ref{rolenextgoal} é possível verificar uma relação de continuidade para a representação (desde que o modelador defina os critérios que levam o agente a tentar atingir um dado objetivo). 

O raciocínio 7 dado por \ref{raciocinio7} apresenta como o modelo se comporta quando é feito uma consulta que não corresponde a realidade, ou seja quando $agente1$ tente alcançar o objetivo $g23$ sendo que este não faz parte dos objetivos que devem ser alcançados por aquele. O resultado objetivo foi o que o autor esperava. 

O autor conclui que o estudo de caso foi representado de forma apropriada pelos predicados e regras presentes nesse texto. Contudo, muitos raciocínios não foram capazes de apresentar todos os cenários possíveis a uma dada circunstância. Porém, todos os cenários resultantes do modelo correspondem a circunstâncias reais da manutenção. Não apenas isso, mas são tanto as circunstâncias mais prováveis como as mais sérias. Assim sendo, escapa ao poder de modelo representar muitos cenários, contudo os cenários mais importantes foram muito bem computados por essa representação.