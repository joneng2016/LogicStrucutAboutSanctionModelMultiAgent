A discussão dos resultados que estão expostos nas sub-seções \ref{mods}, \ref{predic}, \ref{regras} foi feita na própria apresentação dos mesmos. Isso se deve a natureza desses resultados uma vez que não é possível realizar a exposição deles sem discutir os fundamentos conceituais que justifiquem a existência dos mesmos. Essa situação não aplica no texto presente em \ref{studycase} e em \ref{rac} onde os resultados estão apenas expostos mas não foram discutidos. O texto a seguir fará uma discussão desses elementos apresentando as principais dificuldades que foram encontradas na realização desse estudo. 

\subsection{Considerações sobre Critérios Metodológicos ao Caso de Estudo} \label{conscritmetcasoestudo}

A primeira fase consistiu em descrever a manutenção em termos de objetivos que as vezes são organizados em série e as vezes em paralelo. Não houve grandes dificuldades para fazer isso, pois essa atividade é claramente composta de subatividades. O que foi um ponto relativamente complicado de se verificar nesse estudo é que os profissionais não precisam executar os objetivos na estrutura em que o modelo foi apresentado. Inclusive, muitas vezes os técnicos planejam a manutenção de um jeito e ao chegar no ambiente de execução eles mudam o encadeamento dos objetivos. Há um número finito e relativamente pequeno (é difícil definir um número, mas as observações dos pesquisadores permitem concluir que 10 formas diferentes consiste numa posição um tanto conservadora) sobre como esses objetivos podem ser organizados e isso ameniza a falta de previsibilidade de como a manutenção será realizada. 

O problema da organização dos objetivos pode ser resolvida de duas maneiras diferentes. Uma delas o engenheiro de manutenção modela o problema para todos os cenários possíveis. Portanto, se houver 10 formas diferentes de organizar esses objetivos, o engenheiro deverá refletir a cerca dessas 10 formas. Outra forma consiste em definir todas as relações possíveis que o predicado $nextGoal(g_i,g_j)$ permite em uma única estrutura. Nessa implementação do modelo o agente, por intermédio dos seus estados internos, escolhe a qual objetivo ele deverá tentar alcançar. Essa questão não foi levada em consideração no estudo de caso em análise porque os pesquisadores estavam interessados em realizar a análise necessária sobre a possibilidade de usar este modelo em um problema real. É possível argumentar que aquele arranjo de objetivos não é o único possível, contudo não deixa de ser um arranjo real e que pode ser muito bem executado pelos profissionais. 

Os pesquisadores desse estudo não sabem afirmar se o arranjo dos objetivos interferem no predicado $affectsRels(r_k,r_n)$. Para que isso seja analisado se faz necessário aplicar esse modelo para diversas situações diferentes onde todas devem apresentar a problemática do arranjo de objetivos. Se em uma dessas situações a especificação do predicado $affectsRels(r_k,r_n)$ mudar, então a proposição "o arranjo de objetivos afeta o predicado $affectsRels(r_k,r_n)$" é verdadeira, contudo se não mudar não é possível afirmar que essa proposição é falsa. 

A concepção do conceito de "papel" e a concepção da relação entre o agente e o seu papel foram muito adequados para as análises desse modelo. Isso se deu por observar como ocorre a distribuição dos objetivos aos agentes. Isso, pois na manutenção em linha viva todos os profissionais são tidos como executores, contudo existe uma distribuição de tarefas tendo em vista o conhecimento e a experiência de cada profissional ali envolvido. Portanto, para enquadrar essa questão nos moldes do modelo em análise se fez necessário encontrar um padrão de como os objetivos são distribuídos em função das atividades dos agentes. Com base nisso os engenheiros de modelagem concluíram o que está exposto na tabela \ref{agentsroles}.

Os conceitos de artefato e de relação foram adequados para esse caso de estudo não tendo a necessidade de definir nenhuma outra abordagem para isso. Todo o rol de ferramentas e de equipamentos foram definidos como artefatos. A fim de tornar a modelagem mais expressiva os pesquisadores poderiam criar subconjuntos de artefatos definindo um apenas para tratativa de ferramentas e outro apenas para tratativa de equipamentos. Isso não foi feito porque os pesquisadores não quiseram induzir os leitores desse estudo ao erro por entender que essa divisão pertence a estrutura conceitual do modelo propriamente dito. Uma taxonomia dessas seria adequada apenas para esse caso em estudo, caso contrário diminuiria o poder de generalização do modelo. 

A tabela \ref{condition} apresenta as condições que estão relacionadas ao meio. Esse conceito também se apresentou muito apropriado ao modelo porque uma condição não é um agente e não é um artefato mas é algo que está presente no meio da atividade e interfere com grande intensidade no andamento dos processos, portanto desconsiderar esse conceito ou compacta-lo como parte de outras estruturas implicaria uma representação míope da realidade. Os pesquisadores entenderam que essas condições são o suficiente para poder realizar a representação desse modelo.

As tabelas \ref{condition}, \ref{relationEntEnt1} e \ref{relationEntEnt2} apresentam a relação entre entidade ou condição com o risco e a consequência. Nesse estudo foi considerado apenas um único risco, que é o de ser eletrocutado e uma única consequência que é a morte. Contudo, há considerações que devem ser feitas no que tange a realidade, pois essa atividade exibe outros riscos tais como; "queda", "animais peçonhentos", "queimadura" entre outros que, assim como "eletrocutado" podem apresentar outras consequências além da morte. Esses riscos a mais não foram considerados no caso em estudo porque os raciocínios exibidos na subseção \ref{rac} puderam ser feitos sem a necessidade deles. Outro ponto que corroborou com isso consiste no fato de que os engenheiros de modelagem estavam interessados em obter primeiramente uma versão mais simples do modelo para então, se necessário, torná-lo mais complexo. Isso implica realizar algumas escolhas pragmáticas e uma delas consiste na verificação de qual risco é o mais importante e o mais temeroso na atividade. A análise com os profissionais mostram que o risco de 
ser eletrocutado é o principal e é mais preocupante ao executar uma atividade de manutenção em linha viva. Outro ponto reside na verificação das consequências desse risco o que remete a uma pergunta; Um profissional de linha viva ao executar manutenção em uma subestação de energia pode se envolver em um acidente ondo ele é eletrocutado e ainda sim sobreviver? A resposta a essa pergunta é que sim, porém muito improvável. Descargas de equipamentos que operam a 69 kV 1500 kVA  (o que é relativamente baixo) costumam matar o profissional eletrocutado mesmo que 
os disjuntores atuem na ordem de milissegundos. Portanto, existe outras consequências além da morte tal como; queimaduras e perda de membros contudo na grande maioria dos casos o profissional recairá no óbito. 

As tabelas \ref{relationEntEnt1} e \ref{relationEntEnt2} apresentam como se dá a relação entre duas entidades. Essa estrutura se tornou muito útil para fazer diversos raciocínios interessantes que estão presentes nas regras. Portanto os pesquisadores concluem que ela foi adequada, necessária e importante para essa representação e para esse caso de estudo, contudo ela tornou a especificação da modelagem um processo muito custoso porque o engenheiro de modelagem teve de refletir em todas as relações possíveis que são executadas na atividade e, depois disso, teve 
que ver quais relações se enquadravam em cada objetivo. Esse custo também está presente nos raciocínios que devem ser feitos pois dependendo da situação há uma série de relações que devem ser avaliadas. Essa questão nos permite refletir sobre a viabilidade de um modelo assim para situações onde o número de artefatos bem como o número de relações entre esses artefatos tendem ao infinito. Contudo, o fato do engenheiro de modelagem ter de refletir sobre todas as relações bem como seus respectivos riscos permite a realização de uma análise muito mais profunda 
da atividade e de como a segurança dos profissionais pode ser afetada de situação para situação. 

As tabelas \ref{relation1}, \ref{relation2} e \ref{relation3} apresentam a especificação do predicado $affectsRels(r_k,r_n)$. Esse foi um problema sério para os pesquisadores porque houve muitas tentativas de tentar resolve essa questão sem ter que abstrair tanto quanto esse predicado faz. Contudo, realizar um mapeamento minucioso de como se dá as atividades resulta em uma carga de especificação muito grande e que pode apresentar diversas fragilidades no que tange a uma certa consistência lógica (ou seja, um sistema que se contradiz). Portanto, em uma primeira abordagem admitir que a não execução (ou a mal execução) de uma relação afeta negativamente outra relação implica uma visão pragmática e simples para resolver o problema onde um eletricista se envolve em um acidente sobre o qual ele não tem responsabilidade algum. Ainda sim, os pesquisadores desse estudo admitem que esse é um ponto do modelo a ser melhorado a fim de se obter representações consistentes, expressivas e em com relativo baixo custo computacional. 

As tabelas \ref{entities}, \ref{relationsgroup1}, \ref{relationsgroup2} e \ref{conditions} apresentam a compactação das relações e condições cujos vínculos com objetivos foram exibidos, posteriormente, nas tabelas \ref{goalsrelationsentity1} e \ref{goalsrelationsentity2}. Nesse caso de estudo esse procedimento foi muito apropriado porque permitiu observar de forma clara as entidades, condições e relações necessárias para que determinado objetivo possa ser cumprido. Contudo existe um problema nessa abordagem que foi observado após a realização da especificação desse caso de estudo e reside no fato de que esse modelo abre margem para o modelador entrar em contradição com certa facilidade. Isso pode ser demonstrado com o seguinte exemplo; em dada atividade, um objetivo $g_1$ é constituído por $e_1,e_2,e_3$ e é formado pelas relações $possEntityRel(r_{12},e_1,e_2)$ e $possEntityRel(r_{23},e_2,e_3)$. Supondo que ao executar essa modelagem, o modelador faça o seguinte: $rg_1 = \{ r_{12},r_{23} \}$ e $ eg_1 = \{ e_1, e_2 \} $, $ requiresEntity(g_1,eg_1) $ e $ hasRelation(g_1, rg_1) $. Nesse caso o modelador se contradiz porque ele está considerando que $rg_{23}$ é uma relação entre $e_2, e_3$ que deve estar relacionada a $g_1$, mas $e_3$ não está relacionada a $g_1$. A especificação resulta em uma contradição entre os predicados $ requiresEntity(g_1,eg_1) $ e $ hasRelation(g_1, rg_1) $, já que a verdade de um deve necessariamente tornar o outro falso. Para que isso não aconteça, o modelador deverá ter cautela em considerar todas as entidades usadas na relação para um dado objetivo. Esse problema pode ser resolvido por desconsiderar o predicado $requiresEntity(g_n,eg_m)$ tendo em vista que um simples raciocínio usando $hasRelation(g_n,rg_m)$, $r_k \in rg_m$ e $possEntityRel(r_k, e_a,e_b)$ podem ser o suficiente para apresentar todas as entidades de uma relação $ hasRelation(g_n, rg_m) $. Essa concepção não foi considerada por uma questão conservadora, pois quando o modelo estava sendo concebido, os pesquisadores consideraram a possibilidade de existir um cenário onde uma entidade fosse necessária para cumprir com um determinado objetivo mas que não estabelece relação com nenhuma outra entidade no ambiente. Nessa condição um raciocínio envolvendo $hasRelation(g_n,rg_m)$, $r_k \in rg_m$ e $possEntityRel(r_k, e_a,e_b)$ traria apenas as entidades necessárias 
mas não as entidades suficientes. Contudo, após a realização da modelagem sobre esse caso de estudo ficou claro para os pesquisadores que uma situação onde uma entidade não se relaciona com as demais é um cenário absurdo. Isso, pois se a entidade não estabelecer relação alguma com as demais, então essa entidade é desnecessária e - perante as outras - não existe. O modelo como está não apresenta nenhuma condição absurda, contudo torna certa parte da especificação um processo redundante. Para as próximas versões, é possível pensar na eliminação do predicado 
$ requiresEntity(g_n,eg_m) $ e de $ \{ e_i , ... , e_k \} \in eg_m $

\subsection{Considerações sobre Critérios Metodológicos ao Raciocínios} \label{conscritmetrac}

Os raciocínios feitos sobre o modelo são de crucial importância para definir a eficácia desse projeto pois é com base nisso que se torna possível avaliar o quão efetivo vem a ser essa representação. Os raciocínios 1 e 5, dados respectivamente pelas subsubseções \ref{raciocinio1} \ref{raciocinio5} apresentam o problema com bastante expressividade. Tendo em vista o fato de que a Glicerina  é um composto químico relevante para manter o isolamento da parte não condutiva do bastão universal, esquecer de realizar isso gera um potencial acidente para ser eletrocutado em todas as outras situações onde o bastão será usado (a não ser nas situações onde o bastão universal não será usado em condutores energizados). Tanto os predicados como as regras que estão atreladas ao violação de relação e suas respectivas consequências representaram essa condição com sucesso. Nesse caso não aconteceu nenhuma sanção sobre o agente 4, portanto nem toda violação de relação gera necessariamente uma sanção. O predicado $possOfNegConseqFor$ conseguiu trazer com exito a sensação de possibilidade que existe em fenômeno desse gênero. 

Quando os pesquisadores estavam concebendo esse modelo, consideraram a possibilidade de trabalhar problemas dos raciocínios 1 e 5 por meio do conceito de \textit{Probabilidade}. Isso é interessante porque um predicado que consegue expressar momentos estatísticos com excelência apresenta possibilidades de aplicações extremamente elevados. Contudo, trabalhar com probabilidade resulta em diversas complicações de modelagem. Uma dessas complicações consiste no desenvolvimento de técnicas que podem indicar com rigor científico qual é a probabilidade de um acidente acontecer. Entretanto, isso não é o suficiente pois essa probabilidade é condicionada a ocorrência de uma dada relação. O raciocínio 1, por exemplo, demonstra que a não execução de $relPanoGlicerina$ resulta na possibilidade de ocorrer um acidente em $relBastaoGarraGondutor$. Se os pesquisadores estiverem trabalhando com o conceito de probabilidade, então é necessário desenvolver técnicas que verificam a probabilidade de acontecer algo ruim na relação $relBastaoGarraGondutor$ para o caso da relação $relPanoGlicerina$ não for efetivada com sucesso. Contudo, se se a não execução de uma outra relação também afeta $relBastaoGarraGondutor$, então também se faz necessário encontrar essa outra probabilidade. Além de aumentar a complexidade desse modelo, abre diversas indagações no que tange de como fazer isso o que pode ser um potencial campo de investigação científica. Com a finalidade de viabilizar uma primeira versão do modelo, os pesquisadores optaram por usar o conceito de possibilidade em vez de probabilidade. Apesar de diminuir a expressividade do modelo no que tange a questões que existe um componente sobre aleatoriedade, isso simplifica o processo de especificação, facilita o desenvolvimento de raciocínios e evita que o modelo seja estruturado sobre proposições falsas (por exemplo, definir uma probabilidade para uma condição 
de mundo onde isso não está claro). 

O vocabulário definido neste modelo foi apropriado para representar a condição de mundo presente no Raciocínio 2 que está exposto na subsubseção \ref{raciocinio2}. Em uma situação onde não há um pano para poder limpar todas as ferramentas, a manutenção é interrompida e essa situação ficou claramente representada por esse raciocínio onde a geração da violação de entidades corresponde a finalização da manutenção. Há a possibilidade de existir um cenário onde os profissionais criam algum tipo de técnica alternativa para poder transpassar a falta de algum artefato, inclusive se esse não apresentar grande complexidade estrutural como é o caso de um pano. Contudo, os pesquisadores decidiram por não incorporar esse tipo de situação no modelo por conta de complexidades que isso pode trazer a estrutura da representação. Manter o modelo assim permite representar os cenários mais prováveis, tendo vista que a ausência de diversos tipos de artefatos muitas vezes não permite a continuidade da atividade.    

A execução de uma manutenção em linha viva deve seguir a risca as condições ambientais adequadas para essa finalidade. Uma dessas condições é a umidade relativa do ar que deve estar necessariamente inferior a setenta porcento. O raciocínio 3 em \ref{raciocinio3} demonstra esse tipo de situação onde um agente tenta executar uma dada atividade com a umidade relativa dor ar em níveis inapropriados para isso ocasionando o surgimento de uma violação de condição gerando uma sanção no agente que corresponde a ser eletrocutado e, consequentemente, morto. É interessante observar que nem toda violação de condição, no mundo real, resulta necessariamente em uma sanção ao violador. A umidade relativa do ar recai nessa situação, pois pode ser que o profissional cometa essa violação sem se envolver em um acidade. Isso pode ser resolvido por construir regras tratando condições em relação ao predicado $possOfNegConseqFor$. Contudo, a desobediência de condições ambientes normalmente resultam em acidentes. Portanto, essa condição - apesar de não tratar todos os cenários possíveis, trata um bom número dos mesmos. Além disso, essa representação é conservadora no que tange ao evento propriamente dito, pois considera sempre o pior 
cenário possível.

A chave catraca é usada pelo profissional de linha viva para remover um parafuso que está preso ao conector. Uma execução inapropriada dessa relação resulta na ocorrência do eletricista ser eletrocutado e morto. Há diversas formas de como isso pode acontecer sendo que uma delas consiste no profissional se posicionar de forma inapropriada para realizar essa relação e, por consequência, esbarrar tanto com o corpo quanto com a ferramenta em alguma condutor de forma inapropriada. Portanto é de crucial importância que o profissional realize a execução com excelência. Esse comportamento é descrito pelo Raciocínio dado na subsubseção \ref{raciocinio4}. Assim como na situação relacionada condição, a realidade dos fatos pode produzir cenários possíveis "nesse caso" que não são adequadamente representados por esse modelo. Um possível cenário para essa situação consiste no fato do profissional simplesmente não conseguir executar a relação, sem que isso resulte em algum acidente. Contudo, a situação descrita pelo modelo apresenta o pior cenário possível. 

Em um acidente pessoas que não são responsáveis pelos atos cometidos podem sofrer duras consequências por conta disso. Essa situação está demonstrada no raciocínio 5 presente na subsubseção \ref{raciocinio5}.Nessa situação, não passar glicerina no pano pode gerar um acidente ao montar a relação parafuso conector porque o bastão isolante a ser usado nesse  processo não estará em condições operacionais seguras, uma vez que a superfície dessa ferramenta pode 
conter algum tipo de impureza que corrobore com aumento de corrente de fuga em níveis suficientemente altos para matar alguém. O raciocínio 1 apresentou com excelência essa influência que a falta do uso de glicerina tem sobre a possibilidade de ocorrer algo errado no momento onde um profissional removerá o parafuso usando o bastão. Os pesquisadores entendem, portanto que todos os predicados usados para representar essa situação foram necessários sendo que a ausência de um ou de outro descaracteriza completamente essa representação. Nessa condição, se faz necessário saber com qual objetivo $relParafusoConector$ está atrelado e isso é feito por intermédio do $\in$ e de $hasRelation(g_n,rg_m)$. Além disso, não é possível efetuar nenhum tipo de raciocínio sobre essa condição sem levar em consideração se os agentes tentam alcançar esse objetivo, e isso é feito por meio do predicado $starts(agent_n,g_m)$. O fato de ocorrer um acidente ser independente do agente que está executando o objetivo é muito bem representado pelos predicados $possOfNegConseqFor$ e $happensBadEvent$. Pela regra \ref{paybutiamnotguilty}, essa reunião de fatores em conjunto com os riscos associados ao evento dão como verdade para o predicado $possOfNegConseqForEven$ gerando a morte do profissional e a interrupção da manutenção. O que é interessante nessa parte da representação consiste no fato de que até mesmo a possibilidade do acidente não acontecer é algo a ser computado nesse raciocínio. Portanto, de todo o modelo, os pesquisadores entendem que essa é a parte que é representada com maior precisão. 

A presença do raciocínio 6 dado em \ref{raciocinio6} demonstrou que o modelo é capaz de interpretar quando o objetivo $g23$ foi atingido. Em conjunto com o predicado $enabledToStart(ag_i,g_j)$, com a programação dos estados internos do agente e com a regra \ref{rolenextgoal} é possível verificar uma relação de continuidade para a representação. Nessa situação o agente sempre saberá quais objetivos foram finalizados, saberá quais são os próximos objetivos, deverá ter a capacidade de decisão sobre qual objetivo ele deverá tentar alcançar (dando como verdade $tryRach(ag_n,g_m)$) e deverá ter a capacidade de executar o objetivo. 

O raciocínio 7 dado por \ref{raciocinio7} apresenta como o modelo se comporta quando é feito uma consulta que não corresponde a realidade. O resultado objetivo foi o que os pesquisadores esperavam. 

Os pesquisadores observam que os problemas que possuem as mesmas características que as presentes no caso de estudo são representados de forma apropriada pelos predicados e regras presentes nesse texto desde que o interesse consista verificar como se dá situações atreladas a cenários de acidentes. Contudo, muitos raciocínios não foram capazes de apresentar considerações no que tange a todos os cenários possíveis ao qual esses estavam relacionados. Inclusive, mesmo que os pesquisadores conseguissem demonstrar que os sete raciocínios apresentam capacidade de representar todos os cenários, ainda sim não seria 
possível demonstrar que isso é válido ao aplicar essa representação em outro caso de estudo. Contudo, o modelo foi capaz de apresentar situações factíveis 
de acontecerem no mundo real. 

Os pesquisadores não podem afirmar que as situações apresentadas sempre são as mais conservadoras possíveis (até porque o ato de como se dá a modelagem é algo que influência isso), contudo para esse caso em específico os cenários resultantes sempre foram os piores que podem acontecer dentro de um cenário de manutenção com essas características. 