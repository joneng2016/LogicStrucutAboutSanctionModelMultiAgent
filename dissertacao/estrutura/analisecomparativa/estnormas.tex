\textbf{Definição 1.} \textit{Um norma é definida por meio de uma tupla} $N = \langle \mu,\kappa,\chi,\tau,\rho \rangle$

\begin{itemize}
    \item $\mu \in \{obligation,prohibition\}$ representa as modalidades de norma.
    \item $\kappa \in \{action,state\}$ representa o tipo de \textit{trigger} da condição.
    \item $\chi$ representa o conjunto de estados em que uma norma se aplica.
    \item $\tau$ representa a norma da condição de \textit{trigger}
    \item $\rho$ representa a sanção aplicava pela violação do agente.
\end{itemize}

A definição 1 pode ser compreendida sobre o seguinte exemplo; 

\textit{Todos os imigrantes que possuem passaporte válido, devem ser aceitos. A falha resulta na perda de 5 créditos}. 

Dentro da definição 1, o exemplo fica;

\begin{eqnarray}
    \langle obligation,action,valid(Passport),accept(Passport),loss(5)\rangle
\end{eqnarray}

\textit{NormMAS} define um \textit{Registro de ação} que é dado pela definição 2. 

\textbf{Definição 2.} \textit{Um Registro de Ação é definido por meio de uma tupla} $R = \langle \gamma,\alpha,\beta \rangle$

\begin{itemize}
    \item $\gamma$ representa o agente executando uma ação;
    \item $\alpha$ representa a ação sendo executada pelo agente $\gamma$
    \item $\beta$ representa os estados internos do agente $\gamma$ no momento da execução.
\end{itemize}

Para demonstrar como se dá o uso dessa definição pode-se considerar a seguinte sentença;

\textit{O oficial John aprovou passaport 3225. O passaporte 3225 é definido como validado.}

Nessa sentença, $John$ é o agente dado por $\gamma$, o ato de aprovar o passaporte é o $\gamma$ que pode ser definido pelo predicado por $approve(3225)$ e o estado de ser validado por ser dado pelo predicado $valid(3225)$.  

Usando a \textbf{Definição 2}, isso poder ser especificado da seguinte maneira; 

\begin{eqnarray}
    \langle John, approve(3225), valid(3225)\rangle
\end{eqnarray}

O \textit{NormMAS} trabalha com os mesmos conceitos de normas e sanções presentes em \cite{dastaniframework}. Em relação a \textit{Definição 1}, é possível observar conceitos de obrigação, proibição dado por $ \mu $ e conceitos de sanção dado por $\rho$. Isso se assemelha muito as regras \ref{violationcondition}, \ref{violationrelation}, \ref{violationentity}, \ref{consviolationcondition} e \ref{consviolationrelation} que definem uma dada proibição (pois essas regras trabalham sobre o que não deve ser feito) e uma dada sanção (que define uma punição).

No que tange a estrutura dos modelos, o \textit{NormMAS} é mais simples e mais genérico com o foco em definir vocabulário para tratar qualquer tipo de comportamento normativo. Em contraste com isso, o modelo proposto nesse estudo é mais complexo, pois contem uma quantidade de conceitos muito maior, e é mais específico em relação a representar ação de pessoas, que devem trabalhar em grupo (exercendo atividades que apresentam riscos a elas) a fim de atingir um dado objetivo. 