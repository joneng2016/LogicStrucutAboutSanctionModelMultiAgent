\textbf{Definição 1.} \textit{Um norma é definida por meio de uma tupla} $N = \langle \mu,\kappa,\chi,\tau,\rho \rangle$

\begin{itemize}
    \item $\mu \in \{obligation,prohibition\}$ representa as modalidades de norma.
    \item $\kappa \in \{action,state\}$ representa o tipo de \textit{trigger} da condição.
    \item $\chi$ representa o conjunto de estados em que uma norma se aplica.
    \item $\tau$ representa a norma da condição de \textit{trigger}
    \item $\rho$ representa a sanção aplicava pela violação do agente.
\end{itemize}

A definição 1 pode ser compreendida sobre o seguinte exemplo; 

\textit{Todos os imigrantes que possuem passaporte válido, devem ser aceitos. A falha resulta na perda de 5 créditos}. 

Dentro da definição 1, o exemplo fica;

\begin{eqnarray} 
    \langle obligation,action,valid(Passport),accept(Passport),loss(5)\rangle
\end{eqnarray}

\textit{NormMAS} define um \textit{Registro de ação} que é dado pela definição 2. 

\textbf{Definição 2.} \textit{Um Registro de Ação é definido por meio de uma tupla} $R = \langle \gamma,\alpha,\beta \rangle$

\begin{itemize}
    \item $\gamma$ representa o agente executando uma ação;
    \item $\alpha$ representa a ação sendo executada pelo agente $\gamma$
    \item $\beta$ representa os estados internos do agente $\gamma$ no momento da execução.
\end{itemize}

Para demonstrar como se dá o uso dessa definição pode-se considerar a seguinte sentença;

\textit{O oficial John aprovou passaport 3225. O passaporte 3225 é definido como validado.}

Nessa sentença, $John$ é o agente dado por $\gamma$, o ato de aprovar o passaporte é o $\gamma$ que pode ser definido pelo predicado por $approve(3225)$ e o estado de ser validado por ser dado pelo predicado $valid(3225)$.  

Usando a \textbf{Definição 2}, isso poder ser especificado da seguinte maneira; 

\begin{eqnarray} 
    \langle John, approve(3225), valid(3225)\rangle
\end{eqnarray}

As tables a seguir exibem uma comparação entre o \textit{NORRMAS} e o modelo proposto neste estudo.

\begin{table}[H]
\centering
\begin{tabular}{|l|p{0.3\linewidth}|p{0.3\linewidth}|}
\hline          
\textbf{Atributos}
& 
\textbf{NORMMAS}
& 
\textbf{Modelo deste Estudo} 
\\ \hline
Finalidade
&
Representar agentes considerando questões atreladas a Normas, Sanções e Violações
&
Representar cenários de Acidentes 
\\ \hline
Generalização
&
Trata de cenários onde se tem o interesse de representar agentes, normas, violações e sanções. 
&
Trata de cenários específicos para acidentes onde se deseja verificar culpados. 
\\ \hline
Agentes
&
Não define os estados mentais do agente
&
Não define os estados mentais do agente
\\ \hline
Representação de Objetivos
& 
Não possui estrutura para representar objetivos
&
Possui uma estrutura simples para representar objetivos. Essa estrutura considera apenas questões de pré-requisitos. 
\\ \hline
Normas 
&
Presença de tuplas que permite expressar proibições e sanções para os mais diversos casos possíveis. 
&
Presença de regras que permite expressar violações e sanções para cenários de acientes.
\\ \hline
Generalização
&
Trata de cenários onde se tem o interesse de representar agentes, normas, violações e sanções. 
&
Trata de cenários específicos para acidentes onde se deseja verificar culpados.
\\ \hline
Estrutura da Linguagem
& 
Lógica de Primeira Ordem (definições dos conceitos em termos de Tuplas)
&
Teoria de Conjuntos e Lógica de Predicados  
\\ \hline
\end{tabular}
\end{table}