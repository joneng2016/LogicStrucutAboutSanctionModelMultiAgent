\subsection{Modelo MOISE+}

As tables a seguir exibiem uma comparação entre o \textit{MOISE+} e o modelo proposto neste estudo.

\begin{table}[H]
\centering
\begin{tabular}{|l|p{0.3\linewidth}|p{0.3\linewidth}|}
\hline			
\textbf{Atributos}
& 
\textbf{MOISE+}  
& 
\textbf{Modelo deste Estudo} 
\\ \hline
Finalidade  	
& 
Representar uma SMA 
& 
Representar cenários de Acidentes 
\\ \hline
Agentes     	
& 
Não representar os processos mentais do agente 
& 
Não representar os processos mentais do agente 
\\ \hline
Representação de Objetivos   	
& 
Baseado em uma Teoria Robusta para administração de tarefas, possui uma complexa sintaxe para representar objetivos  
&  
Possui uma estrutura simples para representar objetivos que avalia apenas questões de pré-requisitos. 
\\ \hline
Raciocínios		
& 
Possui relações de inferência atreladas a questões deônticas e a tipos de comunicações que os agentes podem fazer 
& 
Possui regras de inferências que possibilitam reproduzir cenários de acidentes entre os agentes. Não apenas isso, mas faz parte da estrutura dessas regras questões atreladas a normas, violações e sanções. Esses raciocínios também representam cenários onde agentes sofrem consequências negativas mesmo não sendo responsáveis causas dos acidentes e fazem isso por levar em consideração questões possibilísticas e influências entre relações. 
\\ \hline
\end{tabular}
\end{table}



\begin{table}[H]
\centering
\begin{tabular}{|l|p{0.3\linewidth}|p{0.3\linewidth}|}
\hline			
\textbf{Atributos}
& 
\textbf{MOISE+}  
& 
\textbf{Modelo deste Estudo} 
\\ \hline		 
Normas    
& 
Faz uso de lógica deôntica para definir o papel do agente em relação ao objetivo 
& 
Faz uso de lógica deôntica para definir o papel do agente em relação ao objetivo contudo também explora outras estruturas lógicas para determinar violações (quando o agente descumpri com o seu respectivo dever) e sanções (penalização por conta de uma dada violação).  
\\ \hline
SMA        		
& 
Presença de conceitos como Agentes, Papéis, Objetivos, Missões e Grupos bem como presença de  operadores (compatibilidade, herança)  
& 
Presença de conceitos como Papéis, Agentes e Objetivos. 
\\ \hline
Estrutura da Linguagem
& 
Lógica de Primeira Ordem 
&
Teoria de Conjuntos e Lógica de Predicados
\\ \hline
\end{tabular}
\end{table}
