\subsection{Modelo MOISE+}

A relação entre duas entidades é algo que se faz parte no levantamento conceitual desse modelo e é formalizado por meio do predicado $possEntityRel(r_l,e_i,e_k)$. O \textit{MOISE+} não apresenta estruturas para formalizar essa relação de maneira tão genérica como é tratada por esse predicado, até porque tanto o conceito de entidade como o conceito de relação entre duas entidades nem mesmo existem nesse arcabouço. Contudo é possível dizer que esse predicado é abordado de forma específica na consideração dos $link's$ entre dois agentes bem como da natureza do mesmo. Isso também se dá com o operador \textit{compatibilidade}. 

Os predicados $adoptsRole(ag_n,\rho_m)$, $hasObligation(\rho_m,g_k)$ e $hasPermission(\rho_m,g_k)$ por serem inspirados na estrutura conceitual do próprio \textit{MOISE+} denotam que podem ser especificados por esse \textit{arcabouços}. Contudo, o \textit{MOISE+} trabalha a relação deônticas entre $\rho$ com um conjunto $m$ que representa missões que devem ser atingidas pelos agentes. Essas missões são \textit{containers} de objetivos.

O conceito que é tradizo pelo predicado $nextGoal(g_i,g_k)$ podem ser tratados pelo \textit{MOISE+} por intermédio da árvore de objetivos que averigua paralelismo e sequencialismo. O \textit{MOISE+} aprofunda a dinâmica desse problema por considerar estruturas subordinação entre objetivos. 

O \textit{MOIISE+} não apresenta condições, em sua estrutura, para especificar o que está levantado nos demais predicados dese modelo computacional. 

Em termos de regras o \textit{MOISE+} é capaz de especificar apenas a regra \ref{reldeonticrole} e isso pode ser feito por meio da relação presente em \ref{deonticRuleMoise}. No que tange as demais regras não foi possível encontrar estruturas conceituais que apresentam capacidade de lidar com as mesmas dentro deste \textit{Arcabouço}.  