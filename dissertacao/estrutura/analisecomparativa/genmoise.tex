\subsection{Modelo MOISE+   }

A estrutura deste modelo está presente na fundamentação teórica na seção \ref{sma} e é com base nisso que se pode afirmar a identificação dos seguintes 
conceitos: papel, grupos, objetivos, missões. 

Tanto o \textit{MOISE+} como a proposta deste estudo apresentam o mesmo conceito de papel que consiste na função do agente dentro de uma \textit{SMA}. 
Contudo, a concepção por trás das relações com os demais conceitos é feita de forma diferente em cada modelo. No \textit{MOISE+} um papel se relaciona
com outro papel por meio do predicado \textit{link} identificado níveis de autoridade. Isso não foi abordado neste modelo porque não era do interesse 
do domínio de estudo verificar como se dá as relações de autoridade entre os agentes uma vez que a natureza das violações e sanções em interesse podem 
ser escritas sem esse termo. O mesmo se dá com o operador que corresponde a compatibilidade. Existe a possibilidade de descutir um caso onde a ocorrência 
de uma violação da natureza delimitada neste estudo se dá por conta de problemas no que diz respeito as concepções de autoridade (subordinado é orientado 
a fazer algo de forma inapropriada). Contudo, os pesquisadores deste estudo entendem que mesmo esses casos podem ser descritos pelo vocabulário em proposta,
talvez não com a mesma expressividade. Isso se dá pelo fato de que a ocorrência de uma violação está associada ao objetivo em sí e não a função do 
profissional. Por exemplo, considere a seguinte situação; Um agente A é supervisor de um agente B e aquele orienta este a fazer algo inapropriado sendo que 
este venha a sofrer consequências ruins por conta disso. Esse problema pode ser modelado da seguinte maneira;

\begin{enumerate}
	\item $hasRole(agenteA,supervisor)$ 
	\item $hasRole(agenteB,subordinado)$
    \item $tryReach(agenteA,g_{instrucao})$
    \item $tryReach(agenteB,g_{instrucao})$
    \item $tryReach(agenteB,g_{execucao})$
	\item $hasObligation(supervisor,g_{instrucao})$
	\item $hasObligation(subordinado,g_{instrucao})$
	\item $hasObligation(subordinado,g_{execucao})$
	\item $nextGoal(g_{instrucao},g_{execucao})$	
    \item $thereIsRelation(r_{passa-informacao},agenteA,agenteB)$
    \item $r_{passa-informacao} \in rg_{instrucao}$
    \item $r_{executar} \in rg_{execucao}$    
    \item $hasRelation(g_{instrucao},rg_{instrucao})$
    \item $hasRelation(g_{execuao},rg_{execuao})$
    \item $happensBadEvent(r_{executar})$
    \item $hasRisk(r_{executar}, cair, cs_{pernaQuebrada})$
    \item $affectsOtherRelations(r_{passa-informacao},r_{executar})$
\end{enumerate}

Os raciocínios, portanto, são: 


\begin{eqnarray}\label{refutmoisea}\nonumber
	hasRelation(g_{instrucao},rg_{instrucao}) \wedge \neg isPresent(r_{passa-informacao}) \\ \nonumber 
    \wedge (r_{passa-informacao} \in rg_{instrucao}) \\ \nonumber
    \wedge tryReach(agenteA,g_{instrucao}) \to \nonumber \\
	violationRelation(agenteA,g_{instrucao},rg_{instrucao})
\end{eqnarray}


\begin{eqnarray}\label{refutmoiseb}
	violationRelation(agenteA,g_{instrucao},rg_{instrucao}) \wedge  \nonumber \\
    affectsOtherRelations(r_{passa-informacao},r_{executar}) \nonumber \\
    \to possibilityHappensBadEvent(r_{executar}) 
\end{eqnarray}

\begin{eqnarray}\label{refutmoisec}
	possibilityHappensBadEvent(r_{executar}) \wedge  \nonumber \\ 
    happensBadEvent(r_{executar}) \wedge \nonumber \\ 
    hasRelation(g_{execuao},rg_{execuao}) \wedge \nonumber \\
    (r_{executar} \in rg_{execucao}) \nonumber \\ 
	\wedge hasRisk(r_{executar}, cair, cs_{pernaQuebrada}) \wedge \nonumber \\ 
    tryReach(agenteB,g_{execucao}) \nonumber \\ 
	\to consequenceOfBadEvent(g_{execucao},agenteB,cair,cs_{pernaQuebrada}) \nonumber \\ 
\end{eqnarray}

Portanto, as relações \ref{refutmoisea}, \ref{refutmoiseb} e \ref{refutmoisec} (o raciocínio que mostra $g_{instrucao}$ ser concluído não foi mostrado) 
demonstram que é possível verificar uma caso de violação por autoridade sem considerar as relações entre os papeis. Uma possível contestação a esse 
raciocínio reside no predicado $isPresent$ pois neste cenário $r_{passa-informacao}$ esteve presente na execução de $g_{instrucao}$. Contudo, esse predicado 
faz referência não somente as relações que não estão presentes bem como as relações que foram mal realizadas (não atingiram o propósito). 

O conceito de grupo está presente no \textit{MOISE+} mas não na proposta deste modelo. Isso recai na mesma condição do que se pretende demonstrar com o 
raciocínio \ref{refutmoisec} sendo que esse conceito não é necessário para os propósitos finais deste estudo. 

Os fundamentois conceituais apresentados em ambos os modelos no que tange a objetivos são os mesmos, que consiste em critérios que devem ser atingidos 
pelos agentes. Contudo, o \textit{MOISE+} lida com essa questão por tratar os objetivos em uma estrutura de grafos definido sub e super objetivos. Ainda 
nessa linha, o \textit{MOISE+} trabalha operadores que relacionam super, subobjetivos que são; sequencial, paralelo e escolha. Apesar de ser uma ideia 
interessante, os pesquisadores entenderam que escrever os objetivos em termos do predicado $nextGoal$ é abrangente, é simples e resolve o problema sendo este 
cenário ideal para uma primeira versão do modelo em proposta. Contudo, é de grande interesse investigar situações onde a abordagem adotada pelos pesquisadores 
não funciona a fim de ter de usar a estrutura em árvore presente no \textit{MOISE+} e verificar como isso se dá no que tange aos demais elementos do modelo 
proposto por eles.

O conceito de \textit{missão} está presente no \textit{MOISE+} mas não no modelo proposto neste estudo. Isso se dá porque a ideia de missão diminui 
a granularidade da relação entre o agente com os seus objetivos. Consequentemente, a verificação de um evento em um dado objetivo necessita mais etapas 
de raciocínio. Por exemplo, se existe uma situação onde um dado agente com papel $\rho_a$ tem uma missão onde $ hasObligation(\rho_a,m_1) $ em que 
$ m_1 = \{ ... , g_a, ...\}$ para que se possa verificar um dado evento que ocorre em $ g_a $, o raciocinador deverá antes consultar $ hasObligation(\rho_a,m
_1) $. Tendo em vista a granularidade das violações e sanções sobre os objetivos, os pesquisadores entederam ser aproprieado criar relacões diretas 
entre o papel $ \rho_a $ com o objetivo $g_1 $ isso elimina a quantidade de predicados que deve ser considerado nas relações de implicação. Se a linha 
adotada fosse a mesma do \textit{MOISE+}, então seria necessário considerar o seguinte termo $ g_i \in m_a $ e uma relação como \ref{rolenextgoal} 
teria de ser escrita da seguinte forma: 

\begin{eqnarray}\label{constmoise01}
	hasRole(ag_n,\rho_m) \wedge hasPermission(\rho_m,m_a) \wedge (g_i \in m_a) \wedge nextGoal(g_i,g_j) \wedge isReached(g_i) \nonumber \\
	\to ableTryReach(ag_i,g_j) \nonumber \\
    ag_i, ag_n \in Agent, \rho_m \in Role, g_j \in Goal, g_i \in Goal, m_a \in Mission
\end{eqnarray}

O \textit{MOISE+} não aborda os conceitos relacionados a norma, violação, sanção e possibilidades presentes na proposta deste estudo. Uma possível 
contestação dessa afirmação reside no fato de que \textit{MOISE+} contem as relaçoes deonticas que terminam permissões e obrigações aos agentes. Contudo,  
assim como na proposta deste estudo, essas relações são usadas para "dizer" ao agente o que ele deve fazer. De certa forma, é uma norma - mas que 
verifica apenas a execução do objetivo em si e não estabele relações violação ou sanção. Então, frente ao \textit{MOISE+} a inovação presente 
neste estudo reside no fato de que existe um vocabulário criado para especificar relações de violação, sanção e possibilidades em problemas 
onde agentes devem atuar de forma colaborativa para cumprir com um dado propósito. 
