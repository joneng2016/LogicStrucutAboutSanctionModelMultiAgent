\newpage

\subsection{Modelo MOISE+}

A Tabela \ref{genmoise1} apresenta uma comparação entre o \textit{MOISE+} e o modelo proposto neste estudo.

\begin{center}
\begin{longtable}[H]{|l|p{0.3\linewidth}|p{0.3\linewidth}|}
\caption{Comparação entre o \textit{MOISE+} e o modelo proposto neste estudo} \label{genmoise1} \\ 
\hline			
\textbf{Atributos}
& 
\textbf{MOISE+}  
& 
\textbf{Modelo deste Estudo} 
\\ \hline
Finalidade  	
& 
Representar a estrutura organizacional de um SMA: estrutural, funcional e deôntica 
& 
Representar cenários de Acidentes com base em uma estrutura organizacional similar ao Moise+

\\ \hline
Agentes     	
& 
Não representar os processos mentais do agente 
& 
Não representar os processos mentais do agente 

\\ \hline
Representação de objetivos   	
& 
Baseado em uma teoria para administração de tarefas, possui uma sintaxe baseada em decomposição, sequência e paralelismo para representar objetivos  
&  
Possui uma estrutura simples para representar objetivos que avalia apenas questões de pré-requisitos. Permite representar sequências e paralelismos
\\ \hline

\newpage

\hline
\textbf{Atributos}
& 
\textbf{MOISE+}  
& 
\textbf{Modelo deste Estudo} 

\\ \hline
Raciocínios		
& 
Possui relações atreladas a questões deônticas e a tipos de comunicações que os agentes podem fazer 
& 
Presença de regras que possibilitam reproduzir cenários de acidentes entre os agentes. Não apenas isso, mas faz parte da estrutura dessas regras questões atreladas a normas, violações e sanções. Esses raciocínios também representam cenários onde agentes sofrem consequências negativas mesmo não sendo responsáveis pelas causas dos acidentes e fazem isso por levar em consideração questões possibilísticas e influências entre relações 
\\ \hline		 
Normas    
& 
Faz uso de lógica deôntica para definir o papel do agente em relação ao objetivo 
& 
Faz uso de lógica deôntica para definir o papel do agente em relação ao objetivo, contudo também explora outras estruturas lógicas para determinar violações (quando o agente descumpre com o seu respectivo dever) e sanções (penalização por conta de uma violação) 


\\ \hline

\newpage

\hline
\textbf{Atributos}
& 
\textbf{MOISE+}  
& 
\textbf{Modelo deste Estudo} 

\\ \hline

SMA        		
& 
Presença de conceitos como agentes, papéis, objetivos, missões e grupos bem como presença de  operadores (compatibilidade, herança)  
& 
Presença de conceitos como papéis, agentes e objetivos

\\ \hline
Estrutura da Linguagem
& 
Lógica de Predicado de Primeira Ordem 
&
Teoria de Conjuntos e Lógica de Predicados de Primeira Ordem

\\ \hline
\end{longtable}
Autoria Própria
\end{center}