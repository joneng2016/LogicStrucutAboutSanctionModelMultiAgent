\subsection{NORMMAS}

A Tabela a seguir exibem uma comparação entre o \textit{NORMMAS} e o modelo proposto neste estudo.

\begin{table}[H]
\centering
\begin{tabular}{|l|p{0.3\linewidth}|p{0.3\linewidth}|}
\hline          
\textbf{Atributos}
& 
\textbf{NORMMAS}
& 
\textbf{Modelo deste Estudo} 
\\ \hline
Finalidade
&
Representar agentes considerando questões atreladas a normas, sanções e violações
&
Representar cenários de acidentes 
\\ \hline
Generalização
&
Trata de cenários onde se tem o interesse de representar agentes, normas, violações e sanções 
&
Trata de cenários específicos para acidentes onde se deseja conhecer as causas do acidente 
\\ \hline
Agentes
&
Não define os estados mentais do agente
&
Não define os estados mentais do agente
\\ \hline
Representação de Objetivos
& 
Não possui estrutura para representar objetivos
&
Possui uma estrutura simples para representar objetivos. Essa estrutura considera apenas questões de pré-requisitos 
\\ \hline
Normas 
&
Presença de tuplas que permitem expressar proibições e sanções para os mais diversos casos possíveis 
&
Presença de regras que permitem expressar violações e sanções para cenários de acidentes
\\ \hline
Generalização
&
Trata de cenários onde se tem o interesse de representar agentes, normas, violações e sanções 
&
Trata de cenários específicos para acidentes onde se deseja verificar as causas do acidente
\\ \hline
Estrutura da Linguagem
& 
Lógica de Primeira Ordem (definições dos conceitos em termos de tuplas)
&
Teoria de Conjuntos e Lógica de Predicados  
\\ \hline
\end{tabular}
\end{table}

\subsection{Comparação Genérica entre os Modelos}

Tendo como base as análises feitas nas seções anteriores, é possível elaborar' na Tabela \ref{comparemodel} que apresenta uma análise comparativa dos arcabouços no que tange a expressividade do modelo computacional proposto nesse texto. Por expressividade, se entende capacidade de expressar, representar o objeto de interesse. Para essa análise 
foi feita a seguinte escala: nenhuma expressividade $\prec$ pouco expressivo $\prec$ expressivo $\prec$ muito expressivo $\prec$ altamente expressivo. O termo nenhuma expressividade não indica que é impossível definir a estrutura em observação dentro do modelo em voga, mas sim que o engenheiro de modelagem terá que criar uma estrutura conceitual \textit{ad hoc}. Sobre o mesmo aspecto reside pouco expressivo, contudo o modelo - neste caso - possui algumas estruturas pré-definidas que diminuem o esforço da especificação. O termo expressivo deixa claro que o modelo permite especificar o objeto de interesse sem que o engenheiro tenha de criar muitos atributos para o domínio de interesse. O termo muito expressivo define que o modelo apresenta diversos conceitos específicos para representar o objeto em interesse, contudo ainda há margem para que o modelador tenha que criar um ou mais atributos. O termo altamente expressivo define o caso onde o modelo especifica o objetivo de interesse muito bem fazendo com que o modelador não precise definir nenhum critério conceitual a mais (ou terá que montar poucas definições).   

\begin{table}[H]
    \centering
    \scalefont{0.6}
    \begin{tabular}{|l|l|l|l|l|}
        \hline
        \textbf{Critérios} & \textbf{MOISE+}        & \textbf{DASTANI}     & \textbf{V3S}         & \textbf{NORMMAS}          \\ \hline
        \textbf{Agente}    & pouco                  & pouco                & muito                & pouco                     \\ \hline
        \textbf{SMA}       & altamente              & pouco                & expressivo           & pouco                     \\ \hline
        \textbf{Artefato}  & nenhuma                & pouco                & expressivo           & pouco                     \\ \hline
        \textbf{Norma}     & nenhuma                & altamente            & pouco                & altamente                 \\ \hline
        \textbf{Violação}  & nenhuma                & altamente            & pouco                & altamente                 \\ \hline
        \textbf{Sanção}    & nenhuma                & altamente            & pouco                & altamente                 \\ \hline
        \textbf{Risco}     & nenhuma                & pouco                & altamente            & pouco                     \\ \hline
        \textbf{P.O.A.E}   & nenhuma                & pouco                & pouco                & pouco                     \\ \hline
        \textbf{Objetivos} & muito                  & pouco                & muito                & pouco                     \\ \hline
        \textbf{C.A}       & nenhuma                & pouco                & pouco                & pouco                     \\ \hline
        \textbf{I.AG.AR}   & nenhuma                & pouco                & pouco                & pouco                    \\ \hline
        \textbf{D.C.A}     & nenhuma                & pouco                & altamente            & pouco 
    \\ \hline
    \end{tabular}
    \caption{Análise comparativa sobre a expressividade desses modelos no que tange aos objetivos deste estudo.}
    \label{comparemodel}
\end{table}

O critério \textbf{Agente} condiz com representação dos estados internos que um agente pode ter. O critério \textbf{SMA} condiz com presença de elementos que são necessários para especificar um \textit{Sistema Multiagente}. O critério \textbf{Artefato} condiz com elementos que correspondem a definição presente na Seção \ref{artefact}. O critérios de \textbf{Norma} corresponde a Regras que devem ser acatadas pelos agentes. \textbf{Violação} define o que corresponde o não cumprimento de uma dada regra. \textbf{Sanção} implica penalidade que está sobre o agente. \textbf{Risco} consiste no evento ruim que tem um potencial de ocorrer sobre o agente. \textbf{P.O.A.E} significa Possibilidade de Ocorrer algo Errado e corresponde a expressar condições onde existe potencial de acontecer algo inapropriado sobre o agente mesmo que esse realize sua função com excelência.  \textbf{Objetivos} implica alvos que devem ser atingidos pelos agentes. \textbf{C.A} consiste em Condições Ambientes que interagem com a atividade executada pelos agentes. \textbf{I.AG.AR} representa as Interações entre Agentes e Artefatos. \textbf{D.C.A} - Descrição de Cenários de Acidentes, consiste na capacidade de desenvolver raciocínios a fim de representar cenários de acidentes.

As subseções anteriores em conjunto com a Tabela \ref{comparemodel} permitem concluir que esse trabalho é inovador no que tange a ter um vocabulário específico para representar cenários de risco e de acidentes (tanto sobre o responsável pelo acidente bem como a vitima) de atividades manuais usando para isso, o conceito de sistema multiagente normativo. Esse vocabulário apresenta limitações as quais serão debatidas na próxima Seção. 