\subsection{Modelo Cartago}

A estrutura do Cartago é exibida em \ref{artefact}. O autor entende que esse arcabouço apresentam estrutura para representar questões referentes aos agentes e artefatos. Tendo em vista o fato de considerar como ocorre as interações entre os etados internos e o corpo do agente com artefatos dispotos ao meio (consolidando conceitos como \textit{workspace}) e tendo em vista o fato de ser o arcabouço que se preocupa em profundidade na descrição de artefatos, o autor entende que esse modelo é o mais expressivo para retratar conceitos atrelados a relações entre os artefatos e agentes, artefatos com artefatos bem  violações podem ocorrer em relação a artefatos e agentes.