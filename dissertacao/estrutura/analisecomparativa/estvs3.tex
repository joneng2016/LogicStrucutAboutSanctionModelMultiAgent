\section{Comparação entre VS3 e o modelo proposto neste estudo}

A Tabela \ref{vs3comparacao1}  apresentam uma comparação entre o \textit{VS3} e o modelo proposto neste estudo.

\begin{center}
\begin{longtable}[H]{|p{0.3\linewidth}|p{0.3\linewidth}|p{0.3\linewidth}|}
\caption{Comparação entre o \textit{VS3} e o modelo proposto neste estudo} \label{vs3comparacao1} \\
\hline			
\textbf{Atributos}
& 
\textbf{VS3}
& 
\textbf{Modelo deste Estudo} 
\\ \hline
Finalidade
&
Representar cenários de acidentes a fim de simulá-los com propósito de treinamento profissional 
&
Representar cenários de acidentes 
\\ \hline
Artefatos-Objetos
&
Representa objetos através da classe (da ontologia atrelada ao módulo \textit{Domain Model}) \textit{V3S-Object}
&
Representa objetos através do conjunto \textit{Artefact}
\\ \hline
Relações entre entidades
&
Representa as relações entre objetos através da classe (da ontologia atrelada ao módulo \textit{Domain Model}) \textit{V3S-Action} através de relacionamentos com a classe \textit{V3S-Object}
&
Representa as relações entre as entidades através do predicado $possEntityRel(r_l,e_i,e_k)$
\\ \hline
Riscos e Acidentes
&
Modelo \textit{MELISSA} (BATU E BCTU)
&
Uso de normas, violações, sanções e lógica possibilística aplicadas a contextos específicos por meio das regras \ref{conditionViol}, \ref{relationViol}, \ref{entityViol}, \ref{consconditionViol} e \ref{consrelationViol}
\\ \hline 
Agentes     	
& 
Uso do \textit{Framework} MASVERP 
& 
Não representar os processos mentais do agente 
\\ \hline 
Estrutura da Linguagem
& 
Diferentes formalismos da computação tais como: ontologias, lógica de primeira ordem, algoritmos e entre outros
&
Teoria de Conjuntos e Lógica de Predicados 
\\ \hline 
Estruturas para Avaliação e Treinamento
& 
HERA
&
Não consta
\\ \hline
\end{longtable}
Autoria Própria
\end{center}

A lista a seguir apresenta uma comparação entre os costrutores do \textit{ACTIVITY-DL} e a parte deste modelo conceitual focada em representar as relações entre os objetivos. Observar Tabela \ref{acticonstruct} 

\begin{enumerate}
	\item \begin{itemize}
			\item \textit{ACTIVITY-DL:}  A $IND$ B.
			\item \textit{Modelo:} $nextGoal(A,B)$,$nextGoal(B,A)$,$nextGoal(A,B) \to F$,$nextGoal(B,A) \to F$, $nextGoal(X,A) \wedge nextGoal(X,B) \wedge nextGoal(A,Y) \wedge nextGoal(B,Y)$, $nextGoal(X,A) \wedge nextGoal(X,B) \wedge nextGoal(A,Y) \wedge nextGoal(B,Y) \to F$   e outras situações que não podem ser expressadas nessa representação.
		\end{itemize}
	\item \begin{itemize}
			\item \textit{ACTIVITY-DL:} A $SEQ$ B.
			\item $nextGoal(A,B)$, $nextGoal(B,A)$ e outras situações que não podem ser expressadas nessa representação. 
		\end{itemize}
	\item \begin{itemize}
			\item \textit{ACTIVITY-DL:} A $SEQ-ORDER$ B.
			\item $nextGoal(A,B)$  e outras situações que não podem ser expressadas nessa representação.
		\end{itemize}
	\item \begin{itemize}
			\item \textit{ACTIVITY-DL:} A $PAR-SIM$ B, A $PAR-SIM$ B, A $PAR-START$ B, A $PAR-END$ B 
			\item $nextGoal(X,A) \wedge nextGoal(X,B) \wedge nextGoal(A,Y) \wedge nextGoal(B,Y) $ e outras situações que não podem ser expressadas nessa representação.  
		\end{itemize}
\end{enumerate} 