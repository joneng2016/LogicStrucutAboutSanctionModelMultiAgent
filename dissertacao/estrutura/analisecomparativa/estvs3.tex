\textit{Activity Model} é estruturado sobre uma linguagem de descrição conhecido por \textit{ACTIVITY-DL}. Um dos elementos dessa linguagem é baseado na álgebra de Allen's que tem como por finalidade definir raciocínios temporais \cite{allenalgebric}. 
As relações definidas por essa álgebra é dada por; 

\begin{enumerate}
	\item $X < Y$ onde $X$: ocorre antes de $Y$ 
	\item $X m Y$,$Y mi X$: $X$ encontra $Y$
	\item $X o Y$, $X oi Y$: $X$ sobrepõem a $Y$
	\item $X s Y$, $Y si X$: $X$ começa $Y$
	\item $X d Y$, $Y di X$: $X$ ocorre durante $Y$	  
	\item $X f Y$, $Y fi X$: $X$ termina junto com $Y$	  	
	\item $X = Y$ $X$ é igual a $Y$	  		
\end{enumerate}

A \textit{Activity Model} define construtores que são semanticamente equivalente a certos operadores da álgebra de Allen's. Esses construtores (atuantes sobre atividades) são definidos pela tabela \ref{acticonstruct}

\begin{table}[H]
\centering
\begin{tabular}{|l|l|l|}
\hline
Construtor & Nome         & Relações de Allen \\ \hline
IND        & Independent  & $A \{ <,>,m,mi,o,oi,s,si,d,di,f,fi,= \} B$\\ \hline
SEQ        & Sequential   & $A \{ <,>,m,mi \} B$\\ \hline
SEQ-ORD    & Ordered      & $A \{ <,>,m \} B$\\ \hline
PAR        & Parallel     & $A \{ o,oi,s,si,d,di,f,fi,= \} B$ \\ \hline
PAR-SIM    & Simultaneous & $A \{ = \} B$\\ \hline
PAR-START  & Start        & $A \{ s,si,= \} B$\\ \hline
PAR-END    & End          & $A \{ f,fi,= \} B$ \\ \hline
\end{tabular}
\caption{Construtores da linguagem \textit{ACTIVITY-DL} \cite{v3sframework}}
\label{acticonstruct}
\end{table}

As relações temporais entre as sub-atividades são especificadas por intermédio de construtores que são formalmente definidos no estudo \cite{allenalgebric}. Essas relações são intermediadas pelo vocábulo \textit{Pré-condição} que tem como por propósito apresentar o contexto sobre qual uma dada atividade deve ser executada. A tabela \ref{precondition} apresenta esses contextos.

\begin{table}[H]
\begin{tabular}{|l|l|p{0.6\linewidth}|}
\hline
\textbf{Categoria} & \textbf{Pré-condição} & \textbf{Descrição}                                                                                                                                                                                                                                              \\ \hline
Condições para perceber  & Nomológico            & Descreve o estado do mundo necessário para que a tarefa seja fisicamente realizável. Condições dependem diretamente das regras de ação definidas no modelo de domínio. Exemplo: Abre a porta se estiver fechada.                                                \\ \hline
Condições para perceber  & Regulamentar          & Descreve o estudo do mundo necessário para uma boa realização da atividade de acordo com prescrito em procedimento. Exemplo: Para desconectar o tubo, a proteções devem ser desgastado.                                                                         \\ \hline
Condições para  Examinar & Contextual            & Descreve o estado de mundo em que a atividade é relevante. Quando essa condição é falsa, então a atividade deve ser ignorada. Exemplo: Limpar o tubo é relevante apenas se o tubo estiver sujo.                                                                 \\ \hline
Condições para  Examinar & Favorável             & Descreve o estado de mundo onde a tarefa é preferencial sobre as demais. Essas condições ajudam a escolher entre várias tarefas quando existe uma alternativa para a realização de uma tarefa decomposta. Exemplo: se o parafuso estiver enferrujado, desarmar. \\ \hline
\end{tabular}
\caption{As pré-condições possíveis para as atividades \cite{v3sframework}}
\label{precondition}
\end{table}


No que tange a questões referentes a segurança e violação, a linguagem \textit{ACTIVITY-DL} deve lidar com atividades em estados de alta degradação bem como com compromissos cognitivos  que são um grande potencial para a geração de risco. Essa condição possibilita a verificação de erros nos seguintes aspectos: atividades de aprendizagem e demonstração de comportamentos similares tendo como base personagens virtuais \cite{v3sframework}. Por conta disso, a linguagem \textit{ACTIVITY-DL} incorpora os conceitos de BCTUs e BATUs cujos fundamentos científicos estão descritos em \ref{risksec}. Ambas tags trabalham com o fato de que, ao menos em partes, o profissional decide por cometer uma dada violação tendo em vista a inviabilidade (ou por não ser prático) efetuar a ação com base no que é definido pelos manuais. 

\textit{Risk Models} é a parte do modelo que define a análise de risco. Existe duas categorias; risco de análise clássico e método de análise de confiabilidade humana. A primeira categoria permite definir uma análise quantitativa de risco, contudo falha ao definir a complexidade dos resultados frente a fatores humanos. Em contrapartida, a segunda categoria considera fatores humanos, contudo falha em definir medidas objetivas sobre questões de segurança \cite{v3sframework}. O \textit{V3S} combina ambas situações usando a abordagem MELISSA \cite{melissaproject} \cite{v3sframework}. Essa abordagem é baseada em três pontos (1) atividades relacionadas em cenários de acidentes, (2) descrição das tarefas de representação e (3) fatores influentes em potencial nas atividades. MELISSA representa os cenário de acidente por meio do gráfico \textit{Bowtie}. Isso consiste na identificação de todos os cenários de acidentes bem como no provisionamento e uma listagem de barreiras para os mesmos. O risco aceitável consiste em escolhas que verificam o número e desempenho dessas barreiras. Os ponto central do gráfico de \textit{Bowtie} consiste em eventos críticos, a parte a esquerda desse gráfico implica as causas do evento e a parte direita do mesmo corresponde as consequências do evento \cite{v3sframework}, \cite{melissaproject}. Essa descrição pode ser analisada na figura \ref{bowtiegraf}. 


\begin{figure}[H]
  \centering
  \includegraphics[width=0.5\linewidth]{figure/bowtie.png} 
  \caption{Gráfico de BowTie do texto \cite{melissaproject}}
  \label{bowtiegraf}
\end{figure}


Com o propósito de gerar reflexões no que tange aos riscos de dada atividade, o \textit{V3S} trabalha com o conceito de personagens virtuais em um ambiente. Os raciocínios a cerca destes personagens são feitos usando um formalismo matemático denominado por redes de Petri, ou - mais especificamente máquinas de estado \cite{v3sframework} tendo como base simular a complexidade, flexibilidade e variabilidade de comportamentos que podem ser verificados em um ser humano. Por conta disso, os cientistas desse estudo decidiram modelar esse comportamento usando sistemas multi-agentes, mas especificamente um \textit{framework} conhecido como MASVERP (tratado na seção \ref{agent}).

O \textit{V3S} tem como por finalidade providenciar um modelo que seja coerente, relevante, variado e eficiente em termos de cenário de treinamento com a finalidade de proporcionar atividades de aprendizagem. Esse modelo também apresenta um módulo que monitora cenários adaptativos conhecidos por \textit{HERA}.Em vez de interromper o usuário de forma sistemática a fim de explicar os erros dele, o framework possibilita que o agente cometa erros e observe suas respectivas consequências no mundo virtual. Portanto, a dinâmica do cenário adaptativo permite trazer situações de treinamento relevantes. 

\textit{HERA}, por intermédio de regras baseadas em modelos pedagógicos, fornece o respaldo ao instrutor. Esse retorno é feito por intermédio dos seguintes critérios pedagógicos; escala de modificação - amplia determinadas partes de um objeto com a finalidade de melhorar a visualização, reificação - verificar como o aprendiz lida com determinados conceitos e abstrações em termos concretos, restrições nos limites das ações do aprendiz que consiste em envio de mensagem ao agente quando ele comete sérios erros e superposição de informações se o aprendiz cometer e argumentar sobre as consequências das ações. \textit{HERA} é integrado ao módulo de reconhecimento que tem como por finalidade usar técnicas que permitem redefinir as relações de atividades usando a linguagem \textit{ACTIVITY-DL} se parametrizando nas açõe,s erros e violações. Essa parte do \textit{V3S} é capaz de distinguir entre os tipos de erros, que são: 1 - erros relacionados a atividades, 2 - erros relacionados ao ato de cumprir com o objetivo, 3 - erros de \textit{BATU}, 4 - erros de função e 5 - erros de ponto de vista.

De todos os modelos verificados neste estudo, o \textit{V3S} é o maior e o mais complexo. Isso não tira um potencial para análise comparativa sendo que o primeiro aspecto a ser verificado consiste numa análise sobre \textit{Domain Model} que define uma ontologia para representar todos os objetos, relações e ações.

Um dos termos presentes no \textit{V3S} é o \textit{V3S-Entity}. Esse vocábulo pode ser comparado, em partes, com o conjunto $ Entity $. A explicação disso se deve ao fato de que o \textit{V3S-Entity} possui uma relação denominada de \textit{is a} com a classe \textit{V3S-Object}. A relação \textit{is a} denota semanticamente que todo o elemento pertencente a \textit{V3S-Object} também é um elemento \textit{V3S-Entity} o que, na representação proposta neste estudo denota uma relação de herança "em termos de UML" ou uma relação de subconjunto $ \subset $. A figura presente em \ref{domainmodel} denota que o termo \textit{Valve} é um \textit{V3S-Object}. O termo \textit{Valve} que em português significa Válvula é algo que, para o modelo conceitual proposto nesse estudo, só pode ser definida como um artefato da classe $Artefact$. Logo, a relação $Artefact \subset Entity $ é equivalente a \textit{V3S-Object is a V3S-Entity}.

Os termos \textit{V3S-Object} $Artefact$ podem ser dados como semanticamente semelhantes (apesar de terem relações diferentes). Contudo o mesmo não pode ser feito para $Entity$ e \textit{V3S-Entity}, apesar do raciocínio presente no paragrafo anterior. Isso se deve ao fato de que \textit{V3S-Entity} não abrange os Agentes do modelo como é o caso de $Entity$. Outro ponto consiste na relação \textit{is a} entre \textit{V3S-Entity} com \textit{V3S-Action}. Antes se faz necessário verificar com qual termologia aquela expressão se assemelha no modelo proposto neste estudo. Para isso se faz necessário analisar a figura \ref{domainmodel} que apresenta \textit{Connect-On} como sendo um \textit{V3S-Action}, logo \textit{V3S-Entity}. É interessante observar que \textit{Connect-On} se relaciona tanto com o objeto \textit{Loading-Arm} bem como com objeto \textit{Valve}. Com base nisso, \textit{V3S-Action} pode ser usado para especificar os elementos de $Relation$. 

O segundo ponto a ser comparado reside sobre o modelo \textit{Activity Models} que basicamente consiste no uso da linguagem \textit{ACTIVITY-DL}. O que existe de equivalente a isso na representação proposta neste estudo são os objetivos descrito por $goal$ e o predicado $nextGoal(g_i,g_j)$ o que permite expressar raciocínios de eventos que podem acontecer tanto em série como em paralelo. Tendo em vista o fato de que \textit{ACTIVITY-DL} permite expressar os seguintes racicínios; tarefas independentes, tarefas sequenciais, tarefas que se iniciam mediante a uma solicitação (pedido ou ordem), tarefas paralelas, tarefas  simultâneas, tarefas que se iniciam no mesmo instante e tarefas que são encerradas no mesmo instante, é possível concluir que essa linguagem é mais que o suficiente para considerar todos os elementos atrelados a relações de objetivos levantados pelo modelo conceitual proposto nesse estudo.  

O \textit{V3S} aprofunda as análises de segurança em \textit{Risk Model}. Essa parte do \textit{V3S} podem ser comparadas com as consequências que os riscos ocasionam sobre os agentes. Isso, pois muitos conceitos considerados nas regras \ref{violationcondition}, \ref{violationrelation}, \ref{violationentity}, \ref{consviolationcondition} e \ref{consviolationrelation} tal como; riscos, consequências podem ser tratados pelo \textit{MELISSA}. Contudo, o \textit{V3S} não trata essas estruturas por meio de violações e sanções. 

Um ponto interessante do \textit{Risk Model} consiste em uma representação MASVERP  de agente \textit{BDI} com o foco em Risco (ver em \ref{agent}), pois \textit{V3S} permite especificar os estados internos do agente a fim de representar a condição emocional sobre a qual uma certo profissional é submetido. 
