\documentclass[openright]{normas-utf-tex} %openright = o capitulo comeca sempre em paginas impares

\special{papersize=210mm,297mm}

\usepackage[alf,abnt-emphasize=bf,bibjustif,recuo=0cm, abnt-etal-cite=2, abnt-etal-list=99]{abntcite} %configuracao correta das referencias bibliograficas.

\usepackage[brazil]{babel} % pacote portugues brasileiro
\usepackage[utf8]{inputenc} % pacote para acentuacao direta
\usepackage{amsmath,amsfonts,amssymb} % pacote matematico
\usepackage{graphicx} % pacote grafico
\usepackage{times} % fonte times
\usepackage[final]{pdfpages} % adicao da ata
\usepackage{float}
\usepackage{scalefnt}
\usepackage{algorithmic}% http://ctan.org/pkg/algorithms

\instituicao{Universidade Tecnológica Federal do Paraná} % nome da instituicao
\programa{Programa de Pós-graduação em Engenharia Elétrica e Informática Industrial} % nome do programa
\area{Informática Industrial} % [Engenharia Biom\'edica] ou [Inform\'atica Industrial] ou [Telem\'atica]

\documento{Dissertação} % [Disserta\c{c}\~ao] ou [Tese]
\nivel{Mestrado} % [Mestrado] ou [Doutorado]
\titulacao{Mestre} % [Mestre] ou [Doutor]

\titulo{Um Modelo Conceitual sobre Cenários de Acidentes em Atividades de Manutenção} % titulo do trabalho em portugues
\title{\MakeUppercase{Title in English}} % titulo do trabalho em ingles

\autor{Jonathan Morris Samara} % autor do trabalho
\cita{Morris Samara, Jonathan} % sobrenome (maiusculas), nome do autor do trabalho

\palavraschave{sistema multiagente, norma, sanção, violação} % palavras-chave do trabalho
\keywords{Keyword 1, Keyword 2, ...} % palavras-chave do trabalho em ingles

\comentario{\UTFPRdocumentodata\ apresentada ao \UTFPRprogramadata\ da \ABNTinstituicaodata\ como requisito parcial para obtenção do grau de ``\UTFPRtitulacaodata\ em Ciências'' -- Área de Concentração: \UTFPRareadata.}

\orientador{Cesar Augusto Tacla} 
\local{Curitiba} % cidade
\data{\the\year} % ano automatico

% desativa hifenizacao mantendo o texto justificado.
% thanks to Emilio C. G. Wille
\tolerance=1
\emergencystretch=\maxdimen
\hyphenpenalty=10000
\hbadness=10000
\sloppy

%---------- Inicio do Documento ----------
\begin{document}

\capa % geracao automatica da capa
\folhaderosto % geracao automatica da folha de rosto

% Lembre-se de que a ficha catalografica eh impressa no verso da folha de rosto
% Ficha catalografica
\fichacatpum{T137}
\fichacatautor{Morris Samara, Jonathan}
\fichacatpgbib{\pageref{bibstart}-\pageref{bibend}}
\fichacatpalcha{1. Sistema Multiagente. 2. Normas. 3. Sanção, Violação}
\fichacatpdois{CDD (22. ed.) 621.3}
\fichacatbib{Biblioteca xxxxxx}
\fichacat

% insercao da ATA
\includepdf{ata.pdf}

% dedicatoria
\begin{dedicatoria}
Texto da dedicat\'oria.
\end{dedicatoria}

% agradecimentos (opcional)
\begin{agradecimentos}
Texto dos agradecimentos.
\end{agradecimentos}

% epigrafe (opcional)
\begin{epigrafe}
Texto da ep\'igrafe.
\end{epigrafe}

%resumo
\begin{resumo}
Texto do resumo (m\'aximo de 500 palavras).
\end{resumo}

%abstract
\begin{abstract}
Abstract text (maximum of 500 words).
\end{abstract}

% listas (opcionais, mas recomenda-se a partir de 5 elementos)
\listadefiguras % geracao automatica da lista de figuras
\listadetabelas % geracao automatica da lista de tabelas
\listadequadros % adivinhe :)
\listadesiglas % geracao automatica da lista de siglas
\listadesimbolos % geracao automatica da lista de simbolos

% sumario
\sumario % geracao automatica do sumario

\setcounter{page}{12}
\chapter{Introdução}
	Hoje em dia é comum que diversas pessoas se submetam a atividades (na grande maior das vezes, profissionais) que trazem algum tipo de risco a vida delas. Trabalhos como de eletricistas, bombeiros, área petro-química, serviços de telecomunicações e de transporte (motoristas de ônibus, carro) são apenas alguns exemplos onde profissionais são  expostos a condições de risco.

Acidentes em situações assim ocorrem pelos mais diversos motivos. Investigas as cadeias de causalidade que resultam nessas situações podem trazer um entendimento de como lidar com circunstâncias assim. Consequentemente, isso traz um potencial para dominuir o número de ocorrências desse gênero. Assim sendo, todas as técnicas possíveis para 
investigar esse tipo de situação é algo de grande interesse para a comunidade em geral. 

A computação é uma ciência que apresenta um grande potencial de contribuição para esse problema. Isso envolve criar representações do mundo real que permitam efetivar algum tipo de raciocínio relevante sobre o assunto. Essa situação levanta diversas reflexões interessantes, que são; Qual paradigma (dos muitos) que pode ser usado para constituir situações assim? Quais modelos já existem para tratar esse tipo de problema e quais são as concepções que eles não conseguem tratar com eficiência? Qual é as raizes dada pela comunidade acadêmica para os conceitos que podem ser usados para tratar esse tipo de problema? Existe a necessidade de criar algum conceito para tratar 
alguma problemática? Quais são as regras de inferências que podem ser construídas para tratativa de questões assim? 

Nesse estudo em específico, os pesquisadores têm o interesse de investigar essa questão usando sisteas multiagentes normativos. A seção \ref{motivation} apresenta os motivos pelos quais é razoável fazer uso de sistemas multiagentes normativos para estudar questões envolvendo riscos e acidentes de trabalho. O entendimento do porque 
esse trabalho é importante é algo que será feito em \ref{relevance}. Os objetivos desse estudo estarão sintetizados na seção \ref{goals}.

	\section{Motivação}	\label{motivation}
		Nesse estudo os pesquisadores assumem que um acidente acontece porque alguém cometeu algum erro. Não é possível afirmar que essa proposição é verdade para todos os casos de acidente, contudo é possível verificar que existe situações de acidentes que se enquadram nessa condição. Acidentes de carro são exemplos disso, pois ocorrem (em muitos casos) porque pelo menos um condutor cometeu algum erro. 

Os pesquisadores entendem que uma representação baseada em sistemas multiagentes normativos é apropriada para tratar situações assim. Modelos baseados nesse tipo de representação tratam os seguintes conceitos; normas (instruções claras que devem ser fornecidas a um agente), violação (ocorre quando um agente descumpri uma norma) e sanção (consequências decorrentes de uma violação desta norma) \cite{dastaniframework} \cite{amodelmultiagentsystemdynamicrelationship}, \cite{ontologynormative}. Nessa linha de raciocínio é possível conceber os profissionais como os agentes, o que eles devem fazer é possível representar por meio de normas, os erros são as violações e as consequências são dadas por sanções.  

No que tange a investigação acadêmica, a motivação para realizar esse trabalho reside em entender como toda estrutura conceitual na área de agentes normativos (concebida pela comunidade acadêmica) pode ser usada para representar cenários de atividades que definem um potencial de acidentes. Outra motivação reside verificar quais são as vantagens e desvantagens perante os modelos que são apropriados para essa condição. Além desses duas motivações, é de grande interesse acadêmico analisar os raciocínios que podem se construídos e qual é o potencial desses no que diz respeito a correspondência com a realidade dos fatos. 
	\section{Relevância} \label{relevance}
		Esse estudo apresenta contribuições para três grandes campos (sendo que dois são da computação) do conhecimento. O primeiro campo é a representação computacional. Isso se da pelo fato de que esse estudo se propõem a conceber uma representação que usa conceitos e estruturas (norma, violação e sanção) para resolver problemas envolvendo situações que apresentam potencial para acidente. Nesse aspecto existe duas condições a serem consideradas; a primeira situação consiste nos modelos que não são estruturados para representar cenários específicos de acidentes porém que podem ser usados para essa finalidade. Nessa situação a inovação e relevância desse estudo reside em aproveitar estruturas desses modelos para construir vocábulos e regras específicas para representar cenários dessa categoria. A segunda situação consiste em analisar modelos cujo foco implica representar cenários de acidentes. Nessa condição esse estudo é relevante porque aborda formas diferentes de fazer isso, o que gera consequências diferentes onde cada conjuntura específica apresenta vantagens e desvantagens únicas.    

O segundo campo para o qual esse estudo é relevante são Sistemas Multiagentes. Existe diversos modelos e aplicações onde esse tipo de sistema é interessante e isso está atrelado a um espectro que vai desde a criação de novas tecnologias a entendimento das dinâmicas de seres humanos em termos computacionais. Assim sendo, esse estudo apresenta uma forma específica de \textit{SMA} normativos a fim de entender, em termos computacionais, a dinâmica de trabalhadores que são expostos a riscos e acidentes. Como será tratado no texto adiante, existe modelos de \textit{SMA} que têm esse propósito. Contudo o modelo apresentado nesse estudo possui características específicas justamente por se basear em agentes normativos e por trazer regras específicas de violação e sanção para questões atreladas a acidentes tendo em vista análises e observações dos pesquisadores (sendo que essas análises foram feitas em estudos relacionados a temática de agentes e relacionados a acidentes de trabalho). É digno de nota que esse estudo não se preocupa em conceber representações que lidam com aspectos internos dos agentes, mas sim com a dinâmica sistêmica dos mesmos.

Entendimento de como se dá a segurança do trabalho é o terceiro campo para o qual esse estudo gera contribuição. O interesse desse texto é primeiramente computacional e não entra no mérito da ciência da Segurança no trabalho propriamente dito. Isso pois o interesse dos pesquisadores está relacionada a construir um modelo usando, em partes, os conceitos presentes nessa área o que não envolve elaborar um novo conceito de segurança propriamente dito. Contudo, o fato de conceber um sistema computacional para lidar com riscos e situações inesperadas em condições de trabalho consiste em um dialogo com a área de segurança por fornecer uma maneira formal de lidar com situações assim. Nessa situação, a relevância nessa área não é tão significativa como nos campos anteriores, contudo não é desprezível e merece ser notada.
	\section{Objetivos} \label{goals}
		Como objetivo geral os pesquisadores propõem, dentro do contexto computacional, um modelo capaz de representar cenários onde os agentes podem cometer erros durante a execução de algum atividade. Tendo em vista esses erros, eles podem prejudicar a si mesmos como seus colegas. Assim sendo, esse estudo investiga esse cenário por meio de um modelo que incorpora normas (instruções claras que devem ser fornecidas a um agente), violação (ocorre quando um agente descumpri uma norma) e sanção (consequências decorrentes de uma violação desta norma). Não apenas isso, mas nesse cenário há a consideração de que essas questões estão atreladas aos artefatos que são usados pelos agentes. Esses artefatos são objetos passivos que estão sujeitos a influencia da ação dos agentes. Os artefatos são usados para representar objetos como ferramentas e máquinas, por exemplo. 

Os pesquisadores consideram, neste estudo, que as violações não estão relacionadas apenas aos artefatos, mas também nas relações entre todas as entidades existentes ao longo do cenário. As entidades compõem todos os elementos (agentes e artefatos) vinculados ao meio. As condições ambientais onde os agentes e os artefatos estão inseridos também são considerados neste estudo. O entendimento que se tem por condição ambiental consiste em um dado processo que existe no ambiente e que, de certa forma, pode influenciar a execução das atividades dos agentes. Como condições ambientais têm-se por exemplo os seguintes eventos; chuva, sol, vento e neve. 

Não é do interesse deste estudo investigar condições normativas cuja caráter punitivo advém de sanções administrativas e jurídicas que ocorre no que tange a decisão de uma certa autoridade. Assim sendo, o modelo resultante deve ser capaz de representar situações onde os agentes sofrem consequências físicas negativas resultantes de uma cadeia de causalidade que foi ocasionada pelo erro de alguém (usando, para isso, o conceito de norma, violação e sanção). Dada essa circunstância, a representação tratada neste estudo deve trabalhar, em sua estrutura interna, o conceito de risco. É digno de nota que o termo risco apresenta um espectro semântico bastante amplo podendo ser usado nos mais diferentes contextos possíveis (ex. risco financeiro de um dado ativo). Por conta disto, é necessário frisar que neste texto o vocábulo risco é usado no mesmo significado atribuído pela comunidade de Engenharia de Segurança. 

Neste estudo, os pesquisadores têm interesse em apresentar um vocabulário específico no que tange ao objeto em interesse a ser representado. Esse modelo, portanto, deve ser capaz de representar organizações tais como trabalhadores em uma obra, industrias, profissionais no âmbito hospitalar e estruturas deste gênero entendendo como se dá as violações em âmbitos específicos (falta de ferramentas, não conseguir executar procedimentos apropriados, executar uma atividade cujo momento não era adequado para isso). 

Para o propósito aqui posto, se não há infinitas maneiras, há ao menos um numero muito variado de formas para construir um modelo com os objetivos aqui definidos. Contudo, esse texto desbrava ao menos uma das formas de se realizar isso, analisa a abordagem sobre um estudo e define uma comparação com os modelos já existentes dada pela comunidade acadêmica. A análise comparativa é estruturada em termos de; conceitos (verificar quais são similares e quais são diferentes), relações entre os conceitos, capacidade de generalização (quanto mais mundos possíveis o modelo é capaz de descrever, mais genérico o é), capacidade de especificação tendo como referência circunstâncias que correspondem ao presente neste texto introdutório e uma verificação sobre como esses critérios se dão dos modelos em relação a um dado estudo de caso aqui presente. 
\chapter{Fundamentação Teórica}
\label{chap:fundteoric}
	Este capítulo apresenta os estudos que fundamentam essa pesquisa. Isso consiste em todos os estudos acadêmicos que dão base para os vocábulos e regras que estruturam o modelo aqui proposto. Esses estudos abrangem as seguintes áreas; Agentes, Artefatos, Sistemas Multiagentes, Normas (em conjunto com Violações e Sanções), Riscos e Lógica Modal. 
	\section{Agentes} \label{agent}
		Não existe uma definição universal para tratar o conceito de agente sendo que esse tópico se encontra em meio a debates e controvérsias. Contudo, existe um entendimento generalizado de que um comportamento \textit{autônomo} é o cerne de noção que se tem por agente \cite{whatisagent}. Apesar disso, a construção de um modelo computacional não pode ser feita sem uma definição. Assim sendo, nesse texto um agente é um sistema computacional que está situado em um dado ambiente e que apresenta comportamento autônomo \cite{whatisagent}. Além disso um agente faz uso do seu comportamento autônomo com o propósito de atingir objetivos que a ele é designado \cite{whatisagent}.

Como a definição de agente faz uso do conceito de ambiente, não é possível ter uma compreensão plena daquele sem considerar aspectos deste. Assim sendo, o termo ambiente  é aquilo que apresenta as propriedades definidas na lista a seguir \cite{artificialinteligencemodermapproach} \cite{whatisagent}; 
\begin{itemize}
    \item \textit{Acessibilidade vs Inacessibilidade}; Um ambiente acessível é aquele em que um agente consegue ter informações claras, precisas e atualizadas no que tange a característica do ambiente.
    \item \textit{Determinístico vs Não-Determinístico}; Um comportamento determinístico é aquele em que uma ação possui um efeito claro e garantido, sem incertezas sobre o estado que irá resultar.
    \item \textit{Episódico vs Não-Episódico}; Um ambiente tende a ser o mais episódico possível tanto quanto o desempenho do agente estiver associado a um episódio discreto e específico no ambiente.
    \item \textit{Estático vs Dinâmico}; Um ambiente é estático se não houver outros processos em parapelo aos eventos associados ao agente.
    \item \textit{Discreto vs Contínuo}; Um ambiente é discreto se existeir um número finito de ações e percepções. 
\end{itemize}

Outro aspecto que está presente na definição de agente é o termo autônomo. Esse termo pode ser compreendido pela seguinte proposição; Uma entidade que possui essa natureza é uma entidade que tem a capacidade de agir por si mesma. Essa entidade não precisa de nenhum outro fator externo (ex. ser humano) para efetuar as suas decisões \cite{whatisagent}.

Há uma série de exemplos que se enquadram dentro dessa definição. Um termostato é um sistema de controle que está em um dado ambiente, como um quarto ou uma sala, \cite{whatisagent}, que gera dois sinais de saída, (um desses sinais indica que a temperatura está ou baixa ou alta demais dependendo da aplicação, e o outro, demonstra que a temperatura está no nível aceitável). O comportamento autônomo do termostato é baseado nas seguintes regras;

\begin{itemize}
    \item Se a temperatura estiver abaixo do nível de temperatura definido, então ligar o atuador.
    \item Se a temperatura estiver dentro do nível estabelecido, então desligar o atuador.
\end{itemize}

Outro exemplo de agentes consiste nos programas \textit{Daemons} em sistemas \textit{UNIX}. Esses algoritmos trabalham em segundo plano e monitoram um dado ambiente de \textit{software}. Com base em certas regras, na ocorrência de um certo evento no ambiente, esses programas realizam uma determinada atuação \cite{whatisagent}.   

Os exemplos presentes neste texto são apropriados dentro do conceito de agentes. Contudo, esses exemplos não estão de acordo com a definição de agentes inteligentes \cite{whatisagent}. Uma entidade que se enquadra dentro das características de um agente inteligente deve necessariamente respeitar a definição já apresentada e deve apresentar as seguintes propriedades; reatividade (capaz de perceber as mudanças
que ocorrem no ambiente e responder a elas de maneira apropriada no que condiz aos objetivos do agente), proatividade (apresenta comportamento orientado a objetivos sendo que o agente toma as decisões a fim de satisfazer os objetivos em interesse) e habilidades sociais (capacidade de interagir com outros agentes a fim de poder satisfazer os próprios objetivos) \cite{whatisagent} \cite{artificialinteligencemodermapproach}.

Os agentes podem ser definidos em categorias. Uma dessas consiste nos agentes que são puramente reativos. Esses agentes tomam decisões considerando apenas informações que estão no instante presente. Por consequência, o comportamento deles ocorre por respostas diretas ao ambiente \cite{whatisagent}. 

Os agentes que possuem estados estão contidos em outra categoria. Esses agentes possuem uma dada estrutura interna de dados que são consideradas em tomadas de decisão \cite{whatisagent}.

Uma outra maneira de analisar um agente se dá por meio das arquiteturas e modelos disponíveis para representar seus estados internos. Para o propósito do estudo que está sendo apresentando neste texto, é o suficiente considerar de forma sucinta quatro dessas arquiteturas. A primeira consiste nos agentes baseados em lógica que realizam deduções em função de uma decisão \cite{logicagent}, a segunda arquitetura consiste nos agentes reativos que tomam decisões com base em um mapeamento de uma certa situação em uma  ação \cite{reactiveagent}, a terceira arquitetura é \textit{BDI} cujo comportamento ocorre da manipulação de estruturas de dados que representam crenças, desejos e intenções do agente \cite{bdi} e a quarta arquitetura consiste em uma estrutura em camadas onde a tomada de decisão acontece por intermédio de diversos níveis de abstração a cerca do ambiente \cite{layeragent} \cite{whatisagent}.  

Existe um modelo de agentes que faz uso da arquitetura BDI com propósito de representar situações de risco. Esse modelo é denominado por \textit{MASVERP} que define o agente como tendo habilidades e objetivos a serem atingidos. Não só isso, mas o agente também pertence a certo círculo social. O autor deste estudo implementou o modelo \textit{COCOM} que define parâmetros físicos e cognitivos a fim de que os agentes articulem suas ações no que tange a execução de objetivos \cite{mavesp}. Esse modelo considera situações que delimitam quando um agente não tem condições de desempenhar uma certa atividade. Essas são; fome, sede, cansaço físico, carga cognitiva, cansaço em relação a vigilância, estresse, motivação para determinar agitação, motivação e condições regulares \cite{mavesp}. 

O \textit{MASVERP} incorpora o conceito de \textit{BTCU} (será tratado na seção \ref{risksec}). Em termos genéricos, \textit{BTCU} consiste em condições limites que são aceitas para que um agente possa realizar a atividade. No \textit{MASVERP} a decisão do agente no que tange a executar ou não uma atividade é parametrizada com base no \textit{BTCU}. 
	\section{Artefatos} \label{artefact}
		\cite{cartago}
\cite{programingagentartefact}

\begin{itemize}
    \item Definição de Artefatos no contexto de Sociedade Multiagente
\end{itemize}
	\section{Sistema Multiagente} \label{sma}
		Um sistema multiagente(SMA) organizado é aquel constituido por agentes autonomos que interagem visando um propósito em comum tendo como consequência um comportamento global \cite{moiseframework} 
\cite{organiationofmultiagentsystem}. Assim sendo, uma organização com essas características deve ser capaz de manifestar conhecimento em comum, cultura, memória, história, distribuição de atividades 
e a capaciade de distinguir um  agente em espeçifico \cite{organiationofmultiagentsystem}. Deste fato é possível identificar o fenômeno "supra-individual" que implica em um comportamento que existe
além dos comportamentos e atributos particulares no que diz respeito as entidades constituintes do sistemas. 

Uma organização de um sistema multiagente deve conter relações sociais no que tange a agentes, institutos e grupos sociais \cite{organiationofmultiagentsystem}. Ainda sobre isso, uma organização 
\textit{SMA} deve apresentar uma \textit{extensão de um espaço abstrato}. Isso implica uma representação dos seguintes conceitos; estrutra espacial, estrutura temporal, símbolos, semântica e 
capacidade de dedução. Há organizações que não se enquadram em todas essas restrições, contudo são suficientes para tratar o problema dentro de uma perspectiva computacional \cite{organiationofmultiagentsystem}.

O subtópico \textit{Conceitos Gerais de uma Organização de Siastems Multiagentes} tem como por finalidade detalhar melhor os elementos presentes da Teoria da Organização dentro do contexto de SMA
em relação a esse estudo. Já o Subtópico \textit{Formalização de Conceitos Específicos para SMA} tem como por objetivo realizar uma verificação analítica dos elementos presentes no modelo de \textit{SMA} 
denomiando por \textit{MOISE+}. Os conceitos que serão analisados são; objetivos, planos e papeis \textit{organiationofmultiagentsystem}.
   

\subsection{Conceitos Gerais de uma Organização de Sistemas Multiagentes}

A finalidade desta subseção consiste trabalhar com uma maior riqueza de detalhes todos os conceitos que constituem a ideia de uma organiozação de um sistema multiagente.  
  
\textbf{Divisão em tipos de atividades:} Uma organização não é uniformemente estruturada. Isso, pois as atividades são distribuidas de forma desigual entre as diferentes entidades.
Dentro do ponto de vista fenomenologico as atividades são sujeitas a classificação e ocorrem com diferentes frequências e em diferentes regiões dentro das definições espaciais da organização \cite{organiationofmultiagentsystem}.

\textbf{Integração:} Dentro de uma organização ocorre a presenção de interdependência entre diferentes espaços de ativiades. Essas, por sua vez, estão relacionadas em uma estrutura única definida
dentro de um contexto alinhado e integrado \cite{organiationofmultiagentsystem}.

\textbf{Composição} Uma organnização é composta por elementos menores. No caso dos multiagentes, os elementos atomicos que estruturam a organização são os agentes \cite{organiationofmultiagentsystem}.

\textbf{Estabilidade/Flexibilidade:} Uma organização apresenta padrões de atividades. Esses padrões possum cateristicas que podem ser enquadradas em dois aspectos; estaveis e flexiveis. 
No que tange as características estaveis, essas são constituidas por elementos/processos que definem o padrão em sí mesmo. Em constraste com isso um comportamento flexível acontece quando o 
sistema é submetido a situações incomuns \cite{organiationofmultiagentsystem} \textbf{?}.

\textbf{Coordenação:} Todo sistema é dependente de algum dado recurso. Assim sendo, se faz necessário que esse recurso seja utilizado de forma inteligente a fim de que possa se manter ao longo 
do tempo. Para que isso, se faz necessário que a organização se comporte como uma amplificadora de recursos a fim de que as estruturas operacionais tenham um comportamente cada vez mais organizado 
\cite{selforganization}, \cite{selforganizatioenvoriment}, \cite{defintionselforganization}. Contudo as incertezas relacioandos aos efeitos combinados resultam
influenciam negativamente nas eficiencias. Portanto, para manter a eficiência organizacional se faz necessário a existência de elementos otimizadores sobre os padrões de atividades \cite{organiationofmultiagentsystem}.

\textbf{Recursividade:} Uma organização é constituída por sub-organizações. Isso ocorre em multiplos níveis de estrutura e se dá por intermédio de um padrão recursivo \cite{organiationofmultiagentsystem}.

\textbf{Representação Multi-Nível e Causaldiade:} Uma organização é estruturada por suborganizações em diferentes níveis estruturas. Isso, por sua vez, resultam em atividades ocorrendo em 
diferentes escalas espaciais, temporais e estruturais. Como consequencia disto, as cadeias causais presentes em estruturas organizacionais são processos multi-níveis \cite{organiationofmultiagentsystem}.

\textbf{Potenciais e Diferenciais:} Diversos são os sistemas físicos onde as forças entre partículas são decorrentes de balanços de potenciais. Como esse comportamento está presente em diversos
sistemas físicos, existe modelos abstratos de sistemas auto-organizaveis que levam em consideração a presença de forças potenciais ediferenciais em organizações \cite{selforganizationdiffforce}. 
Esse conceito é trabalhado dentro de sistemas multiagentes. Um exemplo notório a respeito disto consiste no conceito de \textit{Poder} o qual é entendio como a capacidade de influenciar uma
dada organização \cite{organiationofmultiagentsystem}

\textbf{Regras e Gramáticas:} Organizações podem ser compreendidas como potenciais configurações de atividades e processos. Essas configurações podem ser descritas usando gramática \cite{grammarselforganizationmodel} 
\cite{grammarselforganizationmodel2}. Tanto as gramáticas como as regras que compõem um organização apresentam três interpretações, essas são; como estruturas (especificações procedurais do que deve 
ser feito), como coação as ações defindo o que pode e não pode ser feito e como um compilado das experiênicas \cite{organiationofmultiagentsystem}.

\textbf{Incerteza:} Não é possível conceber o conceito de uma organização sem ao menos entende-la como uma estrutura que distribui informação em sí mesma. Sobre essa ótica, a distrbuição de informação
inequivocamente implica geração de incerteza o que por sua ver se manifesta como um complicante no que tange a comunicação entre as partes bem como a atividade organizacional em si mesma.

\subsection{Formalização de Conceitos Específicos para SMA}
\label{moiseformalizesma}
A apresentação desses conceitos será feita por intermédio de estudos relacionados ao \textit{MOISE+}. Apesar de ser um \textit{framework}, o \textit{MOISE+} trata a rigor acadêmico na ótica
da computação clássica a tratativa dada para os conceitos em interesse a esse estudo. Assim sendo, um estudo aprofundado do modelo, bem como dos textos em referência, satisfaz com exclência 
os fundamentos teóricos para as análises em interesse. 

A constitução do \textit{MOISE+} é estrutura em três categorias de especicação, essas são; estrutural, funcional e deontica. O texto a seguir exibe com maior requize de detalhes cada uma dessas
especificações.  

A especificação estrutural acontece em três níveis, individual, social e coletivo. O nível individual trata de definir os papeis $\rho$ dos agentes. Uma possível entre os papeis acontece por 
intermédio da hereditariedade em que se $\rho'$ é filho de $\rho$. Isso implica afirmar que $\rho'$ é uma especialização de $\rho$. Um exemplo apropriado para isso é o jogo de futebol onde 
existe o papel jogador dado por $\rho$ e existe o papel atacante dado por $\rho'$ \cite{moiseframework} \cite{roleone} \cite{roletwo} \cite{dynamicagenttemporalstruct}. Em termos formais, essa relação é dada pro; 

\begin{eqnarray}\nonumber
\rho_a \sqsubset \rho_b
\end{eqnarray}

O nível social estabelece relações de ligação dado pelo predicado $link(\rho_s,\rho_d,t)$. Existe três possíveis valores para $t$, os quais são $t = \{aut, com, acq\}$. O valor $auth$ significa 
autoridade (neste caso $\rho_s$ exerce autoridade sobre $\rho_d$), o valor $com$ significa comunicação (neste caso $\rho_s$ pode se comunicar com $\rho_d$) e o valor $acq$ significa conhecimento 
($\rho_s$ tem conhecimento da existência de $\rho_d$) \cite{moiseframework} \cite{moiseframeworktwo} \cite{dynamicagenttemporalstruct}. O MOISE+ define as seguintes relações de implicabilidade

\begin{eqnarray}\nonumber
	link(\rho_s,\rho_d,auth) \to link(\rho_s,\rho_d,com) \nonumber \\
	link(\rho_s,\rho_d,com) \to link(\rho_s,\rho_d,acq) 
\end{eqnarray}

O modelo também determina como se dá as relações de hereditariedade para o predicado de $link$, é dado por \cite{moiseframework} \cite{dynamicagenttemporalstruct}; 

\begin{eqnarray}\nonumber
	link(\rho_s,\rho_d,t) \wedge \rho_s' \sqsubset \rho_s' \to link(\rho_s',\rho_d,t) \nonumber \\
	link(\rho_s,\rho_d,t) \wedge \rho_d' \sqsubset \rho_d' \to link(\rho_s,\rho_d',t) 	
\end{eqnarray}


O nível coletivo determina a existência de compatibilidade entre os papeis \cite{moiseframework}. Essa é uma relação reflexiva e transitiva de determina que se um papel $\rho_a$ possui a 
capacidade de realizar um determinado objetivo, então o papel $\rho_b$ também tem essa capacidade. Em termos formais, essa relação se dá da seguinte forma \cite{moiseframework} \cite{deonticOne}.;

\begin{eqnarray}\nonumber
	\rho_a \bowtie \rho_b \wedge \rho_a \neq \rho_b \wedge \rho_a \sqsubset \rho' \to \rho' \bowtie \rho_b 
\end{eqnarray}

O nível coletivo também apresenta o conceito de grupo dado por $gt$ e constituído por;

\begin{eqnarray}\nonumber
	gt = \langle R,SG,L^{intra},L^{inter},C^{intra},C^{inter},np,ng\rangle 
\end{eqnarray}

Em que $R$ é o conjunto dos papeis não abstratos, $SG$ são subgrupos que estão contidos neste grupo, $L^{intra}$ consiste dos $links$ intra-grupos, $L^{inter}$ dos links inter-grupos, 
$C^{intra}$ das relações de compatibilidade intra-grupos e $C^{inter}$ das relações de compatibilidade inter-grupos. O símbolo $np$ denota a cardinalidade mínima e máxima para uma dada 
função e o símbolo $ng$ realiza o mesmo para os subgrupos \cite{moiseframework}. 

A Especificação Funcional tem como por finalidade descrever os objetivos a serem atingidos dentro de uma estrutura de árvore. A figura a seguir define como se dá esse tipo de especificação; 

\begin{figure}[H]
  \centering
  \includegraphics[width=0.8\linewidth]{figure/figmoise} 
  \caption{Arvore de objetivos definido pelo modelo Moise \cite{moiseframework}}
  \label{arvoremoise}
\end{figure}

A figura \ref{arvoremoise} define três tipos de relação de subobjetivos; $sequence$ onde todos os subobjetivos devem necessariamente ser concluídos em sequência, $choice$ onde o agente tem a 
possibilidade de escolher qual objetivo ele deseja seguir e $parallelism$ onde todos os objetivos devem ser concluídos, contudo sem uma sequência definida \cite{taems01} \cite{taems02}. Esta
parte do modelo é baseada em um \textit{framework} de distrubição de atividades denominado por \textit{TAEMS} \cite{TAEMS}.
Como é possível observar na figura, os objetivos são agrupados em conjuntos de missões $m$ \cite{dynamicagenttemporalstruct}. A relação a seguir define isso melhor;

\begin{eqnarray}\nonumber
	m_k = \{ g_n,...,g_m\}
\end{eqnarray}


A Especificação Deôntica define predicados para estabelecer permissões e obrigações entre os papeis e as missões. Toda obrigação implica necessariamente em uma permissão. A relação a seguir 
estabelece isso \cite{moiseframework} \cite{deonticOne}; 

\begin{eqnarray}\nonumber
	obl(\rho,m,tc) \to per(\rho,m,tc) \\
	obl(\rho,m,tc) \wedge \rho \sqsubset \rho' \to obl(\rho',m,tc) \\
	per(\rho,m,tc) \wedge \rho \sqsubset \rho' \to per(\rho',m,tc) \\	
\end{eqnarray}

Onde o predicado $obl$ define uma obrigação e o predicado $per$ define permissão. O argumento $tc$ define uma periodicidade de tempo para o qual a relação deôntica é valida. 

	\section{Normas} \label{normasdastani}
		\cite{amodelmultiagentsystemdynamicrelationship}
\cite{modelingnormsforautnomousagent}
\cite{dastaniframework}
\begin{itemize}
    \item Definir o que é norma
    \item Tratar os tipos de normas
    \item Tratar o conceito de agentes normativos
    \item Tratar o conceito de sistemas multiagentes normativos 
    \item Apresentar o modelo do Dastani
    \item Enfatizar como ele apresenta os conceitos de violação e sanção
\end{itemize}
	\section{Riscos} \label{risksec}
		O primeiro estudo teórico sobre acidentes de trabalho se da no texto \cite{riskoldschool}. Essa pesquisa conclui que os erros em industria não podem ser definidos apenas nas falhas de 
humanos, mas sim como consequência de um comportamento global da instituição. Ainda dentro deste âmbito, esse comportamento advêm de uma forte pressão tendo em vista eficiência e otimização dos processos de produção \cite{riskoldschool} \cite{safety}.

Com base neste entendimento, o estudo \cite{safety} apresenta um \textit{framework} a fim de identificar as redes de causalidade que resultam em acidentes de trabalho. Assim sendo, a gestão de 
segurança se dá com base nos seguintes fatores; 
\begin{itemize}
    \item Políticas; \textit{leis, diretrizes, padrões e regras}
    \item Corporativo; \textit{regras, estratégias, politicas internas, gerenciamento}
    \item Projeto de Equipamentos de Trabalho; \textit{especificação, integração de segurança}
\end{itemize}

O item Política é mais relevante que os itens Corporativo e Projeto de Equipamentos de Trabalho. Esses dois últimos apresentam a mesma importância para uma estrutura de prevenção de acidentes 
bem sucedida. 

A primeira análise a ser feita diz respeito ao nível Corporativo-Projeto de Equipamentos de Trabalho. Muitas vezes a equipe adota atividades paliativas a fim de otimizar os processos de produção. Isso envolve assumir níveis de tolerância no que diz respeito ao desempenho e a segurança. Essa situação está dentro do conceito, para o \textit{framework}, de \textit{atividades
limites}, isso pois tratam de situações que trabalham no limiar com os riscos. Assim sendo, as decisões feitas pelo profissional podem resultar muito facilmente em acidentes ou incidentes \cite{safety}. 
Dentro desta perspectiva que se apresenta o ator \textit{BATU} - \textit{Boundary Activities Tolerated during Use} (Atividades Limites Toleradas Durante o Uso).

Existe dois tipos de \textit{BATU} que devem ser verificados tendo como base os processos de trabalho. Esses são; operacional e gerencial (administrativo - termo identificado no texto original; \textit{managerial}). Aquele faz referência as atividades relacionadas a melhoria da produtividade com o propósito de resultar em aumento das metas produção, qualidade e segurança. Este diz respeito a decisões administrativas independentes dos processos operacionais mas que os impactam.

Outro conceito presente em \cite{safety} é o de \textit{Boundary Conditions Tolerated by Use} - \textit{BCTU} (Condições Limites Toleradas Durante o Uso). O termo condição faz referência a uma situação, um estado, circunstâncias externas às quais pessoas ou até mesmo entidades são afetados no que diz respeito a uma certa decisão. Assim sendo, \textit{BCTU} consiste em uma série de elementos e circunstâncias (ambiental, material, humana, produtos) que por conta de sua natureza ou de como se relaciona com as demais entidades e processos apresenta um certo potencial na geração de situações particulares, tendo em vista causas decorrentes de operações dinâmicas. Tanto os \textit{BATUs bem como os BCTUs} não podem ser analisados diretamente, mas devem ser analisados por intermédio das ações e escolhas dos operadores e dos atores que constituem esse trabalho \cite{safety}. 

Existe dois tipos de \textit{BCTU}. O primeiro consiste no \textit{BCTU} interno que se apresenta como uma concepção global de trabalhos e situações no que tange as relações de política da empresa. Nesta concepção, \textit{BCTU} interno faz referência as diferenças hierarquias em termos de nível e decisões centrais. Em contraste com esse ponto, o \textit{BCTU} externo aponta para o projeto da instalação. Como resultado, há o surgimento de quatro derivações, que são; soluções de segurança - funções de segurança (diz respeito as questões que podem fazer com que um dispositivo de segurança venha a falhar), soluções técnicas - requisições de trabalho (quando as soluções técnicas são incompatíveis com as requisições de trabalho), modelo de projeto - modelo de instalação (se da quando a solução final não é ótima ou está degradada quando comparada com a solução inicial) e condições nominais preventivos - condições reais de operação
\cite{safety}.

As relações entre \textit{BATU}s e \textit{BCTU}s são dinâmicas e são dependentes do processo. Para exemplificar, pode-se considerar o seguinte cenário; O projeto de uma máquina de dobra de papel obriga o operador a adotar uma dada posição que o faz assumir riscos para acessar determinados pontos da máquina.  Assim sendo, as escolhas do projeto da máquina (relacionada ao \textit{BCTU}) não levam em consideração todos os aspectos relacionados a dinâmica profissional-máquina fazendo o que o profissional envolvido tenha que atuar dentro de um certo intervalo de tolerância no que diz respeito a segurança profissional \textit{BATU}.
		
	\section{Lógica Modal} \label{logic}
		A lógica modal consiste em uma linguagem para tratar proposições que necessariamente ocorrem e proposições que possivelmente ocorrem. As proposições dadas como necessárias são aquelas que necessariamente são verdade. Por exemplo; A água sobre 1 atm e entre 0,1 ºC - 99 ºC se apresenta no estado líquido. O conceito de possibilidade é totalmente dependente do conceito de necessidade. Isso pois uma proposição possível é aquela que necessariamente não é falsa \cite{modallogic}. 

A lógica modal é do tipo \textbf{K} e isso significa que nela está condita símbolos $ \sim $ para não, $ \rightarrow$ para "se ... então" e $\Box$ para "Isto é necessário". 

De \textbf{K} e $\Box$, tem-se as seguintes regras;

Sendo que $isTheorem(A,\textbf{K})$ representa "Se A é teorema de \textbf{K}". 

\begin{equation} 
isTheorem(A,\textbf{K}) \rightarrow \Box A
\end{equation}
\label{ktheorema}

\begin{equation} 
 \Box (A \rightarrow B) \rightarrow (\Box A \rightarrow \Box B) 
\end{equation}
\label{boxdist}

O operador $\Diamond$ apresenta o seguinte correspondente semântico; "Isto é possível". A relação entre $\Box$ e $\Diamond$ é dada pela regra que se segue.

\begin{equation} 
 \Diamond A = \sim\Box\sim A
\end{equation}
\label{dianotboxnota}

As relações a seguir apresentam outras regras válidas para essa lógica;

\begin{equation} 
 \Box (A \wedge B)  \rightarrow \Box A \wedge \Box B
\end{equation}
\label{boxand}

\begin{equation} 
 \Box A \vee \Box B \rightarrow \Box (A \vee B)
\end{equation}
\label{boxaor}

\begin{equation} 
 \Box A \rightarrow A
\end{equation}
\label{boxtoa}

\begin{equation} 
    \Box A \rightarrow \Box\Box A
\end{equation}
\label{aboxbox}

\begin{equation} 
    \Diamond A \rightarrow \Box\Diamond A
\end{equation}
\label{diaaboxdiaa}

\begin{equation} 
    \Box\Box...\Box = \Box
\end{equation}
\label{alotbox}

\begin{equation} 
    \Diamond\Diamond...\Diamond = \Diamond
\end{equation}
\label{diamont}

\begin{equation} 
    A \rightarrow \Box\Diamond A
\end{equation}
\label{diamont}
\chapter{Metodologia}
\label{chap:metod}
	Esse capítulo apresenta todos os recursos, técnicas, estratégias e esforços que foram feitos pelos autores com a finalidade de atingir o objetivo dessa pesquisa. As etapas adotadas pelos autores são definidas da seguinte maneira; fazer uma revisão exploratória em conjunto com de campo de uma dada atividade que se enquadre com os critérios dessa pesquisa, formalizar em um modelo conceitual, realizar inferências para avaliar os cenários e explorar os arcabouços possíveis.

    \section{Revisão Exploratória e Análise de Campo} 
        Para realizar um levantamento dos conceitos que estão atrelados a atividades envolvendo acidentes, se faz necessário analisar uma atividade onde esse tipo de situação acontece. Para isso, os pesquisadores optaram por estudar manutenção em linha viva onde eletricistas executam atividades preventivas e corretivas em equipamentos elétricos energizados. Esses profissionais são submetidos a riscos de serem eletrocutados e, consequentemente, mortos. 

A análise das atividades se deu por meio dos seguintes pontos: estudo dos manuais técnicos privativos a uma companhia de energia, análise (teórica e prática) das ferramentas usadas na execução dessas atividades, conversas com profissionais que atuam diretamente na área de manutenção, entrevista com engenheiro de manutenção em linha viva de uma companhia de energia, acompanhamento de um procedimento de manutenção em linha viva (onde o pesquisador esteve em uma certa subestação verificando de perto a execução dos procedimentos), criação de textos descrevendo cenários de manutenção e solicitando a correção por profissionais especializados na área. Além disso, a equipe fez parte de diversos \textit{workshops} onde recebeu treinamentos sobre técnicas e procedimentos em linha viva. 

Nessa etapa, em paralelo aos estudos de como ocorre as manutenções, os pesquisadores fizeram uma revisão exploratória em busca de textos e pesquisas sobre arcabouços que apresentam o potencial para representar esse tipo de situação. Apesar de que não sabiam ao certo qual é o modelo e quais são os conceitos mais apropriados para esse propósito, os pesquisadores entendiam a necessidade de uma estrutura que leva em consideração questões organizacionais, normativas e punitivas. Por conta disso, faz parte do alvo da revisão exploratória, os modelos relacionados a sistemas multiagentes (normativos ou não). Isso se deve ao fato de que as observações em campo apontaram claramente para questões relacionadas a pessoas que trabalham de forma colaborativa, contudo que apresentam o potencial para cometer erros em um certo padrão de normativa. Sobre certas condições os comportamentos inapropriados geram implicações negativas aos agentes causadores dessas situações. Por consequência, esses agentes sofrem os acidentes. Contudo, as observações também mostraram que esses comportamentos podem resultar em consequências inapropriadas a outros profissionais que não cometeram nenhum tipo de erro.  
    \section{Formalização de um Modelo Conceitual} 
        Uma vez identificado quais são esses modelos, e uma vez realizada as observações em campo, os autores formularam conceitos e relações interessantes para o estudo em análise. Feito isso, ocorre a etapa da adaptação, onde essas categorias são redefinidas em estruturas conceituais mais específicas com objetivo de construir um vocabulário especializado para as condições de interesse desse estudo. Como resultado desse processo, os autores obtiveram uma lista de conceitos e suas relações específicas para representar cenários delineados pelos resultados das observações. 

A próxima etapa reside em escrever esses conceitos em algum tipo de formalismo. Nesse estudo, os autores optaram por usar teoria de conjuntos e lógica de predicados. Isso possibilitou estruturar o modelo em voga numa linguagem formal onde cada parte (conjunto, elemento e predicados) é um vocábulo com uma semântica clara. 

Os conjuntos são usados para representar um certo conceito. Os elementos de um conjunto representam os objetos atrelados ao conceito. Por exemplo, supondo que uma loja de Carros venda os seguintes veículos; \textit{Jetta}, \textit{Gol}, \textit{Uno}.
Nesta situação, o conceito de carro é representado pelo conjunto $C$ e os modelos são elementos do mesmo. Portanto, numa linguagem matemática formal tem-se a seguinte situação $C = \{Jetta,Gol,Uno\}$. 

Dentro do conceito matemático, uma relação é uma correspondência entre elementos de conjuntos não vazio, sendo dada por $R \subseteq  A \times B = \{(a,b)| a \in A \wedge b \in B \}$. A Lógica de Predicados foi usada para representar essas
relações.

O \textit{UML} também é uma ferramenta que foi usada para criar representações do modelo. O propósito disto consiste nos seguintes aspectos; apresentar perspectiva global do modelo, definir melhor os critérios existenciais (agregação, composição),
tornar o processo de apresentação mais didático e aproxima-lo de mecanismos de implementação (ex. linguagens de programação). 

Ainda nessa parte da pesquisa os autores construíram as regras que definem como se dá a transição de estados do sistema. Os raciocínios portanto, são definidos com base nessas regras que, em linguagem natural, correspondem a \textit{Se ..., Então }. Em termos formais, os autores optaram por usar o formalismo lógico de implicabilidade dado por $\to$. Também adotaram os seguintes critérios para construir essas regras; elaboração de raciocínios práticos, análise das regras dos outros modelos e verificação da semântica dos vocábulos. 
    \section{Realização de Inferências}    
        A próxima etapa dessa pesquisa se deu por construir as regras que definem como se dá a transição de estados do sistema. Os raciocínios, portanto, são definidos com base nessas regras que, em linguagem natural, correspondem a \textit{Se ..., Então }. Em termos formais, os pesquisadores optaram por usar o formalismo lógico de implicabilidade dado por $\to$. Os pesquisadores adotaram os seguintes critérios para construir essas regras; elaboração de raciocínios práticos, análise das regras dos outros modelos e verificação da semântica dos vocábulos.   

Com a obtenção dos conjuntos, dos predicados e das regras, os pesquisadores analisaram um caso de manutenção específico, definiram esse caso na estrutura do modelo conceitual e realizaram as inferências com a finalidade de verificar se os cenários que podem ser derivados dessas regras correspondem com a realidade dos fatos. 
    \section{Explorar os Arcabouços Possíveis} 
        Com a formulação do modelo conceito, e com o desenvolvimento de inferências que correspondem em partes a realidade, o autor obteve um modelo conceitual criterioso suficiente para avaliar a correspondência com os arcabouços disponíveis. Isso é feito por intermédio de uma análise da estrutura conceitual desses modelos, a fim de averiguar a correspondência dos mesmos com o modelo resultante.

Uma análise da estrutura conceitual consiste em entender a linguagem na qual o modelo é formulado, compreender as regras de sintaxe e averiguar a semântica de todos os elementos. Tendo em vista o fato de que o modelo conceitual possui as suas estruturas semânticas alicerçadas em uma literatura acadêmica que é um ponto em comum com esses arcabouços, a equivalência semântica se torna algo relativamente notório de ser feito (existe alguns modelos onde averiguar essa correspondência semântica é algo relativamente complicado de ser feita, contudo essas questões são tratadas e justificadas na discussão).

Como o modelo conceitual permitiu a formulação de regras e raciocínios, a análise comparativa também verifica a correspondência que o modelo conceitual tem com os arcabouços no que tange a conclusões de riscos e acidentes.
\chapter{Resultados}
\label{chap:resul}
	Essa capítulo tem como por propósito apresentar o modelo, sua utilização para um dado estudo de caso e sua implementação 
em Prolog (linguagem de programação com viés na Lógica Matemática). O modelo foi construindo usando Teoria de Conjuntos
e Lógica de Predicado.

Os conjuntos são usados para representar um dado conceito. Os elementos de um conjunto representam os objetos atrelados ao 
conceito. Por exemplo, supondo que uma lógica de Carros venda os seguintes veículos; \textit{Jetta}, \textit{Gol}, \textit{Uno}.
Nesta situação, o conceito de carro é represdentado pelo conjunto $C$ e os módelos são elementos do mesmo. Portanto, numa linguagem
matemática formal tem-se a seguinte situação $C = \{Jetta,Gol,Uno\}$. 

Dentro do conceito matemático, uma relação é uma correspondencia entre elementos de conjuntos não vazio, sendo dada por
$R \subseteq  A \times B = \{(a,b)| a \in A \wedge b \in B \}$. A Lógica de Predicados foi usada para representar essas
relações. Para representação das regras foi feito uso das relações de inferência $\rightarrow$ "Se ..., então ...".  

O \textit{UML} também é uma ferramenta que foi usada para criar representações do modelo. O propósito disto consiste nos
seguintes aspectos; apresentar perspectiva global do modelo, definir melhor os critérios existenciais (agregação, composição),
tornar o processo de apresentação mais didático e aproxima-lo de mecanismos de implementação (ex. linguagens de programação). 

A figura \ref{module} apresenta a estrutura de módulos (a fim de evitar poluição visual, as relações serão apresentadas
em outra figura). 

\begin{figure}[H]
  \centering
  \includegraphics[width=1\linewidth]{figure/Class.png} 
  \caption{A estrutura geral das classes do modelo}
  \label{module}
\end{figure}

Assim sendo, assumindo que existe $\Omega_{Model}$ (um conjunto global onde todos os outros conjuntos do modelo estão 
contidos nele), os módulos são representados da seguinte maneira; 

\begin{equation} 
    \Omega_{Model} = \{ M_{Risk}, M_{Task}, M_{Entity}, M_{Environment}\}
\end{equation}
\label{modules}

	\section{Estrutura Conceitual} \label{estconceitual}
		Essa parte do texto apresenta a estrutura final do modelo conceitual que é definida em termos de Módulos (exibe informações sobre os módulos e os conceitos neles contidos), Predicados (apresenta os predicados e justifica a sua existência) e Regras (exibe as regras e explica porque sua existência dentro do modelo é necessária). Cada componente desses será exibida em termos de uma subseção. 
		\subsection{Módulos} \label{mods}			
			Os conjuntos que representam os conceitos são organizados em  módulos. Isso ocorre porque muitos conceitos apresentam similaridades, sendo razoável conceber uma taxonomia desses conjuntos organizando-os em conjuntos maiores. A figura \ref{module} apresenta a estrutura de módulos (a fim de evitar poluição visual, as relações serão apresentadas em outra figura). 

\begin{figure}[H]
  \centering
  \includegraphics[width=1\linewidth]{figure/Module.jpeg} 
  \caption{A estrutura geral das classes do modelo}
  \label{module}
\end{figure}

Assim sendo, assumindo que existe $\Omega_{Model}$ (um conjunto global onde todos os outros conjuntos do modelo estão contidos nele), os módulos são representados da seguinte maneira; 

\begin{equation} 
    \Omega_{Model} = \{ M_{Risk}, M_{Task}, M_{Entity}, M_{Environment}\}
\end{equation}
\label{modules}


O mundo sobre o qual este modelo pretende representar trabalha com o fato de que tanto agentes como artefatos possuem algumas propriedades em comum, que é: existem, ocupam lugar no espaço, estão sujeitos ao tempo, apresentam estados e participam de processos. Essa premissa possue os seus fundamentos alicerçados em \ref{agent} e \ref{artefact} e isso será demonstrado com maior rigor no texto que se segue. Tendo em vista a ocorrência de certos conceitos necessários para lidar com essas questões, se fez necessário definir um módulo de entidades para agrupa-los em uma estrutura única. Esse módulo é composto pelos seguintes conceitos;

\begin{equation} 
M_{Entity} = \{ Entity \}
\end{equation}\label{modent}

\textbf{Entity} - O termo entidade é sujeito a profundos debates filosóficos, porém neste texto o termo é usado para referenciar uma "coisa" que pode ser identificada, como uma pessoa, companhia ou um evento \cite{entity}. É de conhecimento que as propriedades anteriormente mencionadas caracterizam as "coisas" que podem ser identificadas, logo são entidades. É digno de nota a existência de entidades que não se adequam a todas essas propriedades. Contudo, essas propriedades fazem referência ao que se caracteriza por entidade, respeitando o conceito padrão \cite{entity} e restringindo para o escopo deste modelo. Isso contempla tanto os agentes como os artefatos, como fica claro na relação \ref{defineentity}. O texto a seguir demonstra como essas propriedades se aplicam a agentes e a artefatos (se isso ficar demonstrado, logo fica demonstrado que são entidades).

\textit{Ocupa lugar no espaço, estão sujeitos ao tempo - } Como definido em \ref{agent} e \ref{artefact}, ambos são situados em ambientes. 
Isso possibilita inferir que se faça necessário a presença de um conceito que se apresente como uma propriedade de estado para agentes e artefatos. Um ambiente, no contexto onde os agentes e artefatos são usados para representar atividades das pessoas, condiz com a relação de espaço e tempo. 

\textit{Participa de Processos -} Processos podem constituir entidades, bem como entidades necessariamente constituem processos por intermédio da ação que aquelas manifestam nesses. No que se verifica ao primeiro caso, é possível usar o ser humano como exemplo - onde
a entidade ser humano é formulada por uma série de processos bio-químicos. Sobre o segundo caso, relações climáticas exemplificam isso, onde a água é uma entidade presente em processos termodinâmicos. 

\textit{Apresenta estados -} O fato de que artefatos bem como agentes apresentam atributos (que podem mudar e podem assumir diferentes valores no que tange aos eventos externos e internos), então ambos também apresentam a concepção de estados (sendo esse termo usado diretamente em certos pontos dos textos presentes tanto em \ref{agent} \ref{artefact}). 


\begin{equation} \label{defineentity} 
 \{ Agent \cup Artefact \} \subset Entity
\end{equation}

\textbf{Agent} - Esse estudo adota a definição de agentes presentes no primeiro parágrafo da seção \ref{agent}. Isso implica em entidades autônomas, ou seja, que apresenta a capacidade de agir por si mesma quando diante de condições onde isso é necessário. A seção \ref{agent}, apresenta o conceito de agentes inteligentes e esse mesmo conceito é adotado neste modelo. 

Não é preocupação deste estudo, delimitar as representações do agente, bem como definir algoritmos para verificar como se da as relações de tomada de decisão dos mesmos. Assim sendo, fica em aberto para o modelador definir como acontece os processos de tomada de decisão, estados internos e modelos de representação que serão usados para definir o comportando do agente. 

\textbf{Artefact} - são entidade que existem para que os agentes possam cumprir com os seus objetivos e que apresentam interface de uso, instruções de operação, funcionalidade e estrutura-comportamento. Essas entidades não são orientados à objetivos e 
não apresentam capacidade de comunicação como definido na seção \ref{artefact}. Predicados que contemplam esses aspectos do artefato serão apresentados mais adiante ao decorrer do texto. 

Os agentes são autônomos e orientados à objetivos sendo esses dois elementos descaracterizantes do que se define como por artefato. Logo, apesar de agentes e artefatos serem entidades, não é possível existir um agente que seja artefato ou um artefato que seja agente, o que é dado pela relação presente na expressão \ref{agentsartefactvoid}. 

\begin{equation} \label{agentsartefactvoid}
    Agent \cap Artefact = \{ \emptyset \}
\end{equation}

Para tornar a apresenação de certos conceitos algo um tanto mais didático, o autor desenvolveu um exemplo entitulado por: \textbf{Exemplo da Redação}, que é o seguinte cenário: "O Professor Aristóteles definiu uma atividade; Escrever uma redação sobre o livro Metafísica. Para isso, o aluno Alexandre o Grande deve escrever um dado texto, deve ler o livro sobre o tópico em definido, deve pegar uma folha, deve pegar um lápis e escrever a redação".  


\textbf{O Módulo de Atividades} - \textit{Task Module} representado por $M_{Task}$ condiz com os conceitos relacionados aos objetivos que devem ser atingidos bem como aos papéis que são assumidos pelos agentes.
\begin{equation}
    M_{Task} = \{ Goal, Role \}
\end{equation}

\textbf{Goal} - faz referência aos objetivos que devem ser atingidos pelos agentes. Os fundamentos semânticos deste conjunto está presente na seção \ref{sma} mais especificamente na subseção \ref{moiseformalizesma}. Neste modelo, um objetivo é descrito em termos de entidades relações e condições necessárias para que uma data atividade possa ser dada como concluída. 

\textbf{Role} - apresenta o papel que um agente pode adotar dentro de um \textit{SMA}. Esse conceito também é importado no \textit{MOISE+} 
\ref{moiseformalizesma} e define as relações deonticas entre os agentes e os objetivos. Para exemplificar, pode-se considerar o \textbf{Exemplo da Redação} onde existe dois agentes $Agent = \{ aristoteles, alexandre \}$, existe dois papéis $Role = \{ professor, aluno\}$. Neste caso, o agente $aristoteles$ é o $professor$ e o agente $alexandre$ é o $aluno$.

\textbf{O Módulo de Ambiente} - \textit{Environment Module} - consiste em conjuntos que representam relações e condições ambientes, que são;

\begin{equation}
    M_{Environment} = \{  Circumstance  \}
\end{equation}

\textbf{Relation} - Uma entidade estabelece relações com outras entidades ao seu redor \cite{entity}. No modelo proposto neste texto. O autor optou por um conjunto para representar os relacionamentos entre as entidades que possibilite identifica-los. Isso facilita o desenvolvimento de raciocínios. O uso 
dos relacionamentos podem ser exemplificado por meio do \textbf{Exemplo da Redação}. Ao definir uma atividade, o professor Aristóteles
, que é uma entidade, estabeleceu uma relação com o seu aluno Alexandre, representado aqui por $relAristotelesAlexandre$. Para cumprir essa tarefa o aluno precisou ler o livro - $relAlexandreLivro$, pegar uma folha - $relAlexandreFolha$,  pegar um lápis $relAlexandreLapis$ e escrever a redação, o que implica em uma relação entre lápis e folha $relLapisFolha$. Portanto, o conjunto de relacionamentos se dá da seguinte maneira;

\begin{eqnarray}\label{Environment}\nonumber
    M_{Environment} = \{ relAristotelesAlexandre, relAlexandreFolha, relAlexandreLivro, \\ \nonumber
     relAlexandreLapis, relLapisFolha \}
\end{eqnarray}

Obviamente, cada entidade do grupo \textbf{Relation} tem um vínculo com elementos do grupo \textit{Entity}, como por exemplo $relAlexandreFolha$ apresenta um vínculo com as entidades $Alexandre$ e $Folha$. Há uma predicado que trata disto e será exibido posteriormente. \textbf{Condition} - Esse conjunto representa as condições que devem ser mantidas para que um objetivo possa ser alcançado. Tendo em vista certas relações de predicado, que serão apresentados com maior riqueza de detalhes mais adiante, se faz necessário definir abstração de alto nível entre \textbf{Condition} e \textbf{Relation}. Assim sendo, o autor assume a existência de um conjunto \textbf{Circumstance} que é dado pela seguinte relação; 

\begin{equation}
    Circumstance = \{ Relation, Condition \} |  Relation \cap Condition = \{ \emptyset \}
\end{equation}


\textbf{Módulo de Risco} - \textit{Risk Module} contém conjuntos que correspondem a conceitos relacionados a temática da segurança.
O módulo de risco é dado pela relação que se segue;

\begin{equation}
    M_{Risk} = \{ Risk, Consequence \}
\end{equation}

\textbf{Risk} - Na seção \ref{risksec} o termo risco é usado para referenciar a um evento que apresenta um potencial de ocorrer, e que gera consequências negativas às pessoas associadas quando acontece. Para exemplificar, pode-se considerar uma condição onde um eletricista está trocando um disjuntor de um quadro elétrico. Nesse processo, o eletricista está sujeito ao risco de ser eletrocutado. Essas consequências negativas também são representadas por um conjunto, o qual é denominado \textbf{Consequence}. O uso deste conjunto pode ser apresentado utilizando esse mesmo exemplo do eletricista, pois a consequência de se submeter a um evento desses implica morte (nem sempre é assim, mas para efeitos didáticos pode-se considerar que o quadro elétrico é de certa potência que a morte é certa para o profissional que for eletrocutado).
		\subsection{Predicados}\label{predic}
			O predicado $possEntityRel(r_l,e_i,e_k) | r_l \in Relation \wedge  e_i, e_k \in Entity$ é usado para, tratar questões de identificar as duas entidades com a sua relação. Esse predicado se lê da seguinte forma: O relacionamento $r_l$ possui a entidade $e_i$ e a entidade $e_k$. Para demonstrar como se dá o uso desse predicado pode-se considerar o \textbf{Exemplo da Redação}. A entidade $alexandre$ apresenta uma relação com $folha$ que é identificada como $relAlexandreFolha$. Portanto, com o predicado, a representação fica; $possEntityRel(relAlexandreFolha,alexandre,folha)$

O predicado $adoptsRole(ag_n,\rho_m) | ag_n \in Agent \wedge \rho_m \in Role$ que tem sua origem nos estudos do \textit{MOISE+} onde cada agente tem uma função dentro do contexto do \textit{SMA}. Esse predicado se lê da seguinte forma: O agente $ag_n$ tem um papel $\rho_m$. Usando o \textbf{Exemplo da Redação} tem-se o seguinte: $adoptsRole(artistoteles,professor)$. 

O predicado $hasObligation(\rho_m,g_j) | \rho_m \in Role, g_j \in Goal $ tem suas origens nos estudos da lógica deôntica também presentes no modelo \textit{MOISE+}. Esse predicado pode ser lido da seguinte maneira: O agente que assumir o papel $\rho_m$ tem a obrigação de concluir o objetivo $g_j$. No exemplo padrão deste texto (\textbf{Exemplo da Redação}), o primeiro objetivo é uma obrigação do professor, portanto: $hasObligation(professor,g_0)$

O predicado $hasPermission(\rho_m, g_j) | \rho_m \in Role, g_j \in Goal $ também está associado aos estudos da lógica deôntica relacionada ao \textit{MOISE+}. A leitura se dá desta maneia: O agente que $\rho_m$ tem a permissão de concluir o objetivo $g_j$. Usando o exemplo padrão como base onde $professor$ tem a obrigação de executar $g_0$, então também tem permissão para isso (isso será melhor explicado em uma regra que será apresentada mais tarde). Portanto; $hasPermission(professor,g_0)$.  

O predicado $instanceOfCond(circ_k)$, $circ_k \in Condition$ informa que $circ_k$ é uma condição, ou seja, pertence ao conjunto Conditions. A existência desse predicado é necessária porque alguns raciocínios de violação necessitam verifcar se o elemento em análise é uma condição. A mesma situação acontece para as relações. Assim sendo o predicado $ instanceOfRel(r_k) \in Relation $ também deve fazer parte da estrutura do modelo.

O predicado $reached(g_k) | g_k \in Goal $ define que todos os agentes, que eram obrigados a alcançar o objetivo $g_k$, concluíram. A existência desse predicado se dá devido ao fato de que em certos raciocínios é necessário identificar que um certo objetivo foi atingido. 

O predicado $stopped(g_n, ag_m) | g_n \in Goal, ag_m \in Agent$ apresenta a seguinte leitura: Toda a atividade foi encerrada com o objetivo $g_n$ por uma ação associada ao agente $ag_m$. Para certos raciocínios se faz necessário identificar o encerramento das atividades como um todo e por quem isso aconteceu. Em certos casos não há a necessidade de identificar o agente $ag_m$. Para isso há uma outra versão deste mesmo predicado escrito da seguinte forma: $stopped(g_n)$ e sua semântica indica que o objetivo $g_n$ teve sua execução encerrada. 

O predicado $nextGoal(g_i,g_j) |g_i, g_j \in Goal$ possui a seguinte semântica: O objetivo $g_i$ tem um próximo objetivo que é $g_j$. Sua necessidade advêm do fato de que este modelo trabalha os mesmos conceitos presentes em \textit{MOISE+} porém sob uma abordagem diferente. Em vez de usar estrutura de super-sub objetivos e definindo operadores de série e paralelo, os objetivos não possuem estruturas, e suas relações se dão por um objetivo apontado para o próximo. Então, para exemplificar pode-se considerar uma atividade descrita por quatro objetivos, sendo que $g_0$ é pré-requisito para $g_1$ e $g_2$. Em contrapartida $g_3$ só começa a ser atingido depois da finalização de $g_1$ e $g_2$. Assim sendo, na linguagem proposta neste estudo, esse problema é escrito da seguinte forma: $nextGoal(g_0,g_1), nextGoal(g_0,g_2), nextGoal(g_1,g_3), nextGoal(g_2,g_3)$.

O predicado $requiresCirc(goal_i,circ_j) | goal \in Goa, circ_j \in Circumstance, Relation \subset Circumstace, Condition \subset Circumstance$ é lido da seguinte forma: Para que o objetivo $goal_i$ possa ser alcançado é necessário a circunstância $circ_j$. Essa circunstância pode ser tanto relações $relation \in Relation$ ou pode ser condições $c_k$.

O predicado $requiresEntity(goal_i, e_j) | g_i \in Goal, e_j \subset Entity $ é lido da seguinte forma: Para que o objetivo $g_i$ seja atingido, as entidades $e_j$ devem estar presentes no instante em que $g_i$ estiver sendo alcançado. O propósito deste predicado reside na necessidade de identificar quais são as entidades que devem estar presentes para que um dado objetivo possa ser executado. No \textbf{Exemplo da Redação}, um objetivo $g_2$ só pode ser atingido se as entidades $\{ aluno, folha, lapis\}$ estiverem presentes no momento em que $g2$ estiver sendo alcançado. Esse cenário é representado da seguinte forma; $requiresEntity(g_2, aluno), requiresEntity(g_2, folha), requiresEntity(g_2, lapis)$.

O predicado $isPresent(circ_i) | circ_i \in Circumstance $ retorna como verdade de $circ_i$ está presente no momento em que este predicado é invocado (retorna falso para o contrário). Em alguns raciocínios é de importância verificar se um elemento está presente durante a tentativa de uma dado agente executar algum objetivo.

O predicado $starts(ag_i,g_j) | ag_i \in Agent \wedge g_j \in Goal $ é lido da seguinte forma: Um agente $ag_i$ está tentando atingir o objetivo $g_j$. Para algumas situações, é de crucial importância identificar quando um agente está tentando atingir um objetivo, sendo necessário a existência de um predicado apenas para esse propósito. Para exemplificar, pode-se considerar o exemplo da redação. Para que o professor $Aristoteles$ possa alcançar o objetivo $g_0$, é necessário a existência de uma tentativa. Neste modelo, essa situação é representada da seguinte maneira: $starts(aristoteles,g_0)$. Seguindo a linha desse predicado, há também o $ableReach(ag_i,g_j) | ag_i \in Agent \wedge g_j \in Goal$ cuja semântica expressa que o agente $ag_i$ está habilitado a tentar buscar o objetivo $g_j$. Contudo, não significa que o agente fará o correspondente de $starts(ag_i,g_j)$. O que definirá a transição de $ableReach(ag_i,g_j)$ para $starts(ag_i,g_j)$ são os estados internos do agente. Contudo, não é do interesse deste estudo aprofundar na dinâmica do agente em si, deixando esse processo em aberto para o programador decidir como resolverá essa questão.

O predicado $ conditionViol(ag_i,g_j,c_k) | ag_i \in Agent \wedge g_k \in Goal \wedge c_k \in Condition$ deve ser lido da seguinte maneira: O agente $ag_i$ comete uma violação de condição no objetivo $g_j$ por tentar realizar uma determinada atividade em que a condição $c_k$ era essencial porém não estava presente. Os fundamentos deste predicado está vinculado com a seção \ref{normasdastani}. No modelo em \ref{normasdastani} para representar aspectos normativos dos agentes, a regra do tipo \textit{Count-as} apresenta quais são as circunstâncias que ocasionam uma violação. Esse predicado cumpre esse propósito para o conjunto \textbf{Conditions}. Para exemplificar, no \textbf{Exemplo da Redação}, se o professor Aristóteles lecionar sem que haja luz suficiente para isso, então ele cometeu uma violação de condição caracterizada da seguinte maneira: $conditionViol(aristoteles,g_0,luz)$

O predicado $ relationViol(ag_i,g_j,r_k) | ag_i \in Agent \wedge g_k \in Goal \wedge r_k \in Relation $ possui a mesma situação presente em $ conditionViol(ag_i,g_j,c_k) $ contudo o foco diz respeito aos relacionamentos. A leitura se dá desta forma: O agente \textit{ag\_i} pratica uma violação de relação no objetivo $g_j$ por executar a atividade sem que a relação $r_k$ esteja presente. O uso deste predicado por ser feito considerando o exemplo em análise com a adição de uma breve descrição de uma situação que possa ocorrer que é o seguinte; A ponta do lápis que Alexandre tenta usar para escrever a redação está quebrada. Portanto, a relação \textit{relLapisFolha} não pode ser feita. Se o agente \textit{Alexandre} alcançar o objetivo $g_2$ sem ter as circunstâncias necessárias para isso, então comente uma violação de relação sendo escrito da seguite forma: $relationViol(alexandre,g_2,r_2)$.

O predicado $ entityViol(ag_i,g_j,e_k) ag_i \in Agent \wedge g_j \in Goal \wedge e_k \in Entity$ advém das mesmas situações dos dois predicados a cima. A leitura deste predicado se dá da seguinte forma: O agente $ag_i$ cometeu uma violação por tentar alcançar o objetivo $g_j$ sem que a entidade $e_k$ esteja presente. No exemplo padrão o uso deste predicado consiste em Alexandre tentar executar $g_2$ sem ter a entidade lápis o que resulta no seguinte: $entityViol(alexandre, g_2, lapis)$.

O predicado $ hasRisk(crts, risk_j, cs_k) | crts \in Circumstance \wedge  risk_k \in Risk \wedge cs_k \in Consequence $ é baseado nos estudos presentes na seção \ref{risksec} e tem a finalidade de definir os riscos associados ao tentar executar alguma atividade sem que $c_k$ ou $r_k$ esteja presente. Portanto, a leitura deste predicado é dada da seguinte forma; A ocorrência de uma violação onde $crts$ ocasiona em um evento associado ao $risk_j$ com a consequência $cs_k$. Ao explicar sobre o conjunto \textbf{Consequence}, foi dado um exemplo sobre um eletricista que tem o potencial de ser eletrocutado. Para esse exemplo, o uso deste predicado apresenta o seguinte formato: $hasRisk(relFerramentaIsolanteBarramento, eletrocutado, morte)$.

O predicado $possOfNegConseqFor(r_l) | r_l \in Relation $ tem seus fundamentos associados ao estudo da lógica modal, presente na seção \ref{logic}, no operador $\Diamond$ cuja semântica denota possibilidade. Contudo, neste estudo esse termo apresenta o seguinte conceito semântico: há a possibilidade de acontecer um evento ruim associado ao risco $r_l$ vinculado ao agente que está associado a essa relação, mesmo que esse agente não tenha cometido nenhum erro durante o procedimento. Esse predicado tem como por finalidade representar situações onde um evento ruim acontece, não pelo erro do profissional diretamente associado a situação, mas sim por outras cadeias causais complexas de serem identificadas e justamente por isso são abstraidas por conceitos de aleatoriedades, idem possibilidades. Para exemplificar pode-se considerar a situação onde um eletricista de linha viva usa um bastão isolante para acessar um barramento altamente energizado. Contudo, esse bastão pode estar com isolamento comprometido. Testes que devem ser feitos antes de fazer uso de uma ferramenta podem eliminar qualquer possibilidade de que os riscos venham a se tornar eventos reais, pois se o bastão em questão estiver em bom estado, então o eletricista não se envolverá em um acidente por esse fator. Se o bastão estiver em mal estado - isso será identificado e a ferramenta será adequadamente substituída. Contudo, considerando um cenário onde por negligencia de profissionais a medição não é feita, surge uma possibilidade do isolamento estar comprometido. Essa situação torna verdade o seguinte predicado $possOfNegConseqFor(relBastaoBarramento)$.

O predicado $affectsRels(r_k,r_n) | \{ r_k, r_n\} \subset Relation $ trabalha em conjunto $possOfNegConseqFor(r_l)$. Esse predicado se lê da seguinte forma: Se $r_k$ não foi realizado, ou se for realizado de forma inapropriada, isso afeta  $r_n$ tornando verdade $possOfNegConseqFor(r_n)$. Ambos são importantes em raciocínios onde se deseja mostrar que a não execução de uma relação não gera consequências negativas imediatas a ninguém, mas resulta em consequências futuras inclusive sobre pessoas que não compartilham da mesma situação. No exemplo do eletricista, a relação $relBastaoBarramento$ é afetada pela não execução da relação $relBastaoMedidor$ (que define a relação entre o aparelho medidor de corrente de fulga e o bastão isolante). Esse exemplo é escrito da seguinte maneira:$affectsRels(relBastaoMedidor,relBastaoBarramento)$

O predicado $negConseqFor(g_k, ag_i,risk_k,cs_m)$ pode ser lido da seguinte maneira; ocorreu um evento ruim no objetivo $g_k$ associado ao agente $ag_i$ e associado ao risco $risk_k$ que atribuiu ao agente $ag_i$ consequências $cs_m$. Esse predicado tem por finalidade traduzir semanticamente as consequências ruins sobre alguém quando ocorre o pior caso. Sobre o exemplo do eletricista que acessa um barramento de alta tensão, esse predicado é escrito da seguinte forma: $negConseqFor(g_{acessoBarramento}, eletricista,eletrocutado,morte)$. Associado a isso há o predicado $happensNegConseqFor(r_m)$ cujo propósito define que um evento ruim aconteceu no relacionamento $r_m$.

O predicado $lastGoal(g_i,\rho_m) | g_i \in Goal, \rho_m \in Role $ apresenta um último objetivo $g_i$ que deve ser alcançado por agentes com determinado papel $\rho_{m}$. Esse predicado tem sua existência justificada em certos raciocínios, que serão demonstrados mais adiante, onde esse tipo de informação é relevante.
		\subsection{Diagrama de Classes}
			Diferente da figura \ref{module}, o foco da figura \ref{classdiagrama} consiste em apresentar como se dá a estrutura de classes e dos relacionamentos. Ambas situações poderiam ser representadas em uma única figura, contudo os pesquisadores decidiram por seccionar em duas, a fim de tornar o processo mais didático. Por esse mesmo motivo não está apresentado neste \textit{UML} todas as classes e propriedades.  

\begin{figure}[H]
  \centering
  \includegraphics[width=1\linewidth]{figure/Class.jpeg} 
  \caption{Diagrama de classes do Modelo }
  \label{classdiagrama}
\end{figure}

Uma vantagem deste tipo de diagrama em relação a representação por conjuntos, consiste na ocorrência de uma sintaxe específica para tratar dois pontos relevantes dentro do contexto computacional que são: cardinalidade e relações existenciais. Um dos predicados interessantes de serem analisados, neste contexto, é \textit{adoptsRole} que define um relacionamento fracamente agregada entre \textit{Agent} e \textit{Role}. Isso, pois, dentro do escopo deste modelo, um agente pode existir sem ter um papel, portanto este não é um critério necessário para definir aquele. A cardinalidade se justifica tendo como base o fato de que um agente pode ter um ou mais papéis. 

As relações \textit{hasObligation}, \textit{hasPermission} se dão por meio de agregações fortes tendo em vista que não há sentido para um papel $\rho$ existir sem que esteja vinculado a ao menos um objetivo. Como um papel se relaciona com diversos objetivos, os engenheiros adotaram a cardinalidade de $1$ para $1 .. *$.

Em \textit{UML} questão de conjunto-subconjunto entre $Circumstance$ com $Relation$ e com $Condition$ é definida por meio de classes que possuem esses mesmos nomes. No \textit{UML}, as classes $Relation$ e $Condition$ são extensões da classe $Circumstance$. Dado essa situação, é possível representar a relação \textit{hasRisk} que ocorre entre $Circumstance$, $Relation$ e $Risk$ e isso é feito por meio do ternário entre essas classes. 

Um objetivo não pode ser definido sem saber quais são as entidades $Entity$, relações $Relation$ e consequências $Consequence$ necessárias para que tal objetivo seja alcançado. Por isso os predicados $requiresCirc$ e $requiresEntity$ estabelecem composição forte de suas respectivas classes com $Goal$. Como um objetivo pode apontar para diversas instâncias dessas classes, os pesquisadores optaram - para cada uma das relações - trabalhar com a cardinalidade 
$1 - 1 .. *$.

No modelo proposto $Relation$ deve estar relacionada com duas entidades. Por esse motivo o predicado $possEntityRel$ faz composição forte com $Entity$ e a sua cardinalidade é dada $1 - 2$. 

A classe $Goal$ possui uma relação consigo mesma dada por $nextGoal$. Essa é uma agregação fraca, pois do contrários seria impossível haver uma única instância desta classe. Isso se deve ao fato de que a primeira instância necessitaria de uma instância de $Goal$ para existir. Contudo, como não há um elemento de $Goal$ antes do primeiro elemento de $Goal$, logo esse primeiro elemento não pode existir. Um objetivo poder ter como próximo um ou mais objetivos, justificando a ocorrência da cardinalidade $1 .. n$. 

O predicado $affectsRels$, por motivos similares a $nextGoal$ deve ter agregação fraca. Como uma relação pode afetar uma ou mais, a cardinalidade adequada para essa circunstância é dada pro $1$ - $1 ..*$
		\subsection{Regras} \label{regras}
			A regra \ref{reldeonticrole} tem os fundamentos teóricos na lógica deôntica e em modelos como \textit{MOISE+}. Assim sendo, todas as relações de obrigação implicam relações de permissão. O que essa regra determina consiste no fato de que se um agente $g_j$ é obrigado a trabalhar sob o objetivo $g_j$, então esse agente também tem a permissão de trabalhar sobre o objetivo $g_j$ 

\begin{eqnarray}\label{reldeonticrole}
	hasObligation(\rho_m,g_j) \to hasPermission(\rho_m,g_j), \nonumber \\
    \rho_m \in Role \wedge g_j \in Goal
\end{eqnarray}

As regras \ref{conditionViol}, \ref{relationViol}, e \ref{entityViol} são fundamentadas em \ref{normasdastani} onde o conceito do que pode ser feito é definido em termos das regras \textit{Count-as}. Essas regras determinam quais são os elementos que resultam em violação. Inspirando-se nesse tipo de estrutura é que o autor trata de trabalhar as referidas regras. 

A complexidade de estudo é extramente ampla, e com certeza existe mais tipos de violações do que as três consideradas a seguir, contudo optou-se por estudar essas violações porque são essenciais para os objetivos deste estudo. Outro questionamento que pode surgir consiste no porque definir três tipos de violações? Isso reside no fato de que essas violações resultam em consequências diferentes, por conta disto em um primeiro momento os engenheiros do modelo decidiram tratá-las em estruturas diferentes. 

A explicação das regras será feita sempre analisando a semântica do predicado que é implicado em relação aos estudos pelos quais elas se fundamentam. Partindo desta premissa, o entendimento da relação \ref{conditionViol} só pode ser feito na ocorrência de uma  investigação sobre quais são os elementos que semanticamente correspondem ao predicado $conditionViol(ag_m,g_i,c_k)$. O primeiro ponto reside em verificar quais são as condições necessárias de $g_i$. Quem tem essa finalidade é o predicado $requiresCirc(g_i,c_k)$. Contudo, saber todas as condições não são o suficientes, pois a violação acontece na ausência de uma condição $c_k$ e isso deve ser verificado nesta relação de implicabilidade. Então se faz necessário considerar um predicado que analisa se $c_k$ está presente no ato da manutenção, e é com esse propósito que $isPresent(c_k)$ faz parte da relação. Contudo, as informações ficam desencontradas se $c_k$ não for uma instância de $Condition$, por isso é importante fazer essa análise também através do predicado $instanceOfCond(c_k)$. Esses são componentes essenciais, porém não são suficientes porque não consideram a condição do agente. Isso, pois afirmar sobre a ocorrência de uma violação de um agente sem considerar se ele esta efetivamente tentando alcançar um objetivo consiste em desconsiderar a semântica daquilo que está sendo implicado. Isso é resolvido por considerar o termo  $starts(ag_m,g_i)$. 

\begin{eqnarray}\label{conditionViol}\nonumber
	requiresCirc(g_i,c_k) \wedge \neg isPresent(c_k) \wedge instanceOfCond(c_k) \wedge starts(ag_m,g_i)  \to \\ \nonumber   
	conditionViol(ag_m,g_i,c_k) \nonumber \\  
    g_i \in Goal, c_k \in Condition, ag_m \in Agent
\end{eqnarray}

O propósito da regra \ref{conditionViol}, quando definido em termos de linguagem natural tem a finalidade de exprimir o seguinte: Se um agente tentar executar um determinado objetivo sem que haja todas as condições ambientes necessárias para isso, então esse agente comete uma violação de condição neste respectivo objetivo. 

A regra \ref{relationViol} define as condições que resultam em uma violação de relação. O predicado $relationViol(ag_m,g_i,r_k)$ considera que a violação se dá por um agente $ag_m$ em um objetivo $g_i$ na relação $r_k$. Portanto, para respeitar a semântica deste predicado se faz necessário considerar ao menos um termo que vincule o objetivo $g_i$ com a relação $r_k$. Para esse propósito é que se considera o termo $requiresCirc(g_i,circ_k)$ pois define quais são as circunstâncias que devem estar presentes para que o objetivo $g_i$ possa ser alcançado. Contudo, só isso não é o suficiente, pois se faz necessário analisar se $r_k$ está contido em $Relation$. Isso se deve ao fato de que o predicado $requiresCirc(g_i,circ_k)$ não permite saber se $circ_k$ está contido em $Relation$ ou se está contido em $Condition$. O predicado $instanceOfRel(r_k)$ resolve essa situação. Outro fator atrelado e importante para que o predicado $relationViol(ag_m,g_i,r_k)$ retorne verdade, reside em saber se o agente em sua tentativa de atingir $g_i$ não executa $r_k$ de forma apropriada. Por conta disso se faz necessário considerar o $isPresent(r_k)$. A semântica de $relationViol$ só é conservada em sua inteireza se a presença do agente também for analisada. Para esse propósito é que se verifica a necessidade do uso de $starts(ag_m,g_i)$ que deverá retornar se o agente está tentando alcançar o objetivo $g_i$.   

\begin{eqnarray}\label{relationViol}\nonumber
	requiresCirc(g_i,r_k)\wedge \neg isPresent(r_k) \wedge instanceOfRel(r_k) \wedge starts(ag_m,g_i) \to \nonumber \\
	relationViol(ag_m,g_i,r_k) \nonumber \\  
    g_i \in Goal, r_k \in Relation, ag_m \in Agent
\end{eqnarray}

Traduzindo a regra \ref{relationViol} para linguagem natural obtêm-se a seguinte expressão: Se um agente tentar alcançar um certo objetivo sem que todas as relações necessárias para isso estejam presentes (considerando as relações do domínio dele, tal como manuseio de uma ferramenta específica, e considerando as relações que são independentes dele), então esse agente comete uma violação de relação. 

A regra \ref{entityViol} tem o propósito de definir quais são as condições que resultam em uma violação de entidade. Como em outras situações, para cumprir com esse propósito é necessário que os fatores implicantes sejam correspondente com $entityViol(ag_m,g_i,e_k)$. Para cumprir com essa finalidade, se faz necessário considerar o predicado $requiresEntity(g_i,eg_n)$ (para avaliar as entidades que devem estar presentes a fim de cumprir com o objetivo $g_i$), $isPresent(e_k)$ (para verificar se a entidade $e_k$ está ou não, presente no momento da execução) e $starts(ag_m,g_i)$ (para avaliar se $ag_m$ começou a tentar alcançar o objetivo $g_i$). A semântica do predicado também considera o momento em que o agente está atuando sobre o objetivo $g_i$, por isso o predicado $starts(ag_m,g_i)$ também é posto na relação de implicabilidade.

\begin{eqnarray}\label{entityViol}\nonumber
	requiresEntity(g_i,eg_n) \wedge \neg isPresent(e_k) \wedge starts(ag_m,g_i) \to \nonumber \\ 
    entityViol(ag_m,g_i,e_k)  \nonumber \\  
    g_i \in Goal, e_k \in Entity, ag_m \in Agent
\end{eqnarray}

Em termos de linguagem natural, a regra \ref{entityViol} se apresenta da seguinte forma: Se um agente tentar alcançar um certo objetivo sem ter todas as entidades presentes para isso, então esse agente cometeu uma violação de entidade.

As regras \ref{consconditionViol} e \ref{consrelationViol} são inspiradas nos estudos presentes na seção \ref{normasdastani} onde as consequências de uma violação são definidas como sanções no que é denominado por $Sanction Rule$. A estrutura dessas regras, em \ref{normasdastani} e em \cite{dastaniframework} é dada como $violation \to ... $. Contudo, este estudo leva em consideração não apenas o termo que se refere a violação, mas também as circunstâncias que são consideradas juntas, que neste caso advêm do predicado $hasRisk$. Assim como em \ref{normasdastani}, o modelo deste estudo define que uma sanção corresponde a uma penalidade que o agente deve pagar. Na estrutura da problemática em análise, a penalidade ocorre pelo fato do agente sofrer fisicamente os efeitos dos seus erros. Esse comportamento é dado pelo predicado $negConseqFor(g_i,ag_m,risk_j,cs_m)$ cujo correspondente semântico define que o agente $ag_m$ sofre o evento associado em $risk_j$, no objetivo $g_i$ a consequência $cs_m$. Se os engenheiros deste modelo considerarem apenas $conditionViol(ag_m,g_i,c_k)$ para a relação \ref{consconditionViol} e $relationViol(ag_m,g_i,r_k)$, o correspondente semântico de $negConseqFor(g_i,ag_m,risk_j,cs_m)$ é desrespeitado, não especificando $risk_j,cs_m$. Contudo, isso é resolvido por levar em consideração o predicado $hasRisk(c_k,risk_j,cs_m)$ para \ref{consconditionViol} e o predicado $hasRisk(r_k,risk_j,cs_m)$ para \ref{consrelationViol}. 

\begin{eqnarray}\label{consconditionViol}\nonumber
	conditionViol(ag_m,g_i,c_k)  \wedge hasRisk(c_k,risk_j,cs_m) \to \nonumber \\ 
	negConseqFor(g_i,ag_m,risk_j,cs_m) \nonumber \\ 
    ag_m \in Agent, g_i \in Goal, c_k \in Condition, risk_k \in Risk, cs_m \in Consequence
\end{eqnarray}

Em termos de linguagem natural, a relação em \ref{consconditionViol} é definida da seguinte maneira: "Uma violação de condição de um determinado agente, em um dado objetivo ocasiona em uma consequência ruim a ele. Essa consequência ruim está associada ao risco da condição violada". 

\begin{eqnarray}\label{consrelationViol}\nonumber
	relationViol(ag_m,g_i,r_k) \wedge hasRisk(r_k,risk_j,cs_m) \to \\ 
	negConseqFor(g_i,ag_m,risk_j,cs_m) \nonumber \\ 
    ag_m \in Agent, g_i \in Goal, r_k \in Relation, risk_k \in Risk, cs_m \in Consequence 
\end{eqnarray}

A regra \ref{consrelationViol}, quando posta em linguagem natural é definida desta forma: "Uma violação de relação de um determinado agente, em um dado objetivo resulta em uma consequência ruim a ele. Essa consequência está atrelada ao risco da relação violada". 

Neste estudo o termo \textit{risco} deve ser analisado com muito cuidado. Isso, pois, dependendo do contexto, a complexidade deste termo é praticamente infinita e neste estudo a concepção deste termo se reduz a dois dos muitos possíveis usos. Neste modelo, risco é analisado como um evento que tem potencial de acontecer, contudo, nas relações de implicação um dos usos do termo risco advém de considerá-lo como evento que acontece apenas na ausência de uma dada condição ou de uma dada relação. O autor optou por essa tratativa ao estudar os conceitos presentes no referencial teórico em \ref{risksec} e ao analisar o estudo de caso (que será apresentado mais tarde). Com base nestes estudos, verificou-se que acidentes acontecem porque profissionais tentam executar uma certa atividade sem ter as condições apropriadas para isso e é à essa circunstância sobre o qual o risco está associado (em \cite{safety}, isso é explicado visando a melhoria da eficiência e da produção). Por exemplo, para poder navegar em alto mar a fim de poder pescar, um barco pesqueiro deve ter a sua disposição uma determinada condição climática. Se a tripulação decidir por navegar sem a presença da condição climática apropriada, então o barco está submetido ao risco de naufragar sob as consequências de morte da tripulação inteira. Portanto é com essa semântica que as relações de implicação \ref{consconditionViol} e \ref{consrelationViol} empregam o conceito de risco. 

Obviamente, existe a possibilidade do barco poder desbravar um mar sem as apropriadas condições e voltar para a terra a salvo. Contudo, considerar situações assim, apesar de serem interessantes, levam a um aprofundamento da complexidade deste modelo. Não que isso seja uma justificativa coerente para não se fazer isso, contudo - neste estudo o interesse reside em uma primeira versão que torne possível a modelagem de condições assim por meio de um vocabulário mais específico. Assim sendo, o autor decidiu por simplificar essa situação e considerar que toda a ação tomada por um agente sem que as condições necessárias estejam presentes, ou as relações apropriadas sejam feitas, resultam em penalidades associadas ao risco da ausência desses elementos.  

Dentro do que condiz ao conceito de sanção que é tratado neste estudo, apenas as regras \ref{consconditionViol} e \ref{consrelationViol} são sanções. Isso se deve ao fato de que essas regras consideram que o equívoco do agente, gerou penalidades a ele mesmo. Apesar de levar em consideração predicados associados a violação, as demais regras não são consideradas como regras de sanção porque elas apresentam uma condição onde o comportamento inapropriado de um agente A, resulta em consequências ruins a outros agentes. Como o erro do agente A não recai sobre si, é um inequívoco, dentro do escopo deste estudo, afirmar que ele sofreu uma sanção por conta disto. 


A regra \ref{entityViolaffect} é usada com o propósito de demonstrar que uma dada violação em uma certa relação afeta outras relações. Isso, pois muitas vezes o ato de não executar uma determinada relação não gera consequências imediatas no instante a ser considerado, contudo essas consequências se manifestam em relações futuras. Não somente isso, mas a regra \ref{entityViolaffect} também considera um dado componente de aleatoriedade que está atrelado com este tipo de raciocínio. O predicado $possOfNegConseqFor(r_n)$ semanticamente corresponde que existe a possibilidade de acontecer algo errado associado ao relacionamento $r_n$. O sentido deste termo é correspondido quando se verifica os elementos que causam este tipo de condição - que no caso desta regra isso envolve a ocorrência de uma violação em $r_k$, sendo que esse relacionamento afeta $r_n$.

\begin{eqnarray}\label{entityViolaffect}
	relationViol(ag_m,g_i,r_k) \wedge affectsRels(r_k,r_n) \nonumber \\
    \to possOfNegConseqFor(r_n)  \nonumber \\
    ag_m \in Agent, g_i \in Goal, r_k,r_n \in Relation, 
\end{eqnarray}

O entendimento desta regra pode ser feito ao considerar um exemplo que já foi mencionado neste texto ao apresentar o correspondente do predicado \textit{possOfNegConseqFor} e o predicado \textit{affectsRels}, onde um eletricista usa um bastão isolante para acessar um dado barramento. Naquela parte do texto o problema é modelado por meio de duas relações; $relBastaoMedidor$ (que define a relação que deve ser feita entre o bastão isolante com um aparelho medidor de correte de fuga) e $relBastaoBarramento$ (que consiste na relação entre o bastão com o barramento elétrico do quadro de energia). Tendo em vista que a ausência de uma medida em $g_{medida}$ afeta a possibilidade de ocorrer algum evento grave em $relBastaoBarramento$, é dado - para esse caso - como verdade o seguinte predicado $affectsRels(relBastaoMedidor, relBastaoBarramento)$. Assim sendo, em um cenário em que ocorre a violação de relação em $g_{medida}$, o seguinte raciocínio pode ser feito: $relationViol(eletricista_{medidor},g_{medida},relBastaoMedidor) \wedge affect(relBastaoMedidor, relBastaoBarramento) $ \\ $\to  possOfNegConseqFor(relBastaoBarramento)$.   

A regra \ref{entityViolaffect} demonstra como um agente pode ser submetido a consequências ruins sem necessariamente ser culpado por isso. Contudo, essa regra denota apenas possibilidade, não demonstrando o que acontece efetivamente quando o agente é submetido ao lado não favorável da possibilidade. Essa situação está atrelada a \ref{paybutiamnotguilty}. Para lidar com as situações onde um agente é submetido a condições ruins, fez-se o uso do predicado $possOfNegConseqFor(g_i,ag_m,risk_j,cs_m) $. Entretanto, diferente das regras \ref{consconditionViol}, \ref{consrelationViol}, essas consequências negativas tem seus correspondentes semânticos em outros predicados. O predicado $possOfNegConseqFor(r_k)$ é invocado com o propósito demonstrar que $r_k$ apresenta a possibilidade da ocorrência de um evento ruim mesmo que o agente que esteja executando essa relação não faça nada de errado. Contudo, esse predicado só denota a possibilidade. Para que o sentido semântico de que a possibilidade de um evento ruim realmente acontece foi considerado o uso do predicado $happensNegConseqFor(r_k)$. Para o contexto desta regra, a semântica deste predicado exibe o seguinte significado: "O evento ruim associado a essa relação realmente aconteceu". Nesta situação que se faz necessário adotar a outra concepção associada ao termo risco que é adotado a este modelo. Nesta regra, esse termo é adotado como um evento em potencial devido a incerteza associada ao evento. 

Para compreender melhor essa situação é possível voltar ao exemplo do eletricista-bastão isolante-quadro de energia. Como já citado anteriormente o fato do agente medidor não executar sua atividade gera uma incerteza sobre a condição do isolamento do bastão. Se a medida for executada com sucesso (partido do pressuposto de que o medidor está em condições apropriadas de funcionamento), a condição do bastão é revelada eliminando qualquer incerteza a respeito disto. Contudo, como está sendo considerado um cenário em que isso não foi feito, a não execução de $relBastaoMedidor$ resultou no surgimento do risco $eletrocutado$ com uma consequência de morte. Esse risco é definido como um potencial evento até que o eletricista de acesso ao barramento faz uso da ferramenta. Por conta disto, se usa o predicado $hasRisk(r_k,risk_j,cs_m)$. Tendo em vista que isso se dá por uma relação que está atrelada a um objetivo, se faz necessário considerar $requiresCirc(g_i,r_k) \wedge (r_k \in rg_n) $ e  $instanceOfRel(r_k)$ . Para verificar a ação do agente nesta situação, o predicado $starts(ag_m,g_i)$ também deve compor a regra.


\begin{eqnarray}\label{paybutiamnotguilty}
	possOfNegConseqFor(r_k) \wedge  happensNegConseqFor(r_k) \wedge requiresCirc(g_i,r_k) \nonumber \\ 
	\wedge instanceOfRel(r_k) \wedge hasRisk(r_k,risk_j,cs_m) \wedge starts(ag_m,g_i) \nonumber \\ 
	\to negConseqFor(g_i,ag_m,risk_j,cs_m) \nonumber \\ 
    r_k \in Relation, g_i \in Goal, risk_k \in Risk, cs_m \in Consequence
\end{eqnarray}

O exemplo em voga pode ser implementado nesta regra da seguinte forma: 

\begin{eqnarray}\nonumber
   possOfNegConseqFor(relBastaoBarramento) \nonumber \\
    \wedge happensNegConseqFor(relBastaoBarramento) \nonumber \\ 
    \wedge requiresCirc(g_{acessoBarramento},relBastaoBarramento) \nonumber \\  
    \wedge instanceOfRel(relBastaoBarramento) \nonumber \\ 
    \wedge hasRisk(relBastaoBarramento,eletrocutado,morte) \nonumber \\  
    \wedge starts(eletricista_{executor},g_{acessoBarramento}) \nonumber \\ 
    \to negConseqFor(g_{acessoBarramento},eletricistaExecutor,eletrocutado,morte) \\ \nonumber
\end{eqnarray}

O exemplo se traduz na situação em que um bastão apresenta uma possibilidade de estar com o seu isolamento comprometido e isso resulta em um risco de eletrocutar o profissional que o usa resultando na morte dele. Portanto, no momento em que a ferramenta é usada o eletricista morre eletrocutado, por que esse bastão pertencia as ferramentas cujo isolamento estava deteriorado. 

A violação de entidade, dada pela regra \ref{consvioent}, diferente das demais, resulta apenas no encerramento da atividade referente ao objetivo onde o método foi invocado. Os engenheiros desse modelo definiram essa regra partindo do pressuposto que a ausência de uma ferramenta, profissional, peça de substituição ou máquina simplesmente gera o impedimento do prosseguimento das atividades. Voltando ao exemplo do eletricista, se o profissional não tiver o bastão isolante para executar a ação, ele simplesmente não consegue dar prosseguimento ao objetivo fazendo com que o procedimento seja encerrado naquele exato instante. 

\begin{eqnarray}\label{consvioent}
	entityViol(ag_m,g_i,e_k) \to stopped(g_i) \nonumber \\  
    ag_m \in Agent, g_i \in Goal, e_k \in Entity \\ \nonumber
\end{eqnarray}

Em termos de linguagem natural, a regra \ref{consvioent} é definida da seguinte forma: Se acontecer uma violação de entidades, então o procedimento é encerrado no objetivo onde aconteceu. 

A regra \ref{badcons} advém do pressuposto de que na ocorrência de uma calamidade onde um profissional sai extremamente ferido ou morto (ocorrência do acidente), os demais envolvidos na manutenção não continuam por executar os procedimentos. 
 
 \begin{eqnarray}\label{badcons}
	negConseqFor(g_k,ag_m,risk_j,cs_m) \to stopped(g_k) \nonumber \\ 
    g_k \in Goal, risk_j \in Risk, cs_m \in Consequence
\end{eqnarray}

Essa regra, no escopo da linguagem natural, pode ser lida desta forma: Se acontecer um evento ruim em que um profissional sai morto ou gravemente ferido, então a manutenção é encerrada no objetivo onde a fatalidade aconteceu. Não há como afirmar que as regras  \ref{consvioent} e \ref{badcons} se aplicam para todo tipo de situação em qualquer procedimento. Operações militares, por exemplo, não se enquadram em situações assim. Isso, pois a morte de um soldado ferido não impede que o resto do batalhão continue em conflito. Contudo, o autor deste estudo entendeu que o pressuposto dessas duas regras englobam diversos cenários que implica no interesse deste estudo, tais como; cenário industrial, subestação, usinas de produção de energia, certas atividades hospitalares e entre outras da mesma natureza.  

A regra estabelecida pela figura \ref{wenStop} define o critério para quando um dado objetivo é considerado como atingido. Isso ocorre quando todos os agentes $ag_n | n = i ... j$ que são obrigados a atingir um certo objetivo $ g_k $ fazem isso sem a ocorrência e uma interrupção $stopped(g_k,ag_n)$. A expressão \ref{wenStop} retrata isso. Diferente das outras regras, o autor prefiriu especificar essa expressão como um algoritmo que avalia se um determinado objetivo foi interrompido agente por agente através de um \textit(foreach) sobre $agentArray$ (um $array$ que trás todos os agentes que tentaram alcançar o objetivo $goal$). 

\begin{figure}[H]
  \centering
  \includegraphics[width=0.6\linewidth]{figure/algrule6.png} 
  \caption{Condição para definir se um dado objetivo foi atingido ou não} \label{wenStop}  
\end{figure}

Se o teste dado por $if(stopped(goa,agent))$ é verdadeiro para pelo menos um dos agentes, então a função $ifNotStopped(agentArray,goal)$ retorna falso. Se essa situação não acontecer para todos os agentes carregados em $agentArray$, então o algoritmo determina um segundo teste, que é dado pela função $allAgenteObliged(agentArray,goal)$. Essa função verifica se todos os agentes que são obrigados a alcançar o objetivo $goal$ estão contidos em $agentArray$. No contexto desse algoritmo, se a função $allAgenteObliged(agentArray,goal)$ retorna falso, então $ifNotStopped$ também retorna falso, caso contrário $ifNotStopped$ retorna verdade. Se a função $ifNotStopped(agentArray,goal)$ retorna verdade, então o predicado $reached(goal)$ é verdade também. O algoritmo apresenta a função $isTrue$ onde o argumento é o predicado $reached(goal)$. A função $isTrue(arg)$ sempre retorna verdade e tem como por objetivo informar que o seu respectivo argumento é verdade. Isso é necessário tendo em vista a diferença do formalismo adotado pelas demais regras em relação a \ref{wenStop}.

Na linguagem natural essa expressão fica da seguinte forma: Se todos os agentes que têm permissão para alcançar um dado objetivo fizeram sem que esse tenha sido interrompido e considerando que um subgrupo deles é constituído por agentes que são obrigados a isso, então o objetivo é dado como alcançado.

A regra \ref{rolenextgoal} apresenta a condição adequada para quando um agente está habilitado para atingir novos objetivos. Para isso, ele deve possuir um papel onde existe uma permissão para que ele possa atingir o próximo objetivo. Isso é traduzido por $ adoptsRole(ag_n,\rho_m) \wedge hasPermission(\rho_m,g_j) $. Não apenas isso, mas o objetivo atual do agente deve ter sido atingido $ reached(g_i) $ e o objetivo em interesse deve estar associado como predicado $nextGoal(g_i,g_j)$. O termo $enabledToStart(ag_i,g_j)$ corresponde semanticamente apenas que o agente está habilitado a buscar novos objetivos mas não significa que isso implicará em $starts(ag_i,g_j)$ pois o que decide esses processos de transição consiste em aspectos que não correspondem a esse modelo. Essa dinâmica é discutida mais tarde na seção de \textit{Predicados de Controle}.

\begin{eqnarray}\label{rolenextgoal}
	adoptsRole(ag_n,\rho_m) \wedge hasPermission(\rho_m,g_j) \wedge nextGoal(g_i,g_j) \wedge reached(g_i) \nonumber \\
	\to enabledToStart(ag_i,g_j) \nonumber \\
    ag_i, ag_n \in Agent, \rho_m \in Role, g_j \in Goal, g_i \in Goal
\end{eqnarray}

Em linguagem natural, a regra \ref{rolenextgoal} exibe o seguinte: "Se um agente que alcançou um objetivo atual tem um papel que lhe dá permissão para buscar o próximo objetivo, então esse agente está habilitado para fazer isso"

A regra \ref{rolelastgoal} apresenta a condição de parada do agente em relação ao seu papel. Isso, pois se o agente, que tem um determinado papel, cumpriu com todos os objetivos designados a ele, então ele deve encerrar sua operação. A verificação do papel é dado por $adoptsRole(ag_n,\rho_m) \wedge hasPermission(\rho_m,g_i)$, a análise semântica sobre o último objetivo associado a um certo papel é dado por $lastGoal(g_i,\rho_m)$ e a verificação se aquele último objetivo foi atingido é dado por $reached(g_i)$. 

\begin{eqnarray}\label{rolelastgoal}
	adoptsRole(ag_n,\rho_m) \wedge hasPermission(\rho_m,g_i) \wedge lastGoal(g_i,\rho_m) \wedge reached(g_i) \nonumber \\
	\to stopped(g_i) \nonumber \\
    ag_n \in Agent, \rho_m \in Role, g_i \in Goal
\end{eqnarray}

Portanto, a regra \ref{rolelastgoal} em linguagem natural é definida da seguinte maneira; Se um agente cumpriu com todos os objetivos associados a permissão do papel dele, então esse agente deve encerrar suas atividades (em relação a esse papel). 

		\subsection{Diagrama de Atividades} \label{umldiagram}
			As Figuras \ref{atividiagram1}, \ref{atividiagram2} e \ref{atividiagram3} (o diagrama foi quebrado em três Figuras distintas com o propósito de melhorar a qualidade da resolução)  apresentam a aplicação das regras em termos de diagrama de atividades. Essas Figuras deve ser entendidas como uma proposta de orientação das regras registradas na subseção \ref{regras}. Há outras maneiras de organizar essas regras em diferentes diagramas de atividades, sendo que a apresentada neste texto é apenas uma dessas. Isso se deve ao fato de que esse estudo pretendo fornecer um modelo e não um algoritmo. 

O primeiro termo desta Figura corresponde a "carregar todos os agentes". Esse elemento é indiferente à estrutura das regras do modelo. A existência desta atividade no diagrama se dá por finalidades de implementação, uma vez que para uma máquina poder processar todos as atividades, primeiramente se faz necessário que informações sobre os agentes sejam carregadas na memória. As atividades "selecione um dos agentes", "carregar os objetivos", "há objetivos que não foram alcançados" e "o agente escolhe por tentar alcançar o objetivo" fazem referência as regras \ref{rolenextgoal} e \ref{reldeonticrole}. Aquela analisa qual é a próxima regra que esta em condições de serem atingidas pelo agente e esta verifica a permissão do agente no que tange a possibilidade de poder adotar o objetivo. 

O ponto de decisão "todas as condições necessárias para esse objetivo estão presentes?" A atividade "violação de condição" fazem referência a regra \ref{conditionViol}, que define uma violação de condição para o caso do agente tentar executar alguma atividade sem que todas as condições estejam presentes naquele instante.

O ponto de decisão "todas as entidades necessárias para esse objetivo estão presentes?" A atividade "violação de relação" fazem referência a regra \ref{entityViol}, pois definem o ocorrido no que diz respeito a ausência de uma entidade ao verificar um dado objetivo em análise. 

\begin{figure}[H]
  \centering
  \includegraphics[width=1.1\linewidth]{figure/diag1.jpg} 
  \caption{Diagrama de atividades do modelo (1)}
  \label{atividiagram1}
\end{figure}


\begin{figure}[H]
  \centering
  \includegraphics[width=1.1\linewidth]{figure/diag2.jpg} 
  \caption{Diagrama de atividades do modelo (2)}
  \label{atividiagram2}
\end{figure}

\begin{figure}[H]
  \centering
  \includegraphics[width=1.1\linewidth]{figure/diag3.jpg} 
  \caption{Diagrama de atividades do modelo (3)}
  \label{atividiagram3}
\end{figure}


O ponto de decisão "todas as relações para esse objetivo estão presentes" e "violação de relação" representam a regra \ref{relationViol}. Isso se deve ao fato de que essas atividades avaliam se uma das relações necessárias para cumprir com o objetivo não está presente, resultando em uma violação de relacionamento. 

As atividades "violação de condição", e "sobre consequências negativas a sua integridade física vinculada a relação/condição" condizem com a regra \ref{consconditionViol} que define as consequências de uma violação de condição. A atividade "violação de relação" em conjunto com a segunda atividade das presentes na sentença anterior fazem referência a regra \ref{consrelationViol}, pois apresenta as sanções relacionadas a uma violação de relacionamento. 

O ponto de decisão "há risco associado a esse relacionamento?", "afeta  outra relação" e "relação afetada ganha possibilidade de ocorrer um evento ruim", apontam para a regra \ref{entityViolaffect} pois ambas situações representam como a ocorrência de uma violação de relacionamento afeta um outro relacionamento. A regra \ref{paybutiamnotguilty} corresponde aos seguintes aspectos do diagrama, "há alguma relação com potencial para gerar algum evento ruim?", "evento ruim acontece?","sobre consequências negativas a sua integridade física vinculadas a relação" tendo em vista a equivalência semântica entre esses elementos sendo que ambas as representações se preocupam com a análise de relações que possuem possibilidades de algum evento ruim surgir sobre o agente do objetivo. 

A regra \ref{consvioent} é representada por "violação de entidade" e pelo elemento que indica o fim do programa. Isso se deve ao fato de que a regra \ref{consvioent} determina o encerramento do processo na ocorrência de uma violação de entidade. 

Os eventos "agente concluiu o objetivo", "os outros agentes que eram obrigados, concluíram esse objetivo" e "verificar agentes para concluir esse objetivo" apontam para as regras \ref{rolenextgoal}, \ref{rolelastgoal}. Esse conjunto de atividades e pontos de decisões apresentam os critérios para definir quando um objetivo foi totalmente alcançado (que é a mesma finalidade dessas regras justificando a equivalência entre ambos formalismos).  A regra \ref{wenStop} é explicitada no diagrama toda vez que a atividade "sobre consequências negativas a integridade física vinculada  a relação/condição" aponta para o fim do programa, tendo em vista que ambras representações definem o encerramento das atividades na ocorrência de feridos. 

O diagrama presente nas Figuras \ref{atividiagram1}, \ref{atividiagram2} e \ref{atividiagram3} apresenta \ref{conditionViol} como a primeira regra de violação a ser executada. O motivo disto reside no fato de que essa regra verifica se o agente respeitou todas as condições do ambiente. Se o agente não fizer isso, ele está sujeito a penalidades físicas encerrando o programa. Ou seja, não abre margens para verificação de outras violações, porque em um caso real, alguém que executa uma atividade sem que todas as condições estejam presentes, então esse alguém está fadado a encerrar qualquer ação em curso. Mesmo que esse alguém estivesse na condição de cometer outras violações, não seria possível fazê-las pois esta primeira violação cometida por ele foi o suficiente para interromper os procedimentos como um todo. Esse mesmo principio fundamenta o sequenciamento das demais regras, sendo que logo em seguida é a \ref{entityViol}, pois se alguma entidade  necessária (ferramenta), não estiver presente no cenário,então não existe possibilidade da continuidade dos procedimentos inviabilizando a realização das relações não fazendo sentido verificar \ref{relationViol}. Contudo, os engenheiros definem essa estrutura apenas como uma proposta que deve ser modificada em função dos interesses da aplicação. Por exemplo, supondo que uma equipe tenha o interesse de usar este modelo para desenvolver jogos sérios com a intenção de analisar todas as violações que podem ser cometidas por um jogador um dado cenário, então para esse caso não há sentido usar esse fluxo de atividades. Em uma condição assim, os engenheiros do jogo devem - usando as mesmas regras - mudar o fluxo do diagrama de atividade para verificar todas as regras de violação antes de analisar se o programa deve ou não ser interrompido.
		\subsection{Predicados de Controle e de Estrutura} \label{cenarios}
			Esse estudo apresenta duas categorias de predicados: \textit{abertos} e \textit{fechados}. Os predicados fechados são aqueles cujo usuário do modelo não possui a liberdade de definir sua estrutura interna por intermédio de outras regras lógicas ou por valores. Isso se dever ao fato de que esses vocábulos tem sua estrutura alicerçada nas concepções deste modelo sendo que são essenciais para que o modelo funcione como foi concebido para ser. Assim sendo, o modelador deve fazer uso delas apenas com o propósito de especificar os objetos de interesse. Em termos práticos não existe dificuldade em identificar esses predicados, pois sua própria natureza não abre margem para que o modelador consiga escrever novos predicados e novas regras para determinar o seu respectivo valor. 

Esses predicados são $thereIsRelation(r_l,e_i,e_k)$, $adoptsRole(ag_n,\rho_m)$, $hasObligation(\rho_m,g_j)$,
$hasPermission(\rho_m, g_j)$, $isReached(g_k)$, $stopIn(g_n, ag_m)$, $nextGoal(g_i,g_j)$, $hasCondition(g_i,cg_n)$,
$hasEntity(g_i,eg_m)$, $hasRelation(g_i,rg_m)$, $violationCondition(ag_i,g_j,c_k) $, $ violationRelation(ag_i,g_j,r_k) $,
$ violationEntity(ag_i,g_j,e_k) $,  $ hasRisk(X, risk_j, cs_k) $, $possibilityHappensBadEvent(r_l)$, 
$affectsOtherRelations(r_k,r_n)$, $consequenceOfBadEvenet(g_k, ag_i,risk_k,cs_m)$  e $lastGoal(g_i,\rho_m)$. 


Para exemplificar, pode-se considerar o predicado $thereIsRelation(r_l,e_i,e_k)$. Se existir uma entidade $A$, uma entidade $B$ e uma relação entre $relAB$, então esse termo é escrito desta forma: $thereIsRelation(relAB,A,B)$. O valor verdade deste predicado não pode ser modificado para a criação de algum cenário e nem pode ser determinado por outras regras. Se modelador fizer isso então estará modificando a estrutura do modelo. Ou seja, esse é um predicado fechado no que tange a aspectos fundamentais aos aspectos semânticos desta representação. 

Por outro lados os predicados \textit{abertos} possuem um correspondente sintático e semântico no modelo mas os seus valores devem ser forçados conforme o cenário que se deseja criar ou conforme outras regras de implicabilidade. A não determinação destes predicados inviabilizam que o modelo seja analisado de forma procedural. Faz parte deste conjunto os seguintes termos: $isPresent(X)$,$tryReach(ag_i,g_j)$ e $happensBadEvent(r_m)$.

Pode-se considerar o seguinte exemplo: Um agente $ag_a$ deve executar o objetivo $g_1$ e $g_2$, os predicados a seguir implementam este modelo para o exemplo:

\begin{itemize}
    \item $nextGoal(g_1,g_2)$
    \item $thereIsRelation(rAB,entA,entB)$
    \item $thereIsRelation(rCE,entC,entE)$
    \item $entA \in eg_1 ,entB \in eg_1$
    \item $entC \in eg_2, entD \in eg_2$
    \item $rAB \in rg_1$,
    \item $rCE \in rg_2 $
    \item $cond_1 \in {cg_1} $
    \item $ hasCondition(g_1,cg_1)$
    \item $ hasCondition(g_2,cg_2 )$
    \item $ hasEntity(g_1,eg_1) $
    \item $ hasEntity(g_2,eg_2) $
    \item $ hasRelation(g_1, rg_1 )$
    \item $ hasRelation(g_2, rg_2)$ 
    \item $ affectsOtherRelations(rAB,rCE) $ 
    \item $hasRisk(cg_1,risk_1,cs_1)$
    \item $hasRisk(rCE,risk_2,cs_2)$
    \item $adoptsRole(ag_a,\rho_1)$
    \item $hasObligation(\rho_1,g_1)$
    \item $hasObligation(\rho_1,g_2)$
\end{itemize}

Apesar de todos os predicados denotarem uma dada condição e de serem o suficientes para definir uma certa representação de mundo, não é possível fazer raciocínio algum. Isso, pois não se sabe quais são as ações dos agentes e não se sabe quais condições e cenários se deseja representar.  

Para isso, se faz necessário definir um cenário de mundo. Por exemplo, pode-se definir o seguinte cenário; $tryReach(ag_a,g_1)$, $\neg isPresent(rAB), tryReach(ag_b,g_2)$ e $ possibilityHappensBadEvent(rCE) \to $ $happensBadEvent(rCE)$. 

Para esse caso é possível obter as seguintes relações de inferência: 

\begin{eqnarray}
    hasRelation(g_1,rg_1)\wedge \neg isPresent(rAB) \wedge (rAB \in rg_1) \wedge tryReach(ag_a,g_1) \to \nonumber \\
    violationRelation(ag_a,g_1,rAB) 
\end{eqnarray}

\begin{eqnarray}
    violationRelation(ag_a,g_1,rAB)  \wedge affectsOtherRelations(rAB,rCE) \nonumber \\
    \to possibilityHappensBadEvent(rCE)  
\end{eqnarray}

\begin{eqnarray}
    possibilityHappensBadEvent(rCE) \to happensBadEvent(rCE) 
\end{eqnarray}

\begin{eqnarray}\label{paybutiamnotguilty}
    possibilityHappensBadEvent(rCE) \wedge  \nonumber \\
    happensBadEvent(rCE) \wedge  \nonumber \\
    hasRelation(g_2,rg_2) \wedge  \nonumber \\
    (rCE \in rg_2) \nonumber \wedge  \nonumber \\
    hasRisk(rCE,risk_2,cs_2) \wedge  \nonumber \\
    tryReach(ag_a,g_2) \nonumber  \nonumber \\
    \to consequenceOfBadEvent(g_2,ag_a,risk_2,cs_2) 
\end{eqnarray}


\begin{eqnarray}\label{badcons}
    consequenceOfBadEvent(g_2,ag_a,risk_2,cs_2) \to stopIn(g_2) 
\end{eqnarray}


Esses raciocínios e conclusões só foram possíveis porque o modelador forçou o valor de três predicados e definiu uma relação de implicação. Isso acontece por conta de três motivos: 1 - Esse é um modelo de \textit{SMA}, 2 - esse modelo apresenta grau de liberdade para escolher a disposição das entidades, condições e relações e 3 - não há como definir a solução de uma possibilidade. 

Para o primeiro caso o predicado $tryReach(ag_i,g_j)$ é resultado de estados internos do agente. Por exemplo, o desenvolvedor pode programar um agente que possui o estado de medo, então sobre certas condições ele resolve não tentar alcançar o objetivo gerando valor falso para esse predicado, ou pode definir um agente que pondera pouco ao decidir se deve ou não tentar alcançar um dado objetivo. Isso pode ser feito por meio de modelos de agentes tais como: agentes lógicos, arquitetura BDI, agentes reativos e agentes em camada. Se for do interesse do modelador, o mesmo pode simplesmente definir o valor verdade para o predicado em certas condições. 

O mesmo se aplica para o $isPresent(X)$ onde desenvolvedor pode definir um cenário que, por meio de estados internos os agentes esqueceram uma determinada ferramenta em um certo local ou, por exemplo, que o agente apresenta um algoritmo para determinar qual ferramenta é a mais apropriada para uma dada condição. Assim como o modelador é livre para gerar diferentes cenários simplesmente por definir valores diferentes para $isPresent(X)$. Por exemplo, supondo que uma equipe está desenvolvendo um jogo sério para avaliar profissionais de uma certa industria. Para avaliar a competência dos trabalhadores, o modelador poderá usar este predicado por adicionar ou remover entidades e condições com base nas necessidades de avaliação.

O terceiro motivo reside no fato de que o predicado $possibilityHappensBadEvent(X)$ denota apenas que existe uma possibilidade de ocorrer algum determinado evento ruim  $happensBadEvent(X)$. Contudo, se esse evento ocorrerá ou não, não é possível definir pois isso depende de questões estatísticas do objeto de estudo. Assim sendo, o usuário deste modelo possui algumas possibilidades de ação, tais como: quando $possibilityHappensBadEvent(X)$  for verdade, então definir $happensBadEvent(X)$ por meio de um número aleatório, para dadas situações onde ocorre $possibilityHappensBadEvent(X)$ tratar $happensBadEvent(X)$ como verdade e para dadas situações tratar $happensBadEvent(X)$ como falso ou definir verdade para $happensBadEvent(X)$ como base em algum estudo probabilístico. Isso dependerá da finalidade dos modeladores. 
	\section{Caso de Estudo} \label{studycase}
			Sete profissionais de linha viva (profissionais que realizam manutenção em equipamentos elétricos energizados) são designados com o propósito de realizar 
a substituição de um isolador de pedestal. Os papeis desses desses profissionais são; 1 supervisor, 5 executores. A manutenção deve ser executada apenas 
sobre as seguintes condições: céu ensolarado e umidade relativa do ar menor que 70 porcento. Todos os profissionais devem possuir os EPI's necessários: 
capacete, óculos de sol, roupa isolante e antichamas, luvas isolantes e botas isolantes. Os profissionais que entram no potencial devem estar vestidos de 
roupa condutiva e cabo guarda. As ferramentas necessárias para resolver esse problema são: bastão garra de diâmetro 64 x 3600 mm, sela de diâmetro 65 , 
colar, corda de fibra sintética, carretilha, chave com catraca, bastão universal, soquete adequado, locador de pino e bastão com soquete multiangular. O método selecionado 
para esse tipo de manutenção é a distância onde o eletricista não acessa diretamente o potencial, mas faz isso por intermédio de um bastão isolante. A substituição do isolador 
de pedestal pode ser escrita nos seguintes objetivos: 

\begin{enumerate}
	\item Limpar, secar e testar corda.
	\item Instalar Bastão Garra na estrutura com o pedestal a ser substituído.
	\item Instalar sela com colar na estrutura
	\item Amarrar o bastão na parte superior da estrutura com a corda.
	\item Amarrar o olhal do bastão ao cavalo da sela atrás de uma corda.
	\item Instalar um segundo conjunto bastão e sela no lado oposto da estrutura.
	\item Enforcar um estropo de Náilon no corpo do isolador.
	\item Colocar a extremidade do estropo no gancho da corda de serviço.
	\item Afrouxar os parafusos do conector que prendem a barra ao isolador.
	\item Terminar de retirar os parafusos com o bastão com o soquete multiangular.
	\item Elevar a barra através da corda que une a sela ao bastão.
	\item Apertar o colar através da porca borboleta.
	\item Segurar firmemente a corda de serviço.
	\item Sacar parafusos da base da coluna.
	\item Baixar o isolador ao solo
	\item Içar o Isolador
	\item Colocar Parafusos na base da coluna.
	\item Baixar a barra para que a mesma apoie no novo isolador.
	\item Colocar os parafusos do conector que prende a barra ao novo isolador. 
	\item Retirar Equipamentos
\end{enumerate}

A tabela \ref{agents} apresenta todos os agentes que fazem parte da manutenção. 
\begin{table}[H]
\scalefont{0.8}
\centering
\begin{tabular}{|l|l|}
\hline
\textbf{símbolo} & \textbf{significado} \\ \hline
agente1 & Um dos agentes participantes da manutenção \\ \hline
agente2 & Um dos agentes participantes da manutenção \\ \hline
agente3 & Um dos agentes participantes da manutenção \\ \hline
agente4 & Um dos agentes participantes da manutenção \\ \hline
agente5 & Um dos agentes participantes da manutenção \\ \hline
agente6 & Um dos agentes participantes da manutenção \\ \hline
agente7 & Um dos agentes participantes da manutenção \\ \hline
\end{tabular}
\caption{Os agentes que constituem uma manutenção}
\label{agents}
\end{table}

 A tabela \ref{roles} apresenta todas as funções que deverão ser exercidas pelos agentes.

\begin{table}[H]
\scalefont{0.8}
\centering
\begin{tabular}{|l|l|}
\hline
\textbf{papel} & \textbf{descrição} \\ \hline
supervisor & Atribui papel a outros profissionais \\ \hline
executor1 & Tem como por finalidade executar certas atividades manuais vinculadas a manutenção \\ \hline
executor2 & Tem como por finalidade executar certas atividades manuais vinculadas a manutenção \\ \hline
executor3 & Tem como por finalidade executar certas atividades manuais vinculadas a manutenção \\ \hline
executor4 & Tem como por finalidade executar certas atividades manuais vinculadas a manutenção \\ \hline
executor5 & Tem como por finalidade executar certas atividades manuais vinculadas a manutenção \\ \hline
\end{tabular}
\caption{Os papeis relevantes para a ocorrência da manutenção}
\label{roles}
\end{table}

A tabela \ref{agentsroles} define a relação $hasRole(ag_n,\rho_m)$ onde $ag_n$ é representado pela coluna agente e $\rho_m$ é representado pela coluna papel.

\begin{table}[H]
\scalefont{0.8}
\centering
\begin{tabular}{|l|l|}
\hline
\textbf{agente} & \textbf{papel} \\ \hline
agente1 & supervisor \\ \hline
agente2 & executor1 \\ \hline
agente3 & executor1 \\ \hline
agente4 & executor2 \\ \hline
agente5 & executor3 \\ \hline
agente6 & executor4 \\ \hline
agente7 & executor5 \\ \hline
\end{tabular}
\caption{Relação $hasRole(ag_n,\rho_m)$}
\label{agentsroles}
\end{table}

A tabela \ref{artefacts1} e \ref{artefacts2} apresentam todos artefatos que fazem parte da descrição deste estudo de caso.

\begin{table}[H]
\scalefont{0.8}
\centering
\begin{tabular}{|l|p{0.8\linewidth}|}
\hline
\textbf{artefato} & \textbf{descrição} \\ \hline
capacete & EPI usado pelo profissional para proteger a cabeça \\ \hline
óculos & Óculos usado para evitar dificuldades de enxergar presentes em dias claros \\ \hline
roupagem & Consiste em roupas isolantes e anti-chamas \\ \hline
luva & Luvas Isolantes \\ \hline
bota & Botas Isolantes para evitar que o profissional seja eletrocutado \\ \hline
bastaoGarra & bastão isolante que possui uma ferramenta em estrutura de garra. 64 X 3600 mm \\ \hline
sela & Possui diâmetro 65 mm, é fixada na torre para sustentar o bastão. \\ \hline
colar & Estrutura que fica fixa na sela, bastão isolante é travado no colar. \\ \hline
corda & Corda Isolante. \\ \hline
carretilha & Carretilha que, em conjunto com a corda, é usada para mover material na vertical. \\ \hline
bastaoUniversal & Bastão isolante que permite o acoplamento de múltiplas ferramentas. \\ \hline
\end{tabular}
\caption{Definindo todos os artefatos presentes na manutenção}
\label{artefacts1}
\end{table} 

\begin{table}[H]
\scalefont{0.8}
\centering
\begin{tabular}{|l|p{0.8\linewidth}|}
\hline
\textbf{artefato} & \textbf{descrição} \\ \hline
soquete & Usado na manipulação de parafusos. \\ \hline
locador & Usado como pino direcional em alinhamento de furo de parafusos, auxiliado na inserção de pinos e parafusos. \\ \hline
bastaoGarra & Bastão Universal que possui uma garra. \\ \hline
isoladorVelho & Isolador de pedestal danificado a ser substituído \\ \hline
isoladorNovo & Isolador de pedestal novo que será posicionado no local do isolador velho. \\ \hline
torre & Estrutura metálica onde fica fixo o isolador \\ \hline
condutor & Em formato de cabo, fica fixo sobre o topo do isolador.e é por onde passa grandes quantidades de energia elétrica. \\ \hline
estropo & pano firme usado para segurar Isolador quando estiver suspenso \\ \hline
pano & pano usado para limpar ferramentas \\ \hline
glicerina & substância usada para limpar as ferramentas adequadamente \\ \hline
condutímetro & Medidor de corrente de fuga sobre o bastão universal. \\ \hline
parafuso & Parafusos prendem o conector condutor-Isolador e também prendem o Isolador a base \\ \hline
conector & Estrutura que tem como por finalidade manter condutor,cabeçote do isolador em conjunto. \\ \hline
\end{tabular}
\caption{Definindo todos os artefatos presentes na manutenção}
\label{artefacts2}
\end{table} 

A tabela \ref{g} apresenta os objetivos dados pela coluna $objetivo$ bem com sua descrição. Essa tabela também apresenta os conjuntos $gp_i$ dado pela coluna pré-requisitos. Assim sendo, essa tabela também apresenta a relação entre os objetivos e seus respectivos pré-requisitos, ou seja, a relação $isPresentRequisite(gp_i,g_j)$. 

\begin{table}[H]
\scalefont{0.8}
\centering
\begin{tabular}{|l|l|p{0.6\linewidth}|}
\hline
\textbf{objetivo} & \textbf{pré-requisito} & \textbf{Descrição} \\ \hline
gSupervisor & g0 & Atribui objetivos aos demais agentes. \\ \hline
g0 &  \O & Vestir os AP'Is \\ \hline
g1 &  gSupervisor & Limpar, secar e testar ferramentas com material isolante. \\ \hline
g2 &  g1 & Medir a corrente de fuga de ferramentas isolantes \\ \hline
g3 &  g2 & Instalar sela com colar na estrutura \\ \hline
g4 &  g3 & Passar o bastão garra por dentro do olhal do colar. \\ \hline
g5 &  g4 & Amarrar o bastão garra na parte superior da estrutura com a corda, fixar no condutor \\ \hline
g6 &  gSupervisor & Amarrar o olhal do bastão garra ao cavalo da sela atrás de uma corda. \\ \hline
g7 &  g6 & Instalar sela com colar no outro lado da estrutura estrutura \\ \hline
g8 &  g7 & Passar o bastão universal por dentro do olhal do colar \\ \hline
g9 &  g8 & Pender carretilha no bastão Universal. \\ \hline
g10 &  g9 & Amarrar o bastão universal na parte superior da estrutura com a corda; \\ \hline
g11 &  g10 & Amarrar o olhal do bastão universal ao cavalo da sela atrás de uma corda. \\ \hline
g12 &  g11,g5 & Rotacionar estrutura olhal garra em 45 graus. \\ \hline
g13 &  g12 & Enforcar um estropo de Náilon no corpo do isolador velho. \\ \hline
g14 &  g13 & Colocar a extremidade do estropo no gancho da corda de serviço. \\ \hline
g15 &  g14 & Afrouxar os parafusos do conector que prendem a barra ao isolador. \\ \hline
g16 &  g15 & Terminar de retirar os parafusos com o bastão com o soquete multiangular. \\ \hline
g17 &  g16 & Elevar o condutor através da corda que une a sela ao bastão. \\ \hline
g18 &  g17 & Apertar o colar através da porca borboleta. \\ \hline
g19 &  g18 & Sacar parafusos da base da coluna. \\ \hline
g20 &  g19 & Segurar firmemente a corda de serviço,baixar o isolador ao solo \\ \hline
g21 &  g20 & Passar Estropo no Isolador Novo \\ \hline
g22 &  g21 & Colocar a extremidade do estropo no gancho da corda de serviço. \\ \hline
g23 &  g22 & Içar o Isolador \\ \hline
g24 &  g23 & Colocar Parafusos na base da coluna. \\ \hline
g25 &  g24 & Baixar o condutor para que a mesma se sustente no novo isolador. \\ \hline
g26 &  g25 & Colocar os parafusos do conector que prende a barra ao novo isolador. \\ \hline
g27 &  g26 & Retirar Equipamentos \\ \hline
\end{tabular}
\caption{Define e descreve os objetivos bem como os respectivos pré-requisitos}
\label{g}
\end{table}

A tabela \ref{condition} apresenta $c_k$ dado pela coluna condição e pela coluna descrição. Essa tabela define a relação $hasRisk(c_k,risk_j,f_m)$ onde $risk_j$ é descrito pela coluna risco e $f_m$ é descrito pela coluna fatalidade. 

\begin{table}[H]
\scalefont{0.8}
\centering
\begin{tabular}{|l|p{0.6\linewidth}|l|l|}
\hline
\textbf{condição} & \textbf{descrição} & \textbf{risco} & \textbf{fatalidade} \\ \hline
umidade70 & Umidade Relativa do Ar deve ser inferior a setenta porcento. & eletrocutado & morte \\ \hline
noVento & Não deve haver vento durante os procedimentos de manutenção. & eletrocutado & morte \\ \hline
noChuva & Não deve haver chuva durante o ato da manutenção & eletrocutado & morte \\ \hline
sol & O dia deve estar ensolarado & eletrocutado & morte \\ \hline
\end{tabular}
\caption{Define as condições necessárias para que a manutenção tenha possibilidade de acontecer}
\label{condition}
\end{table}


A tabela \ref{relation} apresenta três relações onde uma delas é $thereIsRelation(r_l,e_i,e_k)$ onde $r_l$ é definido pela coluna $relacionamento$, $e_i$ e $e_k$ pelas entidades envolvidas. A outra relação é dada por $hasRisk(r_k,risk_j,f_m)$ onde $risk_j$ é dado pela coluna risco e $f_m$ é dado pela coluna fatalidade. A terceira relação é dada por $hasPossibility(gr_n,p_m)$. $X$ é uma variável que pode assumir os seguintes valores $agente1, agente2, agente3,agente4, agente5, agente6$ e $agente7$. Por exemplo, a primeira linha da tabela \ref{relation} é; 


\begin{eqnarray}
	relXCapacete | X,capacete | nenhum | nenhum | false
\end{eqnarray}


Substituindo o $X$ pelos valores, é possível obter todas essas relações; 


\begin{eqnarray}
relAgente1Capacete | Agente1 ,capacete | nenhum | nenhum | false \nonumber \\
relAgente2Capacete | Agente2 ,capacete | nenhum | nenhum | false \nonumber \\ 
relAgente3Capacete | Agente3 ,capacete | nenhum | nenhum | false \nonumber \\ 
relAgente4Capacete | Agente4 ,capacete | nenhum | nenhum | false \nonumber \\
relAgente5Capacete | Agente5 ,capacete | nenhum | nenhum | false \nonumber \\
relAgente6Capacete | Agente6 ,capacete | nenhum | nenhum | false \nonumber \\
relAgente7Capacete | Agente7 ,capacete | nenhum | nenhum | false\nonumber \\
\nonumber \\
\end{eqnarray}

\begin{table}[H]
\scalefont{0.8}
\centering
\begin{tabular}{|l|l|l|l|l|}
\hline
\textbf{relacionamento} & \textbf{entidades envolvidas} & \textbf{risco} & \textbf{fatalidade} & \textbf{possibilidade} \\  \hline
relXCapacete & X,capacete & nenhum & nenhum & false \\ \hline
relXOculos & X,oculos & nenhum & nenhum & false \\ \hline
relXRoupagem & X,roupagem & nenhum & nenhum & false \\ \hline
relXLuva & X,luva & nenhum & nenhum & false \\ \hline
relXBotas & X,bota & nenhum & nenhum & false \\ \hline
relXPano & X,pano & nenhum & nenhum & false \\ \hline
relPanoGlicerina & pano,glicerina & nenhum & nenhum & false \\ \hline
relPanoCorda & pano,corda & nenhum & nenhum & false \\ \hline
relPanoBastoaUniversal & pano,bastaoUniversal & nenhum & nenhum & false \\ \hline
relPanoSoquete & pano,soquete & nenhum & nenhum & false \\ \hline
relPanoBastaoUniversal & pano,bastaoGarra & nenhum & nenhum & false \\ \hline
relXSela & X,sela & nenhum & nenhum & false \\ \hline
relXColar & X,colar & nenhum & nenhum & false \\ \hline
relXBastaoGarra & X,bastaoGarra & nenhum & nenhum & false \\ \hline
relTorreSela & torre,sela & nenhum & nenhum & false \\ \hline
relSelaColar & sela,colar & nenhum & nenhum & false \\ \hline
relColarBastaoGarra & colar,bastaoGarra & nenhum & nenhum & false \\ \hline
relBastaoGarraCondutor & bastaoGarra,condutor & eletrocutado & morte & false \\ \hline
relXBastaoUniversal & X,bastaoUniversal & nenhum & nenhum & false \\ \hline
relCordaBastaoUniversal & corda,bastaoUniversal & nenhum & nenhum & false \\ \hline
relCordaCarretilha & corda,carretilha & nenhum & nenhum & false \\ \hline
relBastaoUniversalCarretilha & bastaoUniversal,carretilha & nenhum & nenhum & false \\ \hline
relBastaoUniversalColar & bastaoUniversal,colar & nenhum & nenhum & false \\ \hline
relBastaoUniversalEstopo & bastaoUniversal,estopo & nenhum & nenhum & false \\ \hline
\end{tabular}
\caption{Define os relacionamentos necessários que a manutenção aconteça}
\label{relation2}
\end{table}


\begin{table}[H]
\scalefont{0.8}
\centering
\begin{tabular}{|l|l|l|l|l|}
\hline
\textbf{relacionamento} & \textbf{entidades envolvidas} & \textbf{risco} & \textbf{fatalidade} & \textbf{possibilidade} \\  \hline
relCordaEstropo & corda,estropo & eletrocutado & morte & false \\ \hline
relEstropoIsoladorVelho & estropo,isoladorVelho & nenhum & nenhum & false \\ \hline
relXChaveCatraca & X,chaveCatraca & nenhum & nenhum & false \\ \hline
relChaveCatracaBastaoUniversal & chaveCatraca,bastaoUniversal & nenhum & nenhum & false \\ \hline
relChaveCatracaParafuso & chaveCatraca,parafuso & eletrocutado & morte & false \\ \hline
relParafusoConector & parafuso,conector & eletrocutado & morte & false \\ \hline
relXBastaoSoquete & X,bastaoSoquete & nenhum & nenhum & false \\ \hline
relSoqueteParafuso & soquete,parafuso & eletrocutado & morte & false \\ \hline
relXCorda & X,corda & eletrocutado & morte & false \\ \hline
relXIsoladorVelho & X,isoladorVelho & nenhum & nenhum & false \\ \hline
relXIsoladorNovo & X,isoladorNovo & nenhum & nenhum & false \\ \hline
relCordaBastaoGarra & corda,bastaoGarra & nenhum & nenhum & false \\ \hline
relBastaoGarraSela & bastaoGarra, sela & nenhum & nenhum & false \\ \hline
relXCarretilha & X,carretilha & nenhum & nenhum & false \\ \hline
relBastaoUniversalCorda & bastaoUniversal,corda & nenhum & nenhum & false \\ \hline
relBastaoUniversalTorre & bastaoUniversal,torre & nenhum & nenhum & false \\ \hline
relEstropoCorda & estropo,corda & eletrocutado & morte & false \\ \hline
relEstropoIsoladorNovo & estropo,isoladorNovo & nenhum & nenhum & false \\ \hline
relBastaoUniversalSela & universal,sela & nenhum & nenhum & false \\ \hline
relBastaoGarraTorre & bastaoGarra,torre & nenhum & nenhum & false \\ \hline
relBastaoUniversalEstropo & bastaoUniversal,estropo & nenhum & nenhum & false \\ \hline
relXColar & X,colar & nenhum & nenhum & false \\ \hline
relParafusoTorre & parafuso,torre & eletrocutado & morte & false \\ \hline
relCondutivimetroCorda & condutímetro,corda & nenhum & nenhum & false \\ \hline
relCondutivimetroBastaoUniversal & condutímetro,bastaoUniversal & nenhum & nenhum & false \\ \hline
relCondutivimetroBastaoGarra & condutímetro,bastaoGarra & nenhum & nenhum & false \\ \hline
relCondutivimetroSoquete & condutímetro,soquete & nenhum & nenhum & false \\ \hline
\end{tabular}
\caption{Define os relacionamentos necessários que a manutenção aconteça}
\label{relation}
\end{table}

As tabelas \ref{relation1},\ref{relation2},\ref{relation3} e \ref{relation4} apresentam a relação $affects(r_k,r_n)$ onde $r_k$ é representado pela coluna relacionamento-errado e $r_n$ é representado pela coluna relacionamento-afetado. 
A coluna nova possibilidade de algo errado tem como por finalidade representar que a possibilidade de ocorrer algum evento ruim atrelado ao relacionamento-afetado mudou de $false$ para $true$.

\begin{table}[H]
\centering
\scalefont{0.6}
\begin{tabular}{|l|l|l|}
\hline
\textbf{relacionamento-errado} & \textbf{relacionamento-afetado} & \textbf{nova possibilidade de algo errado} \\ \hline
relXCapacete & relBastaoGarraCondutor & true \\ \hline
relXCapacete & relCordaEstropo & true \\ \hline
relXCapacete & relChaveCatracaParafuso & true \\ \hline
relXCapacete & relParafusoConector & true \\ \hline
relXCapacete & relSoqueteParafuso & true \\ \hline
relXCapacete & relXCorda & true \\ \hline
relXCapacete & relEstropoCorda & true \\ \hline
relXCapacete & relParafusoTorre & true \\ \hline
relXOculos & relBastaoGarraCondutor & true \\ \hline
relXOculos & relCordaEstropo & true \\ \hline
relXOculos & relChaveCatracaParafuso & true \\ \hline
relXOculos & relParafusoConector & true \\ \hline
relXOculos & relSoqueteParafuso & true \\ \hline
relXOculos & relXCorda & true \\ \hline
relXOculos & relEstropoCorda & true \\ \hline
relXOculos & relParafusoTorre & true \\ \hline
relXLuva & relBastaoGarraCondutor & true \\ \hline
relXLuva & relCordaEstropo & true \\ \hline
relXLuva & relChaveCatracaParafuso & true \\ \hline
relXLuva & relParafusoConector & true \\ \hline
\end{tabular}
\caption{Define o impacto que o erro em um relacionamento gera em outro relacionamento}
\label{relation1}
\end{table}

\begin{table}[H]
\centering
\scalefont{0.6}
\begin{tabular}{|l|l|l|}
\hline
\textbf{relacionamento-errado} & \textbf{relacionamento-afetado} & \textbf{nova possibilidade de algo errado} \\ \hline
relXLuva & relSoqueteParafuso & true \\ \hline
relXLuva & relXCorda & true \\ \hline
relXLuva & relEstropoCorda & true \\ \hline
relXLuva & relParafusoTorre & true \\ \hline
relXBotas & relBastaoGarraCondutor & true \\ \hline
relXBotas & relCordaEstropo & true \\ \hline
relXBotas & relChaveCatracaParafuso & true \\ \hline
relXBotas & relParafusoConector & true \\ \hline
relXBotas & relSoqueteParafuso & true \\ \hline
relXBotas & relXCorda & true \\ \hline
relXBotas & relEstropoCorda & true \\ \hline
relXBotas & relParafusoTorre & true \\ \hline
relXPano & relBastaoGarraCondutor & true \\ \hline
relXPano & relCordaEstropo & true \\ \hline
relXPano & relChaveCatracaParafuso & true \\ \hline
relXPano & relParafusoConector & true \\ \hline
relXPano & relSoqueteParafuso & true \\ \hline
relXPano & relXCorda & true \\ \hline
relXPano & relEstropoCorda & true \\ \hline
relXPano & relParafusoTorre & true \\ \hline
relPanoGlicerina & relBastaoGarraCondutor & true \\ \hline
relPanoGlicerina & relCordaEstropo & true \\ \hline
relPanoGlicerina & relChaveCatracaParafuso & true \\ \hline
relPanoGlicerina & relParafusoConector & true \\ \hline
relPanoGlicerina & relSoqueteParafuso & true \\ \hline
relPanoGlicerina & relXCorda & true \\ \hline
relPanoGlicerina & relEstropoCorda & true \\ \hline
relPanoGlicerina & relParafusoTorre & true \\ \hline
relPanoCorda & relCordaEstropo & true \\ \hline
relPanoCorda & relXCorda & true \\ \hline
relPanoCorda & relEstropoCorda & true \\ \hline
\end{tabular}
\caption{Define o impacto que o erro em um relacionamento gera em outro relacionamento}
\label{relation2}
\end{table}

\begin{table}[H]
\centering
\scalefont{0.6}
\begin{tabular}{|l|l|l|}
\hline
\textbf{relacionamento-errado} & \textbf{relacionamento-afetado} & \textbf{nova possibilidade de algo errado} \\ \hline
relPanoBastaoUniversal & relBastaoGarraCondutor & true \\ \hline
relPanoBastaoUniversal & relChaveCatracaParafuso & true \\ \hline
relPanoBastaoUniversal & relParafusoConector & true \\ \hline
relPanoBastaoUniversal & relParafusoTorre & true \\ \hline
relPanoBastaoUniversal & relBastaoGarraCondutor & true \\ \hline
relPanoSoquete & relBastaoGarraCondutor & true \\ \hline
relPanoSoquete & relCordaEstropo & true \\ \hline
relPanoSoquete & relChaveCatracaParafuso & true \\ \hline
relPanoSoquete & relParafusoConector & true \\ \hline
relPanoSoquete & relSoqueteParafuso & true \\ \hline
relPanoSoquete & relXCorda & true \\ \hline
relPanoSoquete & relEstropoCorda & true \\ \hline
relPanoSoquete & relParafusoTorre & true \\ \hline
relCondutivimetroCorda & relBastaoGarraCondutor & true \\ \hline
relCondutivimetroCorda & relCordaEstropo & true \\ \hline
relCondutivimetroCorda & relChaveCatracaParafuso & true \\ \hline
relCondutivimetroCorda & relParafusoConector & true \\ \hline
relCondutivimetroCorda & relSoqueteParafuso & true \\ \hline
relCondutivimetroCorda & relXCorda & true \\ \hline
relCondutivimetroCorda & relEstropoCorda & true \\ \hline
relCondutivimetroCorda & relParafusoTorre & true \\ \hline
\end{tabular}
\caption{Define o impacto que o erro em um relacionamento gera em outro relacionamento por mudar a possibilidade de algo errado acontecer.}
\label{relation3}
\end{table}



\begin{table}[H]
\centering
\scalefont{0.6}
\begin{tabular}{|l|l|l|}
\hline
\textbf{relacionamento-errado} & \textbf{relacionamento-afetado} & \textbf{nova possibilidade de algo errado} \\ \hline
relCondutivimetroBastaoUniversal & relBastaoGarraCondutor & true \\ \hline
relCondutivimetroBastaoUniversal & relCordaEstropo & true \\ \hline
relCondutivimetroBastaoUniversal & relChaveCatracaParafuso & true \\ \hline
relCondutivimetroBastaoUniversal & relParafusoConector & true \\ \hline
relCondutivimetroBastaoUniversal & relSoqueteParafuso & true \\ \hline
relCondutivimetroBastaoUniversal & relXCorda & true \\ \hline
relCondutivimetroBastaoUniversal & relEstropoCorda & true \\ \hline
relCondutivimetroBastaoUniversal & relParafusoTorre & true \\ \hline
relCondutivimetroBastaoGarra & relBastaoGarraCondutor & true \\ \hline
relCondutivimetroBastaoGarra & relCordaEstropo & true \\ \hline
relCondutivimetroBastaoGarra & relChaveCatracaParafuso & true \\ \hline
relCondutivimetroBastaoGarra & relParafusoConector & true \\ \hline
relCondutivimetroBastaoGarra & relSoqueteParafuso & true \\ \hline
relCondutivimetroBastaoGarra & relXCorda & true \\ \hline
relCondutivimetroBastaoGarra & relEstropoCorda & true \\ \hline
relCondutivimetroBastaoGarra & relParafusoTorre & true \\ \hline
relCondutivimetroSoquete & relBastaoGarraCondutor & true \\ \hline
relCondutivimetroSoquete & relCordaEstropo & true \\ \hline
relCondutivimetroSoquete & relChaveCatracaParafuso & true \\ \hline
relCondutivimetroSoquete & relParafusoConector & true \\ \hline
relCondutivimetroSoquete & relSoqueteParafuso & true \\ \hline
relCondutivimetroSoquete & relXCorda & true \\ \hline
relCondutivimetroSoquete & relEstropoCorda & true \\ \hline
relCondutivimetroSoquete & relParafusoTorre & true \\ \hline
\end{tabular}
\caption{Define o impacto que o erro em um relacionamento gera em outro relacionamento por mudar a possibilidade de algo errado acontecer.}
\label{relation4}
\end{table}

As tabelas \ref{deontic1}, \ref{deontic2}, \ref{deontic3} e \ref{deontic4},  apresentam a relação $hasObligation(\rho_m,g_i)$ onde $\rho_m$ é representado pela coluna papel e $g_i$ é representado pela coluna objetivo. 

\begin{table}[H]
\centering
\scalefont{0.8}
\begin{tabular}{|l|l|}
\hline
\textbf{papel} & \textbf{objetivo} \\ \hline
executor1 & g0 \\ \hline
executor2 & g0 \\ \hline
executor3 & g0 \\ \hline
executor4 & g0 \\ \hline
executor5 & g0 \\ \hline
supervisor & g0 \\ \hline
supervisor & gSupervisor \\ \hline
executor1 & g1 \\ \hline
executor2 & g1 \\ \hline
executor1 & g2 \\ \hline
executor2 & g2 \\ \hline
executor1 & g3 \\ \hline
executor2 & g2 \\ \hline
executor1 & g4 \\ \hline
executor2 & g4 \\ \hline
executor1 & g5 \\ \hline
executor2 & g5 \\ \hline
executor3 & g6 \\ \hline
executor4 & g6 \\ \hline
executor5 & g6 \\ \hline
\end{tabular}
\caption{Objetivos que devem ser atingidos pelo agente que assumir um dada função}
\label{deontic1}
\end{table}

\begin{table}[H]
\centering
\scalefont{0.8}
\begin{tabular}{|l|l|}
\hline
\textbf{papel} & \textbf{objetivo} \\ \hline
executor3 & g7 \\ \hline
executor4 & g7 \\ \hline
executor5 & g7 \\ \hline
executor3 & g8 \\ \hline
executor4 & g8 \\ \hline
executor5 & g8 \\ \hline
executor3 & g9 \\ \hline
executor4 & g9 \\ \hline
executor5 & g9 \\ \hline
executor3 & g10 \\ \hline
executor4 & g10 \\ \hline
executor5 & g10 \\ \hline
executor3 & g11 \\ \hline
executor4 & g11 \\ \hline
executor5 & g11 \\ \hline
executor1 & g12 \\ \hline
executor2 & g12 \\ \hline
executor3 & g12 \\ \hline
executor4 & g12 \\ \hline
executor1 & g13 \\ \hline
executor2 & g13 \\ \hline
executor3 & g13 \\ \hline
executor4 & g13 \\ \hline
executor1 & g14 \\ \h\usepackage{scalefnt}
line
\end{tabular}
\caption{Objetivos que devem ser atingidos pelo agente que assumir um dada função}
\label{deontic2}
\end{table}


\begin{table}[H]
\centering
\scalefont{0.8}
\begin{tabular}{|l|l|}
\hline
\textbf{role} & \textbf{g} \\ \hline
executor2 & g14 \\ \hline
executor3 & g14 \\ \hline
executor4 & g14 \\ \hline
executor2 & g15 \\ \hline
executor3 & g15 \\ \hline
executor4 & g15 \\ \hline
executor5 & g15 \\ \hline
executor2 & g16 \\ \hline
executor3 & g16 \\ \hline
executor4 & g16 \\ \hline
executor5 & g16 \\ \hline
executor1 & g17 \\ \hline
executor3 & g17 \\ \hline
executor4 & g17 \\ \hline
executor5 & g17 \\ \hline
executor1 & g18 \\ \hline
executor3 & g18 \\ \hline
executor4 & g18 \\ \hline
executor5 & g18 \\ \hline
executor1 & g19 \\ \hline
executor3 & g19 \\ \hline
executor4 & g19 \\ \hline
executor5 & g19 \\ \hline
executor1 & g20 \\ \hline
executor3 & g20 \\ \hline
executor4 & g20 \\ \hline
executor5 & g20 \\ \hline
executor1 & g21 \\ \hline
executor3 & g21 \\ \hline
\end{tabular}
\caption{Objetivos que devem ser atingidos pelo agente que assumir um dada função}
\label{deontic3}
\end{table}

\begin{table}[H]
\centering
\scalefont{0.8}
\begin{tabular}{|l|l|}
\hline
\textbf{role} & \textbf{g} \\ \hline
executor4 & g21 \\ \hline
executor5 & g21 \\ \hline
executor1 & g22 \\ \hline
executor2 & g22 \\ \hline
executor3 & g22 \\ \hline
executor5 & g22 \\ \hline
executor1 & g23 \\ \hline
executor2 & g23 \\ \hline
executor3 & g23 \\ \hline
executor5 & g23 \\ \hline
executor1 & g24 \\ \hline
executor2 & g24 \\ \hline
executor3 & g24 \\ \hline
executor5 & g24 \\ \hline
executor1 & g25 \\ \hline
executor2 & g25 \\ \hline
executor3 & g25 \\ \hline
executor4 & g25 \\ \hline
executor1 & g26 \\ \hline
executor2 & g26 \\ \hline
executor3 & g26 \\ \hline
executor4 & g26 \\ \hline
executor1 & g27 \\ \hline
executor2 & g27 \\ \hline
executor3 & g27 \\ \hline
executor4 & g27 \\ \hline
executor5 & g27 \\ \hline
\end{tabular}
\caption{Objetivos que devem ser atingidos pelo agente que assumir um dada função}
\label{deontic4}
\end{table}

A tabela \ref{entities} apresenta as entidades que constituem os conjuntos \textbf{eg}.

\begin{table}[H]
\centering
\scalefont{0.8}
\begin{tabular}{|l|l|}
\hline
\textbf{entidades}                                                                                                    & \textbf{eg} \\ \hline
capacete,óculos,roupagem,luvas,botas X = \{agente que tenta alcançar o objetivo\}                                        & eg0         \\ \hline
pano,glicerina,carretilha,bastaoUniversal,corda,bastaoGarra,X = \{agente que tenta alcançar o objetivo\}               & eg1         \\ \hline
pano,glicerina,carretilha,bastaoUniversal,corda,bastaoGarra,condutímetro,X = \{agente que tenta alcançar o objetivo\} & eg2         \\ \hline
sela,colarX = \{agente que tenta alcançar o objetivo\}                                                                & eg3         \\ \hline
colar,bastaoGarraX = \{agente que tenta alcançar o objetivo\}                                                          & eg4         \\ \hline
corda,bastaoGarra,bastaoGarraTorre,condutorX = \{agente que tenta alcançar o objetivo\}                               & eg5         \\ \hline
bastaoGarra,selaX = \{agente que tenta alcançar o objetivo\}                                                           & eg6         \\ \hline
sela,colarX = \{agente que tenta alcançar o objetivo\}                                                                & eg7         \\ \hline
sela,bastaoUniversal,Colar,X = \{agente que tenta alcançar o objetivo\}                                               & eg8         \\ \hline
bastaoUniversal,carretilha,X = \{agente que tenta alcançar o objetivo\}                                               & eg9         \\ \hline
corda,bastaoUniversal,corda,torre,X = \{agente que tenta alcançar o objetivo\}                                        & eg10        \\ \hline
bastaoUniversal,corda,colar,selaX = \{agente que tenta alcançar o objetivo\}                                          & eg11        \\ \hline
colar,X = \{agente que tenta alcançar o objetivo\}                                                                    & eg12        \\ \hline
bastaoUniversal,estropo,isoladorVelhoX = \{agente que tenta alcançar o objetivo\}                                     & eg13        \\ \hline
bastaoUniversal,corda,estropoX = \{agente que tenta alcançar o objetivo\}                                             & eg14        \\ \hline
chaveCatraca,bastaoUniversal,prafusoX = \{agente que tenta alcançar o objetivo\}                                      & eg15        \\ \hline
bastaoSoquete,parafuso,X = \{agente que tenta alcançar o objetivo\}                                                   & eg16        \\ \hline
bastaoGarra,condutorcordaX = \{agente que tenta alcançar o objetivo\},                                                & eg17        \\ \hline
colar,X = \{agente que tenta alcançar o objetivo\},                                                                   & eg18        \\ \hline
chaveCatraca,bastaoUniversal,prafusobastaoSoquete,parafuso,torreX = \{agente que tenta alcançar o objetivo\}          & eg19        \\ \hline
cordaX = \{agente que tenta alcançar o objetivo\}                                                                     & eg20        \\ \hline
estropo, isoladorNovo,X = \{agente que tenta alcançar o objetivo\}                                                    & eg21        \\ \hline
bastaoUniversal,corda,estropoX = \{agente que tenta alcançar o objetivo\}                                             & eg22        \\ \hline
cordaX = \{agente que tenta alcançar o objetivo\}                                                                     & eg23        \\ \hline
chaveCatraca,bastaoUniversal,prafusobastaoSoquete,parafuso,torreX = \{agente que tenta alcançar o objetivo\}          & eg24        \\ \hline
bastaoGarra,condutorcordaX = \{agente que tenta alcançar o objetivo\},                                                & eg25        \\ \hline
chaveCatraca,bastaoUniversal,prafusoX = \{agente que tenta alcançar o objetivo\}                                      & eg26        \\ \hline
sela,colar,bastaoGarra,bastaoUniversal,bastaoSoquete,corda,carretilha,chaveCatraca,torre,condutor                     & eg27        \\ \hline
\end{tabular}
\caption{Entidades que formam os conjuntos $eg_n$. Cada conjunto destes estão relacionados com um objetivo e determinam as entidades necessárias para que o mesmo tenha codição de ser alcançado.}
\label{entities}
\end{table}


As tabelas \ref{relationsgroup1},\ref{relationsgroup2} apresentam as relações que constituem os conjuntos \textbf{rg}.
\begin{table}[H]
\centering
\scalefont{0.7}
\begin{tabular}{|p{0.8\linewidth}|l|}
\hline
\textbf{relacionamentos}                                                                                                                                                                                                                                                                                                                  & \textbf{rg} \\ \hline
relXcapacete relXoculos relXroupagem relXluva relXbotas                                                                                                                                                                                                                                                                                   & rg0         \\ \hline
relXPano relPanoGlicerina relPanoCorda relPanoBastaoUniversal relPanoBastaoGarra relPanoSoquete                                                                                                                                                                                                                                           & rg1         \\ \hline
relCondutivimetroCorda relCondutivimetroBastaoUniversal relCondutivimetroBastaoGarra relCondutivimetroSoquete                                                                                                                                                                                                                                & rg2         \\ \hline
relCondutivimetro                                                                                                                                                                                                                                                                                                                         & rg3         \\ \hline
relXBastaoGarra relColarBastaoGarra                                                                                                                                                                                                                                                                                                        & rg4         \\ \hline
relXBastaoGarra relXCordarelCordaBastaoGarra relBastaoGarraTorre relBastaoGarraCondutor                                                                                                                                                                                                                                                      & rg5         \\ \hline
relBastaoGarraSela relXBastaoGarra relXSela                                                                                                                                                                                                                                                                                                 & rg6         \\ \hline
relXSela relXColar relTorreSela                                                                                                                                                                                                                                                                                                             & rg7         \\ \hline
relBastaoUniversalColar relXBastaoUniversal                                                                                                                                                                                                                                                                                                & rg8         \\ \hline
relXBastaoUniversal relXCarretilha relBastaoUniversalCarretilha                                                                                                                                                                                                                                                                             & rg9         \\ \hline
relXCorda relXBastaoUniversal relBastaoUniversalCorda relBastaoUniversalTorre                                                                                                                                                                                                                                                             & rg10        \\ \hline
relXCorda relXBastaoUniversal relXColar relBastaoUniversalColar relBastaoUniversalSela                                                                                                                                                                                                                                                    & rg11        \\ \hline
\end{tabular}
\caption{Relacionamentos que formam os conjuntos $rg_n$. Cada conjunto $rg_n$ está relacionado com um objetivo A relação entre $rg_n$ e $goal_m$ determina os relacionaentos necessários para que um dado objetivo tenha condição de ser atingido.}
\label{relationsgroup1}
\end{table}


\begin{table}[H]
\centering
\scalefont{0.7}
\begin{tabular}{|p{0.8\linewidth}|l|}
\hline
\textbf{relacionamentos}                                                                                                                                                                                                                                                                                                                  & \textbf{rg} \\ \hline
relXColar                                                                                                                                                                                                                                                                                                                                 & rg12        \\ \hline
relXBastaoUniversal relBastaoUniversalEstropo relEstropoIsoladorVelho                                                                                                                                                                                                                                                                     & rg13        \\ \hline
relXBastaoUniversal relBastaoUniversalCordarelCordaEstropo relEstropoCorda                                                                                                                                                                                                                                                                & rg14        \\ \hline
relChaveCatracaBastaoUniversal relXChaveCatraca relXBastaoUniversal relChaveCatracaParafuso                                                                                                                                                                                                                                               & rg15        \\ \hline
relXBastaoSoquete relSoqueteParafuso                                                                                                                                                                                                                                                                                                      & rg16        \\ \hline
relXCorda relCordaBastaoGarra relBastaoGarraCondutor                                                                                                                                                                                                                                                                                      & rg17        \\ \hline
relXColar                                                                                                                                                                                                                                                                                                                                 & rg18        \\ \hline
relChaveCatracaBastaoUniversal relXChaveCatraca relXBastaoUniversal relChaveCatracaParafuso relParafusoTorre relXBastaoSoquete relSoqueteParafuso                                                                                                                                                                                           & rg19        \\ \hline
relXCorda                                                                                                                                                                                                                                                                                                                                 & rg20        \\ \hline
relXEstropo relEstropoIsoladorNovo                                                                                                                                                                                                                                                                                                         & rg21        \\ \hline
relXBastaoUniversal relBastaoUniversalCorda relCordaEstropo relEstropoCorda                                                                                                                                                                                                                                                                & rg22        \\ \hline
relXCorda                                                                                                                                                                                                                                                                                                                                 & rg23        \\ \hline
relChaveCatracaBastaoUniversal relXChaveCatraca relXBastaoUniversal relChaveCatracaParafuso relParafusoTorrerelXBastaoSoquete relSoqueteParafuso                                                                                                                                                                                           & rg24        \\ \hline
relXCorda relCordaBastaoGarra relBastaoGarraCondutor                                                                                                                                                                                                                                                                                      & rg25        \\ \hline
relChaveCatracaBastaoUniversal relXChaveCatraca relXBastaoUniversal relChaveCatracaParafuso                                                                                                                                                                                                                                               & rg26        \\ \hline
relXSela relXColarrelXBastaoGarrarelXBastaoUniversal relXBastaoSoquete relXCorda relXCarretilha relXChaveCatraca relColarBastaoGarra relCordaBastaoGarra relBastaoGarraTorre relBastaoGarraCondutor relBastaoUniversalCarretilha relBastaoGarraSela relBastaoUniversalSela relSelaColar relTorreSela relBastaoUniversalCorda relBastaoGarraCorda & rg27        \\ \hline
\end{tabular}
\caption{Relacionamentos que formam os conjuntos $rg_n$. Cada conjunto $rg_n$ está relacionado com um objetivo A relação entre $rg_n$ e $goal_m$ determina os relacionaentos necessários para que um dado objetivo tenha condição de ser atingido.}
\label{relationsgroup2}
\end{table}




A tabela \ref{conditions} apresenta as condições que constituem os conjuntos \textbf{cg}.

\begin{table}[H]
\centering
\scalefont{0.8}
\begin{tabular}{|l|l|}
\hline
\textbf{condições}         & \textbf{cg} \\ \hline
umidade70,noVento,noChuva,sol & cg1         \\ \hline
\end{tabular}
\caption{Todas as condições que constituem o conjunto $cg_n$. Este conjunto está relacionando com um ou mais objetivos e determina quais são as condições que devem ser mantidas para que o agente tenha uma situação razoável para tentar alcançar um certo objetivo}
\label{conditions}
\end{table}


A tabela \ref{goalsrelationsentity} define as relações $hasRelation(g_i,rg_n)$ onde a coluna objetivo é representada por $g_i$, $hasEntity(g_i,eg_m)$, $hasCondition(g_i,cg_n)$.  

		
\begin{table}[H]
\centering
\scalefont{0.6}
\begin{tabular}{|l|l|l|l|}
\hline
\textbf{objetivo}  & \textbf{rg} & \textbf{eg} & \textbf{cg} \\ \hline
goal0 & rg0 & eg0 & cg1 \\ \hline
goal0 & rg0 & eg0 & cg1 \\ \hline
goal1 & rg1 & eg1 & cg1 \\ \hline
goal2 & rg2 & eg2 & cg1 \\ \hline
goal3 & rg3 & eg3 & cg1 \\ \hline
goal4 & rg4 & eg4 & cg1 \\ \hline
goal5 & rg5 & eg5 & cg1 \\ \hline
goal6 & rg6 & eg6 & cg1 \\ \hline
goal7 & rg7 & eg7 & cg1 \\ \hline
goal8 & rg8 & eg8 & cg1 \\ \hline
goal9 & rg9 & eg9 & cg1 \\ \hline
goal10 & rg10 & eg10 & cg1 \\ \hline
goal11 & rg11 & eg11 & cg1 \\ \hline
goal12 & rg12 & eg12 & cg1 \\ \hline
goal13 & rg13 & eg13 & cg1 \\ \hline
goal14 & rg14 & eg14 & cg1 \\ \hline
goal15 & rg15 & eg15 & cg1 \\ \hline
goal16 & rg16 & eg16 & cg1 \\ \hline
goal17 & rg17 & eg17 & cg1 \\ \hline
goal18 & rg18 & eg18 & cg1 \\ \hline
goal19 & rg19 & eg19 & cg1 \\ \hline
goal20 & rg20 & eg20 & cg1 \\ \hline
goal21 & rg21 & eg21 & cg1 \\ \hline
goal22 & rg22 & eg22 & cg1 \\ \hline
goal23 & rg23 & eg23 & cg1 \\ \hline
goal24 & rg24 & eg24 & cg1 \\ \hline
goal25 & rg25 & eg25 & cg1 \\ \hline
goal26 & rg26 & eg26 & cg1 \\ \hline
goal27 & rg27 & eg27 & cg1 \\ \hline
\end{tabular}
\caption{Define a relação entre os objetivos, conjuntos $rg_n$, $eg_n$ e $cg_n$ }
\label{goalsrelationsentity}
\end{table}

	\section{Raciocínio} \label{rac}
			\label{racs}

Uma vez que o modelo foi definido e que foi implementado em um estudo de caso, é possível avaliar as conclusões possíveis em certa condição de mundo. Essa seção demonstra como esse modelo cumpre o proposto por demonstrar certos raciocínios tendo em vista o estudo de caso em análise. 

\subsection{Raciocínio - 1} 
\label{raciocinio1}

O raciocínio a seguir mostra o que acontece se o $agente4$ esquecer de passar glicerina no pano, especificado pela relação $relPanoGlicerina$, designados a ele no objetivo $g1$. 
Todos os predicados vinculados a essa situação são;

\begin{enumerate}
	\item $adoptsRole(agente4,executor2)$ 
	\item $hasObligation(executor2,g1)$
	\item $requiresCirc(g1,relPanoGlicerina)$
	\item $instanceOfRel(relPanoGlicerina)$ 
	\item $relPanoCorda \in rg1$
	\item $ x = agente4 $
	\item $starts(agente4,g1)$
	\item $affectsRels(relPanoGlicerina,relBastaoGarraCondutor)$
	\item $affectsRels(relPanoGlicerina,relCordaEstropo)$  
	\item $affectsRels(relPanoGlicerina,relChaveCatracaParafuso)$
	\item $affectsRels(relPanoGlicerina,relParafusoConector)$ 
	\item $affectsRels(relPanoGlicerina,relSoqueteParafuso)$ 
	\item $affectsRels(relPanoGlicerina,relAgente4Corda)$ 
	\item $affectsRels(relPanoGlicerina,relEstropoCorda)$	
\end{enumerate}

Com base nisso, as relações de implicabilidade resultantes são;

\begin{eqnarray}\nonumber
	requiresCirc(g1,relPanoGlicerina) \wedge \nonumber \\  
	\neg isPresent(relPanoGlicerina) \wedge \nonumber \\   
	instanceOfRel(relPanoGlicerina) \wedge \nonumber \\   
	starts(agente4,g1) \to \nonumber \\
	relationViol(agente4,g1,relPanoGlicerina) \nonumber \\  
\end{eqnarray}

\begin{eqnarray}\nonumber
	relationViol(agente4,g1,relPanoGlicerina) \nonumber \\
	\wedge affectsRels(relPanoGlicerina,relBastaoGarraCondutor) \nonumber \\
    \to possOfNegConseqFor(relBastaoGarraCondutor)  	
\end{eqnarray}

\begin{eqnarray}\nonumber
	relationViol(agente4,g1,relPanoGlicerina) \nonumber \\
	\wedge affectsRels(relPanoGlicerina,relCordaEstropo) \nonumber \\
    \to possOfNegConseqFor(relCordaEstropo)
\end{eqnarray}

\begin{eqnarray}\nonumber
	relationViol(agente4,g1,relPanoGlicerina) \nonumber \\
	\wedge affectsRels(relPanoGlicerina,relParafusoConector) \nonumber \\
    \to possOfNegConseqFor(relParafusoConector)	 
\end{eqnarray}

\begin{eqnarray}\nonumber
	relationViol(agente4,g1,relPanoGlicerina) \nonumber \\
	\wedge affectsRels(relPanoGlicerina,relSoqueteParafuso) \nonumber \\
    \to possOfNegConseqFor(relSoqueteParafuso)	 	
\end{eqnarray}

\begin{eqnarray}\nonumber
	relationViol(agente4,g1,relPanoGlicerina) \nonumber \\
	\wedge affectsRels(relPanoGlicerina,relAgente4Corda) \nonumber \\
    \to possOfNegConseqFor(relAgente4Corda)	 		
\end{eqnarray}

\begin{eqnarray}\nonumber
	relationViol(agente4,g1,relPanoGlicerina) \nonumber \\
	\wedge affectsRels(relPanoGlicerina,relEstropoCorda) \nonumber \\
    \to possOfNegConseqFor(relEstropoCorda)	 			 
\end{eqnarray}



\subsection{Raciocínio - 2} 
\label{raciocinio2}
O raciocínio a seguir mostra o que acontece se o pano não estiver presente no local da manutenção quando os eletricistas alcançarem o $g1$. A lista a seguir exibe todos os predicados necessários para averiguar essa condição de mundo. 


\begin{enumerate}
	\item $adoptsRole(agente2,executor1)$ 
	\item $adoptsRole(agente3,executor1)$	 	
	\item $adoptsRole(agente4,executor2)$	 
	\item $hasObligation(executor1,g1)$
	\item $hasObligation(executor2,g1)$
	\item $starts(agente2,g1)$ 
	\item $starts(agente3,g1)$	 	
	\item $starts(agente4,g1)$	
	\item $requiresEntity(g1,pano)$		
	\item $\neg isPresent(pano)$
\end{enumerate}

\begin{eqnarray}\nonumber
	requiresEntity(g1,pano) \wedge \nonumber \\   
	\neg isPresent(pano) \wedge \nonumber \\ 
	starts(agente2,g1) \to \nonumber \\ 
    entityViol(agente2,g1,pano)  \nonumber \\
\end{eqnarray}

\begin{eqnarray}\nonumber
	requiresEntity(g1,pano) \wedge \nonumber \\   
	\neg isPresent(pano) \wedge \nonumber \\ 
	starts(agente3,g1) \to \nonumber \\ 
    entityViol(agente3,g1,pano)  \nonumber \\
\end{eqnarray}

\begin{eqnarray}\nonumber
	requiresEntity(g1,pano) \wedge \nonumber \\   
	\neg isPresent(pano) \wedge \nonumber \\ 
	starts(agente4,g1) \to \nonumber \\ 
    entityViol(agente4,g1,pano)  \nonumber \\	
\end{eqnarray}

\begin{eqnarray}
	entityViol(agente4,g1,pano) \to stopped(g1)
\end{eqnarray}



\subsection{Raciocínio - 3} 
\label{raciocinio3}

O raciocínio a seguir mostra o que acontece se o $agente5$ tentar alcançar o objetivo $g11$ com a umidade relativa do ar superior a setenta porcento. A lista a seguir exibe todos os predicados necessários para averiguar essa condição de mundo. 

\begin{enumerate}
	\item $adoptsRole(agente5,executor3)$
	\item $hasObligation(executor3,g11)$	
	\item $starts(agente5,g11)$ 
	\item $requiresCirc(g11,umidade70)$
	\item $isIstanceOfCond(umidade70)$
	\item $\neg isPresent(umidade70)$
	\item $hasRisk(umidade70,eletrocutado,morte)$
\end{enumerate}

\begin{eqnarray}
	requiresCirc(g11,umidade70) \\ \nonumber
	\wedge \neg isPresent(umidade70) \\ \nonumber
	\wedge instanceOfCond(umidade70) \\ \nonumber
	\wedge starts(agente5,g11)  \to \\ \nonumber   
	conditionViol(agente5,g11,umidade70) \nonumber \\	
\end{eqnarray}

\begin{eqnarray} \nonumber
	conditionViol(agente5,g11,umidade70) \nonumber \\
	 \wedge hasRisk(umidade70,eletrocutado,morte) \to \nonumber \\ 
	negConseqFor(g11,agente5,eletrocutado,morte) \nonumber \\ 	
\end{eqnarray}

\begin{eqnarray}
	negConseqFor(g11,agente5,eletrocutado,morte) \to stopped(g11)
\end{eqnarray}

\subsection{Raciocínio - 4} 
\label{raciocinio4}
O raciocínio a seguir mostra o que acontece se o $agente3$ errar a forma adequada de realizar o relacionamento $relChaveCatracaParafuso$ no objetivo $g15$. Os predicados envolvidos são;

\begin{enumerate}
	\item $adoptsRole(agente4,executor2)$
	\item $hasObligation(executor4,g15)$	
	\item $starts(agente4,g15)$ 
	\item $requiresCirc(g15,relChaveCatracaParafuso)$
	\item $isInstanceOfRel(relChaveCatracaParafuso)$	
	\item $\neg isPresent(relChaveCatracaParafuso)$
	\item $hasRisk(relChaveCatracaParafuso,eletrocutado,morte)$
\end{enumerate}

\begin{eqnarray}
	requiresCirc(g15,relChaveCatracaParafuso)\wedge \nonumber \\
	\neg isPresent(relChaveCatracaParafuso) \wedge \nonumber \\
	instanceOfRel(relChaveCatracaParafuso) \wedge \nonumber \\
	starts(agente4,g15) \to \nonumber \\
	relationViol(agente4,g15,relChaveCatracaParafuso) \nonumber
\end{eqnarray}

\begin{eqnarray}\nonumber
	relationViol(agente4,g15,relChaveCatracaParafuso) \nonumber \\ 
	 \wedge hasRisk(relChaveCatracaParafuso,eletrocutado,morte) \nonumber \\ 
	\to \nonumber \\ 
	negConseqFor(g15,agente4,eletrocutado,morte)
\end{eqnarray}

\begin{eqnarray}
		negConseqFor(g15,agente4,eletrocutado,morte) \to stopped(g15)
\end{eqnarray}

\subsection{Raciocínio - 5} 
\label{raciocinio5}
A finalidade dessa demonstração consiste em mostrar como um agente pode ser submetido a consequências ruins tendo em vista os erros cometidos por outros profissionais. O raciocínio 1 mostra que o fato do $agente4$ não conseguir realizar o relacionamento $relPanoGlicerina$ resulta na violação $relationViol(agente4,g1,relPanoGlicerina)$. Essa violação, por sua vez, impacta diversas outras relações, em que $possOfNegConseqFor(relParafusoConector)$ é uma delas. Assim sendo, antes do $agente4$ cometer o erro, a possibilidade da ocorrência de um evento ruim acontecer era 0, se o agente realizar a relação $relParafusoConector$ sem cometer violação alguma. Contudo, após a ocorrência do erro cometido pelo $agente4$, existe uma possibilidade de um evento ruim acontecer na relação $relParafusoConector$ mesmo que tudo seja feito de acordo com os conformes. Assim sendo, a lista de predicados e o raciocínio mostra o que acontece dado a seguinte situação; o possível evento ruim presente em $relParafusoConector$ se torna uma realidade;  

\begin{enumerate}	
	\item $requiresCirc(g19,relParafusoConector)$		
	\item $hasObligation(executor3,g19)$
	\item $hasObligation(executor4,g19)$
	\item $hasObligation(executor5,g19)$		
	\item $starts(agente5,g19)$
	\item $starts(agente6,g19)$
	\item $starts(agente7,g19)$									
	\item $adoptsRole(agente5,executor3)$
	\item $adoptsRole(agente6,executor4)$
	\item $adoptsRole(agente7,executor5)$
	\item $hasRisk(relParafusoConector,eletrocutado,morte)$
	\item $possOfNegConseqFor(relParafusoConector)$
	\item $happensNegConseqFor(g19,relParafusoConector)$	
\end{enumerate}


\begin{eqnarray}\nonumber
   possOfNegConseqFor(relParafusoConector) \nonumber \\
    \wedge happensNegConseqFor(relParafusoConector) \nonumber \\ 
    \wedge requiresCirc(g19,relParafusoConector) \nonumber \\  
    \wedge instanceOfRel(relParafusoConector) \nonumber \\ 
    \wedge hasRisk(relParafusoConector,eletrocutado,morte) \nonumber \\  
    \wedge starts(agente5,g19) \nonumber \\ 
    \to negConseqFor(g19,agente5,eletrocutado,morte) \\ \nonumber    
\end{eqnarray}

\begin{eqnarray}
	negConseqFor(g19,agente5,eletrocutado,morte) \to stopped(g19)
\end{eqnarray}

\begin{eqnarray}\nonumber
   possOfNegConseqFor(relParafusoConector) \nonumber \\
    \wedge happensNegConseqFor(relParafusoConector) \nonumber \\ 
    \wedge requiresCirc(g19,relParafusoConector) \nonumber \\  
    \wedge instanceOfRel(relParafusoConector) \nonumber \\ 
    \wedge hasRisk(relParafusoConector,eletrocutado,morte) \nonumber \\  
    \wedge starts(agente6,g19) \nonumber \\ 
    \to negConseqFor(g19,agente6,eletrocutado,morte) \\ \nonumber    
\end{eqnarray}

\begin{eqnarray}
	negConseqFor(g19,agente6,eletrocutado,morte) \to stopped(g19)
\end{eqnarray}

\begin{eqnarray}\nonumber
   possOfNegConseqFor(relParafusoConector) \nonumber \\
    \wedge happensNegConseqFor(relParafusoConector) \nonumber \\ 
    \wedge requiresCirc(g19,relParafusoConector) \nonumber \\  
    \wedge instanceOfRel(relParafusoConector) \nonumber \\ 
    \wedge hasRisk(relParafusoConector,eletrocutado,morte) \nonumber \\  
    \wedge starts(agente7,g19) \nonumber \\ 
    \to negConseqFor(g19,agente7,eletrocutado,morte) \\ \nonumber
\end{eqnarray}

\begin{eqnarray}
	negConseqFor(g19,agente7,eletrocutado,morte) \to stopped(g19)
\end{eqnarray}


\subsection{Raciocínio - 6}
\label{raciocinio6}
Esse raciocínio tem a finalidade de mostrar como se dá os raciocínios para quando se faz necessário verificar se um objetivo foi atingido. O objetivo $g23$ deve ser atingido pelos agentes com as funções de $executor1$,$executor2$,$executor3$ e $executor5$. 

\begin{enumerate}
	\item $stopped(agente2) \to F$	
	\item $stopped(agente3) \to F$
	\item $stopped(agente4) \to F$
	\item $stopped(agente5) \to F$
	\item $stopped(agente7) \to F$
	\item $hasObligation(executor1,g23)$	
	\item $hasObligation(executor2,g23)$	
	\item $hasObligation(executor3,g23)$		
	\item $adoptsRole(agente2,executor1)$
	\item $adoptsRole(agente3,executor1)$
	\item $adoptsRole(agente4,executor2)$
	\item $adoptsRole(agente5,executor3)$
	\item $adoptsRole(agente7,executor5)$		
\end{enumerate}

Essa situação é resolvida pelo algoritmo presente em \ref{wenStop}. A primeira etapa do algoritmo se dá por executar a função $ifNotStopped(agentArray,goal)$. Um dos argumentos, portanto, é um vetor de $Agents$. Para essa situação em específico, esse vetor é descrito da seguinte forma $ag_{array} = \{ agente2,agente3,agente4,agente5,agente7 \}$. O argumento goal é carregado com $g23$. A primera etapa da desta função reside avaliar todos os agentes (um por um) em um \textit(forEach). Nessa avaliação é feito um teste sobre o predicado $stopped(ag)$ que se retornar verdade faz com que $ifNotStopped(agentArray,goal)$ retorne falso. Contudo, para o caso em analise, verificamos que o predicado $stopped(ag_n)$ retorna falso para todos os agentes. O segundo passo consiste na avaliação da função $allAgentObligate(agentArray,goal)$ cujo propósito consite verificar se todos os agentes que são obrigados a alcançar o objetivo em análise estão presentes. Nesse estudo de caso é possível verificar que os agentes 2,3,4,5 e 7 adotaram as funções de executor 1,2,3 e estas são obrigadas a executar o objetivo $g23$. Logo, para esse problema a função $allAgentObligate(agentArray,goal)$ retorna verdade e, por consequência, a função $ifNotStopped(agentArray,goal)$ retorna que $reached(g23)$ é verdade.  


\subsection{Raciocínio - 7}
\label{raciocinio7}
O raciocínio para o caso onde $agente1$ tente alcançar o objetivo $g23$.  

\begin{enumerate}
	\item $adoptsRole(agente1,supervisor)$
	\item $hasObligation(agente1,g23) \to F$										
\end{enumerate}

Isso implica uma afirmação falsa, então esse mundo não é possível segundo o modelo implementado para este estudo de caso.
	\section{Validação} \label{validation}
		O estudo de caso especificado pelo modelo conceitual proposto nesse estudo foi implementado em Prolog. A seguir segue um exemplo de como a regra \ref{reldeonticrole} é escrita em Prolog.

hasPermission(RHO,GOAL) :- hasObligation(RHO,GOAL).

A seguir há um exemplo da especificação $adoptsRole(agente1,supervisor)$ implementada em Prolog.

adoptsRole(agente1,supervisor).

As consultas, em \textit{Prolog}, são feitas por escrever a especificação do implicador ($implicador \to implicado$). Usando o algoritmo \textit{Backtracking}, é possível encontrar todos os predicados que são verdade. Por exemplo, para avaliar o Raciocínio 3 se fazem necessárias fazer as seguintes consultas: $? - stopped(g1)$, $? - entityViol(agente4,g1,pano)$, $? - entityViol(agente3,g1,pano)$, $? - entityViol(agent2,g1,pano)$. 

O apêndice \ref{program} apresenta a implementação das regras e dos raciocínios em Prolog e, no caso do algoritmo, em JavaScript.

\chapter{Discussão}
	O objetivo deste capitulo consiste em discutir os resultados desta pesquisa com base nos seguintes aspectos; comparação com outros modelos \ref{analisecomparativa} e conclusões sobre a inovação deste resultado (na seção \ref{constresult}).
	\section{Análise Comparativa}\label{analisecomparativa}
		A análise comparativa é de suma importância para um estudo como este pois é por intermédio dela que se verifica a inovação do modelo. Para cumprir com esse
propósito foram escolhidos as representações mais relevantes na área de sistemas multiagentes que são: \textit{MOISE+} \cite{moiseframework}, 
\textit{Modelo presente no texto do Dastani} \cite{dastaniframework}, o \textit{V3S} \cite{v3sframework} e o modelo \textit{NormMAS Framework} \cite{normas}. 
Portanto, as próximas seções apresentam os seguintes aspectos: análise da estrutura do modelo para aqueles onde isso não foi feito, 
rastrear os conceitos mostrando os fundamentos dessa análise e comparar com a estrutura do modelo em análise neste estudo. 
		\subsection{Modelo MOISE+   }

A estrutura deste modelo está presente na fundamentação teórica na seção \ref{sma} e é com base nisso que se pode afirmar a identificação dos seguintes 
conceitos: papel, grupos, objetivos, missões. 

Tanto o \textit{MOISE+} como a proposta deste estudo apresentam o mesmo conceito de papel que consiste na função do agente dentro de uma \textit{SMA}. 
Contudo, a concepção por trás das relações com os demais conceitos é feita de forma diferente em cada modelo. No \textit{MOISE+} um papel se relaciona
com outro papel por meio do predicado \textit{link} identificado níveis de autoridade. Isso não foi abordado neste modelo porque não era do interesse 
do domínio de estudo verificar como se dá as relações de autoridade entre os agentes uma vez que a natureza das violações e sanções em interesse podem 
ser escritas sem esse termo. O mesmo se dá com o operador que corresponde a compatibilidade. Existe a possibilidade de descutir um caso onde a ocorrência 
de uma violação da natureza delimitada neste estudo se dá por conta de problemas no que diz respeito as concepções de autoridade (subordinado é orientado 
a fazer algo de forma inapropriada). Contudo, os pesquisadores deste estudo entendem que mesmo esses casos podem ser descritos pelo vocabulário em proposta,
talvez não com a mesma expressividade. Isso se dá pelo fato de que a ocorrência de uma violação está associada ao objetivo em sí e não a função do 
profissional. Por exemplo, considere a seguinte situação; Um agente A é supervisor de um agente B e aquele orienta este a fazer algo inapropriado sendo que 
este venha a sofrer consequências ruins por conta disso. Esse problema pode ser modelado da seguinte maneira;

\begin{enumerate}
	\item $hasRole(agenteA,supervisor)$ 
	\item $hasRole(agenteB,subordinado)$
    \item $tryReach(agenteA,g_{instrucao})$
    \item $tryReach(agenteB,g_{instrucao})$
    \item $tryReach(agenteB,g_{execucao})$
	\item $hasObligation(supervisor,g_{instrucao})$
	\item $hasObligation(subordinado,g_{instrucao})$
	\item $hasObligation(subordinado,g_{execucao})$
	\item $nextGoal(g_{instrucao},g_{execucao})$	
    \item $thereIsRelation(r_{passa-informacao},agenteA,agenteB)$
    \item $r_{passa-informacao} \in rg_{instrucao}$
    \item $r_{executar} \in rg_{execucao}$    
    \item $hasRelation(g_{instrucao},rg_{instrucao})$
    \item $hasRelation(g_{execuao},rg_{execuao})$
    \item $happensBadEvent(r_{executar})$
    \item $hasRisk(r_{executar}, cair, cs_{pernaQuebrada})$
    \item $affectsOtherRelations(r_{passa-informacao},r_{executar})$
\end{enumerate}

Os raciocínios, portanto, são: 


\begin{eqnarray}\label{refutmoisea}\nonumber
	hasRelation(g_{instrucao},rg_{instrucao}) \wedge \neg isPresent(r_{passa-informacao}) \\ \nonumber 
    \wedge (r_{passa-informacao} \in rg_{instrucao}) \\ \nonumber
    \wedge tryReach(agenteA,g_{instrucao}) \to \nonumber \\
	violationRelation(agenteA,g_{instrucao},rg_{instrucao})
\end{eqnarray}


\begin{eqnarray}\label{refutmoiseb}
	violationRelation(agenteA,g_{instrucao},rg_{instrucao}) \wedge  \nonumber \\
    affectsOtherRelations(r_{passa-informacao},r_{executar}) \nonumber \\
    \to possibilityHappensBadEvent(r_{executar}) 
\end{eqnarray}

\begin{eqnarray}\label{refutmoisec}
	possibilityHappensBadEvent(r_{executar}) \wedge  \nonumber \\ 
    happensBadEvent(r_{executar}) \wedge \nonumber \\ 
    hasRelation(g_{execuao},rg_{execuao}) \wedge \nonumber \\
    (r_{executar} \in rg_{execucao}) \nonumber \\ 
	\wedge hasRisk(r_{executar}, cair, cs_{pernaQuebrada}) \wedge \nonumber \\ 
    tryReach(agenteB,g_{execucao}) \nonumber \\ 
	\to consequenceOfBadEvent(g_{execucao},agenteB,cair,cs_{pernaQuebrada}) \nonumber \\ 
\end{eqnarray}

Portanto, as relações \ref{refutmoisea}, \ref{refutmoiseb} e \ref{refutmoisec} (o raciocínio que mostra $g_{instrucao}$ ser concluído não foi mostrado) 
demonstram que é possível verificar uma caso de violação por autoridade sem considerar as relações entre os papeis. Uma possível contestação a esse 
raciocínio reside no predicado $isPresent$ pois neste cenário $r_{passa-informacao}$ esteve presente na execução de $g_{instrucao}$. Contudo, esse predicado 
faz referência não somente as relações que não estão presentes bem como as relações que foram mal realizadas (não atingiram o propósito). 

O conceito de grupo está presente no \textit{MOISE+} mas não na proposta deste modelo. Isso recai na mesma condição do que se pretende demonstrar com o 
raciocínio \ref{refutmoisec} sendo que esse conceito não é necessário para os propósitos finais deste estudo. 

Os fundamentois conceituais apresentados em ambos os modelos no que tange a objetivos são os mesmos, que consiste em critérios que devem ser atingidos 
pelos agentes. Contudo, o \textit{MOISE+} lida com essa questão por tratar os objetivos em uma estrutura de grafos definido sub e super objetivos. Ainda 
nessa linha, o \textit{MOISE+} trabalha operadores que relacionam super, subobjetivos que são; sequencial, paralelo e escolha. Apesar de ser uma ideia 
interessante, os pesquisadores entenderam que escrever os objetivos em termos do predicado $nextGoal$ é abrangente, é simples e resolve o problema sendo este 
cenário ideal para uma primeira versão do modelo em proposta. Contudo, é de grande interesse investigar situações onde a abordagem adotada pelos pesquisadores 
não funciona a fim de ter de usar a estrutura em árvore presente no \textit{MOISE+} e verificar como isso se dá no que tange aos demais elementos do modelo 
proposto por eles.

O conceito de \textit{missão} está presente no \textit{MOISE+} mas não no modelo proposto neste estudo. Isso se dá porque a ideia de missão diminui 
a granularidade da relação entre o agente com os seus objetivos. Consequentemente, a verificação de um evento em um dado objetivo necessita mais etapas 
de raciocínio. Por exemplo, se existe uma situação onde um dado agente com papel $\rho_a$ tem uma missão onde $ hasObligation(\rho_a,m_1) $ em que 
$ m_1 = \{ ... , g_a, ...\}$ para que se possa verificar um dado evento que ocorre em $ g_a $, o raciocinador deverá antes consultar $ hasObligation(\rho_a,m
_1) $. Tendo em vista a granularidade das violações e sanções sobre os objetivos, os pesquisadores entederam ser aproprieado criar relacões diretas 
entre o papel $ \rho_a $ com o objetivo $g_1 $ isso elimina a quantidade de predicados que deve ser considerado nas relações de implicação. Se a linha 
adotada fosse a mesma do \textit{MOISE+}, então seria necessário considerar o seguinte termo $ g_i \in m_a $ e uma relação como \ref{rolenextgoal} 
teria de ser escrita da seguinte forma: 

\begin{eqnarray}\label{constmoise01}
	hasRole(ag_n,\rho_m) \wedge hasPermission(\rho_m,m_a) \wedge (g_i \in m_a) \wedge nextGoal(g_i,g_j) \wedge isReached(g_i) \nonumber \\
	\to ableTryReach(ag_i,g_j) \nonumber \\
    ag_i, ag_n \in Agent, \rho_m \in Role, g_j \in Goal, g_i \in Goal, m_a \in Mission
\end{eqnarray}

O \textit{MOISE+} não aborda os conceitos relacionados a norma, violação, sanção e possibilidades presentes na proposta deste estudo. Uma possível 
contestação dessa afirmação reside no fato de que \textit{MOISE+} contem as relaçoes deonticas que terminam permissões e obrigações aos agentes. Contudo,  
assim como na proposta deste estudo, essas relações são usadas para "dizer" ao agente o que ele deve fazer. De certa forma, é uma norma - mas que 
verifica apenas a execução do objetivo em si e não estabele relações violação ou sanção. Então, frente ao \textit{MOISE+} a inovação presente 
neste estudo reside no fato de que existe um vocabulário criado para especificar relações de violação, sanção e possibilidades em problemas 
onde agentes devem atuar de forma colaborativa para cumprir com um dado propósito. 
	
		\subsection{Modelo DASTANI}

As tabelas \ref{dastanitb1} e \ref{dastanitb2} apresentam uma comparação entre o \textit{DASTANI+} e o modelo proposto neste estudo.

\begin{table}[H]
\centering
\begin{tabular}{|l|p{0.3\linewidth}|p{0.3\linewidth}|}
\hline			
\textbf{Atributos}
& 
\textbf{DASTANI}
& 
\textbf{Modelo deste Estudo} 
\\ \hline
Finalidade
&
Modelo de \textit{SMA} que leva em consideração questões atreladas à normas, sanções e violações. 
&
Representar cenários de Acidentes 
\\ \hline
Agentes
&
Não define os estados mentais do agente
&
Não define os estados mentais do agente
\\ \hline
Generalização
&
Trata de cenários onde se tem o interesse de representar agentes, normas, violações e sanções. 
&
Trata de cenários específicos para acidentes com a finalidade de identificar as causas destes. 
\\ \hline
Normas 
&
Presença de regras que permite expressar violações e sanções para os mais diversos casos possíveis. 
&
Presença de regras que permite expressar violações e sanções para cenários de acientes.
\\ \hline
\end{tabular}
\caption{Comparação entre o \textit{DASTANI+} e o modelo proposto neste estudo}
\label{dastanitb1}
\end{table}


\begin{table}[H]
\centering
\begin{tabular}{|l|p{0.3\linewidth}|p{0.3\linewidth}|}
\hline			
\textbf{Atributos}
& 
\textbf{DASTANI}
& 
\textbf{Modelo deste Estudo}  
\\ \hline
Estrutura da Linguagem
& 
Linguagem de Programação baseada em Lógica de Predicados de Primeira Ordem 
&
Teoria de Conjuntos e Lógica de Predicados de Primeira Ordem
\\ \hline
Raciocínios
& 
Permite representar raciocínios que considera agentes cometendo ou não violações e tendo que lidar com as respectivas sanções. Esses raciocínios levam em consideração conceitos como eventos e ações (estados que geram a transição de eventos)
& 
Presença de regras que possibilitam reproduzir cenários de acidentes entre os agentes. Não apenas isso, mas faz parte da estrutura dessas regras questões atreladas à normas, violações e sanções. Esses raciocínios também representam cenários onde agentes sofrem consequências negativas mesmo não sendo responsáveis pelas causas dos acidentes e fazem isso por levar em consideração questões possibilísticas e influências entre relações.
\\ \hline
\end{tabular}
\caption{Comparação entre o \textit{DASTANI+} e o modelo proposto neste estudo}
\label{dastanitb2}
\end{table}
		\subsection{Modelo V3S}

V3S é um modelo com a finalidade de gerar ambientes para desenvolver treinamentos complexos em ambiente de realidade virtual visando atividades de risco e de emergência. O modelo é composto por três submodelos; \textit{Domain Model}, \textit{Activity Model} e \textit{Risk Model} \cite{v3sframework}. O \textit{Domain Model} é o núcleo do sistema. Todos os objetos, ações e relações são descritos por uma ontologia. 
A figura \ref{domainmodel} exibe a estrutura de classe desta ontologia.

\begin{figure}[H]
  \centering
  \includegraphics[width=0.5\linewidth]{figure/ontologyv3.png} 
  \caption{Ontologia que descreve \textit{Domain Model} no model V3S \cite{v3sframework}}
  \label{domainmodel}
\end{figure}


		\section{Comparação entre VS3 e o modelo proposto neste estudo}

A tabela \ref{vs3comparacao1}  apresentam uma comparação entre o \textit{VS3} e o modelo proposto neste estudo.

\begin{center}
\begin{longtable}[H]{|l|p{0.3\linewidth}|p{0.3\linewidth}|}
\hline			
\textbf{Atributos}
& 
\textbf{VS3}
& 
\textbf{Modelo deste Estudo} 
\\ \hline
Finalidade
&
Representar cenários de Acidentes a fim de simulá-los com propósito de treinamento profissional 
&
Representar cenários de Acidentes 
\\ \hline
Artefatos-Objetos
&
Representa objetos através da classe (da ontologia atrelada ao módulo \textit{Domain Model}) \textit{V3S-Object}
&
Representa objetos através do conjunto \textit{Artefact}
\\ \hline
Relações entre Entidades
&
Representa as relações entre objetos através da classe (da ontologia atrelada ao módulo \textit{Domain Model}) \textit{V3S-Action} através de relacionamentos com a classe \textit{V3S-Object}
&
Representa as relações entre as entidades através do predicado $possEntityRel(r_l,e_i,e_k)$
\\ \hline
Riscos e Acidentes
&
Modelo \textit{MELISSA} (BATU E BCTU)
&
Uso de Normas, Violações, Sanções e Lógica Possibilística aplicadas a contextos específicos por meio das regras \ref{conditionViol}, \ref{relationViol}, \ref{entityViol}, \ref{consconditionViol} e \ref{consrelationViol}
\\ \hline 
Agentes     	
& 
Uso do \textit{framework} MASVERP 
& 
Não representar os processos mentais do agente 
\\ \hline 
Estrutura da Linguagem
& 
Diferentes formalismos da computação tais como: Ontologias, Lógica de Primeira Ordem, Algoritmos e entre outros
&
Teoria de Conjuntos e Lógica de Predicados
\\ \hline 
Estruturas para Avaliação e Treinamento
& 
HERA
&
Não Consta
\\ \hline
\label{vs3comparacao1}
\end{longtable}
\end{center}

A lista a seguir apresenta uma comparação entre os costrutores do \textit{ACTIVITY-DL} e a parte deste modelo conceitual focada em representar as relações entre os objetivos. Observar tabela \ref{acticonstruct} 

\begin{enumerate}
	\item \begin{itemize}
			\item \textit{ACTIVITY-DL:}  A $IND$ B.
			\item \textit{Modelo:} $nextGoal(A,B)$,$nextGoal(B,A)$,$nextGoal(A,B) \to F$,$nextGoal(B,A) \to F$, $nextGoal(X,A) \wedge nextGoal(X,B) \wedge nextGoal(A,Y) \wedge nextGoal(B,Y)$, $nextGoal(X,A) \wedge nextGoal(X,B) \wedge nextGoal(A,Y) \wedge nextGoal(B,Y) \to F$   e outras situações que não podem ser expressadas nessa representação.
		\end{itemize}
	\item \begin{itemize}
			\item \textit{ACTIVITY-DL:} A $SEQ$ B.
			\item $nextGoal(A,B)$, $nextGoal(B,A)$ e outras situações que não podem ser expressadas nessa representação. 
		\end{itemize}
	\item \begin{itemize}
			\item \textit{ACTIVITY-DL:} A $SEQ-ORDER$ B.
			\item $nextGoal(A,B)$  e outras situações que não podem ser expressadas nessa representação.
		\end{itemize}
	\item \begin{itemize}
			\item \textit{ACTIVITY-DL:} A $PAR-SIM$ B, A $PAR-SIM$ B, A $PAR-START$ B, A $PAR-END$ B 
			\item $nextGoal(X,A) \wedge nextGoal(X,B) \wedge nextGoal(A,Y) \wedge nextGoal(B,Y) $ e outras situações que não podem ser expressadas nessa representação.  
		\end{itemize}
\end{enumerate} 
		\section{Modelo NormMAS}
\textit{NormMAS} é um modelo usado para definir comportamento normativo de sistemas multiagentes \cite{normas}. No que tange questões referentes ao 
comportamento normativo, o modelo trabalha com duas definições \cite{normas}.
		
		\textbf{Definição 1.} \textit{Um norma é definida por meio de uma tupla} $N = \langle \mu,\kappa,\chi,\tau,\rho \rangle$

\begin{itemize}
    \item $\mu \in \{obligation,prohibition\}$ representa as modalidades de norma.
    \item $\kappa \in \{action,state\}$ representa o tipo de \textit{trigger} da condição.
    \item $\chi$ representa o conjunto de estados em que uma norma se aplica.
    \item $\tau$ representa a norma da condição de \textit{trigger}
    \item $\rho$ representa a sanção aplicava pela violação do agente.
\end{itemize}

A definição 1 pode ser compreendida sobre o seguinte exemplo; 

\textit{Todos os imigrantes que possuem passaporte válido, devem ser aceitos. A falha resulta na perda de 5 créditos}. 

Dentro da definição 1, o exemplo fica;

\begin{eqnarray} 
    \langle obligation,action,valid(Passport),accept(Passport),loss(5)\rangle
\end{eqnarray}

\textit{NormMAS} define um \textit{Registro de ação} que é dado pela definição 2. 

\textbf{Definição 2.} \textit{Um Registro de Ação é definido por meio de uma tupla} $R = \langle \gamma,\alpha,\beta \rangle$

\begin{itemize}
    \item $\gamma$ representa o agente executando uma ação;
    \item $\alpha$ representa a ação sendo executada pelo agente $\gamma$
    \item $\beta$ representa os estados internos do agente $\gamma$ no momento da execução.
\end{itemize}

Para demonstrar como se dá o uso dessa definição pode-se considerar a seguinte sentença;

\textit{O oficial John aprovou passaport 3225. O passaporte 3225 é definido como validado.}

Nessa sentença, $John$ é o agente dado por $\gamma$, o ato de aprovar o passaporte é o $\gamma$ que pode ser definido pelo predicado por $approve(3225)$ e o estado de ser validado por ser dado pelo predicado $valid(3225)$.  

Usando a \textbf{Definição 2}, isso poder ser especificado da seguinte maneira; 

\begin{eqnarray} 
    \langle John, approve(3225), valid(3225)\rangle
\end{eqnarray}

As tabelas a seguir exibem uma comparação entre o \textit{NORMMAS} e o modelo proposto neste estudo.

\begin{table}[H]
\centering
\begin{tabular}{|l|p{0.3\linewidth}|p{0.3\linewidth}|}
\hline          
\textbf{Atributos}
& 
\textbf{NORMMAS}
& 
\textbf{Modelo deste Estudo} 
\\ \hline
Finalidade
&
Representar agentes considerando questões atreladas a Normas, Sanções e Violações
&
Representar cenários de Acidentes 
\\ \hline
Generalização
&
Trata de cenários onde se tem o interesse de representar agentes, normas, violações e sanções. 
&
Trata de cenários específicos para acidentes onde se deseja as causas do acidente. 
\\ \hline
Agentes
&
Não define os estados mentais do agente
&
Não define os estados mentais do agente
\\ \hline
Representação de Objetivos
& 
Não possui estrutura para representar objetivos
&
Possui uma estrutura simples para representar objetivos. Essa estrutura considera apenas questões de pré-requisitos. 
\\ \hline
Normas 
&
Presença de tuplas que permite expressar proibições e sanções para os mais diversos casos possíveis. 
&
Presença de regras que permite expressar violações e sanções para cenários de acidentes.
\\ \hline
Generalização
&
Trata de cenários onde se tem o interesse de representar agentes, normas, violações e sanções. 
&
Trata de cenários específicos para acidentes onde se desejeva verificar as causas do acidente.
\\ \hline
Estrutura da Linguagem
& 
Lógica de Primeira Ordem (definições dos conceitos em termos de Tuplas)
&
Teoria de Conjuntos e Lógica de Predicados  
\\ \hline
\end{tabular}
\end{table}
		\subsection{Comparação Genérica entre os Modelos}

Tendo como base as análises feitas nas seções anteriores, é possível chegar na tabela \ref{comparemodel} que apresenta uma análise comparativa dos arcabouços no que tange a expressividade do modelo computacional proposto nesse texto. Por expressividade, se entende capacidade de expressar, representar o objeto de interesse. Para essa análise 
foi feita a seguinte escala; nenhuma expressividade $\prec$ pouco expressivo $\prec$ expressivo $\prec$ muito expressivo $\prec$ altamente expressivo. O termo nenhuma expressividade não indica que é impossível definir a estrutura em observação dentro do modelo em voga, mas sim que o engenheiro de modelagem terá que criar uma estrutura conceitual ad hoc. Sobre o mesmo aspecto reside pouco expressivo, contudo o modelo - neste caso - possui algumas estruturas pré-definidas que diminuem o esforço da especificação. O termo expressivo deixa claro que o modelo permite especificar o objeto de interesse sem que o engenheiro tenha de criar muitos atributos para o domínio de interesse. O termo muito expressivo define que o modelo apresenta diversos conceitos específicos para representar o objeto em interesse, contudo ainda há margem para que o engenheiro de conhecimento tenha que criar um ou mais atributos. O termo altamente expressivo define o caso onde o modelo especifica o objetivo de interesse muito bem fazendo com que o engenheiro de conhecimento não precise definir nenhum critério conceitual a mais (ou terá que montar poucas definições).   

\begin{table}[H]
    \centering
    \scalefont{0.6}
    \begin{tabular}{|l|l|l|l|l|}
        \hline
        \textbf{Critérios} & \textbf{MOISE+}        & \textbf{DASTANI}     & \textbf{V3S}         & \textbf{NORMMAS}          \\ \hline
        \textbf{Agente}    & pouco                  & pouco                & muito                & pouco expressivo          \\ \hline
        \textbf{SMA}       & altamente              & pouco                & expressivo           & pouco expressivo          \\ \hline
        \textbf{Artefato}  & nenhuma                & pouco                & expressivo           & pouco expressivo          \\ \hline
        \textbf{Norma}     & nenhuma                & altamente            & pouco                & altamente expressivo      \\ \hline
        \textbf{Violação}  & nenhuma                & altamente            & pouco                & altamente expressivo      \\ \hline
        \textbf{Sanção}    & nenhuma                & altamente            & pouco                & altamente expressivo      \\ \hline
        \textbf{Risco}     & nenhuma                & pouco                & altamente            & pouco expressivo          \\ \hline
        \textbf{P.O.A.E}   & nenhuma                & pouco                & pouco                & pouco expressivo          \\ \hline
        \textbf{Objetivos} & muito                  & pouco                & muito                & pouco expressivo          \\ \hline
        \textbf{C.A}       & nenhuma                & pouco                & pouco                & pouco expressivo          \\ \hline
        \textbf{I.AG.AR}   & nenhuma                & pouco                & pouco                & pouco expressivo          \\ \hline
        \textbf{D.C.A}     & nenhuma                & pouco                & altamente            & pouco expressivo          \\ \hline
    \end{tabular}
    \caption{Análise comparativa sobre a expressividade desses modelos no que tange aos objetivos deste estudo. A sigla P.E.R significa Possibilidade de Algo Errado, a sigla C.A consiste em 
    Condições Ambientes, a sigla I.AG.AR significa Interação entre Agente e Artefato e a sigla D.C.A significa Descrição de Cenário de Acidente}
    \label{comparemodel}
\end{table}

O critério \textbf{Agente} condiz com representação dos estados internos que um agente pode ter. O critério \textbf{SMA} condiz com presença de elementos que são necessário para especificar um \textit{Sistema Multiagente}. O critério \textbf{Artefato} condiz com elementos que correspondem a definição presente na seção \ref{artefact}. O critérios de \textbf{Norma} corresponde a regras que devem ser acatadas pelos agentes. \textbf{Violação} define o que corresponde o não cumprimento de uma dada regra. \textbf{Sanção} implica penalidade que está sobre o agente. \textbf{Risco} consiste no evento ruim que tem um potencial de ocorrer sobre o agente. \textbf{P.O.A.E} significa Possibilidade de Ocorrer algo Errado e corresponde a expressar condições onde existe potencial de acontecer algo inapropriado sobre o agente mesmo que esse realize sua função com excelência.  \textbf{Objetivos} implica alvos que devem ser atingidos pelos agentes. \textbf{C.A} consiste em condições ambientes que interagem com a atividade executada pelos agentes. \textbf{I.AG.AR} representa as interações entre agentes e artefatos. \textbf{D.C.A} - Descrição de Cenários de Acidentes, consiste na capacidade de desenvolver raciocínios a fim de representar cenários de acidentes.

As subseções anteriores em conjunto com a tabela \ref{comparemodel} permite concluir que esse trabalho é inovador no que tange a ter um vocabulário específico para representar cenários de risco e de acidentes (tanto sobre o responsável pelo acidente bem como a vitima) de atividades manuais usando para isso, o conceito de sistema multiagente normativo. Esse vocabulário apresenta limitações cujos quais serão debatidas na próxima seção. 
	\section{Discussão do Modelo Conceitual}\label{constresult}
		A discussão dos resultados que estão expostos nas sub-seções \ref{mods}, \ref{predic}, \ref{regras} foi feita na própria apresentação dos mesmos. Isso se deve a natureza desses resultados uma vez que não é possível realizar a exposição deles sem discutir os fundamentos conceituais que justifiquem a existência dos mesmos. 

O mesmo não se aplica no texto presente em \ref{studycase} e em \ref{rac} onde os resultados estão apenas expostos mas não foram discutidos. O texto a seguir fará uma discussão desses elementos apresentando as principais dificuldades que foram encontradas na realização desse estudo. 

O primeiro ponto a ser analisado reside em escolher um caso de estudo que se adeque as condições do que se tem por interesse estudar. O caso em análise cumpre com essa finalidade pois apresenta um cenário onde profissionais usam ferramentas para trabalhar de forma colaborativa a fim de atingir um determinado objetivo. Não só isso, como esses profissionais são expostos a um dado risco e podem sofrer acidentes que advêm tanto de responsabilidade própria bem como responsabilidade do outro. Outro ponto que contribuiu para isso consiste no fato de que os pesquisadores têm acesso a profissionais da área de manutenção em linha viva tais como eletricistas, técnicos e um engenheiro de manutenção. Isso possibilitou os pesquisadores entenderem a dinâmica da atividade numa profundidade que não seria possível ao analisar outro caso de estudo. Consequentemente houve um contorno de um dos maiores problemas desse tipo de trabalho que é a distância em termos de conhecimento que engenheiro de modelagem tem do domínio a ser modelado. 

Apesar do problema de domínio do conhecimento ter sido contornado com um acesso fácil a profissionais da área, é digno de nota frisar que a compreensão da atividade propriamente dita não foi uma tarefa fácil e exigiu grande esforço dos pesquisadores. Muito possivelmente não seria viável compreender as atividades de linha viva sem a visita técnica para acompanhar um dado procedimento de manutenção, sem os documentos técnicos, sem a entrevista feita ao engenheiro de linha viva e sem o acesso a manuais técnicos da área. 

O caso de estudo em análise é um cenário que é totalmente possível de ser factual, contudo existe diversas outras possibilidades de organizar a mesma manutenção (ou até mesmo outros procedimentos de linha viva). Para a verificação dos modeladores esse cenário foi escolhido porque ele foi o caso mais bem estudado ao longo da pesquisa, porque cumpre com todos os critérios  que são necessários para aplicar esse modelo e porque é o mais simples entre todas as situações levantadas pelos estudiosos.

A primeira fase consistiu em descrever a manutenção em termos de objetivos que as vezes são organizados em série e as vezes em paralelo. Não houve grandes dificuldades para fazer isso, pois essa atividade é claramente composta de subatividades. O que foi um ponto relativamente complicado de se verificar nesse estudo é que os profissionais não precisam executar os objetivos na estrutura em que o modelo foi apresentado. Inclusive, muitas vezes os técnicos planejam a manutenção de um jeito e ao chegar no ambiente de execução eles mudam o encadeamento dos objetivos. Há um número finito e relativamente pequeno (é difícil definir um número, mas as observações dos pesquisadores permitem concluir que 10 formas diferentes consiste numa posição um tanto conservadora) sobre como esses objetivos podem ser organizados e isso ameniza a falta de previsibilidade de como a manutenção será realizada. 

O problema da organização dos objetivos pode ser resolvida de duas maneiras diferentes. Uma delas o engenheiro de manutenção modela o problema para todos os cenários possíveis. Portanto, se houver 10 formas diferentes de organizar esses objetivos, o engenheiro deverá refletir a cerca dessas 10 formas. Outra forma consiste em definir todas as relações possíveis que o predicado $nextGoal(g_i,g_j)$ permite em uma única estrutura. Nessa implementação do modelo o agente, por intermédio dos seus estados internos, escolhe a qual objetivo ele deverá tentar alcançar. Essa questão não foi levada em consideração no estudo de caso em análise porque os pesquisadores estavam interessados em realizar uma prova de conceito sobre a possibilidade de usar este modelo em um problema real. É possível argumentar que aquele arranjo de objetivos não é o único possível, contudo não deixa de ser um arranjo real e que pode ser muito bem executado pelos profissionais. 

Os pesquisadores desse estudo não sabem afirmar se o arranjo dos objetivos interferem em como a especificação do predicado $affectsOtherRelations(r_k,r_n)$ é afetada. Para que isso seja analisado se faz necessário aplicar esse modelo para diversas situações diferentes onde todas devem apresentar a problemática do arranjo de objetivos. Se em uma dessas situações a especificação do predicado $affectsOtherRelations(r_k,r_n)$ mudar, então a proposição "o arranjo de objetivos afeta o predicado $affectsOtherRelations(r_k,r_n)$" é verdadeira, contudo se não mudar não é possível afirmar que essa proposição é falsa. 

A concepção do conceito de "papel" e a concepção da relação entre o agente e o seu papel foram muito adequados para as análises desse modelo. Essa análise se deu por observar como se dá  a distribuição dos objetivos aos agentes. Isso, pois na manutenção em linha viva todos os profissionais são tidos como executores, contudo existe uma distribuição de tarefas tendo em vista o conhecimento e a experiência de cada profissional ali envolvido. Portanto, para enquadrar essa questão nos moldes do modelo em análise se fez necessário encontrar um padrão de como os objetivos são distribuídos em função das atividades dos agentes. Com base nisso os engenheiros de modelagem concluíram o que está exposto na tabela \ref{agentsroles}.

Os conceitos de artefato e de relação foram adequados para esse caso de estudo não tendo a necessidade de definir nenhuma outra abordagem para isso. Todo o rol de ferramentas e de equipamentos foram definidos como artefatos. A fim de tornar a modelagem mais expressiva os pesquisadores poderiam criar subconjuntos de artefatos definindo um apenas para tratativa de ferramentas e outro 
apenas para tratativa de equipamentos. Isso não foi feito porque os pesquisadores não quiseram induzir os leitores desse estudo ao erro por entender que essa divisão pertence a estrutura conceitual do modelo propriamente dito. Uma taxonomia dessas seria adequada apenas para esse caso em estudo, caso contrário diminuiria o poder de generalização do modelo. 

A tabela \ref{condition} apresenta as condições que estão relacionadas ao meio. Esse conceito também se apresentou muito apropriado ao modelo porque uma condição não é um agente e não é um artefato mas é algo que está presente no meio da atividade e interfere com grande intensidade no andamento dos processos, portanto desconsiderar esse conceito ou compacta-lo como parte de outras estruturas implicaria uma representação míope da realidade. Os pesquisadores entenderam que essas condições são o suficiente para poder realizar a representação desse modelo.

As tabelas \ref{condition}, \ref{relationEntEnt1} e \ref{relationEntEnt2} apresentam a relação entre entidade ou condição com o risco e a consequência. Nesse estudo foi considerado apenas um único risco, que é o de ser eletrocutado e uma única consequência que é a morte. Contudo, há considerações que devem ser feitas no que tange a realidade, pois essa atividade exibe outros riscos tais como; "queda", "animais peçonhentos", "queimadura" entre outros que, assim como "eletrocutado" podem apresentar outras consequências além da morte. Esses riscos a mais não foram considerados no caso em estudo porque a prova de conceito exibida na subseção \ref{rac} puderam ser feitas sem a necessidade deles. Outro ponto que corroborou com isso consiste no fato de que os engenheiros de modelagem estavam interessados em obter primeiramente uma versão mais simples do modelo para então, se necessário, torná-lo mais complexo. Isso implica realizar algumas escolhas pragmáticas e uma delas consiste na verificação de qual risco é o mais importante e o mais temeroso na atividade. A análise com os profissionais mostram que o risco de 
ser eletrocutado é o principal e é mais preocupante ao executar uma atividade de manutenção em linha viva. Outro ponto reside na verificação das consequências desse risco o que remete a uma pergunta; Um profissional de linha viva ao executar manutenção em uma subestação de energia pode se envolver em um acidente ondo ele é eletrocutado e ainda sim sobreviver? A resposta a essa pergunta é que sim, porém muito improvável. Descargas de equipamentos que operam a 69 kV 1500 kVA  (o que é relativamente baixo) costumam matar o profissional eletrocutado mesmo que 
os disjuntores atuem na ordem de milissegundos. Portanto, existe outras consequências além da morte tal como; queimaduras e perda de membros contudo na grande maioria dos casos o profissional 
recairá no óbito. 

As tabelas \ref{relationEntEnt1} e \ref{relationEntEnt2} apresentam como se dá a relação entre duas entidades. Essa estrutura se tornou muito útil para fazer diversos raciocínios interessantes que estão presentes nas regras. Portanto os pesquisadores concluem que ela foi adequada, necessária e importante para essa representação e para esse caso de estudo, contudo ela tornou a 
especificação da modelagem um processo muito custoso porque o engenheiro de modelagem teve de refletir em todas as relações possíveis que são executadas na atividade e, depois disso, teve 
que ver quais relações se enquadravam em cada objetivo. Esse custo também está presente nos raciocínios que devem ser feitos pois dependendo da situação há uma série de relações que devem 
ser avaliadas. Essa questão nos permite refletir sobre a viabilidade de um modelo assim para situações onde o número de artefatos bem como o número de relações entre esses artefatos tendem ao 
infinito. Contudo, o fato do engenheiro de modelagem ter de refletir sobre todas as relações bem como seus respectivos riscos permite a realização de uma análise muito mais profunda 
da atividade e de como a segurança dos profissionais pode ser afetada de situação para situação. 

As tabelas \ref{relation1}, \ref{relation2} e \ref{relation3} apresentam a especificação do predicado $affectsOtherRelations(r_k,r_n)$. Esse foi um problema sério para os pesquisadores porque houve muitas tentativas de tentar resolve essa questão sem ter que abstrair tanto quanto esse predicado faz. Contudo, realizar um mapeamento minucioso de como se dá as atividades resulta 
em uma carga de especificação muito grande e que pode apresentar diversas fragilidades no que tange a uma certa consistência lógica (ou seja, um sistema que se contradiz). Portanto, em uma primeira abordagem admitir que a não execução (ou a mal execução) de uma relação afeta negativamente outra relação implica uma visão pragmática e simples para resolver o problema 
onde um eletricista se envolve em um acidente sobre o qual ele não tem responsabilidade algum. Ainda sim, os pesquisadores desse estudo admitem que esse é um ponto do modelo a ser melhorado 
a fim de se obter representações consistentes, expressivas e em com relativo baixo custo computacional. 

As tabelas \ref{entities}, \ref{relationsgroup1}, \ref{relationsgroup2} e \ref{conditions} apresentam a compactação das relações e condições cujos vínculos com objetivos foram exibidos, posteriormente, nas tabelas \ref{goalsrelationsentity1} e \ref{goalsrelationsentity2}. Nesse caso de estudo esse procedimento foi muito apropriado porque permitiu observar de forma clara as entidades, condições e relações necessárias para que determinado objetivo possa ser cumprido. Contudo existe um problema nessa abordagem que foi observado após a realização da especificação desse caso de estudo e reside no fato de que esse modelo abre margem para o modelador entrar em contradição com certa facilidade. Isso pode ser demonstrado com o seguinte exemplo; em dada atividade, um objetivo $g_1$ é constituído por $e_1,e_2,e_3$ e é formado pelas relações $thereIsRelation(r_{12},e_1,e_2)$ e $thereIsRelation(r_{23},e_2,e_3)$. Supondo que ao executar essa modelagem, o modelador faça o seguinte: $rg_1 = \{ r_{12},r_{23} \}$ e $ eg_1 = \{ e_1, e_2 \} $, $ hasEntity(g_1,eg_1) $ e $ hasRelation(g_1, rg_1) $. Nesse caso o modelador se contradiz porque ele está considerando que $rg_{23}$ é uma relação entre $e_2, e_3$ que deve estar relacionada a $g_1$, mas $e_3$ não está relacionada a $g_1$. A especificação resulta em uma contradição entre os predicados $ hasEntity(g_1,eg_1) $ e $ hasRelation(g_1, rg_1) $, já que a verdade de um deve necessariamente tornar o outro falso. Para que isso não aconteça, o modelador deverá ter cautela em considerar todas as entidades usadas na relação para um dado objetivo. Esse problema pode ser resolvido por desconsiderar o predicado $hasEntity(g_n,eg_m)$ tendo em vista que um simples raciocínio usando $hasRelation(g_n,rg_m)$, $r_k \in rg_m$ e $thereIsRelation(r_k, e_a,e_b)$ podem ser o suficiente 
para apresentar todas as entidades de uma relação $ hasRelation(g_n, rg_m) $. Essa concepção não foi considerada por uma questão conservadora, pois quando o modelo estava sendo concebido, os pesquisadores consideraram a possibilidade de existir um cenário onde uma entidade fosse necessária para cumprir com um determinado objetivo mas que não estabelece relação com nenhuma outra entidade no ambiente. Nessa condição um raciocínio envolvendo $hasRelation(g_n,rg_m)$, $r_k \in rg_m$ e $thereIsRelation(r_k, e_a,e_b)$ traria apenas as entidades necessárias 
mas não as entidades suficientes. Contudo, após a realização da modelagem sobre esse caso de estudo ficou claro para os pesquisadores que uma situação onde uma entidade não se relaciona com as demais é um cenário absurdo. Isso, pois se a entidade não estabelecer relação alguma com as demais, então essa entidade é desnecessária e - perante as outras - não existe. O modelo como está não apresenta nenhuma condição absurda, contudo torna certa parte da especificação um processo redundante. Para as próximas versões, é possível pensar na eliminação do predicado 
$ hasEntity(g_n,eg_m) $ e de $ \{ e_i , ... , e_k \} \in eg_m $

Os raciocínios feitos sobre o modelo são de crucial importância para definir a eficácia desse projeto pois é com base nisso que se torna possível avaliar o quão efetivo vem a ser essa representação. Os raciocínios 1 e 5, dados respectivamente pelas subsubseções \ref{raciocinio1} \ref{raciocinio5} apresentam o problema com bastante expressividade. Tendo em vista o fato de que a Glicerina  é um composto químico relevante para manter o isolamento da parte não condutiva do bastão universal, esquecer de realizar isso gera um potencial acidente para ser eletrocutado em todas as outras situações onde o bastão será usado (a não ser nas situações onde o bastão universal não será usado em condutores energizados). Tanto os predicados como as regras que estão atreladas ao violação de relação e suas respectivas consequências representaram essa condição com sucesso. Nesse caso não aconteceu nenhuma sanção sobre o agente 4, portanto nem toda violação de relação gera necessariamente uma sanção. O predicado $possibilityHappensBadEvent$ conseguiu trazer com exito a sensação de possibilidade que existe em fenômeno desse gênero. 

Quando os pesquisadores estavam concebendo esse modelo, consideraram a possibilidade de trabalhar problemas dos raciocínios 1 e 5 por meio do conceito de \textit{Probabilidade}. Isso é interessante porque um predicado que consegue expressar momentos estatísticos com excelência apresenta possibilidades de aplicações extremamente elevados. Contudo, trabalhar com probabilidade resulta em diversas complicações de modelagem. Uma dessas complicações consiste no desenvolvimento de técnicas que podem indicar com rigor científico qual é a probabilidade de um acidente acontecer. Entretanto, isso não é o suficiente pois essa probabilidade é condicionada a ocorrência de uma dada relação. O raciocínio 1, por exemplo, demonstra que a não execução de $relPanoGlicerina$ resulta na possibilidade de ocorrer um acidente em $relBastaoGarraGondutor$. Se os pesquisadores estiverem trabalhando com o conceito de probabilidade, então é necessário desenvolver técnicas que verificam a probabilidade de acontecer algo ruim na relação $relBastaoGarraGondutor$ para o caso da relação $relPanoGlicerina$ não for efetivada com sucesso. Contudo, se se a não execução de uma outra relação também afeta $relBastaoGarraGondutor$, então também se faz necessário encontrar essa outra probabilidade. Além de aumentar a complexidade desse modelo, abre diversas indagações no que tange de como fazer isso o que pode ser um potencial campo de investigação científica. Com a finalidade de viabilizar uma primeira versão do modelo, os pesquisadores optaram por usar o conceito de possibilidade em vez de probabilidade. Apesar de diminuir a expressividade do modelo no que tange a questões que existe um componente sobre aleatoriedade, isso simplifica o processo de especificação, facilita o desenvolvimento de raciocínios e evita que o modelo seja estruturado sobre proposições falsas (por exemplo, definir uma probabilidade para uma condição 
de mundo onde isso não está claro). 

O vocabulário definido neste modelo foi apropriado para representar a condição de mundo presente no Raciocínio 2 que está exposto na subsubseção \ref{raciocinio2}. Em uma situação onde não há um pano para poder limpar todas as ferramentas, a manutenção é interrompida e essa situação ficou claramente representada por esse raciocínio onde a geração da violação de entidades corresponde a finalização da manutenção. Há a possibilidade de existir um cenário onde os profissionais criam algum tipo de técnica alternativa para poder transpassar a falta de algum artefato, inclusive se esse não apresentar grande complexidade estrutural como é o caso de um pano. Contudo, os pesquisadores decidiram por não incorporar esse tipo de situação no modelo por conta de complexidades que isso pode trazer a estrutura da representação. Manter o modelo assim permite representar os cenários mais prováveis, tendo vista que a ausência de diversos tipos de artefatos muitas vezes não permite a continuidade da atividade.    

A execução de uma manutenção em linha viva deve seguir a risca as condições ambientais adequadas para essa finalidade. Uma dessas condições é a umidade relativa do ar que deve estar necessariamente inferior a setenta porcento. O raciocínio 3 em \ref{raciocinio3} demonstra esse tipo de situação onde um agente tenta executar uma dada atividade com a umidade relativa dor ar em níveis inapropriados para isso ocasionando o surgimento de uma violação de condição gerando uma sanção no agente que corresponde a ser eletrocutado e, consequentemente, morto. É interessante observar que nem toda violação de condição, no mundo real, resulta necessariamente em uma sanção ao violador. A umidade relativa do ar recai nessa situação, pois pode ser que o profissional cometa essa violação sem se envolver em um acidade. Isso pode ser resolvido por construir regras tratando condições em relação ao predicado $possibilityHappensBadEvent$. Contudo, a desobediência de condições ambientes normalmente resultam em acidentes. Portanto, essa condição - apesar de não tratar todos os cenários possíveis, trata um bom número dos mesmos. Além disso, essa representação é conservadora no que tange ao evento propriamente dito, pois considera sempre o pior 
cenário possível.

A chave catraca é usada pelo profissional de linha viva para remover um parafuso que está preso ao conector. Uma execução inapropriada dessa relação resulta na ocorrência do eletricista ser eletrocutado e morto. Há diversas formas de como isso pode acontecer sendo que uma delas consiste no profissional se posicionar de forma inapropriada para realizar essa relação e, por consequência, esbarrar tanto com o corpo quanto com a ferramenta em alguma condutor de forma inapropriada. Portanto é de crucial importância que o profissional realize a execução com excelência. Esse comportamento é descrito pelo Raciocínio dado na subsubseção \ref{raciocinio4}. Assim como na situação relacionada condição, a realidade dos fatos pode produzir cenários possíveis "nesse caso" que não são adequadamente representados por esse modelo. Um possível cenário para essa situação consiste no fato do profissional simplesmente não conseguir executar a relação, sem que isso resulte em algum acidente. Contudo, a situação descrita pelo modelo apresenta o pior cenário possível. 

Em um acidente pessoas que não são responsáveis pelos atos cometidos podem sofrer duras consequências por conta disso. Essa situação está demonstrada no raciocínio 5 presente na subsubseção \ref{raciocinio5}.Nessa situação, não passar glicerina no pano pode gerar um acidente ao montar a relação parafuso conector porque o bastão isolante a ser usado nesse  processo não estará em condições operacionais seguras, uma vez que a superfície dessa ferramenta pode 
conter algum tipo de impureza que corrobore com aumento de corrente de fuga em níveis suficientemente altos para matar alguém. O raciocínio 1 apresentou
com excelência essa influência que a falta do uso de glicerina tem sobre a possibilidade de ocorrer algo errado no momento onde um profissional 
removerá o parafuso usando o bastão. Os pesquisadores entendem, portanto que todos os predicados usados para representar essa situação foram necessários 
sendo que a ausência de um ou de outro descaracteriza completamente essa representação. Nessa condição, se faz necessário saber com qual objetivo 
$relParafusoConector$ está atrelado e isso é feito por intermédio do $\in$ e de $hasRelation(g_n,rg_m)$. Além disso, não é possível efetuar nenhum 
tipo de raciocínio sobre essa condição sem levar em consideração se os agentes tentam alcançar esse objetivo, e isso é feito por meio do predicado 
$tryReach(agent_n,g_m)$. O fato de ocorrer um acidente ser independente do agente que está executando o objetivo é muito bem representado pelos predicados 
$possibilityHappensBadEvent$ e $happensBadEvent$. Pela regra \ref{paybutiamnotguilty}, essa reunião de fatores em conjunto com os riscos associados 
ao evento dão como verdade para o predicado $consequenceOfBadEven$ gerando a morte do profissional e a interrupção da manutenção. O que é interessante nessa parte da representação consiste no fato de que até mesmo a possibilidade do acidente não acontecer 
é algo a ser computado nesse raciocínio. Portanto, de todo o modelo, os pesquisadores entendem que essa é a parte que é representada com maior precisão. 

A presença do raciocínio 6 dado em \ref{raciocinio6} demonstrou que o modelo é capaz de interpretar quando o objetivo $g23$ foi atingido. Em 
conjunto com o predicado $ableTryReach(ag_i,g_j)$, com a programação dos estados internos do agente e com a regra \ref{rolenextgoal} é possível 
verificar uma relação de continuidade para a representação. Nessa situação o agente sempre saberá quais objetivos foram finalizados, saberá 
quais são os próximos objetivos, deverá ter a capacidade de decisão sobre qual objetivo ele deverá tentar alcançar 
(dando como verdade $tryRach(ag_n,g_m)$) e deverá ter a capacidade de executar o objetivo. 

O raciocínio 7 dado por \ref{raciocinio7} apresenta como o modelo se comporta quando é feito uma consulta que não corresponde a realidade. O resultado objetivo foi o que os pesquisadores esperavam. 

Os pesquisadores observam que os problemas que possuem as mesmas características que as presentes no caso de estudo são representados de forma apropriada
pelos predicados e regras presentes nesse texto desde que o interesse consista verificar como se dá situações atreladas a cenários de acidentes. Contudo, 
muitos raciocínios não foram capazes de apresentar considerações no que tange a todos os cenários possíveis ao qual esses estavam relacionados. Inclusive, 
mesmo que os pesquisadores conseguissem demonstrar que os sete raciocínios apresentam capacidade de representar todos os cenários, ainda sim não seria 
possível demonstrar que isso é válido ao aplicar essa representação em outro caso de estudo. Contudo, o modelo foi capaz de apresentar situações factíveis 
de acontecerem no mundo real. 

Os pesquisadores não podem afirmar que as situações apresentadas sempre são as mais conservadoras possíveis (até porque o ato de como se dá a modelagem é algo que influência isso), contudo para esse caso em específico os cenários resultantes sempre foram os piores que podem acontecer dentro de um cenário de manutenção com essas características. 
\label{chap:anacomp}
\chapter{Conclusão}
\label{chpa:conc}
	Esse capítulo tem como por finalidade realizar uma consideração sobre todo estudo, verificar quais são as conclusões que podem ser feitas e discernir sobre perspectivas futuras para essa pesquisa.
	\section{Avaliação do Objetivo}
		A primeira parte da sentença que define o objetivo geral desse estudo é "Obter um modelo conceitual". Os pesquisadores entendem que essa parte do objetivo foi atingido com êxito por que parte dos resultados apresentam uma estrutura formalizada em termos de conjuntos e predicados que definem a estrutura de um modelo. 

A segunda parte da sentença sobre do objetivo aqui posto é; "que define os conceitos, relações e operações". Esse estudo também cumpriu com esse objetivo porque os conjuntos fazem referência a conceitos (respaldado por observações empíricas e análise da literatura) e tanto as relações como as operações são conceitualizadas por meio dos predicados. Algo que agrega a essa observação reside no fato de que essas estruturas foram usadas para usar um dado caso de estudo.

A terceira parte da sentença sobre o objetivo geral é; "para representar cenários de ambientes de acidentes". Os pesquisadores entendem que esse objetivo foi alcançado com sucesso porque os fundamentos dos vocábulos (conjuntos e predicados) bem como as regras são baseados em observações de cenários de acidentes reais e  em estudos que lidam com elementos correspondente a essa problemática. Outro argumento que pode ser usado em prol da veracidade da proposição reside no fato de que a estrutura conceitual resultante foi usada para representar um dado cenário de acidente. Tanto a formalização bem como os raciocínios correspondem, não em totalidade mas ao menos nos elementos principais, boa parte da realidade no que tange ao estudo de caso. 

Os pesquisadores concluem que esse modelo conceitual apresenta uma formalização e conceitualização da problemática que há em situações onde existe riscos de acidentes (permitindo, portanto, entender a profundidade do tema), mapeia a rede de conceitos que devem ser especificados por algum dado arcabouço e possibilita arbitrar a utilização especifica de um dado arcabouço.
	\section{Trabalhos Futuros}
		Esse estudo abre margem para muitos trabalhos futuros. Alguns desses residem no fato de que este texto apresenta análises de certos arcabouços na representação do modelo conceitual aqui posto. Tendo em vista isso, para cada arcabouço é possível derivar um estudo futuro a fim de usar o modelo conceitual aqui posto para conceber a formulação de requisitos e especificações. 

Outro estudo futuro possível reside numa reformulação e em um refinamento da estrutura conceitual aqui posta a fim de analisar com maior riqueza de detalhes a discussão presente em \ref{conscritmetcasoestudo} sobre $hasEntity(g_j,e_i)$ e $hasRelation(g_j,r_i)$. A invetigação dos conceitos presentens nesse modelo para formular análises de probabilidades discutido em \ref{conscritmetrac} também apresenta um grande potencial para estudos futuros.

O texto presente em \ref{conscritmetrac} discute, para cada um dos cinco primeiros raciocínios, elementos que foram muito bem representados assim como elementos que não foram representados adequadamente. Verificar como cada um desses problemas podem ser resolvidos também resultam em estudos futuros. 

\clearpage % this is need for add +1 to pageref of bibstart used in 'ficha catalografica'.
\label{bibstart}
\bibliography{reflatex} % geracao automatica das referencias a partir do arquivo reflatex.bib
\label{bibend}

\apendice
\chapter{Nome do Ap\^endice}

\anexo
\chapter{Nome do Anexo}


\end{document}

compnormas