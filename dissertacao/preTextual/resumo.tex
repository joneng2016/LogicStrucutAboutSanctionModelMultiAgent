% resumo em português
\begin{resumo}
 Diversas pessoas são submetidas a algum tipo de risco ao executar atividades profissionais relacionadas à área da eletricidade, petroquímica, transportes e outras. Com base em entrevista com um Engenheiro de Manutenção, participação \textbf{in loco} em uma atividade de manutenção na área de transmissão de energia elétrica e revisão bibliográfica exploratória, esse estudo sintetiza as informações coletadas, constrói e avalia um modelo conceitual destinado à construção de sistemas computacionais de simulação de riscos e acidentes com fins diversos, tais como, treinamento de operadores ou estudo de riscos associados a uma atividade laboral. O modelo conceitual construído fundamenta-se nos conceitos de agentes, artefatos, sistemas multiagentes normas e riscos/consequências. Atividades podem ser expressas em termos de objetivos com pré-requisitos e efeitos. A preocupação central é a modelagem de riscos e de suas consequências. Deste modo, pré-requisitos de atividades que não estão presentes quando os agentes decidem executá-las levam a violações que podem causar consequências para o objetivo atual e/ou subsequentes (ex. parada) assim como para os agentes envolvidos (ex. acidente). O modelo, formalizado em Lógica de Primeira Ordem, foi implementado em PROLOG e aplicado a um estudo de caso permitindo avaliar se as inferências produzidas correspondiam ao esperado bem como compreender o estado do problema particular modelado. Além disso, esta formalização permitiu realizar uma análise comparativa (em relação a cenários de acidentes) entre quatro modelos, MOISE+, Modelo de Agentes Normativos de DASTANI, V3S e NORMMAS. Como conclusão, o modelo é uma síntese de vários modelos existentes reunindo em um só as características relevantes para tal tipo de modelagem. Também, se revelou capaz de representar diversos cenários atrelados a atividades profissionais de risco, em especial, mostrou-se capaz de produzir resultados coerentes com o esperado no estudo de caso.

 \vspace{\onelineskip}
    
 \noindent
 \textbf{Palavras-chaves}: agentes, atividades, objetivos, normas, riscos, violação, consequências, modelo conceitual.
\end{resumo}

% resumo em inglês
\begin{resumo}[Abstract]
 \begin{otherlanguage*}{english}
Many people are at risk when performing professional activities related to electricity, petrochemicals, transportation and others. Based on an interview with a Maintenance Engineer, on-site participation in a maintenance activity in the area of ​​electricity transmission and exploratory literature review, this study synthesizes the information collected, builds and assesses a conceptual model for the construction of computer simulation systems. hazards and accidents for various purposes, such as operator training or study of risks associated with a labor activity. The constructed conceptual model is based on the concepts of agents, artifacts, multi-agent systems, norms and risks / consequences. Activities can be expressed in terms of goals with prerequisites and effects. The central concern is risk modeling and its consequences. Thus, prerequisites for activities that are not present at the time agents decide to perform them lead to violations that may have consequences for the current and / or subsequent goals (e.g. stopping) as well as for the agents involved (e.g. accident). The model, formalized in First Order Logic, was implemented in PROLOG and applied to a case study allowing to evaluate if the inferences produced corresponded to the expected ones as well as to assess the state of the problem modeled. This representation made it possible to perform a comparative analysis (in relation to accident scenarios modeling) between four models, MOISE +, DASTANI Normative Agents Model, V3S and NORMS. In conclusion, the model is a synthesis of several existing models bringing together in one the relevant characteristics for this type of modeling. Besides, if it was able to represent several scenarios related to risky professional activities, in particular, it was able to produce coherent results with the expected ones in the case study.
   \vspace{\onelineskip}
 
   \noindent 
   \textbf{Key-words}: agents, activities, goals, norms, risks, violation, consequences, conceptual model.
 \end{otherlanguage*}
\end{resumo}

