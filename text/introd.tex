\documentclass[12pt]{article}

\usepackage{sbc-template}
\usepackage[utf8]{inputenc}
\usepackage{graphicx,url}
\usepackage[brazil]{babel}   
\usepackage[usestackEOL]{stackengine}
\usepackage[latin1]{inputenc}  
\usepackage{scalefnt}
\usepackage{float}

\usepackage{amssymb}
\usepackage{amsmath}
\usepackage{graphicx}

\sloppy

\title{Estrutura conceitual do Modelo para Agentes Normativos}

\begin{document} 

\section{Objetivo}

Um dos interesses deste estudo reside na investigação, dentro do contexto computacional, de cenários onde os agentes cometem erros durante a execução de algum atividade. Tendo em vista esses erros, eles podem prejudicar a si mesmos como seus colegas. Assim sendo, esse estudo investiga esse cenário por meio de um modelo que incorpora normas (instruções claras que devem ser fornecidas a um agente), violação (ocorre quando um agente descumpri uma norma) e sanção (consequências decorrentes de uma violação desta norma). Não apenas isso, mas nesse cenário há a consideração de que essas questões estão atreladas aos artefatos que são usados pelos agentes. Esses artefatos são objetos passivos que estão sujeitos a influencia da ação dos agentes. Os artefatos são usados para representar objetos como ferramentas e máquinas, por exemplo. 

O autor considerou, neste estudo, que as violações não estão relacionadas apenas aos artefatos, mas também nas relações entre todas as entidades existentes ao longo do cenário. As entidades compõem todos os elementos ativos (agentes) bem como os elementos passivos (artefatos) vinculados ao meio. Assim sendo, entidade é todo o ente que existe por si mesmo. As condições ambientais onde os agentes e os artefatos estão inseridos também são considerados neste estudo. O entendimento que se tem por condição ambiental consiste em um dado processo que existe no ambiente e que, de certa forma, pode influenciar a execução das atividades dos agentes. Como condições ambientais têm-se por exemplo os seguintes eventos; chuva, sol, vento e neve. 

Não é do interesse deste estudo investigar condições normativas cuja caráter punitivo advém de sanções administrativas e jurídicas que ocorre no que tange a decisão de uma certa autoridade. Assim sendo, o modelo resultante deve ser capaz de representar situações onde os agentes sofrem consequências físicas negativas resultantes de uma cadeia de causalidade que foi ocasionada pelo erro de alguém (usando, para isso, o conceito de norma, violação e sanção). Dada essa circunstância, a representação tratada neste estudo deve trabalhar, em sua estrutura interna, o conceito de risco. É digno de nota que o termo risco apresenta um espectro semântico bastante amplo podendo ser usado nos mais diferentes contextos possíveis (ex. risco financeiro de um dado ativo). Por conta disto, é necessário frisar que neste texto o vocábulo risco é usado no mesmo significado atribuído pela comunidade de Engenharia de Segurança. 

As relações causais, em muitos casos, são complexas demais para serem mapeadas em seu mínimos detalhes. Por conta disso, muitos estudos realizam uma tratativa probabilística dessas relações. O modelo a que se propõem esse estudo não pretende tratar nenhuma das duas linhas. Para isso, o autor decidiu por fazer uso do conceito de possibilidade. Essa situação é usada para compreender casos onde um agente sofre consequências negativas por meio de eventos que possuem uma certa natureza aleatória. O cenário neste estudo se restringe apenas aos mundos onde esses eventos ruins (apresentam consequências ruins aos profissionais envolvidos) se tornam possíveis dado ao vacilo de algum profissional em uma atividade anterior.   


Muitas atividades praticas são definidas em termos de objetivos. Não apenas isso, mas essa linha é muito bem verificada pela acadêmica científica e está presente em modelos de \textit{SMA} dos mais diversos, tal como o \textit{MOISE}. Assim sendo, a representação a que se propõem neste estudo orienta os agentes em termos de objetivos que devem ser atingidos. Um objetivo define um estado de mundo $S_g$. Assim sendo, se o mundo está um estado $S_a$ onde $S_a \neq S_g$, os agentes devem fazer o possível para que o estado do mundo seja $S_g$. Neste estudo, um estado mundo é dado por todos os estados de todas as entidades.

Não é possível definir um agente dentro de uma sociedade sem ao menos definir o seu papel (ou função). Por exemplo, se há o interesse em construir um modelo de \textit{SMA} que seja apropriado para descrever o cenário de um hospital, essa representação pode ser entendida como demasiadamente pobre se não levar em consideração a função do médico. Assim sendo, o modelo presente neste texto deve considerar o papel do agente. 

Agentes são máquinas de estado que possuem autonomia para tomar decisões. Em termos genéricos, é possível classificar dois tipos de agentes; reativos e cognitivos. Os agentes reativos apenas reagem a estímulos do meio. A outra classe de agentes são os cognitivos, possuem estados internos e tem a capacidade de realizar raciocínios. O tipo de agente (bem como sua estrutura interna no que diz respeito a tomada de decisões) não faz parte do objeto de estudo desta pesquisa. Assim sendo, o modelo em interesse não delimita o vocabulário para estruturar a concepção do agente propriamente dito. 


Neste estudo, o autor tem interesse em apresentar um vocabulário específico no que tange aos conceitos presentes nesta seção. Esse modelo, portanto, deve ser capaz de representar organizações tais como trabalhadores em uma obra, industrias, profissionais no âmbito hospitalar e estruturas deste gênero entendendo como se dá as violações em âmbitos específicos (falta de ferramentas, não conseguir executar procedimentos apropriados, executar uma atividade cujo momento não era adequado para isso). 

A fim de se obter uma estrutura representacional formal com aspecto de especificidade altamente notório, este estudo faz uso da teoria dos modelos (onde os conceitos são escritos em termos de conjuntos) e lógica de predicados (usado para definir as relações entre os conceitos). Não apenas isso, há o interesse em definir regras que tem como por finalidade exprimir a transição de estados possessível de mundo. Assim sendo, essas regras, neste modelo, são definidas em termos de relações de implicabilidade que são delineadas pelas \textit{Cláusulas de Horn}.


Para o propósito aqui posto, se não há infinitas maneiras, há ao menos um numero muito variado de formas para construir um modelo com os objetivos aqui definidos. Contudo, esse texto desbrava ao menos uma das formas de se realizar isso, analisa a abordagem sobre um estudo e define uma comparação com os modelos já existentes dada pela comunidade acadêmica. A análise comparativa é estruturada em termos de; conceitos (verificar quais são similares e quais são diferentes), relações entre os conceitos, capacidade de generalização (quanto mais mundos possíveis o modelo é capaz de descrever, mais genérico o é), capacidade de especificação tendo como referência circunstâncias que correspondem ao presente neste texto introdutório e uma verificação sobre como esses critérios se dão dos modelos em relação a um dado estudo de caso aqui presente. 

\bibliographystyle{sbc}
\bibliography{sbc-template}
\end{document} 
	
